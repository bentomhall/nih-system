\clearpage
\section{Monstrosities}\label{sec:monstrosities}
Monstrosities are the catch-all category for things that aren't shaped roughly like a person. Does it speak but look like a \hyperref[sec:beasts]{beast}? Does it have magical abilities but not fit into another category? Was it created by a mage in a long-ago day but now breed normally? Then it's probably a monstrosity.

\begin{DndMonster}{Ankheg}
\DndMonsterType{Large monstrosity}
\DndMonsterBasics[armor-class={14 (natural armor), 11 while prone}, hit-points={39 (6d10 + 6)}, speed={30 ft., burrow 10 ft.}]
\MonsterStats{+3}{+0}{+1}{-5}{+1}{-2}
\DndMonsterDetails[saving-throws={}, skills={}, damage-immunities={}, damage-resistances={}, damage-vulnerabilities={}, condition-immunities={}, senses={darkvision 60 ft., tremorsense 60 ft., passive Perception 11}, languages={—}, challenge={2:1}]
\DndMonsterSection{Actions}
\DndMonsterMelee[
    name=Bite,
    mod=+5,
    dmg=\DndDice{2d6+3},
    dmg-type=slashing,
    plus-dmg=\DndDice{1d6},
    plus-dmg-type=acid,
    extra={If the target is a Large or smaller creature, it is grappled (escape DC 13). Until this grapple ends, the ankheg can bite only the grappled creature and has advantage on attack rolls to do so.}
]
\DndMonsterAction{Spit} The ankheg spits acid in a line that is 30 feet long and 5 feet wide, provided that it has no creature grappled. Each creature in that line must make a DC 13 Dexterity saving throw, taking 10 (3d6) acid damage on a failed save, or half as much damage on a successful one.
\end{DndMonster}

\begin{DndMonster}{Basilisk}
	\DndMonsterType{Medium monstrosity}
	\DndMonsterBasics[armor-class={15 (natural armor)}, hit-points={52 (8d8 + 16)}, speed={20 ft.}]
	\MonsterStats{+3}{-1}{+2}{-4}{-1}{-2}
	\DndMonsterDetails[saving-throws={}, skills={}, damage-immunities={}, damage-resistances={}, damage-vulnerabilities={}, condition-immunities={}, senses={darkvision 60 ft., passive Perception 9}, languages={—}, challenge={2:3}]
	\DndMonsterAction{Petrifying Gaze} If a creature starts its turn within 30 feet of the basilisk and the two of them can see each other, the basilisk can force the creature to make a DC 12 Constitution saving throw if the basilisk isn't incapacitated. On a failed save, the creature magically begins to turn to stone and is restrained. It must repeat the saving throw at the end of its next turn. On a success, the effect ends. On a failure, the creature is petrified until freed by the \nameref{inc:restoration} incantation (greater only) or other magic.
	
	A creature that isn't surprised can avert its eyes to avoid the saving throw at the start of its turn. If it does so, it can't see the basilisk until the start of its next turn, when it can avert its eyes again. If it looks at the basilisk in the meantime, it must immediately make the save.
	
	If the basilisk sees its reflection within 30 feet of it in bright light, it mistakes itself for a rival and targets itself with its gaze.
	
	\DndMonsterSection{Actions}
	\DndMonsterMelee[
			name=Bite,
			mod=+5,
			dmg=\DndDice{2d6 + 3},
			dmg-type=piercing,
			plus-dmg=\DndDice{2d6},
			plus-dmg-type=poison
	]
\end{DndMonster}
	
\begin{DndMonster}[float*=hb,width=\textwidth+8pt]{Behir}
	\begin{multicols}{2}
	\DndMonsterType{Huge monstrosity}
	\DndMonsterBasics[armor-class={17 (natural armor)}, hit-points={168 (16d12 + 64)}, speed={50 ft., climb 40 ft.}]
	\MonsterStats{+6}{+3}{+4}{-2}{+2}{+1}
	\DndMonsterDetails[saving-throws={}, skills={Perception +6, Stealth +7}, damage-immunities={lightning}, damage-resistances={}, damage-vulnerabilities={}, condition-immunities={}, senses={darkvision 90 ft., passive Perception 16}, languages={Draconic}, challenge={13:10}]
	\DndMonsterSection{Actions}
	\DndMonsterAction{Multiattack} The behir makes two attacks: one with its bite and one to constrict.
	\DndMonsterAttack[
		name=Bite,
		distance=melee,
		type=weapon,
		mod=+10,
		reach=10,
		dmg=\DndDice{3d10 + 6},
		dmg-type=piercing
	]
	\DndMonsterAttack[
		name=Constrict,
		distance=melee,
		type=weapon,
		mod=+10,
		reach=5,
		dmg=\DndDice{2d10 + 6},
		dmg-type=bludgeoning,
		extra={ plus 17 (2d10 + 6) slashing damage. The target is grappled (escape DC 16) if the behir isn't already constricting a creature, and the target is restrained until this grapple ends.}
	]
	
	\DndMonsterAction{Lightning Breath (Recharge 5-6)} 
	The behir exhales a line of lightning that is 20 feet long and 5 feet wide. Each creature in that line must make a DC 16 Dexterity saving throw, taking 66 (12d10) lightning damage on a failed save, or half as much damage on a successful one.
	
	\DndMonsterAction{Swallow}
	The behir makes one bite attack against a Medium or smaller target it is grappling. If the attack hits, the target is also swallowed, and the grapple ends. While swallowed, the target is blinded and restrained, it has total cover against attacks and other effects outside the behir, and it takes 21 (6d6) acid damage at the start of each of the behir's turns. A behir can have only one creature swallowed at a time.
	If the behir takes 30 damage or more on a single turn from the swallowed creature, the behir must succeed on a DC 14 Constitution saving throw at the end of that turn or regurgitate the creature, which falls prone in a space within 10 feet of the behir. If the behir dies, a swallowed creature is no longer restrained by it and can escape from the corpse by using 15 feet of movement, exiting prone.
	\end{multicols}
\end{DndMonster}

\begin{DndMonster}{Land Shark}
	\DndMonsterType{Large monstrosity}
	\DndMonsterBasics[armor-class={17 (natural armor)}, hit-points={94 (9d10 + 45)}, speed={40 ft., burrow 40 ft.}]
	\MonsterStats{+4}{+0}{+5}{-4}{+0}{-3}
	\DndMonsterDetails[saving-throws={}, skills={Perception +6}, damage-immunities={}, damage-resistances={}, damage-vulnerabilities={}, condition-immunities={}, senses={darkvision 60 ft., tremorsense 60 ft., passive Perception 16}, languages={—}, challenge={5:6}]
	\DndMonsterAction{Standing Leap} The land shark's long jump is up to 30 feet and its high jump is up to 15 feet, with or without a running start.
	
	\DndMonsterSection{Actions}
	\DndMonsterAttack[
		name=Bite,
		distance=melee,
		type=weapon,
		mod=+7,
		reach=5,
		dmg=\DndDice{4d12 + 4},
		dmg-type=piercing
	]
	\DndMonsterAction{Deadly Leap}
	If the land shark jumps at least 15 feet as part of its movement, it can then use this action to land on its feet in a space that contains one or more other creatures. Each of those creatures must succeed on a DC 16 Strength or Dexterity saving throw (target's choice) or be knocked prone and take 14 (3d6 + 4) bludgeoning damage plus 14 (3d6 + 4) slashing damage. On a successful save, the creature takes only half the damage, isn't knocked prone, and is pushed 5 feet out of the land shark's space into an unoccupied space of the creature's choice. If no unoccupied space is within range, the creature instead falls prone in the land shark's space.
\end{DndMonster}

\begin{DndMonster}{Chimera}
	\DndMonsterType{Large monstrosity}
	\DndMonsterBasics[armor-class={14 (natural armor)}, hit-points={114 (12d10 + 48)}, speed={30 ft., fly 60 ft.}]
	\MonsterStats{+4}{+0}{+4}{-4}{+2}{+0}
	\DndMonsterDetails[saving-throws={}, skills={Perception +8}, damage-immunities={}, damage-resistances={}, damage-vulnerabilities={}, condition-immunities={}, senses={darkvision 60 ft., passive Perception 18}, languages={understands Draconic but can't speak}, challenge={8:6}]
	\DndMonsterSection{Actions}
	\DndMonsterAction{Multiattack} The chimera makes three attacks: one with its bite, one with its horns, and one with its claws. When its fire breath is available, it can use the breath in place of its bite or horns.
	\DndMonsterAttack[
		name=Bite,
		distance=melee,
		type=weapon,
		mod=+7,
		reach=5,
		dmg=\DndDice{2d6 + 4},
		dmg-type=piercing
	]
	\DndMonsterAttack[
		name=Horns,
		distance=melee,
		type=weapon,
		mod=+7,
		reach=5,
		dmg=\DndDice{1d12 + 4},
		dmg-type=bludgeoning
	]
	\DndMonsterAttack[
		name=Claws,
		distance=melee,
		type=weapon,
		mod=+7,
		reach=5,
		dmg=\DndDice{2d6 + 4},
		dmg-type=slashing
	]
	\DndMonsterAction{Fire Breath (Recharge 5-6)}
	The dragon head exhales fire in a 15-foot cone. Each creature in that area must make a DC 15 Dexterity saving throw, taking 31 (7d8) fire damage on a failed save, or half as much damage on a successful one.
\end{DndMonster}

\begin{DndMonster}{Chuul}
	\DndMonsterType{Large monstrosity}
	\DndMonsterBasics[armor-class={16 (natural armor)}, hit-points={93 (11d10 + 33)}, speed={30 ft., swim 30 ft.}]
	\MonsterStats{+4}{+0}{+3}{-3}{+0}{-3}
	\DndMonsterDetails[saving-throws={}, skills={Perception +4}, damage-immunities={poison}, damage-resistances={}, damage-vulnerabilities={}, condition-immunities={poisoned}, senses={darkvision 60 ft., passive Perception 14}, languages={understands Deep Speech but can't speak}, challenge={4:5}]
	\DndMonsterAction{Amphibious} The chuul can breathe air and water.

	\DndMonsterAction{Sense Magic} The chuul senses magic within 120 feet of it at will. This trait otherwise works like the \nameref{inc:sense-aura} incantation but isn't itself magical.

	\DndMonsterSection{Actions}
	\DndMonsterAction{Multiattack} The chuul makes two pincer attacks. If the chuul is grappling a creature, the chuul can also use its tentacles once.
	\DndMonsterAttack[
		name=Pincer,
		distance=melee,
		type=weapon,
		mod=+6,
		reach=10,
		dmg=\DndDice{2d6 + 4},
		dmg-type=bludgeoning,
		extra={. The target is grappled (escape DC 14) if it is a Large or smaller creature and the chuul doesn't have two other creatures grappled.}
	]
	\DndMonsterAction{Tentacles}
	One creature grappled by the chuul must succeed on a DC 13 Constitution saving throw or be poisoned for 1 minute. Until this poison ends, the target is paralyzed. The target can repeat the saving throw at the end of each of its turns, ending the effect on itself on a success.
\end{DndMonster}

\begin{DndMonster}{Cloaker}
\begin{multicols}{2}
\DndMonsterType{Large monstrosity}
\DndMonsterBasics[armor-class={14 (natural armor)}, hit-points={78 (12d10 + 12)}, speed={10 ft., fly 40 ft.}]
\MonsterStats{+3}{+2}{+1}{+1}{+1}{+2}
\DndMonsterDetails[saving-throws={}, skills={Stealth +5}, damage-immunities={}, damage-resistances={}, damage-vulnerabilities={}, condition-immunities={}, senses={darkvision 60 ft., passive Perception 11}, languages={Deep Speech, Undercommon}, challenge={6:8}]
\DndMonsterAction{Damage Transfer} While attached to a creature, the cloaker takes only half the damage dealt to it (rounded down), and that creature takes the other half.

\DndMonsterAction{False Appearance} While the cloaker remains motionless without its underside exposed, it is indistinguishable from a dark leather cloak.

\DndMonsterAction{Light Sensitivity} While in bright light, the cloaker has disadvantage on attack rolls and Wisdom (Perception) checks that rely on sight.

\DndMonsterSection{Actions}
\DndMonsterAction{Multiattack} The cloaker makes two attacks: one with its bite and one with its tail.
\DndMonsterAttack[
	name=Bite,
	distance=melee,
	type=weapon,
	mod=+6,
	reach=5,
	dmg=\DndDice{2d6 + 3},
	dmg-type=piercing,
	extra={, and if the target is Large or smaller, the cloaker attaches to it. If the cloaker has advantage against the target, the cloaker attaches to the target's head, and the target is blinded and unable to breathe while the cloaker is attached. While attached, the cloaker can make this attack only against the target and has advantage on the attack roll. The cloaker can detach itself by spending 5 feet of its movement. A creature, including the target, can take its action to detach the cloaker by succeeding on a DC 16 Strength check.}
]
\DndMonsterAttack[
	name=Tail,
	distance=melee,
	type=weapon,
	mod=+6,
	reach=10,
	dmg=\DndDice{1d8 + 3},
	dmg-type=slashing
]
\DndMonsterAction{Moan}
Each creature within 60 feet of the cloaker that can hear its moan and that isn't an aberration must succeed on a DC 13 Wisdom saving throw or become frightened until the end of the cloaker's next turn. If a creature's saving throw is successful, the creature is immune to the cloaker's moan for the next 24 hours

\DndMonsterAction{Phantasms (Recharges after a Short or Long Rest)}
The cloaker magically creates three illusory duplicates of itself if it isn't in bright light. The duplicates move with it and mimic its actions, shifting position so as to make it impossible to track which cloaker is the real one. If the cloaker is ever in an area of bright light, the duplicates disappear.
Whenever any creature targets the cloaker with an attack or a harmful spell while a duplicate remains, that creature rolls randomly to determine whether it targets the cloaker or one of the duplicates. A creature is unaffected by this magical effect if it can't see or if it relies on senses other than sight.
A duplicate has the cloaker's AC and uses its saving throws. If an attack hits a duplicate, or if a duplicate fails a saving throw against an effect that deals damage, the duplicate disappears.
\end{multicols}
\end{DndMonster}

\begin{DndMonster}{Cockatrice}
\DndMonsterType{Small monstrosity}
\DndMonsterBasics[armor-class={11}, hit-points={27 (6d6 + 6)}, speed={20 ft., fly 40 ft.}]
\MonsterStats{-2}{+1}{+1}{-4}{+1}{-3}
\DndMonsterDetails[saving-throws={}, skills={}, damage-immunities={}, damage-resistances={}, damage-vulnerabilities={}, condition-immunities={}, senses={darkvision 60 ft., passive Perception 11}, languages={—}, challenge={1/4:1/2}]
\DndMonsterSection{Actions}
\DndMonsterAttack[
	name=Bite,
	distance=melee,
	type=weapon,
	mod=+3,
	reach=5,
	dmg=\DndDice{1d4 + 1},
	dmg-type=piercing,
	extra={, and the target must succeed on a DC 11 Constitution saving throw against being magically petrified. On a failed save, the creature begins to turn to stone and is restrained. It must repeat the saving throw at the end of its next turn. On a success, the effect ends. On a failure, the creature is petrified for 24 hours.}
]
\end{DndMonster}

\begin{DndMonster}[width=\textwidth + 8pt]{Darkmantle}
	\DndMonsterType{Small monstrosity}
	\DndMonsterBasics[armor-class={11}, hit-points={22 (5d6 + 5)}, speed={10 ft., fly 30 ft.}]
	\MonsterStats{+3}{+1}{+1}{-4}{+0}{-3}
	\DndMonsterDetails[saving-throws={}, skills={Stealth +3}, damage-immunities={}, damage-resistances={}, damage-vulnerabilities={}, condition-immunities={}, senses={blindsight 60 ft., passive Perception 10}, languages={—}, challenge={1:1/2}]
	\DndMonsterAction{Echolocation} The darkmantle can't use its blindsight while deafened.
	
	\DndMonsterAction{False Appearance} While the darkmantle remains motionless, it is indistinguishable from a cave formation such as a stalactite or stalagmite.
	
	\DndMonsterSection{Actions}
	\DndMonsterAttack[
		name=Crush,
		distance=melee,
		type=weapon,
		mod=+5,
		reach=5,
		dmg=\DndDice{1d6 + 3},
		dmg-type=bludgeoning,
		extra={, and the darkmantle attaches to the target. If the target is Medium or smaller and the darkmantle has advantage on the attack roll, it attaches by engulfing the target's head, and the target is also blinded and unable to breathe while the darkmantle is attached in this way. While attached to the target, the darkmantle can attack no other creature except the target but has advantage on its attack rolls. The darkmantle's speed also becomes 0, it can't benefit from any bonus to its speed, and it moves with the target. A creature can detach the darkmantle by making a successful DC 13 Strength check as an action. On its turn, the darkmantle can detach itself from the target by using 5 feet of movement.}
	]
	
	\DndMonsterAction{Darkness Aura (1/Day)}
	A 15-foot radius of magical darkness extends out from the darkmantle, moves with it, and spreads around corners. The darkness lasts as long as the darkmantle maintains concentration, up to 10 minutes (as if concentrating on a spell). Darkvision can't penetrate this darkness, and no natural light can illuminate it. If any of the darkness overlaps with an area of light created by a spell of 2nd level or lower, the spell creating the light is dispelled.
\end{DndMonster}

\begin{DndMonster}{Death Dog}
\DndMonsterType{Medium monstrosity}
\DndMonsterBasics[armor-class={12}, hit-points={39 (6d8 + 12)}, speed={40 ft.}]
\MonsterStats{+2}{+2}{+2}{-4}{+1}{-2}
\DndMonsterDetails[saving-throws={}, skills={Perception +5, Stealth +4}, damage-immunities={}, damage-resistances={}, damage-vulnerabilities={}, condition-immunities={}, senses={darkvision 120 ft., passive Perception 15}, languages={—}, challenge={1:1}]
\DndMonsterAction{Two-Headed} The dog has advantage on Wisdom (Perception) checks and on saving throws against being blinded, charmed, deafened, frightened, stunned, or knocked unconscious.

\DndMonsterSection{Actions}
\DndMonsterAction{Multiattack} The dog makes two bite attacks.
\DndMonsterAttack[
	name=Bite,
	distance=melee,
	type=weapon,
	mod=+4,
	reach=5,
	dmg=\DndDice{1d6 + 2},
	dmg-type=piercing,
	extra={. If the target is a creature, it must succeed on a DC 12 Constitution saving throw against disease or become poisoned until the disease is cured. Every 24 hours that elapse, the creature must repeat the saving throw, reducing its hit point maximum by 5 (1d10) on a failure. This reduction lasts until the disease is cured. The creature dies if the disease reduces its hit point maximum to 0.}
]
\end{DndMonster}

\begin{DndMonster}{Doppelganger}
	\DndMonsterType{Medium monstrosity (shapechanger)}
	\DndMonsterBasics[armor-class={14}, hit-points={52 (8d8 + 16)}, speed={30 ft.}]
	\MonsterStats{+0}{+4}{+2}{+0}{+1}{+2}
	\DndMonsterDetails[saving-throws={}, skills={Deception +6, Insight +3}, damage-immunities={}, damage-resistances={}, damage-vulnerabilities={}, condition-immunities={charmed}, senses={darkvision 60 ft., passive Perception 11}, languages={Common}, challenge={2:2}]
	\DndMonsterAction{Shapechanger} The doppelganger can use its action to polymorph into a Small or Medium humanoid it has seen, or back into its true form. Its statistics, other than its size, are the same in each form. Any equipment it is wearing or carrying isn't transformed. It reverts to its true form if it dies.
	
	\DndMonsterAction{Ambusher} The doppelganger has advantage on attack rolls against any creature it has surprised.
	
	\DndMonsterAction{Surprise Attack} If the doppelganger surprises a creature and hits it with an attack during the first round of combat, the target takes an extra 10 (3d6) damage from the attack.
	
	\DndMonsterSection{Actions}
	\DndMonsterAction{Multiattack} The doppelganger makes two melee attacks.
	\DndMonsterAttack[
		name=Slam,
		distance=melee,
		type=weapon,
		mod=+6,
		reach=5,
		dmg=\DndDice{1d6 + 4},
		dmg-type=bludgeoning
	]
	\DndMonsterAction{Read Thoughts}
	The doppelganger magically reads the surface thoughts of one creature within 60 feet of it. The effect can penetrate barriers, but 3 feet of wood or dirt, 2 feet of stone, 2 inches of metal, or a thin sheet of lead blocks it. While the target is in range, the doppelganger can continue reading its thoughts, as long as the doppelganger's concentration isn't broken (as if concentrating on a spell). While reading the target's mind, the doppelganger has advantage on Wisdom (Insight) and Charisma (Deception, Intimidation, and Persuasion) checks against the target.
\end{DndMonster}
	
\begin{DndMonster}{Drider}
	\begin{multicols}{2}
	\DndMonsterType{Large monstrosity}
	\DndMonsterBasics[armor-class={19 (natural armor)}, hit-points={123 (13d10 + 52)}, speed={30 ft., climb 30 ft.}]
	\MonsterStats{+3}{+3}{+4}{+1}{+2}{+1}
	\DndMonsterDetails[saving-throws={}, skills={Perception +5, Stealth +9}, damage-immunities={}, damage-resistances={}, damage-vulnerabilities={}, condition-immunities={}, senses={darkvision 120 ft., passive Perception 15}, languages={Abyssal, one other}, challenge={9:8}]
	\DndMonsterAction{Fey Ancestry} The drider has advantage on saving throws against being charmed, and magic can't put the drider to sleep.
	
	\DndMonsterAction{Innate Spellcasting} The drider's innate spellcasting ability is Wisdom (spell save DC 13). The drider can innately cast the following spells, requiring no material components:
	\begin{itemize}
		\item[] \textbf{Concentration} \nameref{spell:faerie-fire} (1x), \nameref{spell:darkness} (1x)
		\item[] \textbf{Other} \nameref{spell:dancing-lights}
	\end{itemize}
	
	\DndMonsterAction{Spider Climb} The drider can climb difficult surfaces, including upside down on ceilings, without needing to make an ability check.
	
	\DndMonsterAction{Sunlight Sensitivity} While in sunlight, the drider has disadvantage on attack rolls, as well as on Wisdom (Perception) checks that rely on sight.
	
	\DndMonsterAction{Web Walker} The drider ignores movement restrictions caused by webbing.
	
	\DndMonsterSection{Actions}
	\DndMonsterAction{Multiattack} The drider makes three attacks, either with its longsword or its longbow. It can replace one of those attacks with a bite attack.
	\DndMonsterAttack[
		name=Bite,
		distance=melee,
		type=weapon,
		mod=+6,
		reach=5,
		dmg=\DndDice{2d8 + 3},
		dmg-type=piercing,
		extra={ plus 9 (2d8) poison damage.}
	]
	\DndMonsterAttack[
		name=Longsword,
		distance=melee,
		type=weapon,
		mod=+6,
		reach=5,
		dmg=\DndDice{2d10 + 3},
		dmg-type=slashing,
		plus-dmg=\DndDice{1d8},
		plus-dmg-type=poison
	]
	\DndMonsterAttack[
		name=Longbow,
		distance=ranged,
		type=weapon,
		mod=+6,
		range=150/600,
		dmg=\DndDice{2d8 + 3},
		dmg-type=piercing,
		extra={ plus 4 (1d8) poison damage.}
	]
	\end{multicols}
\end{DndMonster}

\begin{DndMonster}[width=\textwidth + 8pt]{Ettercap}
	\begin{multicols}{2}
	\DndMonsterType{Medium monstrosity}
	\DndMonsterBasics[armor-class={13 (natural armor)}, hit-points={44 (8d8 + 8)}, speed={30 ft., climb 30 ft.}]
	\MonsterStats{+2}{+2}{+1}{-2}{+1}{-1}
	\DndMonsterDetails[saving-throws={}, skills={Perception +3, Stealth +4, Survival +3}, damage-immunities={}, damage-resistances={}, damage-vulnerabilities={}, condition-immunities={}, senses={darkvision 60 ft., passive Perception 13}, languages={—}, challenge={2:1}]
	\DndMonsterAction{Spider Climb} The ettercap can climb difficult surfaces, including upside down on ceilings, without needing to make an ability check.

	\DndMonsterAction{Web Sense} While in contact with a web, the ettercap knows the exact location of any other creature in contact with the same web.

	\DndMonsterAction{Web Walker} The ettercap ignores movement restrictions caused by webbing.

	\DndMonsterSection{Actions}
	\DndMonsterAction{Multiattack} The ettercap makes two attacks: one with its bite and one with its claws.
	\DndMonsterAttack[
		name=Bite,
		distance=melee,
		type=weapon,
		mod=+4,
		reach=5,
		dmg=\DndDice{1d8 + 2},
		dmg-type=piercing,
		extra={ plus 4 (1d8) poison damage. The target must succeed on a DC 11 Constitution saving throw or be poisoned for 1 minute. The creature can repeat the saving throw at the end of each of its turns, ending the effect on itself on a success.}
	]
	\DndMonsterAttack[
		name=Claws,
		distance=melee,
		type=weapon,
		mod=+4,
		reach=5,
		dmg=\DndDice{2d4 + 2},
		dmg-type=slashing
	]
	\DndMonsterAttack[
		name=Web (Recharge 5-6),
		distance=ranged,
		type=weapon,
		mod=+4,
		range=30/60,
		dmg=\DndDice{1d6},
		dmg-type=poison,
		extra={. The creature is restrained by webbing. As an action, the restrained creature can make a DC 11 Strength check, escaping from the webbing on a success. The effect also ends if the webbing is destroyed. The webbing has AC 10, 5 hit points, vulnerability to fire damage, and immunity to bludgeoning, poison, and psychic damage.}
	]
	\end{multicols}
\end{DndMonster}

\begin{DndMonster}{Grick}
	\DndMonsterType{Medium monstrosity}
	\DndMonsterBasics[armor-class={14 (natural armor)}, hit-points={40 (9d8)}, speed={30 ft., climb 30 ft.}]
	\MonsterStats{+2}{+2}{+0}{-4}{+2}{-3}
	\DndMonsterDetails[saving-throws={}, skills={}, damage-immunities={}, damage-resistances={}, damage-vulnerabilities={}, condition-immunities={}, senses={darkvision 60 ft., passive Perception 12}, languages={—}, challenge={2:1}]
	\DndMonsterAction{Stone Camouflage} The grick has advantage on Dexterity (Stealth) checks made to hide in rocky terrain.
	
	\DndMonsterSection{Actions}
	\DndMonsterAction{Multiattack} The grick makes one attack with its tentacles. If that attack hits, the grick can make one beak attack against the same target.
	\DndMonsterAttack[
		name=Tentacles,
		distance=melee,
		type=weapon,
		mod=+4,
		reach=5,
		dmg=\DndDice{2d6 + 2},
		dmg-type=slashing
	]
	\DndMonsterAttack[
		name=Beak,
		distance=melee,
		type=weapon,
		mod=+4,
		reach=5,
		dmg=\DndDice{1d6 + 2},
		dmg-type=piercing
	]
\end{DndMonster}
	
\begin{DndMonster}{Griffon}
	\DndMonsterType{Large monstrosity}
	\DndMonsterBasics[armor-class={12}, hit-points={59 (7d10 + 21)}, speed={30 ft., fly 80 ft.}]
	\MonsterStats{+4}{+2}{+3}{-4}{+1}{-1}
	\DndMonsterDetails[saving-throws={}, skills={Perception +5}, damage-immunities={}, damage-resistances={}, damage-vulnerabilities={}, condition-immunities={}, senses={darkvision 60 ft., passive Perception 15}, languages={—}, challenge={3:2}]
	\DndMonsterAction{Keen Sight} The griffon has advantage on Wisdom (Perception) checks that rely on sight.
	
	\DndMonsterSection{Actions}
	\DndMonsterAction{Multiattack} The griffon makes two attacks: one with its beak and one with its claws.
	\DndMonsterAttack[
		name=Beak,
		distance=melee,
		type=weapon,
		mod=+6,
		reach=5,
		dmg=\DndDice{1d8 + 4},
		dmg-type=piercing
	]
	\DndMonsterAttack[
		name=Claws,
		distance=melee,
		type=weapon,
		mod=+6,
		reach=5,
		dmg=\DndDice{2d6 + 4},
		dmg-type=slashing
	]
\end{DndMonster}

\begin{DndMonster}{Harpy}
	\DndMonsterType{Medium monstrosity}
	\DndMonsterBasics[armor-class={11}, hit-points={38 (7d8 + 7)}, speed={20 ft., fly 40 ft.}]
	\MonsterStats{+1}{+1}{+1}{-2}{+0}{+1}
	\DndMonsterDetails[saving-throws={}, skills={}, damage-immunities={}, damage-resistances={}, damage-vulnerabilities={}, condition-immunities={}, senses={passive Perception 10}, languages={Common}, challenge={1:1}]
	\DndMonsterSection{Actions}
	\DndMonsterAction{Multiattack} The harpy makes two attacks: one with its claws and one with its club.
	\DndMonsterAttack[
		name=Claws,
		distance=melee,
		type=weapon,
		mod=+3,
		reach=5,
		dmg=\DndDice{2d4 + 1},
		dmg-type=slashing
	]
	\DndMonsterAttack[
		name=Club,
		distance=melee,
		type=weapon,
		mod=+3,
		reach=5,
		dmg=\DndDice{1d4 + 1},
		dmg-type=bludgeoning
	]
	\DndMonsterAction{Luring Song}
	The harpy sings a magical melody. Every beast, monstrosity, humanoid, or giant within 300 feet of the harpy that can hear the song must succeed on a DC 11 Wisdom saving throw or be charmed until the song ends. The harpy must take a bonus action on its subsequent turns to continue singing. It can stop singing at any time. The song ends if the harpy is incapacitated.
	
	While charmed by the harpy, a target is incapacitated and ignores the songs of other harpies. If the charmed target is more than 5 feet away from the harpy, the target must move on its turn toward the harpy by the most direct route, trying to get within 5 feet. It doesn't avoid opportunity attacks, but before moving into damaging terrain, such as lava or a pit, and whenever it takes damage from a source other than the harpy, the target can repeat the saving throw. A charmed target can also repeat the saving throw at the end of each of its turns. If the saving throw is successful, the effect ends on it.
	
	A target that successfully saves is immune to this harpy's song for the next 24 hours.
\end{DndMonster}
		
\begin{DndMonster}{Hippogriff}
	\DndMonsterType{Large monstrosity}
	\DndMonsterBasics[armor-class={11}, hit-points={25 (4d10 + 4)}, speed={40 ft., fly 60 ft.}]
	\MonsterStats{+3}{+1}{+1}{-4}{+1}{-1}
	\DndMonsterDetails[saving-throws={}, skills={Perception +5}, damage-immunities={}, damage-resistances={}, damage-vulnerabilities={}, condition-immunities={}, senses={passive Perception 15}, languages={—}, challenge={2:1/2}]
	\DndMonsterAction{Keen Sight} The hippogriff has advantage on Wisdom (Perception) checks that rely on sight.
	
	\DndMonsterSection{Actions}
	\DndMonsterAction{Multiattack} The hippogriff makes two attacks: one with its beak and one with its claws.
	\DndMonsterAttack[
		name=Beak,
		distance=melee,
		type=weapon,
		mod=+5,
		reach=5,
		dmg=\DndDice{1d10 + 3},
		dmg-type=piercing
	]
	\DndMonsterAttack[
		name=Claws,
		distance=melee,
		type=weapon,
		mod=+5,
		reach=5,
		dmg=\DndDice{2d6 + 3},
		dmg-type=slashing
	]
\end{DndMonster}
		
\begin{DndMonster}{Hydra}
	\DndMonsterType{Huge monstrosity}
	\DndMonsterBasics[armor-class={15 (natural armor)}, hit-points={172 (15d12 + 75)}, speed={30 ft., swim 30 ft.}]
	\MonsterStats{+5}{+1}{+5}{-4}{+0}{-2}
	\DndMonsterDetails[saving-throws={}, skills={Perception +6}, damage-immunities={}, damage-resistances={}, damage-vulnerabilities={}, condition-immunities={}, senses={darkvision 60 ft., passive Perception 16}, languages={—}, challenge={8:10}]
	\DndMonsterAction{Hold Breath} The hydra can hold its breath for 1 hour.
	
	\DndMonsterAction{Multiple Heads} The hydra has five heads. While it has more than one head, the hydra has advantage on saving throws against being blinded, charmed, deafened, frightened, stunned, and knocked unconscious.
	
	Whenever the hydra takes 25 or more damage in a single turn, one of its heads dies. If all its heads die, the hydra dies.
	
	At the end of its turn, it grows two heads for each of its heads that died since its last turn, unless it has taken fire damage since its last turn. The hydra regains 10 hit points for each head regrown in this way.
	
	\DndMonsterAction{Reactive Heads} For each head the hydra has beyond one, it gets an extra reaction that can be used only for opportunity attacks.
	
	\DndMonsterAction{Wakeful} While the hydra sleeps, at least one of its heads is awake.
	
	\DndMonsterSection{Actions}
	\DndMonsterAction{Multiattack} The hydra makes as many bite attacks as it has heads.
	\DndMonsterAttack[
		name=Bite,
		distance=melee,
		type=weapon,
		mod=+8,
		reach=10,
		dmg=\DndDice{1d10 + 5},
		dmg-type=piercing
	]

	\DndMonsterSection{Variants}
	\DndMonsterAction{Cryohydra} Cryohydras gain a breath weapon (recharge 6 unless they have more than 5 heads, in which case recharge 4-6)--as an action, they breath a line of cold 40ft long and 10 ft wide from all their heads. Targets in the area must make a DC 15 Constitution saving throw, taking 3d6 cold damage per head on a failure or half as much on a success. This increases the OR by 2.
	\DndMonsterAction{Pyrohydra} Pyrohydras gain a breath weapon (recharge 6 unless they have more than 5 heads, in which case recharge 4-6)--as an action, they breath a line of fire 40ft long and 10 ft wide from all their heads. Targets in the area must make a DC 15 Constitution saving throw, taking 3d6 fire damage per head on a failure or half as much on a success. This increases the OR by 2. Their regeneration is stopped by cold, not fire.
	\DndMonsterAction{Venom Hydra} Venom Hydras gain a breath weapon (recharge 6 unless they have more than 5 heads, in which case recharge 4-6)--as an action, they breath a line of poison 40ft long and 10 ft wide from all their heads. Targets in the area must make a DC 15 Constitution saving throw, taking 3d6 poison damage per head on a failure or half as much on a success. This increases the OR by 2.
\end{DndMonster}

\begin{DndMonster}[width=\textwidth + 8pt]{Kraken}
	\begin{multicols}{2}
	\DndMonsterType{Gargantuan monstrosity (titan)}
	\DndMonsterBasics[armor-class={18 (natural armor)}, hit-points={472 (27d20 + 189)}, speed={20 ft., swim 60 ft.}]
	\MonsterStats{+10 (+17)}{+0 (+7)}{+7 (+14)}{+6 (+13)}{+4 (+11)}{+5}
	\DndMonsterDetails[saving-throws={Str +17, Dex +7, Con +14, Int +13, Wis +11}, skills={}, damage-immunities={lightning}, damage-resistances={}, damage-vulnerabilities={}, condition-immunities={frightened, paralyzed}, senses={truesight 120 ft., passive Perception 14}, languages={understands Abyssal, Celestial, Infernal, and Primordial but can't speak, telepathy 120 ft.}, challenge={20++}]
	\DndMonsterAction{Amphibious} The kraken can breathe air and water.
	
	\DndMonsterAction{Freedom of Movement} The kraken ignores difficult terrain, and magical effects can't reduce its speed or cause it to be restrained. It can spend 5 feet of movement to escape from nonmagical restraints or being grappled.
	
	\DndMonsterAction{Siege Monster} The kraken deals double damage to objects and structures.
	
	\DndMonsterSection{Actions}
	\DndMonsterAction{Multiattack} The kraken makes three tentacle attacks, each of which it can replace with one use of Fling.
	\DndMonsterAttack[
		name=Bite,
		distance=melee,
		type=weapon,
		mod=+17,
		reach=5,
		dmg=\DndDice{3d8 + 10},
		dmg-type=piercing,
		extra={. If the target is a Large or smaller creature grappled by the kraken, that creature is swallowed, and the grapple ends. While swallowed, the creature is blinded and restrained, it has total cover against attacks and other effects outside the kraken, and it takes 42 (12d6) acid damage at the start of each of the kraken's turns. \\ If the kraken takes 50 damage or more on a single turn from a creature inside it, the kraken must succeed on a DC 25 Constitution saving throw at the end of that turn or regurgitate all swallowed creatures, which fall prone in a space within 10 feet of the kraken. If the kraken dies, a swallowed creature is no longer restrained by it and can escape from the corpse using 15 feet of movement, exiting prone.}
	]
	\DndMonsterAttack[
		name=Tentacle,
		distance=melee,
		type=weapon,
		mod=+17,
		reach=30,
		dmg=\DndDice{3d6 + 10},
		dmg-type=bludgeoning,
		extra={, and the target is grappled (escape DC 18). Until this grapple ends, the target is restrained. The kraken has ten tentacles, each of which can grapple one target.}
	]
	\DndMonsterAction{Fling}
	One Large or smaller object held or creature grappled by the kraken is thrown up to 60 feet in a random direction and knocked prone. If a thrown target strikes a solid surface, the target takes 3 (1d6) bludgeoning damage for every 10 feet it was thrown. If the target is thrown at another creature, that creature must succeed on a DC 18 Dexterity saving throw or take the same damage and be knocked prone.
	\DndMonsterAction{Lightning Storm (recharge 5-6)}
	The kraken magically creates three bolts of lightning, each of which can strike a target the kraken can see within 120 feet of it. A target must make a DC 23 Dexterity saving throw, taking 22 (4d10) lightning damage on a failed save, or half as much damage on a successful one.
	
	\DndMonsterSection{Legendary Actions}
	The Kraken can take 3 legendary actions, choosing from the options below. Only one legendary action option can be used at a time and only at the end of another creature's turn. The Kraken regains spent legendary actions at the start of its turn.
	\begin{DndMonsterLegendaryActions}
	\DndMonsterLegendaryAction{Tentacle Attack or Fling}{The kraken makes one tentacle attack or uses its Fling.}
	\DndMonsterLegendaryAction{Lightning Storm (Costs 2 Actions)}{The kraken uses Lightning Storm.}
	\DndMonsterLegendaryAction{Ink Cloud (Costs 3 Actions)}{While underwater, the kraken expels an ink cloud in a 60-foot radius. The cloud spreads around corners, and that area is heavily obscured to creatures other than the kraken. Each creature other than the kraken that ends its turn there must succeed on a DC 23 Constitution saving throw, taking 16 (3d10) poison damage on a failed save, or half as much damage on a successful one. A strong current disperses the cloud, which otherwise disappears at the end of the kraken's next turn.}
	\end{DndMonsterLegendaryActions}
	\end{multicols}
\end{DndMonster}

\begin{DndMonster}{Manticore}
\DndMonsterType{Large monstrosity}
\DndMonsterBasics[armor-class={14 (natural armor)}, hit-points={68 (8d10 + 24)}, speed={30 ft., fly 50 ft.}]
\MonsterStats{+3}{+3}{+3}{-2}{+1}{-1}
\DndMonsterDetails[saving-throws={}, skills={}, damage-immunities={}, damage-resistances={}, damage-vulnerabilities={}, condition-immunities={}, senses={darkvision 60 ft., passive Perception 11}, languages={Common}, challenge={3:3}]
\DndMonsterAction{Tail Spike Regrowth} The manticore has twenty-four tail spikes. Used spikes regrow when the manticore finishes a long rest.

\DndMonsterSection{Actions}
\DndMonsterAction{Multiattack} The manticore makes three attacks: one with its bite and two with its claws or three with its tail spikes.
\DndMonsterAttack[
	name=Bite,
	distance=melee,
	type=weapon,
	mod=+5,
	reach=5,
	dmg=\DndDice{1d8 + 3},
	dmg-type=piercing
]
\DndMonsterAttack[
	name=Claw,
	distance=melee,
	type=weapon,
	mod=+5,
	reach=5,
	dmg=\DndDice{1d6 + 3},
	dmg-type=slashing
]
\DndMonsterAttack[
	name=Tail Spike,
	distance=ranged,
	type=weapon,
	mod=+5,
	range=100/200,
	dmg=\DndDice{1d8 + 3},
	dmg-type=piercing
]
\end{DndMonster}

\begin{DndMonster}[float*=b,width=\textwidth + 8pt]{Medusa}
\begin{multicols}{2}
\DndMonsterType{Medium monstrosity}
\DndMonsterBasics[armor-class={15 (natural armor)}, hit-points={127 (17d8 + 51)}, speed={30 ft.}]
\MonsterStats{+0}{+2}{+3}{+1}{+1}{+2}
\DndMonsterDetails[saving-throws={}, skills={Deception +5, Insight +4, Perception +4, Stealth +5}, damage-immunities={}, damage-resistances={}, damage-vulnerabilities={}, condition-immunities={}, senses={darkvision 60 ft., passive Perception 14}, languages={Common}, challenge={5:6}]
\DndMonsterAction{Petrifying Gaze} When a creature that can see the medusa's eyes starts its turn within 30 feet of the medusa, the medusa can force it to make a DC 14 Constitution saving throw if the medusa isn't incapacitated and can see the creature. If the saving throw fails by 5 or more, the creature is instantly petrified. Otherwise, a creature that fails the save begins to turn to stone and is restrained. The restrained creature must repeat the saving throw at the end of its next turn, becoming petrified on a failure or ending the effect on a success. The petrification lasts until the creature is freed by the \nameref{inc:restoration} (greater only).

Unless surprised, a creature can avert its eyes to avoid the saving throw at the start of its turn. If the creature does so, it can't see the medusa until the start of its next turn, when it can avert its eyes again. If the creature looks at the medusa in the meantime, it must immediately make the save.

If the medusa sees itself reflected on a polished surface within 30 feet of it and in an area of bright light, the medusa is, due to its curse, affected by its own gaze.

\DndMonsterSection{Actions}
\DndMonsterAction{Multiattack} The medusa makes either three melee attacks—one with its snake hair and two with its shortsword—or two ranged attacks with its longbow.
\DndMonsterAttack[
	name=Snake Hair,
	distance=melee,
	type=weapon,
	mod=+5,
	reach=5,
	dmg=\DndDice{1d4 + 2},
	dmg-type=piercing,
	extra={ plus 14 (4d6) poison damage.}
]
\DndMonsterAttack[
	name=Shortsword,
	distance=melee,
	type=weapon,
	mod=+5,
	reach=5,
	dmg=\DndDice{1d6 + 2},
	dmg-type=piercing
]
\DndMonsterAttack[
	name=Longbow,
	distance=ranged,
	type=weapon,
	mod=+5,
	range=150/600,
	dmg=\DndDice{1d8 + 2},
	dmg-type=piercing,
	extra={ plus 7 (2d6) poison damage.}
]
\end{multicols}
\end{DndMonster}

\begin{DndMonster}{Mimic}
	\DndMonsterType{Medium monstrosity (shapechanger)}
	\DndMonsterBasics[armor-class={12 (natural armor)}, hit-points={58 (9d8 + 18)}, speed={15 ft.}]
	\MonsterStats{+3}{+1}{+2}{-3}{+1}{-1}
	\DndMonsterDetails[saving-throws={}, skills={Stealth +5}, damage-immunities={acid}, damage-resistances={}, damage-vulnerabilities={}, condition-immunities={prone}, senses={darkvision 60 ft., passive Perception 11}, languages={—}, challenge={1:2}]
	\DndMonsterAction{Shapechanger} The mimic can use its action to polymorph into an object or back into its true, amorphous form. Its statistics are the same in each form. Any equipment it is wearing or carrying isn't transformed. It reverts to its true form if it dies.

	\DndMonsterAction{Adhesive (Object Form Only)} The mimic adheres to anything that touches it. A Huge or smaller creature adhered to the mimic is also grappled by it (escape DC 13). Ability checks made to escape this grapple have disadvantage.

	\DndMonsterAction{False Appearance (Object Form Only)} While the mimic remains motionless, it is indistinguishable from an ordinary object.

	\DndMonsterAction{Grappler} The mimic has advantage on attack rolls against any creature grappled by it.

	\DndMonsterSection{Actions}
	\DndMonsterAttack[
		name=Pseudopod,
		distance=melee,
		type=weapon,
		mod=+5,
		reach=5,
		dmg=\DndDice{1d8 + 3},
		dmg-type=bludgeoning,
		extra={. If the mimic is in object form, the target is subjected to its Adhesive trait.}
	]
	\DndMonsterAttack[
		name=Bite,
		distance=melee,
		type=weapon,
		mod=+5,
		reach=5,
		dmg=\DndDice{1d8 + 3},
		dmg-type=piercing,
		extra={ plus 4 (1d8) acid damage.}
	]
\end{DndMonster}

\begin{DndMonster}[float*=b,width=\textwidth + 8pt]{Guardian Naga}
	\begin{multicols}{2}
	\DndMonsterType{Large monstrosity}
	\DndMonsterBasics[armor-class={18 (natural armor)}, hit-points={127 (15d10 + 45)}, speed={40 ft.}]
	\MonsterStats{+4}{+4 (+8)}{+3 (+7)}{+3 (+7)}{+4 (+8)}{+4 (+8)}
	\DndMonsterDetails[saving-throws={Dex +8, Con +7, Int +7, Wis +8, Cha +8}, skills={}, damage-immunities={poison}, damage-resistances={}, damage-vulnerabilities={}, condition-immunities={charmed, poisoned}, senses={darkvision 60 ft., passive Perception 14}, languages={Celestial, Common}, challenge={10:8}]

	\DndMonsterAction{Spellcasting} The naga's spellcasting ability is Wisdom (spell save DC 16, +8 to hit with spell attacks), and it needs only verbal components to cast its spells. It can cast the following spells:
	\begin{itemize}
		\item[] \textbf{Concentration} \nameref{spell:banishment} (2x), \nameref{spell:shield-of-faith} (1x), \nameref{spell:bestow-curse} (1x), \nameref{spell:hold-person} (2x)
		\item[] \textbf{Healing} \nameref{spell:cure-wounds} (3x)
		\item[] \textbf{Damage} \nameref{spell:flame-strike} (1x)
	\end{itemize}

	\DndMonsterSection{Actions}
	\DndMonsterAttack[
		name=Bite,
		distance=melee,
		type=weapon,
		mod=+8,
		reach=10,
		dmg=\DndDice{1d8 + 4},
		dmg-type=piercing,
		extra={, and the target must make a DC 15 Constitution saving throw, taking 45 (10d8) poison damage on a failed save, or half as much damage on a successful one.}
	]
	\DndMonsterAttack[
		name=Spit Poison,
		distance=ranged,
		type=weapon,
		mod=+8,
		range=15/30,
		dmg=\DndDice{10d8},
		dmg-type=poison
	]
	\DndMonsterAttack[
		name=Holy Lance,
		distance=ranged,
		type=spell,
		mod=+8,
		range=60,
		dmg=\DndDice{3d8 + 4},
		dmg-type=radiant
	]
	\end{multicols}
\end{DndMonster}

\begin{DndMonster}[float*=b,width=\textwidth + 8pt]{Spirit Naga}
	\begin{multicols}{2}
	\DndMonsterType{Large monstrosity}
	\DndMonsterBasics[armor-class={15 (natural armor)}, hit-points={160 (20d10 + 50)}, speed={40 ft.}]
	\MonsterStats{+4}{+3}{+2}{+3}{+2}{+3}
	\DndMonsterDetails[saving-throws={}, skills={}, damage-immunities={poison}, damage-resistances={}, damage-vulnerabilities={}, condition-immunities={charmed, poisoned}, senses={darkvision 60 ft., passive Perception 12}, languages={Abyssal, Common}, challenge={10:9}]

	\DndMonsterAction{Spellcasting} The naga's spellcasting ability is Intelligence (spell save DC 14, +6 to hit with spell attacks), and it needs only verbal components to cast its spells. It can cast the following spells:
	\begin{itemize}
		\item[] \textbf{Concentration} \nameref{spell:dominate} (Legendary, 1x), \nameref{spell:hold-person} (3x)
		\item[] \textbf{Damage} \nameref{spell:blight} (2x), \nameref{spell:lightning-bolt} (2x)
		\item[] \textbf{Other} \nameref{spell:dimension-door} (1x)
	\end{itemize}

	\DndMonsterSection{Actions}
	\DndMonsterAttack[
		name=Bite,
		distance=melee,
		type=weapon,
		mod=+7,
		reach=10,
		dmg=\DndDice{1d6 + 4},
		dmg-type=piercing,
		extra={, and the target must make a DC 13 Constitution saving throw, taking 31 (7d8) poison damage on a failed save, or half as much damage on a successful one.}
	]
	\DndMonsterAttack[
		name=Frost Blast,
		distance=ranged,
		type=spell,
		mod=+6,
		range=90,
		dmg=\DndDice{2d8+3},
		dmg-type=cold,
		extra={ and the target's speed is reduced by 10 feet until the end of their next turn.}
	]
	\end{multicols}
\end{DndMonster}

\begin{DndMonster}{Otyugh}
	\DndMonsterType{Large monstrosity}
	\DndMonsterBasics[armor-class={14 (natural armor)}, hit-points={114 (12d10 + 48)}, speed={30 ft.}]
	\MonsterStats{+3}{+0}{+4 (+7)}{-2}{+1}{-2}
	\DndMonsterDetails[saving-throws={}, skills={}, damage-immunities={}, damage-resistances={}, damage-vulnerabilities={}, condition-immunities={}, senses={darkvision 120 ft., passive Perception 11}, languages={Otyugh}, challenge={6:6}]
	\DndMonsterAction{Limited Telepathy} The otyugh can magically transmit simple messages and images to any creature within 120 feet of it that can understand a language. This form of telepathy doesn't allow the receiving creature to telepathically respond.
	
	\DndMonsterSection{Actions}
	\DndMonsterAction{Multiattack} The otyugh makes three attacks: one with its bite and two with its tentacles.
	\DndMonsterAttack[
		name=Bite,
		distance=melee,
		type=weapon,
		mod=+6,
		reach=5,
		dmg=\DndDice{2d8 + 3},
		dmg-type=piercing,
		extra={. If the target is a creature, it must succeed on a DC 15 Constitution saving throw against disease or become poisoned until the disease is cured. Every 24 hours that elapse, the target must repeat the saving throw, reducing its hit point maximum by 5 (1d10) on a failure. The disease is cured on a success. The target dies if the disease reduces its hit point maximum to 0. This reduction to the target's hit point maximum lasts until the disease is cured.}
	]
	\DndMonsterAttack[
		name=Tentacle,
		distance=melee,
		type=weapon,
		mod=+6,
		reach=10,
		dmg=\DndDice{1d8 + 3},
		dmg-type=bludgeoning,
		extra={ plus 4 (1d8) piercing damage. If the target is Medium or smaller, it is grappled (escape DC 13) and restrained until the grapple ends. The otyugh has two tentacles, each of which can grapple one target.}
	]
	\DndMonsterAction{Slam}
	The otyugh slams creatures grappled by it into each other or a solid surface. Each creature must succeed on a DC 14 Constitution saving throw or take 10 (2d6 + 3) bludgeoning damage and be stunned until the end of the otyugh's next turn. On a successful save, the target takes half the bludgeoning damage and isn't stunned.
\end{DndMonster}

\begin{DndMonster}{Owlbear}
	\DndMonsterType{Large monstrosity}
	\DndMonsterBasics[armor-class={13 (natural armor)}, hit-points={59 (7d10 + 21)}, speed={40 ft.}]
	\MonsterStats{+5}{+1}{+3}{-4}{+1}{-2}
	\DndMonsterDetails[saving-throws={}, skills={Perception +3}, damage-immunities={}, damage-resistances={}, damage-vulnerabilities={}, condition-immunities={}, senses={darkvision 60 ft., passive Perception 13}, languages={—}, challenge={3:2}]
	\DndMonsterAction{Keen Sight and Smell} The owlbear has advantage on Wisdom (Perception) checks that rely on sight or smell.
	
	\DndMonsterSection{Actions}
	\DndMonsterAction{Multiattack} The owlbear makes two attacks: one with its beak and one with its claws.
	\DndMonsterAttack[
		name=Beak,
		distance=melee,
		type=weapon,
		mod=+7,
		reach=5,
		dmg=\DndDice{1d10 + 5},
		dmg-type=piercing
	]
	\DndMonsterAttack[
		name=Claws,
		distance=melee,
		type=weapon,
		mod=+7,
		reach=5,
		dmg=\DndDice{2d8 + 5},
		dmg-type=slashing
	]
\end{DndMonster}

%MARKER
\begin{DndMonster}{Phase Spider}
	\DndMonsterType{Large monstrosity}
	\DndMonsterBasics[armor-class={13 (natural armor)}, hit-points={32 (5d10 + 5)}, speed={30 ft., climb 30 ft.}]
	\MonsterStats{+2}{+2}{+1}{-2}{+0}{-2}
	\DndMonsterDetails[saving-throws={}, skills={Stealth +6}, damage-immunities={}, damage-resistances={}, damage-vulnerabilities={}, condition-immunities={}, senses={darkvision 60 ft., passive Perception 10}, languages={—}, challenge={3:1}]
	\DndMonsterAction{Ethereal Jaunt} As a bonus action, the spider can magically shift from the Material Plane to the Ethereal Plane, or vice versa.

	\DndMonsterAction{Spider Climb} The spider can climb difficult surfaces, including upside down on ceilings, without needing to make an ability check.

	\DndMonsterAction{Web Walker} The spider ignores movement restrictions caused by webbing.

	\DndMonsterSection{Actions}
	\DndMonsterAttack[
		name=Bite,
		distance=melee,
		type=weapon,
		mod=+4,
		reach=5,
		dmg=\DndDice{1d10 + 2},
		dmg-type=piercing,
		extra={, and the target must make a DC 11 Constitution saving throw, taking 18 (4d8) poison damage on a failed save, or half as much damage on a successful one. If the poison damage reduces the target to 0 hit points, the target is stable but poisoned for 1 hour, even after regaining hit points, and is paralyzed while poisoned in this way.}
	]
\end{DndMonster}

\begin{DndMonster}{Purple Worm}
	\DndMonsterType{Gargantuan monstrosity}
	\DndMonsterBasics[armor-class={18 (natural armor)}, hit-points={247 (15d20 + 90)}, speed={50 ft., burrow 30 ft.}]
	\MonsterStats{+9}{-2}{+6}{-5}{-1}{-3}
	\DndMonsterDetails[saving-throws={Con +11, Wis +4}, skills={}, damage-immunities={}, damage-resistances={}, damage-vulnerabilities={}, condition-immunities={}, senses={blindsight 30 ft., tremorsense 60 ft., passive Perception 9}, languages={—}, challenge={16:14}]
	\DndMonsterAction{Tunneler} The worm can burrow through solid rock at half its burrow speed and leaves a 10-foot-diameter tunnel in its wake.
	
	\DndMonsterSection{Actions}
	\DndMonsterAction{Multiattack} The worm makes two attacks: one with its bite and one with its stinger.
	\DndMonsterAttack[
		name=Bite,
		distance=melee,
		type=weapon,
		mod=+9,
		reach=10,
		dmg=\DndDice{3d8 + 9},
		dmg-type=piercing,
		extra={. If the target is a Large or smaller creature, it must succeed on a DC 19 Dexterity saving throw or be swallowed by the worm. A swallowed creature is blinded and restrained, it has total cover against attacks and other effects outside the worm, and it takes 21 (6d6) acid damage at the start of each of the worm's turns. If the worm takes 30 damage or more on a single turn from a creature inside it, the worm must succeed on a DC 21 Constitution saving throw at the end of that turn or regurgitate all swallowed creatures, which fall prone in a space within 10 feet of the worm. If the worm dies, a swallowed creature is no longer restrained by it and can escape from the corpse by using 20 feet of movement, exiting prone.}
	]
	\DndMonsterAttack[
		name=Tail Stinger,
		distance=melee,
		type=weapon,
		mod=+9,
		reach=10,
		dmg=\DndDice{3d6 + 9},
		dmg-type=piercing,
		extra={, and the target must make a DC 19 Constitution saving throw, taking 42 (12d6) poison damage on a failed save, or half as much damage on a successful one.}
	]
\end{DndMonster}

\begin{DndMonster}{Remorhaz}
\DndMonsterType{Huge monstrosity}
\DndMonsterBasics[armor-class={17 (natural armor)}, hit-points={195 (17d12 + 85)}, speed={30 ft., burrow 20 ft.}]
\MonsterStats{+7}{+1}{+5}{-3}{+0}{-3}
\DndMonsterDetails[saving-throws={}, skills={}, damage-immunities={cold, fire}, damage-resistances={}, damage-vulnerabilities={}, condition-immunities={}, senses={darkvision 60 ft., tremorsense 60 ft., passive Perception 10}, languages={—}, challenge={10:11}]
\DndMonsterAction{Heated Body} A creature that touches the remorhaz or hits it with a melee attack while within 5 feet of it takes 10 (3d6) fire damage.

\DndMonsterSection{Actions}
\DndMonsterAttack[
	name=Bite,
	distance=melee,
	type=weapon,
	mod=+11,
	reach=10,
	dmg=\DndDice{6d10 + 7},
	dmg-type=piercing,
	extra={ plus 10 (3d6) fire damage. If the target is a creature, it is grappled (escape DC 17). Until this grapple ends, the target is restrained, and the remorhaz can't bite another target.}
]
\DndMonsterAction{Swallow}
The remorhaz makes one bite attack against a Medium or smaller creature it is grappling. If the attack hits, that creature takes the bite's damage and is swallowed, and the grapple ends. While swallowed, the creature is blinded and restrained, it has total cover against attacks and other effects outside the remorhaz, and it takes 21 (6d6) acid damage at the start of each of the remorhaz's turns.\\nIf the remorhaz takes 30 damage or more on a single turn from a creature inside it, the remorhaz must succeed on a DC 15 Constitution saving throw at the end of that turn or regurgitate all swallowed creatures, which fall prone in a space within 10 feet of the remorhaz. If the remorhaz dies, a swallowed creature is no longer restrained by it and can escape from the corpse using 15 feet of movement, exiting prone.
\end{DndMonster}

\begin{DndMonster}[width=\textwidth + 8pt]{Roc}
\begin{multicols}{2}
\DndMonsterType{Gargantuan monstrosity}
\DndMonsterBasics[armor-class={15 (natural armor)}, hit-points={248 (16d20 + 80)}, speed={20 ft., fly 120 ft.}]
\MonsterStats{+9}{+0}{+5}{-4}{+0}{-1}
\DndMonsterDetails[saving-throws={Dex +4, Con +9, Wis +4, Cha +3}, skills={Perception +4}, damage-immunities={}, damage-resistances={}, damage-vulnerabilities={}, condition-immunities={}, senses={passive Perception 14}, languages={—}, challenge={11 (7,200 XP)}]
\DndMonsterAction{Keen Sight} The roc has advantage on Wisdom (Perception) checks that rely on sight.

\DndMonsterSection{Actions}
\DndMonsterAction{Multiattack} The roc makes two attacks: one with its beak and one with its talons.
\DndMonsterAttack[
	name=Beak,
	distance=melee,
	type=weapon,
	mod=+13,
	reach=10,
	dmg=\DndDice{4d8 + 9},
	dmg-type=piercing
]
\DndMonsterAttack[
	name=Talons,
	distance=melee,
	type=weapon,
	mod=+13,
	reach=5,
	dmg=\DndDice{4d6 + 9},
	dmg-type=slashing,
	extra={, and the target is grappled (escape DC 19). Until this grapple ends, the target is restrained, and the roc can't use its talons on another target.}
]
\end{multicols}
\end{DndMonster}

\subsection{Roper}
\begin{DndMonster}[width=\textwidth + 8pt]{Roper}
\begin{multicols}{2}
\DndMonsterType{Large monstrosity}
\DndMonsterBasics[armor-class={20 (natural armor)}, hit-points={93 (11d10 + 33)}, speed={10 ft., climb 10 ft.}]
\MonsterStats{+4}{-1}{+3}{-2}{+3}{-2}
\DndMonsterDetails[saving-throws={}, skills={Perception +6, Stealth +5}, damage-immunities={}, damage-resistances={}, damage-vulnerabilities={}, condition-immunities={}, senses={darkvision 60 ft., passive Perception 16}, languages={—}, challenge={5 (1,800 XP)}]
\DndMonsterAction{False Appearance} While the roper remains motionless, it is indistinguishable from a normal cave formation, such as a stalagmite.

\DndMonsterAction{Grasping Tendrils} The roper can have up to six tendrils at a time. Each tendril can be attacked (AC 20; 10 hit points; immunity to poison and psychic damage). Destroying a tendril deals no damage to the roper, which can extrude a replacement tendril on its next turn. A tendril can also be broken if a creature takes an action and succeeds on a DC 15 Strength check against it.

\DndMonsterAction{Spider Climb} The roper can climb difficult surfaces, including upside down on ceilings, without needing to make an ability check.

\DndMonsterSection{Actions}
\DndMonsterAction{Multiattack} The roper makes four attacks with its tendrils, uses Reel, and makes one attack with its bite.
\DndMonsterAttack[
	name=Bite,
	distance=melee,
	type=weapon,
	mod=+7,
	reach=5,
	dmg=\DndDice{4d8 + 4},
	dmg-type=piercing
]
\DndMonsterAttack[
	name=Tendril,
	distance=melee,
	type=weapon,
	mod=+7,
	reach=50,
	dmg=\DndDice{1d8+4},
	dmg-type=bludgeoning,
	extra={. The target is grappled (escape DC 15). Until the grapple ends, the target is restrained and has disadvantage on Strength checks and Strength saving throws, and the roper can't use the same tendril on another target.}
]
\DndMonsterAction{Reel}
The roper pulls each creature grappled by it up to 25 feet straight toward it.
\end{multicols}
\end{DndMonster}

\begin{DndMonster}{Stirge}
	\DndMonsterType{Tiny monstrosity}
	\DndMonsterBasics[armor-class={14 (natural armor)}, hit-points={2 (1d4)}, speed={10 ft., fly 40 ft.}]
	\MonsterStats{-3}{+3}{+0}{-4}{-1}{-2}
	\DndMonsterDetails[saving-throws={}, skills={}, damage-immunities={}, damage-resistances={}, damage-vulnerabilities={}, condition-immunities={}, senses={darkvision 60 ft., passive Perception 9}, languages={—}, challenge={1/4:0}]
	\DndMonsterSection{Actions}
	\DndMonsterAttack[
		name=Blood Drain,
		distance=melee,
		type=weapon,
		mod=+5,
		reach=5,
		dmg=\DndDice{1d4 + 3},
		dmg-type=piercing,
		extra={, and the stirge attaches to the target. While attached, the stirge doesn't attack. Instead, at the start of each of the stirge's turns, the target loses 5 (1d4 + 3) hit points due to blood loss. The stirge can detach itself by spending 5 feet of its movement. It does so after it drains 10 hit points of blood from the target or the target dies. A creature, including the target, can use its action to detach the stirge.}
	]
\end{DndMonster}

% \begin{DndMonster}[width=\textwidth + 8pt]{Tarrasque} rewrite me
% 	\begin{multicols}{2}
% 	\DndMonsterType{Gargantuan monstrosity (titan)}
% 	\DndMonsterBasics[armor-class={25 (natural armor)}, hit-points={676 (33d20 + 330)}, speed={40 ft.}]
% 	\MonsterStats{+10}{+0}{+10}{-4}{+0}{+0}
% 	\DndMonsterDetails[saving-throws={Int +5, Wis +9, Cha +9}, skills={}, damage-immunities={fire, poison; bludgeoning, piercing, and slashing from nonmagical attacks}, damage-resistances={}, damage-vulnerabilities={}, condition-immunities={charmed, frightened, paralyzed, poisoned}, senses={blindsight 120 ft., passive Perception 10}, languages={—}, challenge={30 (155,000 XP)}]
% 	\DndMonsterAction{Legendary Resistance (3/Day)} If the tarrasque fails a saving throw, it can choose to succeed instead.
	
% 	\DndMonsterAction{Magic Resistance} The tarrasque has advantage on saving throws against spells and other magical effects.
	
% 	\DndMonsterAction{Reflective Carapace} Any time the tarrasque is targeted by a \textit{magic missile} spell, a line spell, or a spell that requires a ranged attack roll, roll a d6. On a 1 to 5, the tarrasque is unaffected. On a 6, the tarrasque is unaffected, and the effect is reflected back at the caster as though it originated from the tarrasque, turning the caster into the target.
	
% 	\DndMonsterAction{Siege Monster} The tarrasque deals double damage to objects and structures.
	
% 	\DndMonsterSection{Actions}
% 	\DndMonsterAction{Multiattack} The tarrasque can use its Frightful Presence. It then makes five attacks: one with its bite, two with its claws, one with its horns, and one with its tail. It can use its Swallow instead of its bite.
% 	\DndMonsterAttack[
% 		name=Bite,
% 		distance=melee,
% 		type=weapon,
% 		mod=+19,
% 		reach=10,
% 		dmg=\DndDice{4d12 + 10},
% 		dmg-type=piercing,
% 		extra={. If the target is a creature, it is grappled (escape DC 20). Until this grapple ends, the target is restrained, and the tarrasque can't bite another target.}
% 	]
% 	\DndMonsterAttack[
% 		name=Claw,
% 		distance=melee,
% 		type=weapon,
% 		mod=+19,
% 		reach=15,
% 		dmg=\DndDice{4d8 + 10},
% 		dmg-type=slashing
% 	]
% 	\DndMonsterAttack[
% 		name=Horns,
% 		distance=melee,
% 		type=weapon,
% 		mod=+19,
% 		reach=10,
% 		dmg=\DndDice{4d10 + 10},
% 		dmg-type=piercing
% 	]
% 	\DndMonsterAttack[
% 		name=Tail,
% 		distance=melee,
% 		type=weapon,
% 		mod=+19,
% 		reach=20,
% 		dmg=\DndDice{4d6 + 10},
% 		dmg-type=bludgeoning,
% 		extra={. If the target is a creature, it must succeed on a DC 20 Strength saving throw or be knocked prone.}
% 	]
% 	\DndMonsterAction{Frightful Presence}
% 	Each creature of the tarrasque's choice within 120 feet of it and aware of it must succeed on a DC 17 Wisdom saving throw or become frightened for 1 minute. A creature can repeat the saving throw at the end of each of its turns, with disadvantage if the tarrasque is within line of sight, ending the effect on itself on a success. If a creature's saving throw is successful or the effect ends for it, the creature is immune to the tarrasque's Frightful Presence for the next 24 hours.
% 	\DndMonsterAction{Swallow}
% 	The tarrasque makes one bite attack against a Large or smaller creature it is grappling. If the attack hits, the target takes the bite's damage, the target is swallowed, and the grapple ends. While swallowed, the creature is blinded and restrained, it has total cover against attacks and other effects outside the tarrasque, and it takes 56 (16d6) acid damage at the start of each of the tarrasque's turns.\\nIf the tarrasque takes 60 damage or more on a single turn from a creature inside it, the tarrasque must succeed on a DC 20 Constitution saving throw at the end of that turn or regurgitate all swallowed creatures, which fall prone in a space within 10 feet of the tarrasque. If the tarrasque dies, a swallowed creature is no longer restrained by it and can escape from the corpse by using 30 feet of movement, exiting prone.
	
% 	\DndMonsterSection{Legendary Actions}
% 	The Tarrasque can take 3 legendary actions, choosing from the options below. Only one legendary action option can be used at a time and only at the end of another creature's turn. The Tarrasque regains spent legendary actions at the start of its turn.
% 	\begin{DndMonsterLegendaryActions}
% 	\DndMonsterLegendaryAction{Attack}{The tarrasque makes one claw attack or tail attack.}
% 	\DndMonsterLegendaryAction{Move}{The tarrasque moves up to half its speed.}
% 	\DndMonsterLegendaryAction{Chomp (Costs 2 Actions)}{The tarrasque makes one bite attack or uses its Swallow.}
% 	\end{DndMonsterLegendaryActions}
% 	\end{multicols}
% \end{DndMonster}

\begin{DndMonster}{Winter Wolf}
	\DndMonsterType{Large monstrosity}
	\DndMonsterBasics[armor-class={13 (natural armor)}, hit-points={75 (10d10 + 20)}, speed={50 ft.}]
	\MonsterStats{+4}{+1}{+2}{-2}{+1}{-1}
	\DndMonsterDetails[saving-throws={}, skills={Perception +5, Stealth +3}, damage-immunities={cold}, damage-resistances={}, damage-vulnerabilities={}, condition-immunities={}, senses={passive Perception 15 }, languages={Common, Giant, Winter Wolf }, challenge={3:3}]
	\DndMonsterAction{Keen Hearing and Smell} The wolf has advantage on Wisdom (Perception) checks that rely on hearing or smell.

	\DndMonsterAction{Pack Tactics} The wolf has advantage on an attack roll against a creature if at least one of the wolf's allies is within 5 feet of the creature and the ally isn't incapacitated.

	\DndMonsterAction{Snow Camouflage} The wolf has advantage on Dexterity (Stealth) checks made to hide in snowy terrain.

	\DndMonsterSection{Actions}
	\DndMonsterAttack[
		name=Bite,
		distance=melee,
		type=weapon,
		mod=+6,
		reach=5,
		dmg=\DndDice{2d6 + 4},
		dmg-type=piercing,
		extra={. If the target is a creature, it must succeed on a DC 14 Strength saving throw or be knocked prone.}
	]

	\DndMonsterAction{Frost Breath (recharge 5-6)}
	The wolf exhales a blast of freezing wind in a 15-foot cone. Each creature in that area must make a DC 12 Dexterity saving throw, taking 18 (4d8) cold damage on a failed save, or half as much damage on a successful one.
\end{DndMonster}

\begin{DndMonster}{Worg}
\DndMonsterType{Large monstrosity}
\DndMonsterBasics[armor-class={13 (natural armor)}, hit-points={26 (4d10 + 4)}, speed={50 ft.}]
\MonsterStats{+3}{+1}{+1}{-2}{+0}{-1}
\DndMonsterDetails[saving-throws={}, skills={Perception +4}, damage-immunities={}, damage-resistances={}, damage-vulnerabilities={}, condition-immunities={}, senses={darkvision 60 ft., passive Perception 14}, languages={Goblin, Worg}, challenge={1:1/2}]
\DndMonsterAction{Keen Hearing and Smell} The worg has advantage on Wisdom (Perception) checks that rely on hearing or smell.

\DndMonsterSection{Actions}
\DndMonsterAttack[
	name=Bite,
	distance=melee,
	type=weapon,
	mod=+5,
	reach=5,
	dmg=\DndDice{2d6 + 3},
	dmg-type=piercing,
	extra={. If the target is a creature, it must succeed on a DC 13 Strength saving throw or be knocked prone.}
]
\end{DndMonster}