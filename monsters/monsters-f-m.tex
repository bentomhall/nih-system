\section{Monsters (F)}\label{sec:monster-f}

\subsection{Fungus, Shrieker}
\begin{DndMonster}[width=\textwidth + 8pt]{Fungus, Shrieker}
\begin{multicols}{2}
\DndMonsterType{Medium plant}
\DndMonsterBasics[armor-class={5}, hit-points={13 (3d8)}, speed={0 ft.}]
\MonsterStats{-5}{-5}{+0}{-5}{-4}{-5}
\DndMonsterDetails[saving-throws={}, skills={}, damage-immunities={}, damage-resistances={}, damage-vulnerabilities={}, condition-immunities={blinded, deafened, frightened}, senses={blindsight 30 ft. (blind beyond this radius), passive Perception 6}, languages={—}, challenge={0 (10 XP)}]
\DndMonsterAction{False Appearance} While the shrieker remains motionless, it is indistinguishable from an ordinary fungus.

\DndMonsterAction{Shriek} When bright light or a creature is within 30 feet of the shrieker, it emits a shriek audible within 300 feet of it. The shrieker continues to shriek until the disturbance moves out of range and for 1d4 of the shrieker's turns afterward.

\DndMonsterSection{Actions}
\end{multicols}
\end{DndMonster}

\subsection{Fungus, Violet}
\begin{DndMonster}[width=\textwidth + 8pt]{Fungus, Violet}
\begin{multicols}{2}
\DndMonsterType{Medium plant}
\DndMonsterBasics[armor-class={5}, hit-points={18 (4d8)}, speed={5 ft.}]
\MonsterStats{-4}{-5}{+0}{-5}{-4}{-5}
\DndMonsterDetails[saving-throws={}, skills={}, damage-immunities={}, damage-resistances={}, damage-vulnerabilities={}, condition-immunities={blinded, deafened, frightened}, senses={blindsight 30 ft. (blind beyond this radius), passive Perception 6}, languages={—}, challenge={1/4 (50 XP)}]
\DndMonsterAction{False Appearance} While the violet fungus remains motionless, it is indistinguishable from an ordinary fungus.

\DndMonsterSection{Actions}
\DndMonsterAction{Multiattack} The fungus makes 1d4 Rotting Touch attacks.
\DndMonsterAttack[
	name=Rotting Touch,
	distance=melee,
	type=weapon,
	mod=+2,
	reach=10,
	dmg=\DndDice{1d8},
	dmg-type=necrotic
]
\end{multicols}
\end{DndMonster}

\FloatBarrier
\section{Monsters (G)}\label{sec:monsters-g}

\subsection{Gargoyle}


\subsection{Gelatinous Cube}
\begin{DndMonster}[width=\textwidth + 8pt]{Gelatinous Cube}
\begin{multicols}{2}
\DndMonsterType{Large ooze}
\DndMonsterBasics[armor-class={6}, hit-points={84 (8d10 + 40)}, speed={15 ft.}]
\MonsterStats{+2}{-4}{+5}{-5}{-2}{-5}
\DndMonsterDetails[saving-throws={}, skills={}, damage-immunities={}, damage-resistances={}, damage-vulnerabilities={}, condition-immunities={blinded, charmed, deafened, exhaustion, frightened, prone}, senses={blindsight 60 ft. (blind beyond this radius), passive Perception 8}, languages={—}, challenge={2 (450 XP)}]
\DndMonsterAction{Ooze Cube} The cube takes up its entire space. Other creatures can enter the space, but a creature that does so is subjected to the cube’s Engulf and has disadvantage on the saving throw.\\nCreatures inside the cube can be seen but have total cover.\\nA creature within 5 feet of the cube can take an action to pull a creature or object out of the cube. Doing so requires a successful DC 12 Strength check, and the creature making the attempt takes 10 (3d6) acid damage.\\nThe cube can hold only one Large creature or up to four Medium or smaller creatures inside it at a time.\\n

\DndMonsterAction{Transparent} Even when the cube is in plain sight, it takes a successful DC 15 Wisdom (Perception) check to spot a cube that has neither moved nor attacked. A creature that tries to enter the cube’s space while unaware of the cube is surprised by the cube.

\DndMonsterSection{Actions}
\DndMonsterAttack[
	name=Pseudopod,
	distance=melee,
	type=weapon,
	mod=+4,
	reach=5,
	dmg=\DndDice{3d6},
	dmg-type=acid
]
\DndMonsterAction{Engulf}
The cube moves up to its speed. While doing so, it can enter Large or smaller creatures’ spaces. Whenever the cube enters a creature’s space, the creature must make a DC 12 Dexterity saving throw.\\nOn a successful save, the creature can choose to be pushed 5 feet back or to the side of the cube. A creature that chooses not to be pushed suffers the consequences of a failed saving throw.\\nOn a failed save, the cube enters the creature’s space, and the creature takes 10 (3d6) acid damage and is engulfed. The engulfed creature can’t breathe, is restrained, and takes 21 (6d6) acid damage at the start of each of the cube’s turns. When the cube moves, the engulfed creature moves with it.\\nAn engulfed creature can try to escape by taking an action to make a DC 12 Strength check. On a success, the creature escapes and enters a space of its choice within 5 feet of the cube.
\end{multicols}
\end{DndMonster}

\subsection{Ghast}
\begin{DndMonster}[width=\textwidth + 8pt]{Ghast}
\begin{multicols}{2}
\DndMonsterType{Medium undead}
\DndMonsterBasics[armor-class={13}, hit-points={36 (8d8)}, speed={30 ft.}]
\MonsterStats{+3}{+3}{+0}{+0}{+0}{-1}
\DndMonsterDetails[saving-throws={}, skills={}, damage-immunities={poison}, damage-resistances={necrotic}, damage-vulnerabilities={}, condition-immunities={charmed, exhaustion, poisoned}, senses={darkvision 60 ft., passive Perception 10}, languages={Common}, challenge={2 (450 XP)}]
\DndMonsterAction{Stench} Any creature that starts its turn within 5 feet of the ghast must succeed on a DC 10 Constitution saving throw or be poisoned until the start of its next turn. On a successful saving throw, the creature is immune to the ghast's Stench for 24 hours.

\DndMonsterAction{Turning Defiance} The ghast and any ghouls within 30 feet of it have advantage on saving throws against effects that turn undead.

\DndMonsterSection{Actions}
\DndMonsterAttack[
	name=Bite,
	distance=melee,
	type=weapon,
	mod=+3,
	reach=5,
	dmg=\DndDice{2d8 + 3},
	dmg-type=piercing
]
\DndMonsterAttack[
	name=Claws,
	distance=melee,
	type=weapon,
	mod=+5,
	reach=5,
	dmg=\DndDice{2d6 + 3},
	dmg-type=slashing,
	extra={. If the target is a creature other than an undead, it must succeed on a DC 10 Constitution saving throw or be paralyzed for 1 minute. The target can repeat the saving throw at the end of each of its turns, ending the effect on itself on a success.}
]
\end{multicols}
\end{DndMonster}

\subsection{Ghost}
\begin{DndMonster}[width=\textwidth + 8pt]{Ghost}
\begin{multicols}{2}
\DndMonsterType{Medium undead}
\DndMonsterBasics[armor-class={11}, hit-points={45 (10d8)}, speed={0 ft., fly 40 ft. (hover)}]
\MonsterStats{-2}{+1}{+0}{+0}{+1}{+3}
\DndMonsterDetails[saving-throws={}, skills={}, damage-immunities={cold, necrotic, poison}, damage-resistances={acid, fire, lightning, thunder; bludgeoning, piercing, and slashing from nonmagical attacks}, damage-vulnerabilities={}, condition-immunities={charmed, exhaustion, frightened, grappled, paralyzed, petrified, poisoned, prone, restrained}, senses={darkvision 60 ft., passive Perception 11}, languages={any languages it knew in life}, challenge={4 (1,100 XP)}]
\DndMonsterAction{Ethereal Sight} The ghost can see 60 feet into the Ethereal Plane when it is on the Material Plane, and vice versa.

\DndMonsterAction{Incorporeal Movement} The ghost can move through other creatures and objects as if they were difficult terrain. It takes 5 (1d10) force damage if it ends its turn inside an object.

\DndMonsterSection{Actions}
\DndMonsterAttack[
	name=Withering Touch,
	distance=melee,
	type=weapon,
	mod=+5,
	reach=5,
	dmg=\DndDice{4d6 + 3},
	dmg-type=necrotic
]

\DndMonsterAction{Etherealness}
The ghost enters the Ethereal Plane from the Material Plane, or vice versa. It is visible on the Material Plane while it is in the Border Ethereal, and vice versa, yet it can't affect or be affected by anything on the other plane.
\DndMonsterAction{Horrifying Visage}
Each non-undead creature within 60 feet of the ghost that can see it must succeed on a DC 13 Wisdom saving throw or be frightened for 1 minute. If the save fails by 5 or more, the target also ages 1d4 \texttimes 10 years. A frightened target can repeat the saving throw at the end of each of its turns, ending the frightened condition on itself on a success. If a target's saving throw is successful or the effect ends for it, the target is immune to this ghost's Horrifying Visage for the next 24 hours. The aging effect can be reversed with a \textit{greater restoration} spell, but only within 24 hours of it occurring.

One humanoid that the ghost can see within 5 feet of it must succeed on a DC 13 Charisma saving throw or be possessed by the ghost; the ghost then disappears, and the target is incapacitated and loses control of its body. The ghost now controls the body but doesn't deprive the target of awareness. The ghost can't be targeted by any attack, spell, or other effect, except ones that turn undead, and it retains its alignment, Intelligence, Wisdom, Charisma, and immunity to being charmed and frightened. It otherwise uses the possessed target's statistics, but doesn't gain access to the target's knowledge, class features, or proficiencies.

The possession lasts until the body drops to 0 hit points, the ghost ends it as a bonus action, or the ghost is turned or forced out by an effect like the \textit{dispel evil and good} spell. When the possession ends, the ghost reappears in an unoccupied space within 5 feet of the body. The target is immune to this ghost's Possession for 24 hours after succeeding on the saving throw or after the possession ends.
\end{multicols}
\end{DndMonster}

\subsection{Ghoul}
\begin{DndMonster}[width=\textwidth + 8pt]{Ghoul}
\begin{multicols}{2}
\DndMonsterType{Medium undead}
\DndMonsterBasics[armor-class={12}, hit-points={22 (5d8)}, speed={30 ft.}]
\MonsterStats{+1}{+2}{+0}{-2}{+0}{-2}
\DndMonsterDetails[saving-throws={}, skills={}, damage-immunities={poison}, damage-resistances={}, damage-vulnerabilities={}, condition-immunities={charmed, exhaustion, poisoned}, senses={darkvision 60 ft., passive Perception 10}, languages={Common}, challenge={1 (200 XP)}]
\DndMonsterSection{Actions}
\DndMonsterAttack[
	name=Bite,
	distance=melee,
	type=weapon,
	mod=+2,
	reach=5,
	dmg=\DndDice{2d6 + 2},
	dmg-type=piercing
]
\DndMonsterAttack[
	name=Claws,
	distance=melee,
	type=weapon,
	mod=+4,
	reach=5,
	dmg=\DndDice{2d4 + 2},
	dmg-type=slashing,
	extra={. If the target is a creature other than an elf or undead, it must succeed on a DC 10 Constitution saving throw or be paralyzed for 1 minute. The target can repeat the saving throw at the end of each of its turns, ending the effect on itself on a success.}
]
\end{multicols}
\end{DndMonster}

\subsection{Grick}
\begin{DndMonster}[width=\textwidth + 8pt]{Grick}
\begin{multicols}{2}
\DndMonsterType{Medium monstrosity}
\DndMonsterBasics[armor-class={14 (natural armor)}, hit-points={27 (6d8)}, speed={30 ft., climb 30 ft.}]
\MonsterStats{+2}{+2}{+0}{-4}{+2}{-3}
\DndMonsterDetails[saving-throws={}, skills={}, damage-immunities={}, damage-resistances={bludgeoning, piercing, and slashing from nonmagical attacks}, damage-vulnerabilities={}, condition-immunities={}, senses={darkvision 60 ft., passive Perception 12}, languages={—}, challenge={2 (450 XP)}]
\DndMonsterAction{Stone Camouflage} The grick has advantage on Dexterity (Stealth) checks made to hide in rocky terrain.

\DndMonsterSection{Actions}
\DndMonsterAction{Multiattack} The grick makes one attack with its tentacles. If that attack hits, the grick can make one beak attack against the same target.
\DndMonsterAttack[
	name=Tentacles,
	distance=melee,
	type=weapon,
	mod=+4,
	reach=5,
	dmg=\DndDice{2d6 + 2},
	dmg-type=slashing
]
\DndMonsterAttack[
	name=Beak,
	distance=melee,
	type=weapon,
	mod=+4,
	reach=5,
	dmg=\DndDice{1d6 + 2},
	dmg-type=piercing
]
\end{multicols}
\end{DndMonster}

\subsection{Griffon}
\begin{DndMonster}[width=\textwidth + 8pt]{Griffon}
\begin{multicols}{2}
\DndMonsterType{Large monstrosity}
\DndMonsterBasics[armor-class={12}, hit-points={59 (7d10 + 21)}, speed={30 ft., fly 80 ft.}]
\MonsterStats{+4}{+2}{+3}{-4}{+1}{-1}
\DndMonsterDetails[saving-throws={}, skills={Perception +5}, damage-immunities={}, damage-resistances={}, damage-vulnerabilities={}, condition-immunities={}, senses={darkvision 60 ft., passive Perception 15}, languages={—}, challenge={2 (450 XP)}]
\DndMonsterAction{Keen Sight} The griffon has advantage on Wisdom (Perception) checks that rely on sight.

\DndMonsterSection{Actions}
\DndMonsterAction{Multiattack} The griffon makes two attacks: one with its beak and one with its claws.
\DndMonsterAttack[
	name=Beak,
	distance=melee,
	type=weapon,
	mod=+6,
	reach=5,
	dmg=\DndDice{1d8 + 4},
	dmg-type=piercing
]
\DndMonsterAttack[
	name=Claws,
	distance=melee,
	type=weapon,
	mod=+6,
	reach=5,
	dmg=\DndDice{2d6 + 4},
	dmg-type=slashing
]
\end{multicols}
\end{DndMonster}

\FloatBarrier
\section{Monsters (H)}\label{sec:monsters-h}

\subsection{Harpy}
\begin{DndMonster}[width=\textwidth + 8pt]{Harpy}
\begin{multicols}{2}
\DndMonsterType{Medium monstrosity}
\DndMonsterBasics[armor-class={11}, hit-points={38 (7d8 + 7)}, speed={20 ft., fly 40 ft.}]
\MonsterStats{+1}{+1}{+1}{-2}{+0}{+1}
\DndMonsterDetails[saving-throws={}, skills={}, damage-immunities={}, damage-resistances={}, damage-vulnerabilities={}, condition-immunities={}, senses={passive Perception 10}, languages={Common}, challenge={1 (200 XP)}]
\DndMonsterSection{Actions}
\DndMonsterAction{Multiattack} The harpy makes two attacks: one with its claws and one with its club.
\DndMonsterAttack[
	name=Claws,
	distance=melee,
	type=weapon,
	mod=+3,
	reach=5,
	dmg=\DndDice{2d4 + 1},
	dmg-type=slashing
]
\DndMonsterAttack[
	name=Club,
	distance=melee,
	type=weapon,
	mod=+3,
	reach=5,
	dmg=\DndDice{1d4 + 1},
	dmg-type=bludgeoning
]
\DndMonsterAction{Luring Song}
The harpy sings a magical melody. Every humanoid and giant within 300 feet of the harpy that can hear the song must succeed on a DC 11 Wisdom saving throw or be charmed until the song ends. The harpy must take a bonus action on its subsequent turns to continue singing. It can stop singing at any time. The song ends if the harpy is incapacitated.

While charmed by the harpy, a target is incapacitated and ignores the songs of other harpies. If the charmed target is more than 5 feet away from the harpy, the target must move on its turn toward the harpy by the most direct route, trying to get within 5 feet. It doesn't avoid opportunity attacks, but before moving into damaging terrain, such as lava or a pit, and whenever it takes damage from a source other than the harpy, the target can repeat the saving throw. A charmed target can also repeat the saving throw at the end of each of its turns. If the saving throw is successful, the effect ends on it.

A target that successfully saves is immune to this harpy's song for the next 24 hours.
\end{multicols}
\end{DndMonster}

\subsection{Hippogriff}
\begin{DndMonster}[width=\textwidth + 8pt]{Hippogriff}
\begin{multicols}{2}
\DndMonsterType{Large monstrosity}
\DndMonsterBasics[armor-class={11}, hit-points={19 (3d10 + 3)}, speed={40 ft., fly 60 ft.}]
\MonsterStats{+3}{+1}{+1}{-4}{+1}{-1}
\DndMonsterDetails[saving-throws={}, skills={Perception +5}, damage-immunities={}, damage-resistances={}, damage-vulnerabilities={}, condition-immunities={}, senses={passive Perception 15}, languages={—}, challenge={1 (200 XP)}]
\DndMonsterAction{Keen Sight} The hippogriff has advantage on Wisdom (Perception) checks that rely on sight.

\DndMonsterSection{Actions}
\DndMonsterAction{Multiattack} The hippogriff makes two attacks: one with its beak and one with its claws.
\DndMonsterAttack[
	name=Beak,
	distance=melee,
	type=weapon,
	mod=+5,
	reach=5,
	dmg=\DndDice{1d10 + 3},
	dmg-type=piercing
]
\DndMonsterAttack[
	name=Claws,
	distance=melee,
	type=weapon,
	mod=+5,
	reach=5,
	dmg=\DndDice{2d6 + 3},
	dmg-type=slashing
]
\end{multicols}
\end{DndMonster}



\subsection{Hydra}
\begin{DndMonster}[width=\textwidth + 8pt]{Hydra}
\begin{multicols}{2}
\DndMonsterType{Huge monstrosity}
\DndMonsterBasics[armor-class={15 (natural armor)}, hit-points={172 (15d12 + 75)}, speed={30 ft., swim 30 ft.}]
\MonsterStats{+5}{+1}{+5}{-4}{+0}{-2}
\DndMonsterDetails[saving-throws={}, skills={Perception +6}, damage-immunities={}, damage-resistances={}, damage-vulnerabilities={}, condition-immunities={}, senses={darkvision 60 ft., passive Perception 16}, languages={—}, challenge={8 (3,900 XP)}]
\DndMonsterAction{Hold Breath} The hydra can hold its breath for 1 hour.

\DndMonsterAction{Multiple Heads} The hydra has five heads. While it has more than one head, the hydra has advantage on saving throws against being blinded, charmed, deafened, frightened, stunned, and knocked unconscious.

Whenever the hydra takes 25 or more damage in a single turn, one of its heads dies. If all its heads die, the hydra dies.

At the end of its turn, it grows two heads for each of its heads that died since its last turn, unless it has taken fire damage since its last turn. The hydra regains 10 hit points for each head regrown in this way.

\DndMonsterAction{Reactive Heads} For each head the hydra has beyond one, it gets an extra reaction that can be used only for opportunity attacks.

\DndMonsterAction{Wakeful} While the hydra sleeps, at least one of its heads is awake.

\DndMonsterSection{Actions}
\DndMonsterAction{Multiattack} The hydra makes as many bite attacks as it has heads.
\DndMonsterAttack[
	name=Bite,
	distance=melee,
	type=weapon,
	mod=+8,
	reach=10,
	dmg=\DndDice{1d10 + 5},
	dmg-type=piercing
]
\end{multicols}
\end{DndMonster}

\FloatBarrier
\section{Monsters (I)}\label{sec:monsters-i}

\subsection{Invisible Stalker}


\FloatBarrier
\section{Monsters (J)}
\subsection{Jelly, Ochre}
\begin{DndMonster}[width=\textwidth + 8pt]{Ochre Jelly}
\begin{multicols}{2}
\DndMonsterType{Large ooze}
\DndMonsterBasics[armor-class={8}, hit-points={45 (6d10 + 12)}, speed={10 ft., climb 10 ft.}]
\MonsterStats{+2}{-2}{+2}{-4}{-2}{-5}
\DndMonsterDetails[saving-throws={}, skills={}, damage-immunities={lightning, slashing}, damage-resistances={acid}, damage-vulnerabilities={}, condition-immunities={blinded, charmed, deafened, exhaustion, frightened, prone}, senses={blindsight 60 ft. (blind beyond this radius), passive Perception 8}, languages={—}, challenge={2 (450 XP)}]
\DndMonsterAction{Amorphous} The jelly can move through a space as narrow as 1 inch wide without squeezing.

\DndMonsterAction{Spider Climb} The jelly can climb difficult surfaces, including upside down on ceilings, without needing to make an ability check.

\DndMonsterSection{Actions}
\DndMonsterAttack[
	name=Pseudopod,
	distance=melee,
	type=weapon,
	mod=+4,
	reach=5,
	dmg=\DndDice{2d6 + 2},
	dmg-type=bludgeoning,
	extra={ plus 3 (1d6) acid damage.}
]
\DndMonsterAction{Split}
When a jelly that is Medium or larger is subjected to lightning or slashing damage, it splits into two new jellies if it has at least 10 hit points. Each new jelly has hit points equal to half the original jelly’s, rounded down. New jellies are one size smaller than the original jelly.
\end{multicols}
\end{DndMonster}


\FloatBarrier
\section{Monsters (K)}\label{sec:monsters-k}

\subsection{Kraken}
\begin{DndMonster}[width=\textwidth + 8pt]{Kraken}
\begin{multicols}{2}
\DndMonsterType{Gargantuan monstrosity (titan)**}
\DndMonsterBasics[armor-class={18 (natural armor)}, hit-points={472 (27d20 + 189)}, speed={20 ft., swim 60 ft.}]
\MonsterStats{+10}{+0}{+7}{+6}{+4}{+5}
\DndMonsterDetails[saving-throws={Str +17, Dex +7, Con +14, Int +13, Wis +11}, skills={}, damage-immunities={lightning; bludgeoning, piercing, and slashing from nonmagical attacks}, damage-resistances={}, damage-vulnerabilities={}, condition-immunities={frightened, paralyzed}, senses={truesight 120 ft., passive Perception 14}, languages={understands Abyssal, Celestial, Infernal, and Primordial but can't speak, telepathy 120 ft.}, challenge={23 (50,000 XP)}]
\DndMonsterAction{Amphibious} The kraken can breathe air and water.

\DndMonsterAction{Freedom of Movement} The kraken ignores difficult terrain, and magical effects can't reduce its speed or cause it to be restrained. It can spend 5 feet of movement to escape from nonmagical restraints or being grappled.

\DndMonsterAction{Siege Monster} The kraken deals double damage to objects and structures.

\DndMonsterSection{Actions}
\DndMonsterAction{Multiattack} The kraken makes three tentacle attacks, each of which it can replace with one use of Fling.
\DndMonsterAttack[
	name=Bite,
	distance=melee,
	type=weapon,
	mod=+17,
	reach=5,
	dmg=\DndDice{3d8 + 10},
	dmg-type=piercing,
	extra={. If the target is a Large or smaller creature grappled by the kraken, that creature is swallowed, and the grapple ends. While swallowed, the creature is blinded and restrained, it has total cover against attacks and other effects outside the kraken, and it takes 42 (12d6) acid damage at the start of each of the kraken's turns. \\ If the kraken takes 50 damage or more on a single turn from a creature inside it, the kraken must succeed on a DC 25 Constitution saving throw at the end of that turn or regurgitate all swallowed creatures, which fall prone in a space within 10 feet of the kraken. If the kraken dies, a swallowed creature is no longer restrained by it and can escape from the corpse using 15 feet of movement, exiting prone.}
]
\DndMonsterAttack[
	name=Tentacle,
	distance=melee,
	type=weapon,
	mod=+17,
	reach=30,
	dmg=\DndDice{3d6 + 10},
	dmg-type=bludgeoning,
	extra={, and the target is grappled (escape DC 18). Until this grapple ends, the target is restrained. The kraken has ten tentacles, each of which can grapple one target.}
]
\DndMonsterAction{Fling}
One Large or smaller object held or creature grappled by the kraken is thrown up to 60 feet in a random direction and knocked prone. If a thrown target strikes a solid surface, the target takes 3 (1d6) bludgeoning damage for every 10 feet it was thrown. If the target is thrown at another creature, that creature must succeed on a DC 18 Dexterity saving throw or take the same damage and be knocked prone.
\DndMonsterAction{Lightning Storm (recharge 5-6)}
The kraken magically creates three bolts of lightning, each of which can strike a target the kraken can see within 120 feet of it. A target must make a DC 23 Dexterity saving throw, taking 22 (4d10) lightning damage on a failed save, or half as much damage on a successful one.

\DndMonsterSection{Legendary Actions}
The Kraken can take 3 legendary actions, choosing from the options below. Only one legendary action option can be used at a time and only at the end of another creature's turn. The Kraken regains spent legendary actions at the start of its turn.
\begin{DndMonsterLegendaryActions}
\DndMonsterLegendaryAction{Tentacle Attack or Fling}{The kraken makes one tentacle attack or uses its Fling.}
\DndMonsterLegendaryAction{Lightning Storm (Costs 2 Actions)}{The kraken uses Lightning Storm.}
\DndMonsterLegendaryAction{Ink Cloud (Costs 3 Actions)}{While underwater, the kraken expels an ink cloud in a 60-foot radius. The cloud spreads around corners, and that area is heavily obscured to creatures other than the kraken. Each creature other than the kraken that ends its turn there must succeed on a DC 23 Constitution saving throw, taking 16 (3d10) poison damage on a failed save, or half as much damage on a successful one. A strong current disperses the cloud, which otherwise disappears at the end of the kraken's next turn.}
\end{DndMonsterLegendaryActions}
\end{multicols}
\end{DndMonster}
\FloatBarrier
\section{Monsters (L)}\label{sec:monsters-l}

\subsection{Lich}
\begin{DndMonster}[width=\textwidth + 8pt]{Lich}
\begin{multicols}{2}
\DndMonsterType{Medium undead}
\DndMonsterBasics[armor-class={17 (natural armor)}, hit-points={135 (18d8 + 54)}, speed={30 ft.}]
\MonsterStats{+0}{+3}{+3}{+5}{+2}{+3}
\DndMonsterDetails[saving-throws={Con +10, Int +12, Wis +9}, skills={Arcana +18, History +12, Insight +9, Perception +9}, damage-immunities={poison; bludgeoning, piercing, and slashing from nonmagical attacks}, damage-resistances={cold, lightning, necrotic}, damage-vulnerabilities={}, condition-immunities={charmed, exhaustion, frightened, paralyzed, poisoned}, senses={truesight 120 ft., passive Perception 19}, languages={Common plus up to five other languages}, challenge={21 (33,000 XP)}]
\DndMonsterAction{Legendary Resistance (3/Day)} If the lich fails a saving throw, it can choose to succeed instead.

\DndMonsterAction{Rejuvenation} If it has a phylactery, a destroyed lich gains a new body in 1d10 days, regaining all its hit points and becoming active again. The new body appears within 5 feet of the phylactery.

\DndMonsterAction{Spellcasting} The lich is an 18th-level spellcaster. Its spellcasting ability is Intelligence (spell save DC 20, +12 to hit with spell attacks). The lich has the following arcanist spells prepared:

\DndMonsterAction{Turn Resistance} The lich has advantage on saving throws against any effect that turns undead.

\DndMonsterSection{Actions}
\DndMonsterAttack[
	name=Paralyzing Touch,
	distance=melee,
	type=spell,
	mod=+12,
	reach=5,
	dmg=\DndDice{3d6},
	dmg-type=cold,
	extra={. The target must succeed on a DC 18 Constitution saving throw or be paralyzed for 1 minute. The target can repeat the saving throw at the end of each of its turns, ending the effect on itself on a success.}
]

\DndMonsterSection{Legendary Actions}
The Lich can take 3 legendary actions, choosing from the options below. Only one legendary action option can be used at a time and only at the end of another creature's turn. The Lich regains spent legendary actions at the start of its turn.
\begin{DndMonsterLegendaryActions}
\DndMonsterLegendaryAction{Cantrip}{The lich casts a cantrip.}
\DndMonsterLegendaryAction{Paralyzing Touch (Costs 2 Actions)}{The lich uses its Paralyzing Touch.}
\DndMonsterLegendaryAction{Frightening Gaze (Costs 2 Actions)}{The lich fixes its gaze on one creature it can see within 10 feet of it. The target must succeed on a DC 18 Wisdom saving throw against this magic or become frightened for 1 minute. The frightened target can repeat the saving throw at the end of each of its turns, ending the effect on itself on a success. If a target's saving throw is successful or the effect ends for it, the target is immune to the lich's gaze for the next 24 hours.}
\DndMonsterLegendaryAction{Disrupt Life (Costs 3 Actions)}{Each living creature within 20 feet of the lich must make a DC 18 Constitution saving throw against this magic, taking 21 (6d6) necrotic damage on a failed save, or half as much damage on a successful one.}
\end{DndMonsterLegendaryActions}
\end{multicols}
\end{DndMonster}
\FloatBarrier
\section{Monsters (M)}\label{sec:monsters-m}

\subsection{Manticore}
\begin{DndMonster}[width=\textwidth + 8pt]{Manticore}
\begin{multicols}{2}
\DndMonsterType{Large monstrosity}
\DndMonsterBasics[armor-class={14 (natural armor)}, hit-points={68 (8d10 + 24)}, speed={30 ft., fly 50 ft.}]
\MonsterStats{+3}{+3}{+3}{-2}{+1}{-1}
\DndMonsterDetails[saving-throws={}, skills={}, damage-immunities={}, damage-resistances={}, damage-vulnerabilities={}, condition-immunities={}, senses={darkvision 60 ft., passive Perception 11}, languages={Common}, challenge={3 (700 XP)}]
\DndMonsterAction{Tail Spike Regrowth} The manticore has twenty-four tail spikes. Used spikes regrow when the manticore finishes a long rest.

\DndMonsterSection{Actions}
\DndMonsterAction{Multiattack} The manticore makes three attacks: one with its bite and two with its claws or three with its tail spikes.
\DndMonsterAttack[
	name=Bite,
	distance=melee,
	type=weapon,
	mod=+5,
	reach=5,
	dmg=\DndDice{1d8 + 3},
	dmg-type=piercing
]
\DndMonsterAttack[
	name=Claw,
	distance=melee,
	type=weapon,
	mod=+5,
	reach=5,
	dmg=\DndDice{1d6 + 3},
	dmg-type=slashing
]
\DndMonsterAttack[
	name=Tail Spike,
	distance=ranged,
	type=weapon,
	mod=+5,
	range=100/200,
	dmg=\DndDice{1d8 + 3},
	dmg-type=piercing
]
\end{multicols}
\end{DndMonster}

\subsection{Medusa}
\begin{DndMonster}[width=\textwidth + 8pt]{Medusa}
\begin{multicols}{2}
\DndMonsterType{Medium monstrosity}
\DndMonsterBasics[armor-class={15 (natural armor)}, hit-points={127 (17d8 + 51)}, speed={30 ft.}]
\MonsterStats{+0}{+2}{+3}{+1}{+1}{+2}
\DndMonsterDetails[saving-throws={}, skills={Deception +5, Insight +4, Perception +4, Stealth +5}, damage-immunities={}, damage-resistances={}, damage-vulnerabilities={}, condition-immunities={}, senses={darkvision 60 ft., passive Perception 14}, languages={Common}, challenge={6 (2,300 XP)}]
\DndMonsterAction{Petrifying Gaze} When a creature that can see the medusa's eyes starts its turn within 30 feet of the medusa, the medusa can force it to make a DC 14 Constitution saving throw if the medusa isn't incapacitated and can see the creature. If the saving throw fails by 5 or more, the creature is instantly petrified. Otherwise, a creature that fails the save begins to turn to stone and is restrained. The restrained creature must repeat the saving throw at the end of its next turn, becoming petrified on a failure or ending the effect on a success. The petrification lasts until the creature is freed by the \textit{greater restoration} spell or other magic.

Unless surprised, a creature can avert its eyes to avoid the saving throw at the start of its turn. If the creature does so, it can't see the medusa until the start of its next turn, when it can avert its eyes again. If the creature looks at the medusa in the meantime, it must immediately make the save.

If the medusa sees itself reflected on a polished surface within 30 feet of it and in an area of bright light, the medusa is, due to its curse, affected by its own gaze.

\DndMonsterSection{Actions}
\DndMonsterAction{Multiattack} The medusa makes either three melee attacks—one with its snake hair and two with its shortsword—or two ranged attacks with its longbow.
\DndMonsterAttack[
	name=Snake Hair,
	distance=melee,
	type=weapon,
	mod=+5,
	reach=5,
	dmg=\DndDice{1d4 + 2},
	dmg-type=piercing,
	extra={ plus 14 (4d6) poison damage.}
]
\DndMonsterAttack[
	name=Shortsword,
	distance=melee,
	type=weapon,
	mod=+5,
	reach=5,
	dmg=\DndDice{1d6 + 2},
	dmg-type=piercing
]
\DndMonsterAttack[
	name=Longbow,
	distance=ranged,
	type=weapon,
	mod=+5,
	range=150/600,
	dmg=\DndDice{1d8 + 2},
	dmg-type=piercing,
	extra={ plus 7 (2d6) poison damage.}
]
\end{multicols}
\end{DndMonster}

\subsection{Mimic}
\begin{DndMonster}[width=\textwidth + 8pt]{Mimic}
\begin{multicols}{2}
\DndMonsterType{Medium monstrosity (shapechanger)}
\DndMonsterBasics[armor-class={12 (natural armor)}, hit-points={58 (9d8 + 18)}, speed={15 ft.}]
\MonsterStats{+3}{+1}{+2}{-3}{+1}{-1}
\DndMonsterDetails[saving-throws={}, skills={Stealth +5}, damage-immunities={acid}, damage-resistances={}, damage-vulnerabilities={}, condition-immunities={prone}, senses={darkvision 60 ft., passive Perception 11}, languages={—}, challenge={2 (450 XP)}]
\DndMonsterAction{Shapechanger} The mimic can use its action to polymorph into an object or back into its true, amorphous form. Its statistics are the same in each form. Any equipment it is wearing or carrying isn't transformed. It reverts to its true form if it dies.

\DndMonsterAction{Adhesive (Object Form Only)} The mimic adheres to anything that touches it. A Huge or smaller creature adhered to the mimic is also grappled by it (escape DC 13). Ability checks made to escape this grapple have disadvantage.

\DndMonsterAction{False Appearance (Object Form Only)} While the mimic remains motionless, it is indistinguishable from an ordinary object.

\DndMonsterAction{Grappler} The mimic has advantage on attack rolls against any creature grappled by it.

\DndMonsterSection{Actions}
\DndMonsterAttack[
	name=Pseudopod,
	distance=melee,
	type=weapon,
	mod=+5,
	reach=5,
	dmg=\DndDice{1d8 + 3},
	dmg-type=bludgeoning,
	extra={. If the mimic is in object form, the target is subjected to its Adhesive trait.}
]
\DndMonsterAttack[
	name=Bite,
	distance=melee,
	type=weapon,
	mod=+5,
	reach=5,
	dmg=\DndDice{1d8 + 3},
	dmg-type=piercing,
	extra={ plus 4 (1d8) acid damage.}
]
\end{multicols}
\end{DndMonster}

\subsection{Mummy}
\begin{DndMonster}[width=\textwidth + 8pt]{Mummy}
\begin{multicols}{2}
\DndMonsterType{Medium undead**}
\DndMonsterBasics[armor-class={11 (natural armor)}, hit-points={58 (9d8 + 18)}, speed={20 ft.}]
\MonsterStats{+3}{-1}{+2}{-2}{+0}{+1}
\DndMonsterDetails[saving-throws={Wis +2}, skills={}, damage-immunities={necrotic, poison}, damage-resistances={bludgeoning, piercing, and slashing from nonmagical attacks}, damage-vulnerabilities={fire}, condition-immunities={charmed, exhaustion, frightened, paralyzed, poisoned}, senses={darkvision 60 ft., passive Perception 10}, languages={the languages it knew in life}, challenge={3 (700 XP)}]
\DndMonsterSection{Actions}
\DndMonsterAction{Multiattack} The mummy can use its Dreadful Glare and makes one attack with its rotting fist.
\DndMonsterAttack[
	name=Rotting Fist,
	distance=melee,
	type=weapon,
	mod=+5,
	reach=5,
	dmg=\DndDice{2d6 + 3},
	dmg-type=bludgeoning,
	extra={ plus 10 (3d6) necrotic damage. If the target is a creature, it must succeed on a DC 12 Constitution saving throw or be cursed with mummy rot. The cursed target can't regain hit points, and its hit point maximum decreases by 10 (3d6) for every 24 hours that elapse. If the curse reduces the target's hit point maximum to 0, the target dies, and its body turns to dust. The curse lasts until removed by the \textit{remove curse} spell or other magic.}
]
\DndMonsterAction{Mummy's Curse (recharge 5-6)}
The mummy targets one creature it can see within 60 feet of it. If the target can see the mummy, it must succeed on a DC 11 Wisdom saving throw against this magic or become frightened until the end of the mummy's next turn. If the target fails the saving throw by 5 or more, it is also paralyzed for the same duration. A target that succeeds on the saving throw is immune to the Dreadful Glare of all mummies (but not mummy lords) for the next 24 hours.
\end{multicols}
\end{DndMonster}

\subsection{Mummy Lord}
\begin{DndMonster}[width=\textwidth + 8pt]{Mummy Lord}
\begin{multicols}{2}
\DndMonsterType{Medium undead}
\DndMonsterBasics[armor-class={17 (natural armor)}, hit-points={97 (13d8 + 39)}, speed={20 ft.}]
\MonsterStats{+4}{+0}{+3}{+0}{+4}{+3}
\DndMonsterDetails[saving-throws={Con +8, Int +5, Wis +9, Cha +8}, skills={History +5, Religion +5}, damage-immunities={necrotic, poison; bludgeoning, piercing, and slashing from nonmagical attacks}, damage-resistances={}, damage-vulnerabilities={fire}, condition-immunities={charmed, exhaustion, frightened, paralyzed, poisoned}, senses={darkvision 60 ft., passive Perception 14}, languages={the languages it knew in life}, challenge={15 (13,000 XP)}]
\DndMonsterAction{Magic Resistance} The mummy lord has advantage on saving throws against spells and other magical effects.

\DndMonsterAction{Rejuvenation} A destroyed mummy lord gains a new body in 24 hours if its heart is intact, regaining all its hit points and becoming active again. The new body appears within 5 feet of the mummy lord's heart.

\DndMonsterAction{Spellcasting} The mummy lord is a 10th-level spellcaster. Its spellcasting ability is Wisdom (spell save DC 17, +9 to hit with spell attacks). The mummy lord has the following priest spells prepared:

Cantrips (at will): \textit{sacred flame}, \textit{thaumaturgy}

1st level (4 slots): \textit{command}, \textit{guiding bolt}, \textit{shield of faith}

2nd level (3 slots): \textit{hold person}, \textit{silence}, \textit{spiritual weapon}

3rd level (3 slots): \textit{animate dead}, \textit{dispel magic}

4th level (3 slots): \textit{divination}, \textit{guardian of faith}

5th level (2 slots):  \textit{contagion}, \textit{insect plague}

6th level (1 slot): \textit{harm}

\DndMonsterSection{Actions}
\DndMonsterAction{Multiattack} The mummy can use its Dreadful Glare and makes one attack with its rotting fist.
\DndMonsterAttack[
	name=Rotting Fist,
	distance=melee,
	type=weapon,
	mod=+9,
	reach=5,
	dmg=\DndDice{3d6 + 4},
	dmg-type=bludgeoning,
	extra={ plus 21 (6d6) necrotic damage. If the target is a creature, it must succeed on a DC 16 Constitution saving throw or be cursed with mummy rot. The cursed target can't regain hit points, and its hit point maximum decreases by 10 (3d6) for every 24 hours that elapse. If the curse reduces the target's hit point maximum to 0, the target dies, and its body turns to dust. The curse lasts until removed by the \textit{remove curse} spell or other magic.}
]
\DndMonsterAction{Mummy's Curse (recharge 5-6)}
The mummy lord targets one creature it can see within 60 feet of it. If the target can see the mummy lord, it must succeed on a DC 16 Wisdom saving throw against this magic or become frightened until the end of the mummy's next turn. If the target fails the saving throw by 5 or more, it is also paralyzed for the same duration. A target that succeeds on the saving throw is immune to the Dreadful Glare of all mummies and mummy lords for the next 24 hours.

\DndMonsterSection{Legendary Actions}
The Mummy Lord can take 3 legendary actions, choosing from the options below. Only one legendary action option can be used at a time and only at the end of another creature's turn. The Mummy Lord regains spent legendary actions at the start of its turn.

\begin{DndMonsterLegendaryActions}

\DndMonsterLegendaryAction{Attack}{The mummy lord makes one attack with its rotting fist or uses its Dreadful Glare.}
\DndMonsterLegendaryAction{Blinding Dust}{Blinding dust and sand swirls magically around the mummy lord. Each creature within 5 feet of the mummy lord must succeed on a DC 16 Constitution saving throw or be blinded until the end of the creature's next turn.}
\DndMonsterLegendaryAction{Blasphemous Word (Costs 2 Actions)}{The mummy lord utters a blasphemous word. Each non-undead creature within 10 feet of the mummy lord that can hear the magical utterance must succeed on a DC 16 Constitution saving throw or be stunned until the end of the mummy lord's next turn.}
\DndMonsterLegendaryAction{Channel Negative Energy (Costs 2 Actions)}{The mummy lord magically unleashes negative energy. Creatures within 60 feet of the mummy lord, including ones behind barriers and around corners, can't regain hit points until the end of the mummy lord's next turn.}
\DndMonsterLegendaryAction{Whirlwind of Sand (Costs 2 Actions)}{The mummy lord magically transforms into a whirlwind of sand, moves up to 60 feet, and reverts to its normal form. While in whirlwind form, the mummy lord is immune to all damage, and it can't be grappled, petrified, knocked prone, restrained, or stunned. Equipment worn or carried by the mummy lord remain in its possession.}
\end{DndMonsterLegendaryActions}
\end{multicols}
\end{DndMonster}
