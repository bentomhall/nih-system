\section{Rogue}\label{class:rogue}

Rogues are those who focus their training on fighting smarter, not harder. Not for them is the clash of two-handed weapons on heavy armor--they fight nimbly and evade blows, waiting for the perfect moment to strike where it hurts. Out of combat they have a broad arsenal of capabilities--they are the undisputed masters of skills and skill tricks. At higher levels, they gain overtly magical effects, often tied to the spaces Between. Subtlety and stealth often characterize a rogue.

Rogues are both damage dealers and controllers, capable of inflicting conditions on enemies. They do not do well under direct pressure, preferring to attack opportunistically where their allies are commanding the enemy's attention, either up close or from range.

Design Discussion: Less overtly criminal. Focuses on precision. Tons of skill tricks. Subclasses give overt magical abilities. SDCT 3/7/5/5

\subsection{Class Features}

As a rogue, you have the following class features.

\subsubsection{Hit Points}

\textbf{Hit Dice:} 1d8 per rogue level

\textbf{Hit Points at 1st Level:} 8 + your Constitution modifier

\textbf{Hit Points at Higher Levels:} 1d8 (or 5) + your Constitution modifier per rogue level after 1st

\subsubsection{Proficiencies}

\textbf{Armor:} Light armor

\textbf{Weapons:} Simple weapons, hand crossbows, longswords, rapiers, shortswords, acid vials, flasks of alchemist fire

\textbf{Tools:} Thieves' tools

\textbf{Saving Throws:} Dexterity, Intelligence

\textbf{Skills:} Choose four from Acrobatics, Athletics, Deception, Insight, Intimidation, Investigation, Perception, Performance, Persuasion, Sleight of Hand, and Stealth

\subsubsection{Equipment}

You start with the following equipment, in addition to the equipment granted by your background:
\begin{itemize}
	\item (\textit{a}) a rapier or (\textit{b}) a shortsword
	\item (\textit{a}) a shortbow and quiver of 20 arrows or (\textit{b}) a shortsword
	\item (\textit{a}) a burglar's pack, (\textit{b}) a dungeoneer's pack, or (\textit{c}) an explorer's pack
  \item (\textit{a}) Leather armor, two daggers, and thieves' tools
\end{itemize}

\begin{figure*}[htb]
\begin{DndTable}[header=The Rogue\label{tbl:rogue}]{lllXcccc}
	\textbf{Level} & \textbf{Proficiency} & \textbf{Sneak Attack} & \textbf{Features} & \textbf{Skill Tricks} & \textbf{Stamina} & \textbf{Aether} & \textbf{Aether Limit} \\
	1st & +2 & 1d6 & Expertise, Sneak Attack & -- & 1 + CON & 1 & 1 \\
	2nd & +2 & 1d6 & Cunning Action, Skill Tricks & 1 & 2 + CON & 1 & 1 \\
	3rd & +2 & 2d6 & Roguish Archetype & 2 & 3 + CON & 2 & 1 \\
	4th & +2 & 2d6 & Ability Score Improvement & 3 & 4 + CON & 2 & 1 \\
	5th & +3 & 3d6 & Uncanny Dodge & 3 & 5 + CON & 3 & 2 \\
	6th & +3 & 3d6 & Expertise, Roguish Archetype Feature & 4 & 6 + CON & 3 & 2 \\
	7th & +3 & 4d6 & Evasion, Improved Skill Tricks & 4 & 7 + CON & 4 & 2 \\
	8th & +3 & 4d6 & Ability Score Improvement & 5 & 8 + CON & 4 & 2 \\
	9th & +4 & 5d6 & Roguish Archetype Feature & 5 & 9 + CON & 5 & 2 \\
	10th & +4 & 5d6 & Ability Score Improvement & 6 & 10 + CON & 5 & 3 \\
	11th & +4 & 6d6 & Reliable Talent, Expert Skill Tricks & 6 & 11 + CON & 6 & 3 \\
	12th & +4 & 6d6 & Ability Score Improvement & 7 & 12 + CON & 6 & 3 \\
	13th & +5 & 7d6 & Roguish Archetype Feature & 7 & 13 + CON & 7 & 3 \\
	14th & +5 & 7d6 & Blindsense & 7 & 14 + CON & 7 & 3 \\
	15th & +5 & 8d6 & Slippery Mind, Master Skill Tricks & 7 & 15 + CON & 8 & 3 \\
	16th & +5 & 8d6 & Ability Score Improvement & 8 & 16 + CON & 8 & 3 \\
	17th & +6 & 9d6 & Roguish Archetype Feature & 8 & 17 + CON & 9 & 4 \\
	18th & +6 & 9d6 & Elusive & 8 & 18 + CON & 9 & 4 \\
	19th & +6 & 10d6 & Ability Score Improvement & 9 & 19 + CON & 10 & 4 \\
	20th & +6 & 11d6 & Stroke of Luck & 9 & 20 + CON & 10 & 4 \\
\end{DndTable}
\end{figure*}

\subsubsection{Expertise}

At 1st level, choose two of your skill proficiencies, or one of your skill proficiencies and your proficiency with thieves' tools. Your proficiency bonus is doubled for any ability check you make that uses either of the chosen proficiencies.

At 6th level, you can choose two more of your proficiencies (in skills or with thieves' tools) to gain this benefit.

\subsubsection{Sneak Attack}

Beginning at 1st level, you know how to strike subtly and exploit a foe's distraction. Once per turn, you can deal an extra 1d6 damage to one creature you hit with an attack if you have advantage on the attack roll. The attack must use a finesse or a ranged weapon.

You don't need advantage on the attack roll if another enemy of the target is within 5 feet of it, that enemy isn't \nameref{condition:incapacitated}, and you don't have disadvantage on the attack roll.

The amount of the extra damage increases as you gain levels in this class, as shown in the Sneak Attack column of the Rogue table.

\subsubsection{Cunning Action}

Starting at 2nd level, your quick thinking and agility allow you to move and act quickly. You can take a bonus action on each of your turns in combat. This action can be used only to take the Dash, Disengage, or Hide action.

\subsubsection{Skill Tricks}

Starting at 2nd level, you've learned additional ways to employ your abilities. You learn one \nameref{sec:skill-tricks-basic} of your choice, even if you don't have proficiency in that skill. See \nameref{ch:skill-tricks} for more details and the rules governing skill tricks.

You gain additional Skill Tricks as shown in the Skill Tricks column of the Rogue table. When you reach 7th level, you can learn \nameref{sec:skill-tricks-advanced}; at 11th level \nameref{sec:skill-tricks-expert}; and 17th level \nameref{sec:skill-tricks-master}. When you learn advanced, expert, or master skill tricks in this way, you do not have to meet any prerequisites.

When you gain access to a new Skill Trick, you can also swap any Skill Trick you know for a new one you could otherwise learn at that point.

\subsubsection{Roguish Archetype}

At 3rd level, you choose an archetype that you emulate in the exercise of your rogue abilities: , all detailed at the end of the class description. Your archetype choice grants you features at 3rd level and then again at 6th, 9th, 13th, and 17th level.

\subsubsection{Ability Score Improvement}

When you reach 4th level, and again at 8th, 10th, 12th, 16th, and 19th level, you can increase one ability score of your choice by 1. As normal, you can't increase an ability score above 20 using this feature.

You can also pick a Skill Trick (included in the skill tricks column of the Rogue table) but you must meet the prerequisites for skill tricks learned in this way. See \nameref{ch:skill-tricks} for that list.

\subsubsection{Uncanny Dodge}

Starting at 5th level, when an attacker that you can see hits you with an attack, you can use your reaction to halve the attack's damage against you. You can use this ability even if you have used your reaction already by expending 2 STA.

\subsubsection{Evasion}

Beginning at 7th level, you can nimbly dodge out of the way of certain area effects, such as a red dragon's fiery breath or an \nameref{spell:ice-storm} spell. When you are subjected to an effect that allows you to make a Dexterity saving throw to take only half damage, you instead take no damage if you succeed on the saving throw, and only half damage if you fail.

\subsubsection{Reliable Talent}

By 11th level, you have refined your chosen skills until they approach perfection. Whenever you make an ability check that lets you add your proficiency bonus, you can treat a d20 roll of 9 or lower as a 10. If you expend 3 STA, you can treat a d20 roll of 14 or lower as a 15.

\subsubsection{Blindsense}

Starting at 14th level, if you are able to hear, you are aware of the location of any hidden or invisible creature within 10 feet of you.

\subsubsection{Slippery Mind}

By 15th level, you have acquired greater mental strength. You gain proficiency in Wisdom saving throws.

\subsubsection{Elusive}

Beginning at 18th level, you are so evasive that attackers rarely gain the upper hand against you. No attack roll has advantage against you while you aren't incapacitated.

\subsubsection{Stroke of Luck}

At 20th level, you have an uncanny knack for succeeding when you need to. If your attack misses a target within range, you can turn the miss into a hit. Alternatively, if you fail an ability check, you can treat the d20 roll as a 20. Either option requires spending 6 STA.

\section{Roguish Archetypes}

Rogues have many features in common, including their emphasis on perfecting their skills, their precise and deadly approach to combat, and their increasingly quick reflexes. But different rogues steer those talents in varying directions, embodied by the rogue archetypes. Your choice of archetype is a reflection of your focus—not necessarily an indication of your chosen profession, but a description of your preferred techniques.

\subsection{Shadowdancer}

Shadowdancers are infiltration and covert work specialists. They've trained their souls to the degree that they can truly become one with the shadows, wrapping the substance of the Shadow plane around themselves and slipping between the cracks just as that liminal plane "slips between the cracks" of the other planes. This lends them both extraordinary grace in combat as well as enhanced stealth abilities.

\begin{figure}[htb]
\begin{DndTable}[header=Shadowdancer]{lX}
	\textbf{Rogue Level} & \textbf{Features} \\
	3 & Shadow Strike, Stalk the Shadows \\
	6 & Improved Uncanny Dodge \\
	9 & Hide in Plain Sight \\
	13 & Shrouded Nature \\
	17 & Ethereal Body \\
\end{DndTable}
\end{figure}

\subsubsection{Shadow Strike}
Starting at 3rd level when you take this archetype, your strikes while in dimmed lighting are harder to stop. You have advantage on any attack made while you or your target are in any lighting condition other than sunlight.

Additionally, when you hit with a weapon attack and have advantage on the attack, you can spend 1 STA to add your proficiency bonus to the damage dealt. If you do so, you can choose to have the entire attack deal necrotic damage.

\subsubsection{Stalk the Shadows}
Starting at 3rd level when you take this archetype, you can magically transport yourself between the shadows. Spend 1 AET and a bonus action to teleport to an area of shadow large enough to fit your body that you can see within 60 ft. You must be in an area of shadow large enough to fit your body already. For this ability, "shadow" includes any dimly-lit or unlit area as well as the shadows cast by objects and other obstructions. The vertical dimension of the shadow doesn't matter.

\subsubsection{Improved Uncanny Dodge}
Starting at 6th level, you can shunt incoming attacks partially into Shadow more effectively. When you use your Uncanny Dodge, the effect lasts until the end of the current turn instead of only for that attack.

\subsubsection{Hide in Plain Sight}
Starting at 9th level, you can wrap the stuff of Shadow around yourself. You can spend 1 AET to attempt to hide even when directly observed and/or in bright light. If you succeed at the attempt and do not reach total obscurement by the beginning of your next turn, you immediately become unhidden.

\subsubsection{Shrouded Nature}
Starting at 13th level, you have learned to shroud yourself in shadow even in bright light. As an action, you can expend 2 AET to create an aura of shadow around yourself for one hour. This magical aura leaves you lightly obscured and upgrades other sources of light obscurement to heavy obscurement. In addition, you always count as being in shadow for the purpose of Stalk the Shadows; neither your starting or ending point need to be shadowed by any external source.  

\subsubsection{Ethereal Body}
Starting at 17th level, you have learned to transition to the Border Shadow more easily than most. As a bonus action while you are affected by Shrouded Nature, you can spend an additional 1 AET to go fully ethereal. While you are ethereal you cannot be seen by any creature on the Mortal plane that does not have truesight, but you can see the Mortal plane. You can pass through walls and most barriers other than ones that explicitly affect the Border Shadow. You cannot affect the Mortal while ethereal. You can exit the Border Shadow as a bonus action.

When you go ethereal, you can bring your gear, the objects you are carrying, and any unconscious creatures. You cannot bring a conscious creature with you, willing or not.

\subsection{Trickster}

Tricksters hone their skills with less of a supernatural flair, but are no less effective for that. They gain better uses of mundane and magical objects and specialize in imposing conditions on their foes in combat.

\begin{figure}[htb]
	\begin{DndTable}[header=Trickster]{lX}
		\textbf{Rogue Level} & \textbf{Features} \\
		3 & Fast Hands, Expanded Sneak Attack \\
		6 & Confident Disguise, Distraction \\
		9 & Trick Attack \\
		13 & Magical Impersonation \\
		17 & Deceive the Universe \\
	\end{DndTable}
\end{figure}

\subsubsection{Expanded Sneak Attack}
Starting at 3rd level when you take this archetype, you can sneak attack with any melee weapon that does not have the heavy property in addition to any finesse or ranged weapon.

When you hit and apply Sneak Attack, you can spend 1 STA to treat any damage die that was less than half the maximum value as half the maximum value instead.

\subsubsection{Fast Hands}
Starting at 3rd level when you take this archetype, you can do any one of the following as part of your Cunning Action in addition to the normal effect:
\begin{itemize}
	\item Administer a potion to someone else.
	\item Draw or stow any number of weapons or other items on your person.
	\item Apply poison to a weapon.
	\item Set a trap, scatter caltrops, pour out oil, or similarly use any mundane piece of equipment that does not require an attack roll.
	\item Make a Wisdom (Medicine) check to stabilize someone.
	\item Flip any number of switches, pull any number of levers, or otherwise interact with the environment in ways that do not require ability checks.
\end{itemize}

\subsubsection{Confident Disguise}
Starting at 6th level, you can pull off disguises and impersonations more effectively. You have advantage on any attempt to persuade or deceive someone into believing you are someone else. If you spend 1 AET per person deceived, you can magically convince people against whom you succeeded that whatever documents you present (including blank paper) confirm your story in all details. If the documents are examined outside of your presence, the ruse will be revealed.

\subsubsection{Distraction}
Starting at 6th level, you've become even better at providing distractions for others. As an action, you can spend 1+ STA to attempt a distraction that affects a number of creatures of your choice that can see or hear you equal to the stamina expended. The targets must make a Wisdom (Perception) check opposed by your Charisma (Deception) check. On a failure, they are considered blind and deaf to everything but you until the end of your next turn. To maintain the deception, you must take the action again on your next turn(s), redoing the opposed check each time, but you do not need to expend more stamina.

\subsubsection{Trick Attack}
At 9th level, you learn to impose conditions on those that you hit with your Sneak Attacks. Each condition has a cost in either AET or STA, a saving throw necessary to apply the condition (or --- for those that are applied automatically by spending the resource), and a duration. Notation of "Save Ends" indicates that the target can re-attempt the saving throw at the end of their turns, ending the effect on a success. The DC for all of these saving throws is 8 + your proficiency bonus + your Dexterity modifier. If you have expertise in Sleight of Hand, the DC increases by half your proficiency bonus. You can only apply one condition each time you apply Sneak Attack.

\begin{DndTable}[header=Trick Attack]{lllX}
	\textbf{Condition} & \textbf{Cost} & \textbf{Save} & \textbf{Duration} \\
	\nameref{condition:blinded} & 4 STA & CON & 1 round \\
	\nameref{condition:broken} & 2 AET + 3 STA & WIS & Save Ends \\
	\nameref{condition:charmed} & 2 AET + 1 STA & WIS & Save Ends \\
	\nameref{condition:deafened} & 2 STA & CON & 1 round \\
	\nameref{condition:frightened} & 2 AET + 1 STA & WIS & 1 round \\
	\nameref{condition:prone} & 2 STA & STR & --- \\
	\nameref{condition:shaken} & 1 STA & --- & 1 round \\
	\nameref{condition:staggered} & 3 STA & CON & Save Ends \\
	\nameref{condition:stunned} & 2 AET + 2 STA & CON & 1 round \\
	\nameref{condition:unconscious} & 2 AET + 8 STA & CON & Save Ends \\
\end{DndTable}

\subsubsection{Magical Impersonation}
Starting at 13th level, you've learned to mold your aether to mimic that of other creatures. You gain the following benefits:
\begin{itemize}
	\item You can attune to and use any magic item, regardless of the normal requirements, and your attunement limit increases by 1.
	\item Spells and other effects that are triggered by perceptible conditions, creature type, or other similar criteria are only triggered if you choose them to be, whether or not you would normally qualify. You must spend 1 AET to gain this benefit, at the time you make the choice.
	\item You can impersonate the master of golems, automated defenses, and other constructs unless that is tied to a particular item rather than a person. You must spend 1 AET per minute you attempt to do this.
	\item You can activate magic items keyed to another person by spending 2 AET and 2 STA.
\end{itemize}

\subsubsection{Deceive the Universe}
Starting at 17, you can attempt to cast any spell or use any incantation (without needing the Ritual Scroll). Spend any combination of AET and STA equal to the base cost of the spell and make a Charisma (Deception) check against the DC listed in the table below. This attempt takes the same time as the regular casting time, but does not require any expensive components (consumed or not). On a success, the spell or incantation takes effect, using Charisma as the requisite spellcasting modifier where appropriate. Once you use this feature once, you incure 2 levels of \nameref{condition:exhaustion} every time you use it again until you finish a long rest.

\begin{DndTable}[header=Spell/Incantation Difficulty]{Xl}
	\textbf{Type} & \textbf{DC} \\
	Spell & 8 + base cost \\
	Incantations & --- \\
	Common & 10 \\
	Uncommon & 15 \\
	Rare & 18 \\
	Very Rare & 20 \\
	Legendary & 25
\end{DndTable}
