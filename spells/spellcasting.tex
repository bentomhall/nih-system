\chapter{Spellcasting}\label{ch:spellcasting}
Magic permeates fantasy gaming worlds and often appears in the form of a spell.

This chapter provides the rules for casting spells. Different character classes have distinctive ways of learning and preparing their spells, and monsters use spells in unique ways. Regardless of its source, a spell follows the rules here.

\section{What Is a Spell?}

A spell is a discrete magical effect, a single shaping of the magical energies that suffuse the multiverse into a specific, limited expression. In casting a spell, a character carefully plucks at the invisible strands of raw magic suffusing the world, pins them in place in a particular pattern, sets them vibrating in a specific way, and then releases them to unleash the desired effect—in most cases, all in the span of seconds.

Spells can be versatile tools, weapons, or protective wards. They can deal damage or undo it, impose or remove conditions (see appendix A), drain life energy away, and restore life to the dead.

Uncounted thousands of spells have been created over the course of the multiverse's history, and many of them are long forgotten. Some might yet lie recorded in crumbling spellbooks hidden in ancient ruins or trapped in the minds of dead gods. Or they might someday be reinvented by a character who has amassed enough power and wisdom to do so.

\subsection{Aether}
Every character has a pool of Aether which they draw on to produce magical effects. Spellcasting also draws on this pool, and spellcasters tend to have much larger Aether pools than non-spellcasters.

Some items have their own pools of aether that can be used to cast spells from them. These are separate from the character's aether.

\subsubsection{Aether Limit}
Each class has a limit on how much aether they can channel into any one thing. This is denoted on their class table, and increases with level.

\subsubsection{Aether Recovery}
Aether pools are completely recovered on completing a Long Rest. In addition, some classes have features that let them recover a limited portion more quickly, and some items grant reserves that can be used to restore some aether to a pool.

\subsection{Known Spells}

Before a spellcaster can use a spell, he or she must have the spell firmly fixed in mind, or must have access to the spell in a magic item. This process varies for different classes, as detailed in their descriptions. In general, no spellcaster can learn a spell whose base aether cost is higher than his Aether Limit, which grows with their class level. 

In every case, the number of spells a caster can have fixed in mind at any given time depends on the character's level.

\subsection{Casting Spells}
Casting a spell requires expending a certain amount of Aether from your personal reserves, as indicated in the spell. You cannot cast a spell that requires expending more aether than your Aether Limit or that requires more Aether than you have remaining. 

\subsubsection{Overcasting a spell}

When a spellcaster casts a spell using more aether than the base costs, the spell has a heightened effect and can bypass defenses that block or counter lower-powered spells. For example, if a creature is immune to spells that cost less than 5 aether and Torvald casts magic missile (which requires 2 aether) and expends 5 aether, the creature's defenses do not block that spell. This is called overcasting the spell.

Some spells, such as \textit{magic missile} and \textit{cure wounds}, have more powerful effects when overcast, as detailed in a spell's description.

\subsection{Casting in Armor}
Because of the mental focus and precise gestures required for spellcasting, armor inhibits spellcasting. You cannot cast spells with material or somatic components while wearing any physical armor (not including \nameref{spell:mage-armor}).

\subsection{Cantrips}

A cantrip is a spell that can be cast at will, without spending aether and without being prepared in advance. Repeated practice has fixed the spell in the caster's mind and infused the caster with the magic needed to produce the effect over and over. A cantrip costs 0 aether and cannot be overcast. Some cantrips gain more power as the caster becomes stronger; this is detailed in the entry itself.

\section{Casting a Spell}

When a character casts any spell, the same basic rules are followed, regardless of the character's class or the spell's effects.

Each spell description begins with a block of information, including the spell's name, aether cost, casting time, range, components, and duration. The rest of a spell entry describes the spell's effect.

\subsection{Casting Time}

Most spells require a single action to cast, but some spells require a bonus action, a reaction, or much more time to cast. You can't spend aether to cast a spell more than once per turn regardless of action costs.

\subsubsection{Bonus Action}

A spell cast with a bonus action is especially swift. Note that spending aether to cast a spell as a bonus action precludes using any other aether-using ability that turn. You can still cast cantrips, however.

\subsubsection{Reactions}

Some spells can be cast as reactions. These spells take a fraction of a second to bring about and are cast in response to some event. If a spell can be cast as a reaction, the spell description tells you exactly when you can do so and whether it interrupts the trigger or happens afterward.

\subsubsection{Longer Casting Times}

Certain spells require more time to cast: minutes or even hours. When you cast a spell with a casting time longer than a single action or reaction, you must spend your action each turn casting the spell, and you must maintain your concentration while you do so (see “Concentration” below). If your concentration is broken, the spell fails, but you don't expend aether. If you want to try casting the spell again, you must start over.

\subsection{Spell Range}
See \nameref{subsec:magic-range}.

\subsection{Components}

A spell's components are the physical requirements you must meet in order to cast it. Each spell's description indicates whether it requires verbal (V), somatic (S), or material (M) components. If you can't provide one or more of a spell's components, you are unable to cast the spell.

\subsubsection{Verbal (V)}

Most spells require the chanting of mystic words. The words themselves aren't the source of the spell's power; rather, the particular combination of sounds, with specific pitch and resonance, sets the threads of magic in motion. Thus, a character who is gagged or in an area of silence, such as one created by the \nameref{spell:silence} spell, can't cast a spell with a verbal component.

These mystic words are not in a recognizable language and can be immediately recognized as spellcasting by anyone who can hear the chanting (unless they are unintelligent or particularly ignorant). They cannot be hidden except by large amounts of ambient noise or intervening solid objects.

\subsubsection{Somatic (S)}

Spellcasting gestures might include a forceful gesticulation or an intricate set of gestures. If a spell requires a somatic component, the caster must have free use of at least one hand to perform these gestures.

These gestures cannot be performed in stealth. Anyone who can see the caster can see the gestures and may recognize them as components of spellcasting. Particularly ignorant people and animals may not recognize them as such.

\subsubsection{Material (M)}

Casting some spells requires particular objects, specified in parentheses in the component entry. A character can use a \textbf{component pouch} or a \textbf{spellcasting focus} (found in “\nameref{ch:equipment}") in place of the components specified for a spell. But if a cost is indicated for a component, a character must have that specific component before he or she can cast the spell.

If a spell states that a material component is consumed by the spell, the caster must provide this component for each casting of the spell.

A spellcaster must have a hand free to access a spell's material components—--or to hold a spellcasting focus—--but it can be the same hand that he or she uses to perform somatic components.

\subsection{Duration}

A spell's duration is the length of time the spell persists. A duration can be expressed in rounds, minutes, hours, or even years. Some spells specify that their effects last until the spells are dispelled or destroyed.

\subsubsection{Instantaneous}

Many spells are instantaneous. The spell harms, heals, creates, or alters a creature or an object in a way that can't be dispelled, because its magic exists only for an instant.

\subsubsection{Concentration}

Many spells require concentration. See \nameref{sec:concentration} for details.

\subsection{Spell Saving Throws}

Many spells specify that a target can make a saving throw to avoid some or all of a spell's effects. The spell specifies the ability that the target uses for the save and what happens on a success or failure. The GM may decide to tell you which creatures failed or succeeded on the saving throw, but this is not mandatory.

The DC to resist one of your spells equals 8 + your spellcasting ability modifier + your proficiency bonus + any special modifiers.

\subsection{Targets}
See \nameref{sec:targets} for the general rules that apply to all abilities, including spells and legendary effects.

\subsection{Spell Attack Rolls}

Some spells require the caster to make an attack roll to determine whether the spell effect hits the intended target. Your attack bonus with a spell attack equals your spellcasting ability modifier + your proficiency bonus.

\subsection{Combining Magical Effects}
See \nameref{sec:stacking} for the general rules.

The effects of different spells add together while the durations of those spells overlap. The effects of the same spell cast multiple times don't combine, however. Instead, the most potent effect—such as the highest bonus—from those castings applies while their durations overlap.

For example, if two priests cast \nameref{spell:bless} on the same target, that character gains the spell's benefit only once; he or she doesn't get to roll two bonus dice.

\section{Legendary Effects}
See \nameref{sec:legendary-effects} for the full details. 

Legendary effects are those whose power is too great to learn or cast as a normal spell. These can only be accessed via class features, feats, and special boons. They do not consume aether and effects that interact with spells do not interact with legendary effects unless they explicitly say they do.

\begin{DndSidebar}[float=b]{Conversion from 5e spells}
    These are starting points. Unlike fixed spell levels, spells can have any integer base aether cost. And some spells will end up moving up or down a category.
    \begin{DndTable}{XX}
        \textbf{Spell level (5e)} & \textbf{Starting Aether Cost} \\ 
        1                & 2                    \\
        2                & 3                    \\
        3                & 5                    \\
        4                & 8                    \\
        5                & 12                   \\
        6+               & legendary            \\        
    \end{DndTable}
\end{DndSidebar}

\begin{DndSidebar}[float=hb]{Starting values for aether/limit}
    Starting points. Note that everyone gets aether. "Martial" is those who don't explicitly have a Spellcasting trait.
    \begin{DndTable}{XXXXXX}
        \textbf{Level} & \textbf{Full} & \textbf{Half} & \textbf{Martial} & \textbf{Limit (F/H/M)}\\  
        1     & 4    & 2  & 1       & 2/1/1         \\
        2     & 8    & 4  & 1       & 3/2/1         \\
        3     & 12   & 6  & 2       & 4/3/1         \\             
        4     & 16   & 8  & 2       & 5/3/1         \\
        5     & 20   & 10 & 3       & 6/4/2         \\
        6     & 24   & 12 & 3       & 7/5/2         \\
        7     & 28   & 14 & 4       & 8/5/2         \\
        8     & 32   & 16 & 4       & 9/6/2         \\
        9     & 36   & 18 & 5       & 10/7/2        \\
        10    & 40   & 20 & 5       & 11/7/3        \\
        11    & 44   & 22 & 6       & 12/8/3        \\
        12    & 48   & 24 & 6       & 13/9/3        \\
        13    & 52   & 26 & 7       & 13/9/3        \\
        14    & 56   & 28 & 7       & 14/10/3        \\
        15    & 60   & 30 & 8       & 14/11/3        \\
        16    & 64   & 32 & 8       & 15/11/3        \\
        17    & 68  & 34  & 9       & 15/12/4       \\
        18    & 72  & 36  & 9       & 16/13/4       \\
        19    & 76  & 38  & 10      & 16/13/4       \\
        20    & 80  & 40  & 10      & 17/14/4       \\
    \end{DndTable}
\end{DndSidebar}

\subsection{Guidance on control effects}
Control effects are those effects (whether from spells or features or monster abilities) that remove or constrain the ability of a creature to take the actions they desire. They break down into two major categories--soft control merely constrains or diminishes the effect of certain actions, while hard control outright denies control/actions.

Soft control is normal and good, as long as there's a diversity of saving throws being called for. This is things like grappling, restraining, slowing (as the spell), poisoning, etc. Still nasty, but people still get to play.

Hard control is dangerous, because it's such a game-changer. Taking a hard enemy completely out of the fight dramatically shifts the balance. And when applied against player characters, it can become un-fun quite quickly, because it's, in essence, telling the player "no, you don't get to play". Traditionally, it's also targeted Wisdom, which is a secondary (at best) stat for martial types.

Different tables have different points at which this is all acceptable, but as a general guideline, try to avoid hard control unless it takes multiple failed saves (or really badly failed saves) to take effect. And hard control effects should be much higher level than soft control effects.

\subparagraph{Escalating Conditions} In some cases, it makes sense to "chain" conditions together--the first save (or a close failure) might incur a lesser condition, while a subsequent failed save (or a really bad failure) might incur a greater condition.

\begin{figure}
	\begin{DndTable}{llllX}
		\textbf{Save} & \textbf{Weak} & \textbf{Regular} & \textbf{Strong} & \textbf{Notes} \\
		WIS & \nameref{condition:shaken} & \nameref{condition:frightened} & \nameref{condition:broken} & Broken should rarely apply to PCs. \\
		STR & \nameref{condition:grappled} & \nameref{condition:restrained} & \nameref{condition:petrified} & Petrified should require multiple failed saves. Prone can work as a weak version where appropriate. \\
		CON & \nameref{condition:staggered} & \nameref{condition:stunned} & \nameref{condition:paralyzed} & Be very careful with paralyzed. \\
		CHA & --- & \nameref{condition:charmed} & Dominated & Dominated isn't a condition, but... \\
		CON/WIS & --- & \nameref{condition:incapacitated} & \nameref{condition:unconscious} & \\
		CON & --- & \nameref{condition:poisoned} & \nameref{condition:exhaustion} &
	\end{DndTable}
\end{figure}

DEX and INT generally don't cause conditions--instead they deal damage (usually allowing a save for half damage). CON, WIS, and DEX are considered "major" saving throws--every class gets proficiency in one of these. Since CON is so rarely disregarded, even if it isn't prioritized, it's the dumping ground for miscellaneous saving throws.

Generally, monsters have the lowest Dexterity and Intelligence saving throw modifiers and the highest Constitution saving throw modifiers. There are many outliers, however.

\import{./}{spell-lists.tex}
\section{Spells, Alphabetical}
\import{./}{spells-a-g.tex}
\import{./}{spells-h-p.tex}
\import{./}{spells-q-z.tex}
\clearpage
\import{./}{legendary-effects.tex}