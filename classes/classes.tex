\chapter{Classes}
\label{ch:classes}

\begin{DndComment}{Commentary on Classes}
Classes are archetypes for playing the game. They represent a tiny slice of the wild and wonderful variation in the world. While playing NIH System, you may encounter creatures, including other "normal" humanoids who have abilities reminicient of class abilities and those with entirely other abilities that no class offers. Even if they are called "warrior" or "rogue" or "oathbound", they may not have all the abilities of a member of that class and may infact have others unattainable in game. Every individual is different, but the classes represent packages of abilities balanced and suited for play as an adventurer. They are not "real" in the context of the fictional world.
\end{DndComment}

\section{Meta Class Design}
This is an interim section. Will it make it into the final? Who knows.

\subsection{SDCT}
Imagine you have 20 points to allocate among four combat categories: Support, Damage, Control, and Toughness. This is arbitrary, but 0 is the lowest and 20 is the highest. This is more for designing \textit{inside} a class, not really comparing a class. And not very formalized. In principle, one class's "1" might be another class's "5" or "10" (although the latter is unlikely). Ideally, they'd all be comparable. Nobody should be 0 and nobody should be 20, because that would mean you can't provide anything on those areas or that you can't provide anything on any \textit{other} area. The game is designed around tight \textit{thematic} specialization but only loose \textit{mechanical} specialization. 

A class with high (S)upport is good at preventing damage to others, increasing others' efforts, healing damage taken, etc. When you want the opportunity to say "together we stand" or "oh no you don't do that to him", pick up a high Support class. A class with high (D)amage is good at putting out damage. When you want to rack up the big numbers and watch enemies drop, pick a high Damage class. C standst for Control. Control is the flip side of Support--you're not making allies better, you're making enemies worse. That might be directly debuffing them or providing zones or even punishing them for trying to go after your allies. Both the "sticky tank" and the "chess master" fit into this category. (T)oughness is basically durability. You can stand in harm's way and laugh. This may come from good armor and defensive abilities or just massive amounts of health and a healthy regeneration.

\import{./}{arcanist.tex}
\clearpage
\import{./}{armsman.tex}
\clearpage
\import{./}{brawler.tex}
\clearpage
\import{./}{oathbound.tex}
\clearpage
\import{./}{priest.tex}
\clearpage
\import{./}{ranger.tex}
\clearpage
\import{./}{rogue.tex}
\clearpage
\import{./}{shaman.tex}
\clearpage
\import{./}{spellblade.tex}
\clearpage
\import{./}{warden.tex}
\clearpage
\import{./}{warlock.tex}
\clearpage
