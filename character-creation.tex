\chapter{Character Creation}
\label{ch:character-creation}
Creating a character follows a specific process.
\begin{enumerate}
    \item Pick a class.
    \item Pick a lineage.
    \item Pick a culture.
    \item Pick or design a background. Including deciding your motivations for adventuring.
    \item Assign ability scores.
    \item Pick proficiencies. If your background proficiencies overlap your class proficiencies, pick a new one.
    \item Pick starting gear from class.
    \item Calculate dependent values. This is HP, AC, attack bonuses, saving throw modifiers, saving throw DCs, Stamina, Aether, etc.
\end{enumerate}

\section{Alignment}
\label{sec:alignment}
Alignment is not in effect, except descriptively. Instead, come up with two adjectives that describe your character's default reaction to things. 


\section{Languages}
\label{sec:languages}
Your culture indicates the languages your character can speak by default, and your background might give you access to one or more additional languages of your choice. Note these languages on your character sheet.

Choose your languages from the Standard Languages table, or choose one that is common in your campaign. With your GM's permission, you can instead choose a language from the Exotic Languages table or a secret language, such as thieves' cant or the tongue of druids.

Some of these languages are actually families of languages with many dialects. For example, the Primordial language includes the Auran, Aquan, Ignan, and Terran dialects, one for each of the four elemental planes. Creatures that speak different dialects of the same language can communicate with one another.

\begin{DndTable}[header=Standard Languages\label{tbl:standard-languages}]{XXXX}
    \textbf{Language} & \textbf{Common Name} & \textbf{Typical Speakers} & \textbf{Script} \\
    Common & -- & Most folks & Reformed Imperial \\
    Tumni & Dwarven & Dwarves & Modern Runic \\
    Yonwach & High Elven & Gwerin & Aelven \\
    Metsae & Wood Elven & Ihmisi & Aelven \\
    Too-til & Giantish & Giants, jazuu & Modern Runic \\
    Ard-teang & Orcish & Orcs & Mixed Imperial/Aelven \\
    Ngyon toi & Goblin & Goblins & None, tr. Imperial           
\end{DndTable}

\begin{DndTable}[header=Exotic Languages\label{tbl:exotic-languages}]{XXXX}
    \textbf{Language} & \textbf{Common Name} & \textbf{Typical Speakers} & \textbf{Script} \\
    Abyssal & Demonic & Demons, cultists & Lucian \\
    Celestial & -- & Lucians acting on divine business & Lucian \\
    Draconic & -- & Dragons, dragonborn & tr. mixed Aelven/Runic \\
    Iath Neidr & Snakefolk & Ophidians & Modified Aelven \\
    Jinzi & Eastern Imperial & Eastern Noefrans & Jinzi \\
    Kamigami & Druidic, Sylvan & Druids, fey & tr. Aelven \\
    Lucian & Infernal & Astral residents & Lucian \\
    Primordial & Elemental & Elementals & Archaic Runic \\
    Tiborian & Old Imperial & Scholars & Imperial \\            
\end{DndTable}

\import{./}{backgrounds.tex}

\section{Advancement}\label{sec:advancement}
\subsection{Beyond 1st Level}\label{subsec:beyond-first-level}

As your character goes on adventures and overcomes challenges, he or she gains experience, represented by experience points. A character who reaches a specified experience point total advances in capability. This advancement is called **gaining a level**.

When your character gains a level, his or her class often grants additional features, as detailed in the class description. Some of these features allow you to increase your ability scores, either increasing two scores by 1 each or increasing one score by 2. You can't increase an ability score above 20. In addition, every character's proficiency bonus increases at certain levels.

Each time you gain a level, you gain 1 additional Hit Die. Roll that Hit Die, add your Constitution modifier to the roll, and add the total to your hit point maximum. Alternatively, you can use the fixed value shown in your class entry, which is the average result of the die roll (rounded up).

When your Constitution modifier increases by 1, your hit point maximum increases by 1 for each level you have attained. For example, if your 7th-level fighter has a Constitution score of 18, when he reaches 8th level, he increases his Constitution score from 17 to 18, thus increasing his Constitution modifier from +3 to +4. His hit point maximum then increases by 8.

The Character Advancement table summarizes the XP you need to advance in levels from level 1 through level 20, and the proficiency bonus for a character of that level. Consult the information in your character's class description to see what other improvements you gain at each level.

\begin{DndComment}
    The expectation is that 1 XP ~ 1 session of serious play, regardless of what's accomplished, fought, or done during the session. GMs can provide extra XP for good play or not award XP if the party just sits around and dithers, but should do so sparingly. Two paths are presented--slow and fast. Slow expects it to take a number of sessions (XP) equal to your current level until level 6, at which point it stabilizes as 6/level. Fast caps at 4. A Fast-path advancement will take a party 1-20 in just over a year of weekly play (71 sessions, about 14 months), while a slow advancement path will take just shy of two years (100 sessions). The goal here is to keep Tier 1 quite fast, and then slow down to a fixed pace.
\end{DndComment}

\begin{DndTable}[header=Experience and Leveling\label{tbl:xp-level}]{XXXX}
    \textbf{Experience Points (Fast)} & \textbf{Experience Points (Slow)} & \textbf{Level} & \textbf{Proficiency Bonus} \\
    0 & 0 & 1 & +2 \\
    1 & 1 & 2 & +2 \\
    3 & 3 & 3 & +2 \\
    7 & 7 & 4 & +2 \\
    11 & 11 & 5 & +3 \\
    15 & 16 & 6 & +3 \\
    19 & 22 & 7 & +3 \\
    23 & 28 & 8 & +3 \\
    27 & 34 & 9 & +4 \\
    31 & 40 & 10 & +4 \\
    35 & 46 & 11 & +4 \\
    39 & 52 & 12 & +4 \\
    43 & 58 & 13 & +5 \\
    47 & 64 & 14 & +5 \\
    51 & 70 & 15 & +5 \\
    55 & 76 & 16 & +5 \\
    59 & 82 & 17 & +6 \\
    63 & 88 & 18 & +6 \\
    67 & 94 & 19 & +6 \\
    71 & 100 & 20 & +6 \\
\end{DndTable}