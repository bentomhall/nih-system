\chapter{Introduction}
\section{Core Principles of NIH}

\begin{DndSidebar}[float=b]{What is Magic?}
    That word is used throughout these documents, and deserves a little more reflection. "Magic", as used here, is all those things that separate the fictional world in which the game takes place from the real world in which the players act. Spells? Magic. Dragons? Magic. Heroic mortals breaking "normal" constraints? Magic. \textbf{It's magic all the way down.} Unlike Dungeons and Dragons, I attempt to be more precise in my use of this term. So abilities that counter \textit{specific forms} of magic will be written precisely. Thus, \textit{dispel magic} (the Dungeons and Dragons spell) becomes \nameref{spell:unbind}, an spell that specifically focuses on undoing active aetheric effects (a subset of all magic including spell effects, magic items, and sustained wards of various types).
\end{DndSidebar}

\subsection{Core Assumptions}
The core assumptions are that:
\begin{enumerate}
\item \textbf{The world is thoroughly magical.} Magic is in and through everything and everyone. There is no mundane (in the sense of "bound by all the principles of real world"), at least as far as adventurers and adventuring goes.
\item \textbf{Adventuring is normalized.} Adventurers are a known social "group", even if a disreputable one.
\item \textbf{The world responds to mortal efforts.} In these sorts of worlds, training really hard physically \textit{is} a form of access to magical power just as much as learning magic gestures and words in a book. But both of these plateau quite hard for most people.
\item \textbf{The world has its own logic, not real-world science.} Players should be able to rely on \textit{surface} similarity between the real world and the fictional one. Things will fall when dropped. Water flows downhill. Paper burns and fire hurts. Fire turns water to steam. If it looks like a cow, walks like a cow, and moos like a cow, it's probably a cow. But the \textit{reasons} why these happen are not at all guaranteed to be the same. The further and deeper you get into modern scientific understanding of the world, the less this will apply to the fictional world. A fire spell will burn flammable things...but not because of exothermic oxidation reactions. There might not even be oxygen, and fires might burn just fine even in a "vacuum"...if such a thing even exists. Atoms, molecules, cells, DNA--these sorts of things are not at all guaranteed to exist.
\item \textbf{PCs are among the few that break the normal (soft) limits.} Most people in the world are relatively low power, and will never grow much further. PCs are not bounded in the same way.
\item \textbf{World-ending threats are rare, but problems are common.} Most threats a party will face don't have whole-world-changing consequences. But they do change the local world.
\item \textbf{PCs change the world...but not by pushing buttons.} It's expected that the outcome of the PCs adventure will be changes to the status quo. The world \textit{should} react to their actions. But the PCs don't have powers that allow them to directly do that on the large scale. Large-scale changes happen because of the relationships the PCs form with others, the movements they support, and the people they affect.
\item \textbf{PCs are usually the underdogs.} Either due to numbers (the enemy has an army) or due to individual power. PCs generally win against significant foes not by overwhelming the opponents with bigger numbers but by teamwork, guile, good strategy, finding the opponents' weak spots, building alliances with others, etc.
\item \textbf{Accuracy is bounded.} What does this mean? It means that attack bonuses, armor class, and saving throw and ability check DCs and modifiers are not assumed to grow significantly (relative to the d20's effect) with level. They may change and grow, but it should be hard if not impossible to "move off the d20" permanently in most cases. Monsters that hit PCs at level 1 should still be able to hit some appreciable fraction of the time against level 20 PCs, even if that fraction is smaller. Unsaveable saving throws shouldn't usually happen unless the PC or monster has a strongly negative modifier. Neither should "unmissable" saving throws unless class features or magic is involved, and then rarely. PCs and monsters scale mostly in three ways--(a) having more health to absorb hits and stay standing, (b) dealing more damage (usually via more attacks rather than bigger single attacks, but this varies), and (c) having abilities that give different ways to approach the problem entirely (horizontal growth). As a result of this, magic items no longer give +1 (etc) to AC, saving throw DCs, or attack rolls. Such passive stacking bonuses are rare if they exist at all.   
\end{enumerate}

\subsection{How to Play}
NIH requires one player to assume the role of Game Master (hereafter GM), while the remaining players (usually 2-4 in number) create and control individual characters, called Player Characters (PCs) or "the party". It's assumed that the PCs will work together cooperatively--the basic unit of the game is the party, not the individual. The GM's role is several-fold:
\begin{enumerate}
    \item He voices and decides the actions for all the non-Player Character characters (NPCs, for short). If the party is fighting someone, the GM makes decisions for that antagonist. This doesn't mean the GM is antagonistic toward the \textit{players} or is trying to kill the PCs, but the characters he or she controls most certainly may be antagonists.
    \item The GM is the voice of the setting and the narrative. Nothing happens in-game until he narrates it, and he is the eyes through which the players experience the world and the story they are collectively creating. Which makes it imperative that the GM is not biased toward or against any of the players and should, when speaking as the "voice of the world", never lie to the players. NPCs may lie and try to deceive, the GM as the GM should not. Of course, when illusions and compulsions are in play, what a PC sees or experiences may not be real. But what is described should be what the PC experiences.
    \item The GM is the rules engine for the game. These rules are inputs and guidelines, not mandates. It's the GM's role to decide how, when, and even if the rules apply and to adjust on the fly. Many GMs rely on group consensus for rule modifications and rulings, but at the end of the day, the GM is the final decider. If a player says that his PC acts a certain way, the GM is responsible for deciding how to execute that action and what success or failure looks like, as well as narrating the result. 
    \item Often, the GM is responsible for coming up with the world and/or the set of events surrounding the PCs. Unlike a video game, there may not be a "main quest" that the PCs must follow, but the GM is the one placing things in the world for them to find and interact with.
\end{enumerate}

\subsubsection{The Basic Game Loop}
The most basic, most generic pattern of play is as follows:
\begin{itemize}
    \item The GM describes a scenario, including what the PCs see, hear, smell, etc.
    \item He or she then asks either a single player or the group "What do you do"? (or something similar).
    \item That player or group of players then describes what action or actions they want their PC to attempt. This description may be vague ("I attack the giant with my sword!") or specific ("Gerrold lunges forward, slashing upward at the giant's leg with his sword"), but must always provide
    \begin{enumerate}
        \item An indication of what the player wants to have happen (a goal)
        \item An indication of how the player character is achieving that goal (a method)
        \item And any pertinent facts, such as abilities being used, amounts of bribes being offered, etc. that may change the resolution of the action.
    \end{enumerate}
    \item The GM, often after discussing details with the player, decides how the action will be resolved. This often involves some sort of die roll for randomization purposes. These rules are full of resolution mechanics such as attack rolls, ability checks, saving throws, etc. Many actions don't need any explicit resolution method--they just succeed. It is not expected that you have to roll to tie your shoes in the morning. Actions that have little opposition (chance of failure), are a core fictional competency of the character in question (the sailor can climb masts in calm weather), or where failure doesn't have meaningful consequences that change the situation (picking a practice lock in the safety of your home) rarely, if ever, call for active resolution.
    \item Once the action is resolved, the GM narrates the changes to the situation and the loop continues.
\end{itemize}

Many times, multiple actions can be resolved simultaneously and the GM may ask multiple players for their actions and decide how to order their resolutions. The first person to speak doesn't necessarily go first--that depends on the entire situation.

At times when exact sequencing is important (such as combat), the GM may call for Initiative. When in initiative order, players take actions from highest initiative downward. In the game world, they're all acting within the same short period of time--one complete pass through the initiative order, called a \textbf{round} represents about 6 seconds of game time, but they are sequenced for ease of play.

\subsection{The World: Quartus}
Throughout these rules, you will see references to Quartus and Noefra. Quartus is the main inhabited planet of the Dreams of Hope setting, and Noefra is the north-eastern continent, which serves as the default setting for this system. Other worlds can be used, and Quartus itself has other places to set stories. Dreams of Hope is a long-running, "living world" setting, where each adventuring party makes changes based on their actions, where PCs retire to become NPCs at the end of their adventures and future parties can interact with them. If you wish to ignore all of that, feel free. The world is yours.

A full description of the world will not fit in these margins, but can be found at the \href{https://wiki.admiralbenbo.org}[Dreams of Hope Wiki page]. Everything there is licensed CC-BY 4.0 unless specified otherwise. The current year is 252 AC (After Cataclysm); changes after about 250 AC are marked as such.

\subsubsection{Cosmology}
Dreams of Hope is divided into several planes of existence, all constrained to fit within a spherical shell about the same size as the Inner Solar System ($\approx$2 Astronomical Units in radius). There are four major planes and one aberrant plane, although the Elemental is further sub-divided, as is Shadow:
\begin{itemize}
	\item The Mortal plane is the foundation on which all the other planes rest. It is the source of all aether, and the home of most mortal beings who produce said aether.
	\item The Astral plane is the "heavens", the home of most of the gods, ascendants, and angels (as well as devils!), but it is \textit{not} the place of the afterlife. A luminous plane of drifting, super-earth-sized inhabited plates, its nature is hard to comprehend for most mortals.
	\item The Elemental plane is composed of 12 sub-planes forming a radial "pie" shape, 3 for each of the base elements. They are fixed in space, and as the planets of the Mortal orbit through their influence, they cause the seasons. In order from the beginning of spring, those planes are
	\begin{itemize}
		\item Clay, being Earth + Water with Earth dominant.
		\item Stone, being pure Earth.
		\item Coal, being Earth + Fire with Earth dominant.
		\item Lava, being Fire + Earth with Fire dominant.
		\item Flames, being pure Fire.
		\item Lightning, being Fire + Air, with Fire dominant.
		\item Ash or Smoke, being Air + Fire, with Air dominant.
		\item Wind, being pure Air.
		\item Cloud, being Air + Water, with Air dominant.
		\item Ice, being Water + Air with Water dominant.
		\item Ocean, being pure Water.
		\item Mud, being Water + Earth with Water dominant.
	\end{itemize}
	\item Shadow, being the liminal plane that acts as the interface between the Mortal and all the other planes. It serves as the afterlife as well as the home of many of the fey and a hunting ground for demons. Its geography reflects the geography and especially areas of magical significance in the Mortal, although distorted as if through a funhouse mirror. It is composed of four sub-planes:
	\begin{itemize}
		\item Border Shadow, being an empty region where one can transition between planes easily. Other names for this include the Ethereal. Teleportation as well as many spells touch this plane. Movement is by thought, and the space itself reacts to your stray thoughts.
		\item Beastholm, being the place of quiet somnilence and rest. As well as depression and ennui. Home to fantastic animals and plants, it is somber but quite dangerous.
		\item Mirrorhaven, being the place of excitement, energy, illusion, and mania. Full of color and light, akin to a hallucinatory trip. Underneath its cheerful, almost cartoonish surface lurks many dangers.
		\item The Waste, being the result of abyssal corruption, cuts across the layers in some areas. A wasteland of cracked earth and demonic plants and animals, here beings of the Abyss hunt for souls to devour and brave souls (as well as devils from the Astral) hunt the demons in turn.
	\end{itemize}
	\item The Abyss is an aberration. A cyst, a wound in reality, it is not the same size as the rest of the universe. Instead, it is approximately the volume of Quartus and orbits in a complex cloverleaf, causing its influence to wax and wane unpredictably except to the savants. At its heart is the Oblivion Gate, the ever-hungry living black hole that sends fragments of itself, called \textbf{jotnar} to devour everything. These jotnar, when they infect a soul, convert it into a demon; thus the residents of the Abyss (by choice or otherwise) are demons. Others may journey there, but too-long residence is hazardous. Demons can only exist natively in the Wastes and in the Abyss, but can be summoned elsewhere...with the result that the area they are summoned in is contaminated by jotnar energies. Undead are also the result of jotnar infestation, this time of the dead bodies of mortals and the dying remains of mortals.
\end{itemize}

\subparagraph*{The Mortal Plane}
The primary plane upon which all others rest is the Mortal plane, called so because it is home to most of the mortal souls. This is the normal plane of matter, energy, humans, planets, etc. It consists of the central star, Eua, and four planets named in order from the star outward:
\begin{enumerate}
	\item Eua, the star. Unlike Sol, this is \textit{not} a burning ball of gas undergoing fusion reactions. It's instead a great glowing crystal, radiating luminous- and fire-aspected aether throughout the plane. Visually, it is very similar to Sol, at least as seen from Quartus. Slightly smaller, but still yellow. It does not give off nearly as much heat--the thermal input for the planets comes from the influence of the elemental planes, not the star itself, which provides mostly light.
	\item Primus, the rapidly-shifting, elementally-dominated world bereft of much normal life, located about 0.25 AU from the star (less than the radius of Mercury in our solar system).
	\item Secundus and Tertius, the twin oppositional planets, both slightly smaller than Earth; Secundus is dominated by Earth and Air and consists of vast deserts of barren sand with floating islands of rock. Tertius is dominated by Water + Fire and is dominated by a world-girdling jungle of strange plants and animals. The two are very close and orbit their common barycenter--a sufficiently strong flier can pass from one to the other in a matter of hours. They orbit Eua at about 0.5 AU.
	\item Quartus, with its two (originally 3) moons. An Earthlike planet, it has zero orbital inclination and a perfectly circular orbit. Seasonal variation comes from elemental influence. It is the home of most of the intelligent mortal life in the setting, and is the primary setting for the game. For practical purposes, it can be thought of as Earthlike, except with two moons. It orbits at almost exactly 1 AU, with a year of 384 days of 24 hours (in the conventional reckoning) each. The moons are Quella, a large red moon with a period of 32 days, and Tekki, a smaller (but closer, so their visual size is similar) bone-white moon with a period of 8 days.
	\item The Crystal Sphere forms the boundary of the universe. A sphere of some unimaginably tough, transparent crystal located at 2 AU from Eua, it encompasses all of creation. Outside there is only the Dark Beyond and the memetic thought-forms that inhabit it. Those forms often leak through the boundary (rather the least of them do), where the angels fight to destroy them lest they infect and destroy all life.
\end{enumerate}

The stars are \textif{not} fixed--they are beacons used by angelic legions in their never-ending fight against the entities of the Dark Beyond.

Quartus has five continents, but the game defaults to the north-eastern continent of Noefra and more particularly the western half of that continent. Noefra is the most mixed as to the lineages of mortals, having representatives of most, if not all the lineages.

One key event in recent history is the Cataclysm, a time about 250 years ago when misuse of an artifact, combined with the invasion of a primordial entity of chaos, caused the elemental planes to shift, all the gods to die or become depowered, spells to stop working, and massive natural disasters to sweep the lands, killing roughly 70\% of the population of Noefra (and slightly less on other continents). All Noefran calendars use this date as the starting point for their enumeration of years.

