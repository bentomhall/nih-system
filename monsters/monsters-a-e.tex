\section{Monsters (A)}




\subsection{Ankheg}
\begin{DndMonster}[width=\textwidth + 8pt]{Ankheg}
\DndMonsterType{Large monstrosity}
\DndMonsterBasics[armor-class={14 (natural armor), 11 while prone}, hit-points={39 (6d10 + 6)}, speed={30 ft., burrow 10 ft.}]
\MonsterStats{+3}{+0}{+1}{-5}{+1}{-2}
\DndMonsterDetails[saving-throws={}, skills={}, damage-immunities={}, damage-resistances={}, damage-vulnerabilities={}, condition-immunities={}, senses={darkvision 60 ft., tremorsense 60 ft., passive Perception 11}, languages={—}, challenge={2}]
\DndMonsterSection{Actions}
\DndMonsterMelee[
    name=Bite,
    mod=+5,
    dmg=\DndDice{2d6+3},
    dmg-type=slashing,
    plus-dmg=\DndDice{1d6},
    plus-dmg-type=acid,
    extra={If the target is a Large or smaller creature, it is grappled (escape DC 13). Until this grapple ends, the ankheg can bite only the grappled creature and has advantage on attack rolls to do so.}
]
\DndMonsterAction{Spit} The ankheg spits acid in a line that is 30 feet long and 5 feet wide, provided that it has no creature grappled. Each creature in that line must make a DC 13 Dexterity saving throw, taking 10 (3d6) acid damage on a failed save, or half as much damage on a successful one.
\end{DndMonster}

\subsection{Azer}

\FloatBarrier

\section{Monsters (B)}\label{sec:monsters-b}
\subsection{Basilisk}
\begin{DndMonster}[width=\textwidth + 8pt]{Basilisk}
\begin{multicols}{2}
\DndMonsterType{Medium monstrosity}
\DndMonsterBasics[armor-class={15 (natural armor)}, hit-points={52 (8d8 + 16)}, speed={20 ft.}]
\MonsterStats{+3}{-1}{+2}{-4}{-1}{-2}
\DndMonsterDetails[saving-throws={}, skills={}, damage-immunities={}, damage-resistances={}, damage-vulnerabilities={}, condition-immunities={}, senses={darkvision 60 ft., passive Perception 9}, languages={—}, challenge={3}]
\DndMonsterAction{Petrifying Gaze} If a creature starts its turn within 30 feet of the basilisk and the two of them can see each other, the basilisk can force the creature to make a DC 12 Constitution saving throw if the basilisk isn't incapacitated. On a failed save, the creature magically begins to turn to stone and is restrained. It must repeat the saving throw at the end of its next turn. On a success, the effect ends. On a failure, the creature is petrified until freed by the \nameref{inc:restoration} incantation (greater only) or other magic.

A creature that isn't surprised can avert its eyes to avoid the saving throw at the start of its turn. If it does so, it can't see the basilisk until the start of its next turn, when it can avert its eyes again. If it looks at the basilisk in the meantime, it must immediately make the save.

If the basilisk sees its reflection within 30 feet of it in bright light, it mistakes itself for a rival and targets itself with its gaze.

\DndMonsterSection{Actions}
\DndMonsterMelee[
    name=Bite,
    mod=+5,
    dmg=\DndDice{2d6 + 3},
    dmg-type=piercing,
    plus-dmg=\DndDice{2d6},
    plus-dmg-type=poison
]
\end{multicols}
\end{DndMonster}

\begin{DndMonster}[width=\textwidth + 8pt]{Behir}
\begin{multicols}{2}
\DndMonsterType{Huge monstrosity}
\DndMonsterBasics[armor-class={17 (natural armor)}, hit-points={168 (16d12 + 64)}, speed={50 ft., climb 40 ft.}]
\MonsterStats{+6}{+3}{+4}{-2}{+2}{+1}
\DndMonsterDetails[saving-throws={}, skills={Perception +6, Stealth +7}, damage-immunities={lightning}, damage-resistances={}, damage-vulnerabilities={}, condition-immunities={}, senses={darkvision 90 ft., passive Perception 16}, languages={Draconic}, challenge={11}]
\DndMonsterSection{Actions}
\DndMonsterAction{Multiattack} The behir makes two attacks: one with its bite and one to constrict.
\DndMonsterAttack[
	name=Bite,
	distance=melee,
	type=weapon,
	mod=+10,
	reach=10,
	dmg=\DndDice{3d10 + 6},
	dmg-type=piercing
]
\DndMonsterAttack[
	name=Constrict,
	distance=melee,
	type=weapon,
	mod=+10,
	reach=5,
	dmg=\DndDice{2d10 + 6},
	dmg-type=bludgeoning,
	extra={ plus 17 (2d10 + 6) slashing damage. The target is grappled (escape DC 16) if the behir isn't already constricting a creature, and the target is restrained until this grapple ends.}
]

\DndMonsterAction{Lightning Breath (Recharge 5-6)} 
The behir exhales a line of lightning that is 20 feet long and 5 feet wide. Each creature in that line must make a DC 16 Dexterity saving throw, taking 66 (12d10) lightning damage on a failed save, or half as much damage on a successful one.

\DndMonsterAction{Swallow}
The behir makes one bite attack against a Medium or smaller target it is grappling. If the attack hits, the target is also swallowed, and the grapple ends. While swallowed, the target is blinded and restrained, it has total cover against attacks and other effects outside the behir, and it takes 21 (6d6) acid damage at the start of each of the behir's turns. A behir can have only one creature swallowed at a time.
If the behir takes 30 damage or more on a single turn from the swallowed creature, the behir must succeed on a DC 14 Constitution saving throw at the end of that turn or regurgitate the creature, which falls prone in a space within 10 feet of the behir. If the behir dies, a swallowed creature is no longer restrained by it and can escape from the corpse by using 15 feet of movement, exiting prone.
\end{multicols}
\end{DndMonster}

% \subsection{Bugbear}
% \begin{DndMonster}[width=\textwidth + 8pt]{Bugbear}
% \begin{multicols}{2}
% \DndMonsterType{Medium humanoid (goblinoid)}
% \DndMonsterBasics[armor-class={16 (hide armor, shield)}, hit-points={27 (5d8 + 5)}, speed={30 ft.}]
% \MonsterStats{+2}{+2}{+1}{-1}{+0}{-1}
% \DndMonsterDetails[saving-throws={}, skills={Stealth +6, Survival +2}, damage-immunities={}, damage-resistances={}, damage-vulnerabilities={}, condition-immunities={}, senses={darkvision 60 ft., passive Perception 10}, languages={Common, Goblin}, challenge={1 (200 XP)}]
% \DndMonsterAction{Brute} A melee weapon deals one extra die of its damage when the bugbear hits with it (included in the attack).

% \DndMonsterAction{Surprise Attack} If the bugbear surprises a creature and hits it with an attack during the first round of combat, the target takes an extra 7 (2d6) damage from the attack.

% \DndMonsterSection{Actions}
% \DndMonsterAttack[
% 	name=Morningstar,
% 	distance=melee,
% 	type=weapon,
% 	mod=+4,
% 	reach=5,
% 	dmg=\DndDice{2d8 + 2},
% 	dmg-type=piercing
% ]
% \DndMonsterAttack[
% 	name=Javelin,
% 	distance=both,
% 	type=weapon,
% 	mod=+4,
% 	reach=5,
% 	dmg=\DndDice{2d6 + 2},
% 	dmg-type=piercing,
% 	extra={ in melee or 5 (1d6 + 2) piercing damage at range.}
% ]
% \end{multicols}
% \end{DndMonster}

\begin{DndMonster}[width=\textwidth + 8pt]{Bulette}
\begin{multicols}{2}
\DndMonsterType{Large monstrosity}
\DndMonsterBasics[armor-class={17 (natural armor)}, hit-points={94 (9d10 + 45)}, speed={40 ft., burrow 40 ft.}]
\MonsterStats{+4}{+0}{+5}{-4}{+0}{-3}
\DndMonsterDetails[saving-throws={}, skills={Perception +6}, damage-immunities={}, damage-resistances={}, damage-vulnerabilities={}, condition-immunities={}, senses={darkvision 60 ft., tremorsense 60 ft., passive Perception 16}, languages={—}, challenge={5}]
\DndMonsterAction{Standing Leap} The bulette's long jump is up to 30 feet and its high jump is up to 15 feet, with or without a running start.

\DndMonsterSection{Actions}
\DndMonsterAttack[
	name=Bite,
	distance=melee,
	type=weapon,
	mod=+7,
	reach=5,
	dmg=\DndDice{4d12 + 4},
	dmg-type=piercing
]
\DndMonsterAction{Deadly Leap}
If the bulette jumps at least 15 feet as part of its movement, it can then use this action to land on its feet in a space that contains one or more other creatures. Each of those creatures must succeed on a DC 16 Strength or Dexterity saving throw (target's choice) or be knocked prone and take 14 (3d6 + 4) bludgeoning damage plus 14 (3d6 + 4) slashing damage. On a successful save, the creature takes only half the damage, isn't knocked prone, and is pushed 5 feet out of the bulette's space into an unoccupied space of the creature's choice. If no unoccupied space is within range, the creature instead falls prone in the bulette's space.
\end{multicols}
\end{DndMonster}

\FloatBarrier
\section{Monsters (C)}\label{sec:monsters-c}

% \subsection{Centaur}
% \begin{DndMonster}[width=\textwidth + 8pt]{Centaur}
% \begin{multicols}{2}
% \DndMonsterType{Large monstrosity}
% \DndMonsterBasics[armor-class={12}, hit-points={45 (6d10 + 12)}, speed={50 ft.}]
% \MonsterStats{+4}{+2}{+2}{-1}{+1}{+0}
% \DndMonsterDetails[saving-throws={}, skills={Athletics +6, Perception +3, Survival +3}, damage-immunities={}, damage-resistances={}, damage-vulnerabilities={}, condition-immunities={}, senses={passive Perception 13}, languages={Elvish, Sylvan}, challenge={2 (450 XP)}]
% \DndMonsterAction{Charge} If the centaur moves at least 30 feet straight toward a target and then hits it with a pike attack on the same turn, the target takes an extra 10 (3d6) piercing damage.

% \DndMonsterSection{Actions}
% \DndMonsterAction{Multiattack} The centaur makes two attacks: one with its pike and one with its hooves or two with its longbow.
% \DndMonsterAttack[
% 	name=Pike,
% 	distance=melee,
% 	type=weapon,
% 	mod=+6,
% 	reach=10,
% 	dmg=\DndDice{1d10 + 4},
% 	dmg-type=piercing
% ]
% \DndMonsterAttack[
% 	name=Hooves,
% 	distance=melee,
% 	type=weapon,
% 	mod=+6,
% 	reach=5,
% 	dmg=\DndDice{2d6 + 4},
% 	dmg-type=bludgeoning
% ]
% \DndMonsterAttack[
% 	name=Longbow,
% 	distance=ranged,
% 	type=weapon,
% 	mod=+4,
% 	range=150/600,
% 	dmg=\DndDice{1d8 + 2},
% 	dmg-type=piercing
% ]
% \end{multicols}
% \end{DndMonster}

\subsection{Chimera}
\begin{DndMonster}[width=\textwidth + 8pt]{Chimera}
\begin{multicols}{2}
\DndMonsterType{Large monstrosity}
\DndMonsterBasics[armor-class={14 (natural armor)}, hit-points={114 (12d10 + 48)}, speed={30 ft., fly 60 ft.}]
\MonsterStats{+4}{+0}{+4}{-4}{+2}{+0}
\DndMonsterDetails[saving-throws={}, skills={Perception +8}, damage-immunities={}, damage-resistances={}, damage-vulnerabilities={}, condition-immunities={}, senses={darkvision 60 ft., passive Perception 18}, languages={understands Draconic but can't speak}, challenge={6 (2,300 XP)}]
\DndMonsterSection{Actions}
\DndMonsterAction{Multiattack} The chimera makes three attacks: one with its bite, one with its horns, and one with its claws. When its fire breath is available, it can use the breath in place of its bite or horns.
\DndMonsterAttack[
	name=Bite,
	distance=melee,
	type=weapon,
	mod=+7,
	reach=5,
	dmg=\DndDice{2d6 + 4},
	dmg-type=piercing
]
\DndMonsterAttack[
	name=Horns,
	distance=melee,
	type=weapon,
	mod=+7,
	reach=5,
	dmg=\DndDice{1d12 + 4},
	dmg-type=bludgeoning
]
\DndMonsterAttack[
	name=Claws,
	distance=melee,
	type=weapon,
	mod=+7,
	reach=5,
	dmg=\DndDice{2d6 + 4},
	dmg-type=slashing
]
\DndMonsterAction{Fire Breath (Recharge 5-6)}
The dragon head exhales fire in a 15-foot cone. Each creature in that area must make a DC 15 Dexterity saving throw, taking 31 (7d8) fire damage on a failed save, or half as much damage on a successful one.
\end{multicols}
\end{DndMonster}

\begin{DndMonster}[width=\textwidth + 8pt]{Chuul}
\begin{multicols}{2}
\DndMonsterType{Large monstrosity}
\DndMonsterBasics[armor-class={16 (natural armor)}, hit-points={93 (11d10 + 33)}, speed={30 ft., swim 30 ft.}]
\MonsterStats{+4}{+0}{+3}{-3}{+0}{-3}
\DndMonsterDetails[saving-throws={}, skills={Perception +4}, damage-immunities={poison}, damage-resistances={}, damage-vulnerabilities={}, condition-immunities={poisoned}, senses={darkvision 60 ft., passive Perception 14}, languages={understands Deep Speech but can't speak}, challenge={4}]
\DndMonsterAction{Amphibious} The chuul can breathe air and water.

\DndMonsterAction{Sense Magic} The chuul senses magic within 120 feet of it at will. This trait otherwise works like the \nameref{inc:sense-aura} spell but isn't itself magical.

\DndMonsterSection{Actions}
\DndMonsterAction{Multiattack} The chuul makes two pincer attacks. If the chuul is grappling a creature, the chuul can also use its tentacles once.
\DndMonsterAttack[
	name=Pincer,
	distance=melee,
	type=weapon,
	mod=+6,
	reach=10,
	dmg=\DndDice{2d6 + 4},
	dmg-type=bludgeoning,
	extra={. The target is grappled (escape DC 14) if it is a Large or smaller creature and the chuul doesn't have two other creatures grappled.}
]
\DndMonsterAction{Tentacles}
One creature grappled by the chuul must succeed on a DC 13 Constitution saving throw or be poisoned for 1 minute. Until this poison ends, the target is paralyzed. The target can repeat the saving throw at the end of each of its turns, ending the effect on itself on a success.
\end{multicols}
\end{DndMonster}

\subsection{Cloaker}
\begin{DndMonster}[width=\textwidth + 8pt]{Cloaker}
\begin{multicols}{2}
\DndMonsterType{Large monstrosity}
\DndMonsterBasics[armor-class={14 (natural armor)}, hit-points={78 (12d10 + 12)}, speed={10 ft., fly 40 ft.}]
\MonsterStats{+3}{+2}{+1}{+1}{+1}{+2}
\DndMonsterDetails[saving-throws={}, skills={Stealth +5}, damage-immunities={}, damage-resistances={}, damage-vulnerabilities={}, condition-immunities={}, senses={darkvision 60 ft., passive Perception 11}, languages={Deep Speech, Undercommon}, challenge={8}]
\DndMonsterAction{Damage Transfer} While attached to a creature, the cloaker takes only half the damage dealt to it (rounded down), and that creature takes the other half.

\DndMonsterAction{False Appearance} While the cloaker remains motionless without its underside exposed, it is indistinguishable from a dark leather cloak.

\DndMonsterAction{Light Sensitivity} While in bright light, the cloaker has disadvantage on attack rolls and Wisdom (Perception) checks that rely on sight.

\DndMonsterSection{Actions}
\DndMonsterAction{Multiattack} The cloaker makes two attacks: one with its bite and one with its tail.
\DndMonsterAttack[
	name=Bite,
	distance=melee,
	type=weapon,
	mod=+6,
	reach=5,
	dmg=\DndDice{2d6 + 3},
	dmg-type=piercing,
	extra={, and if the target is Large or smaller, the cloaker attaches to it. If the cloaker has advantage against the target, the cloaker attaches to the target's head, and the target is blinded and unable to breathe while the cloaker is attached. While attached, the cloaker can make this attack only against the target and has advantage on the attack roll. The cloaker can detach itself by spending 5 feet of its movement. A creature, including the target, can take its action to detach the cloaker by succeeding on a DC 16 Strength check.}
]
\DndMonsterAttack[
	name=Tail,
	distance=melee,
	type=weapon,
	mod=+6,
	reach=10,
	dmg=\DndDice{1d8 + 3},
	dmg-type=slashing
]
\DndMonsterAction{Moan}
Each creature within 60 feet of the cloaker that can hear its moan and that isn't an aberration must succeed on a DC 13 Wisdom saving throw or become frightened until the end of the cloaker's next turn. If a creature's saving throw is successful, the creature is immune to the cloaker's moan for the next 24 hours

\DndMonsterAction{Phantasms (Recharges after a Short or Long Rest)}
The cloaker magically creates three illusory duplicates of itself if it isn't in bright light. The duplicates move with it and mimic its actions, shifting position so as to make it impossible to track which cloaker is the real one. If the cloaker is ever in an area of bright light, the duplicates disappear.
Whenever any creature targets the cloaker with an attack or a harmful spell while a duplicate remains, that creature rolls randomly to determine whether it targets the cloaker or one of the duplicates. A creature is unaffected by this magical effect if it can't see or if it relies on senses other than sight.
A duplicate has the cloaker's AC and uses its saving throws. If an attack hits a duplicate, or if a duplicate fails a saving throw against an effect that deals damage, the duplicate disappears.
\end{multicols}
\end{DndMonster}

\subsection{Cockatrice}
\begin{DndMonster}[width=\textwidth + 8pt]{Cockatrice}
\DndMonsterType{Small monstrosity}
\DndMonsterBasics[armor-class={11}, hit-points={27 (6d6 + 6)}, speed={20 ft., fly 40 ft.}]
\MonsterStats{-2}{+1}{+1}{-4}{+1}{-3}
\DndMonsterDetails[saving-throws={}, skills={}, damage-immunities={}, damage-resistances={}, damage-vulnerabilities={}, condition-immunities={}, senses={darkvision 60 ft., passive Perception 11}, languages={—}, challenge={1/2}]
\DndMonsterSection{Actions}
\DndMonsterAttack[
	name=Bite,
	distance=melee,
	type=weapon,
	mod=+3,
	reach=5,
	dmg=\DndDice{1d4 + 1},
	dmg-type=piercing,
	extra={, and the target must succeed on a DC 11 Constitution saving throw against being magically petrified. On a failed save, the creature begins to turn to stone and is restrained. It must repeat the saving throw at the end of its next turn. On a success, the effect ends. On a failure, the creature is petrified for 24 hours.}
]
\end{DndMonster}

\FloatBarrier
\section{Monsters (D)}\label{sec:monsters-d}

\begin{DndMonster}[width=\textwidth + 8pt]{Darkmantle}
\begin{multicols}{2}
\DndMonsterType{Small monstrosity}
\DndMonsterBasics[armor-class={11}, hit-points={22 (5d6 + 5)}, speed={10 ft., fly 30 ft.}]
\MonsterStats{+3}{+1}{+1}{-4}{+0}{-3}
\DndMonsterDetails[saving-throws={}, skills={Stealth +3}, damage-immunities={}, damage-resistances={}, damage-vulnerabilities={}, condition-immunities={}, senses={blindsight 60 ft., passive Perception 10}, languages={—}, challenge={1/2 (100 XP)}]
\DndMonsterAction{Echolocation} The darkmantle can't use its blindsight while deafened.

\DndMonsterAction{False Appearance} While the darkmantle remains motionless, it is indistinguishable from a cave formation such as a stalactite or stalagmite.

\DndMonsterSection{Actions}
\DndMonsterAttack[
	name=Crush,
	distance=melee,
	type=weapon,
	mod=+5,
	reach=5,
	dmg=\DndDice{1d6 + 3},
	dmg-type=bludgeoning,
	extra={, and the darkmantle attaches to the target. If the target is Medium or smaller and the darkmantle has advantage on the attack roll, it attaches by engulfing the target's head, and the target is also blinded and unable to breathe while the darkmantle is attached in this way. While attached to the target, the darkmantle can attack no other creature except the target but has advantage on its attack rolls. The darkmantle's speed also becomes 0, it can't benefit from any bonus to its speed, and it moves with the target. A creature can detach the darkmantle by making a successful DC 13 Strength check as an action. On its turn, the darkmantle can detach itself from the target by using 5 feet of movement.}
]

\DndMonsterAction{Darkness Aura (1/Day)}
A 15-foot radius of magical darkness extends out from the darkmantle, moves with it, and spreads around corners. The darkness lasts as long as the darkmantle maintains concentration, up to 10 minutes (as if concentrating on a spell). Darkvision can't penetrate this darkness, and no natural light can illuminate it. If any of the darkness overlaps with an area of light created by a spell of 2nd level or lower, the spell creating the light is dispelled.
\end{multicols}
\end{DndMonster}

\begin{DndMonster}[width=\textwidth + 8pt]{Doppelganger}
\begin{multicols}{2}
\DndMonsterType{Medium monstrosity (shapechanger)}
\DndMonsterBasics[armor-class={14}, hit-points={52 (8d8 + 16)}, speed={30 ft.}]
\MonsterStats{+0}{+4}{+2}{+0}{+1}{+2}
\DndMonsterDetails[saving-throws={}, skills={Deception +6, Insight +3}, damage-immunities={}, damage-resistances={}, damage-vulnerabilities={}, condition-immunities={charmed}, senses={darkvision 60 ft., passive Perception 11}, languages={Common}, challenge={3 (700 XP)}]
\DndMonsterAction{Shapechanger} The doppelganger can use its action to polymorph into a Small or Medium humanoid it has seen, or back into its true form. Its statistics, other than its size, are the same in each form. Any equipment it is wearing or carrying isn't transformed. It reverts to its true form if it dies.

\DndMonsterAction{Ambusher} The doppelganger has advantage on attack rolls against any creature it has surprised.

\DndMonsterAction{Surprise Attack} If the doppelganger surprises a creature and hits it with an attack during the first round of combat, the target takes an extra 10 (3d6) damage from the attack.

\DndMonsterSection{Actions}
\DndMonsterAction{Multiattack} The doppelganger makes two melee attacks.
\DndMonsterAttack[
	name=Slam,
	distance=melee,
	type=weapon,
	mod=+6,
	reach=5,
	dmg=\DndDice{1d6 + 4},
	dmg-type=bludgeoning
]
\DndMonsterAction{Read Thoughts}
The doppelganger magically reads the surface thoughts of one creature within 60 feet of it. The effect can penetrate barriers, but 3 feet of wood or dirt, 2 feet of stone, 2 inches of metal, or a thin sheet of lead blocks it. While the target is in range, the doppelganger can continue reading its thoughts, as long as the doppelganger's concentration isn't broken (as if concentrating on a spell). While reading the target's mind, the doppelganger has advantage on Wisdom (Insight) and Charisma (Deception, Intimidation, and Persuasion) checks against the target.
\end{multicols}
\end{DndMonster}



\subsection{Drider}
\begin{DndMonster}[width=\textwidth + 8pt]{Drider}
\begin{multicols}{2}
\DndMonsterType{Large monstrosity}
\DndMonsterBasics[armor-class={19 (natural armor)}, hit-points={123 (13d10 + 52)}, speed={30 ft., climb 30 ft.}]
\MonsterStats{+3}{+3}{+4}{+1}{+2}{+1}
\DndMonsterDetails[saving-throws={}, skills={Perception +5, Stealth +9}, damage-immunities={}, damage-resistances={}, damage-vulnerabilities={}, condition-immunities={}, senses={darkvision 120 ft., passive Perception 15}, languages={Elvish,}, challenge={6}]
\DndMonsterAction{Fey Ancestry} The drider has advantage on saving throws against being charmed, and magic can't put the drider to sleep.

\DndMonsterAction{Innate Spellcasting} The drider's innate spellcasting ability is Wisdom (spell save DC 13). The drider can innately cast the following spells, requiring no material components:
At will: \textit{dancing lights}
1/day each: \textit{darkness}, \textit{faerie fire}

\DndMonsterAction{Spider Climb} The drider can climb difficult surfaces, including upside down on ceilings, without needing to make an ability check.

\DndMonsterAction{Sunlight Sensitivity} While in sunlight, the drider has disadvantage on attack rolls, as well as on Wisdom (Perception) checks that rely on sight.

\DndMonsterAction{Web Walker} The drider ignores movement restrictions caused by webbing.

\DndMonsterSection{Actions}
\DndMonsterAction{Multiattack} The drider makes three attacks, either with its longsword or its longbow. It can replace one of those attacks with a bite attack.
\DndMonsterAttack[
	name=Bite,
	distance=melee,
	type=weapon,
	mod=+6,
	reach=5,
	dmg=\DndDice{1d4},
	dmg-type=piercing,
	extra={ plus 9 (2d8) poison damage.}
]
\DndMonsterAttack[
	name=Longsword,
	distance=melee,
	type=weapon,
	mod=+6,
	reach=5,
	dmg=\DndDice{1d8 + 3},
	dmg-type=slashing,
	extra={, or 8 (1d10 + 3) slashing damage if used with two hands.}
]
\DndMonsterAttack[
	name=Longbow,
	distance=ranged,
	type=weapon,
	mod=+6,
	range=150/600,
	dmg=\DndDice{1d8 + 3},
	dmg-type=piercing,
	extra={ plus 4 (1d8) poison damage.}
]
\end{multicols}
\end{DndMonster}



\section{Monsters (E)}\label{sec:monsters-e}



\subsection{Ettercap}
\begin{DndMonster}[width=\textwidth + 8pt]{Ettercap}
\begin{multicols}{2}
\DndMonsterType{Medium monstrosity}
\DndMonsterBasics[armor-class={13 (natural armor)}, hit-points={44 (8d8 + 8)}, speed={30 ft., climb 30 ft.}]
\MonsterStats{+2}{+2}{+1}{-2}{+1}{-1}
\DndMonsterDetails[saving-throws={}, skills={Perception +3, Stealth +4, Survival +3}, damage-immunities={}, damage-resistances={}, damage-vulnerabilities={}, condition-immunities={}, senses={darkvision 60 ft., passive Perception 13}, languages={—}, challenge={2 (450 XP)}]
\DndMonsterAction{Spider Climb} The ettercap can climb difficult surfaces, including upside down on ceilings, without needing to make an ability check.

\DndMonsterAction{Web Sense} While in contact with a web, the ettercap knows the exact location of any other creature in contact with the same web.

\DndMonsterAction{Web Walker} The ettercap ignores movement restrictions caused by webbing.

\DndMonsterSection{Actions}
\DndMonsterAction{Multiattack} The ettercap makes two attacks: one with its bite and one with its claws.
\DndMonsterAttack[
	name=Bite,
	distance=melee,
	type=weapon,
	mod=+4,
	reach=5,
	dmg=\DndDice{1d8 + 2},
	dmg-type=piercing,
	extra={ plus 4 (1d8) poison damage. The target must succeed on a DC 11 Constitution saving throw or be poisoned for 1 minute. The creature can repeat the saving throw at the end of each of its turns, ending the effect on itself on a success.}
]
\DndMonsterAttack[
	name=Claws,
	distance=melee,
	type=weapon,
	mod=+4,
	reach=5,
	dmg=\DndDice{2d4 + 2},
	dmg-type=slashing
]
\DndMonsterAttack[
	name=Web (Recharge 5-6),
	distance=ranged,
	type=weapon,
	mod=+4,
	range=30/60,
	dmg=\DndDice{1d6},
	dmg-type=poison,
	extra={. The creature is restrained by webbing. As an action, the restrained creature can make a DC 11 Strength check, escaping from the webbing on a success. The effect also ends if the webbing is destroyed. The webbing has AC 10, 5 hit points, vulnerability to fire damage, and immunity to bludgeoning, poison, and psychic damage.}
]
\end{multicols}
\end{DndMonster}