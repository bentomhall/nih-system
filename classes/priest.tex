\section{Priest}

Design Goals: The priest is the basic full-caster divine class, replacing the cleric. Their UCT is Miracles--basically Divine Intervention, but starting earlier. Their subclasses give bonuses to various types of spells, basically adding riders. SDCT: 7/4/5/4

Subclasses:
\begin{itemize}
	\item Life Domain: healing-focused support.
	\item War domain: Gets armor and weapons and an ersatz Extra Attack.
	\item Knowledge Domain: gets extra skill tricks.
\end{itemize}

\subsection{Class Features}

As a priest, you gain the following class features.

\subsection{Hit Points}

\textbf{Hit Dice:} 1d8 per priest level

\textbf{Hit Points at 1st Level:} 8 + your Constitution modifier

\textbf{Hit Points at Higher Levels:} 1d8 (or 5) + your Constitution modifier per priest level after 1st

\subsection{Proficiencies}

\textbf{Armor:} Light armor, medium armor, shields

\textbf{Weapons:} Simple weapons

\textbf{Tools:} None

\textbf{Saving Throws:} Wisdom, Charisma

\textbf{Skills:} Choose two from History, Insight, Medicine, Persuasion, and Religion

\subsection{Equipment}

You start with the following equipment, in addition to the equipment granted by your background:
\begin{itemize}
\item (\textit{a}) a mace or (\textit{b}) a warhammer (if proficient)
\item (\textit{a}) scale mail, (\textit{b}) leather armor, or (\textit{c}) chain mail (if proficient)
\item (\textit{a}) a light crossbow and 20 bolts or (\textit{b}) any simple weapon
\item (\textit{a}) a priest's pack or (\textit{b}) an explorer's pack
\item A shield and a holy symbol
\end{itemize}

\onecolumn
\begin{DndTable}[header=The Priest\label{tbl:priest}]{XXXXXXXXX}
 Level & Proficiency Bonus & Features                                          & Cantrips Known & Spells Known & Miracles per day & Stamina & Aether & Aether Limit  \\
 1st   & +2                & Spellcasting, Divine Domain                       & 3   & 2   & -   & 1 & 4 & 2 \\
 2nd   & +2                & Miracles, Divine Domain Feature								   & 3   & 3   & 1   & 1 & 8 & 3 \\
 3rd   & +2                & Channel Divine Power						                   & 3   & 4   & 1   & 2 & 12 & 4 \\
 4th   & +2                & Ability Score Improvement                         & 4   & 5   & 1   & 2 & 16 & 5 \\
 5th   & +3                & Divine Overflow                                 	 & 4   & 6   & 1   & 3 & 20 & 6 \\
 6th   & +3                & Divine Domain Feature  													 & 4   & 7   & 2   & 3 & 24 & 7 \\
 7th   & +3                & Improved Channel, Improved Miracles							 & 4   & 8   & 2   & 4 & 28 & 8 \\
 8th   & +3                & Ability Score Improvement,    										 & 4   & 9   & 2   & 4 & 32 & 9 \\
 9th   & +4                & Divine Domain Feature                             & 4   & 10   & 2  & 5 & 36 & 10 \\
 10th  & +4                & Improved Divine Overflow                          & 5   & 10   & 2  & 5 & 40 & 11 \\
 11th  & +4                & Lesser Legendary Effect                           & 5   & 11   & 2  & 6 & 44 & 12 \\
 12th  & +4                & Ability Score Improvement                         & 5   & 11   & 2  & 6 & 48 & 13 \\
 13th  & +5                & Lesser Legendary Effect (2)                       & 5   & 12   & 2  & 7 & 52 & 13 \\
 14th  & +5                & Radiating Overflow									               & 5   & 12   & 2  & 7 & 56 & 14 \\
 15th  & +5                & Lesser Legendary Effect (3)                       & 5   & 13   & 2  & 8 & 60 & 14 \\
 16th  & +5                & Ability Score Improvement                         & 5   & 13   & 2  & 8 & 64 & 15 \\
 17th  & +6                & Divine Domain Feature, Greater Legendary Effect   & 5   & 14   & 3  & 9 & 68 & 15 \\
 18th  & +6                & Greater Legendary Effect (2)                         & 5   & 14   & 3  & 9 & 72 & 16 \\
 19th  & +6                & Ability Score Improvement                         & 5   & 15   & 3  & 10 & 76 & 16 \\
 20th  & +6                & Supreme Miracles								                   & 5   & 15   & 3  & 10 & 80 & 17 \\
\end{DndTable}
\twocolumn

\subsection{Spellcasting}

As a conduit for divine power, you can cast priest spells.

\subsection{Cantrips}

At 1st level, you know three cantrips of your choice from the priest spell list. You learn additional priest cantrips of your choice at higher levels, as shown in the Cantrips Known column of the Priest table.

\subsubsection{Preparing and Casting Spells}

\nameref{tbl:priest} table shows how much aether (AET) you have to cast your spells and do other magical tasks. To cast a spell that requires aether, you must expend aether equal to its cost or greater. You regain all expended aether when you finish a long rest. It also shows your Aether Limit, which is the maximum aether you can expend on a single action.

You know a certain number of priest spells, choosing from the priest spell list. You can trade out any known spell for any other spell you can learn from that list when you finish a long rest. When you do so, choose a number of priest spells equal to your Wisdom modifier + your priest level (minimum of one spell). To prepare a spell you must be able to cast it without exceeding your Aether Limit.

\subsection{Spellcasting Ability}

Wisdom is your spellcasting ability for your priest spells. The power of your spells comes from your devotion to your deity. You use your Wisdom whenever a priest spell refers to your spellcasting ability. In addition, you use your Wisdom modifier when setting the saving throw DC for a priest spell you cast and when making an attack roll with one.

\textbf{Spell save DC} = 8 + your proficiency bonus + your Wisdom modifier

\textbf{Spell attack modifier} = your proficiency bonus + your Wisdom modifier

\subsection{Ritual Casting}

You learn a common incantation (see \nameref{ch:incantations} for the list) of your choice. When you reach 5th level, you learn an uncommon incantation of your choice, and at 11th level you learn a rare incantation of your choice. You can cast any incantation you learned from this feature without needing a Ritual Scroll in hand.

\subsection{Spellcasting Focus}

You can use a holy symbol (see \nameref{ch:equipment}) as a spellcasting focus for your priest spells.

\subsection{Divine Domain}

Choose one domain related to your deity: Knowledge, Life, Light, Nature, Tempest, Trickery, or War. Each domain is detailed at the end of the class description, and each one provides examples of gods associated with it. Your choice grants you domain spells and other features when you choose it at 1st level. It also grants you additional ways to use Channel Divinity when you gain that feature at 2nd level, and additional benefits at 6th, 8th, and 17th levels.

\subsection{Domain Spells}

Each domain has a list of spells—its domain spells— that you gain at the priest levels noted in the domain description. Once you gain a domain spell, you always have it prepared, and it doesn't count against the number of spells you can prepare each day.

If you have a domain spell that doesn't appear on the priest spell list, the spell is nonetheless a priest spell for you.

\subsection{Miracles}

At 2nd level, your relationship with your Ascended patron has grown to the point that you can make impromptu pleas for direct assistance and have them answered based on your faith. As an action, you state the nature of the assistance you desire and roll a d20 and add your Wisdom modifier. This is not an ability check and cannot be modified by any other feature. The result determines the outcome:

\begin{DndTable}[header=Miracle Outcomes]{XX}
	\textbf{Check Result} & \textbf{Outcome} \\
	Less than 5 & No intervention \\
	5-9 & No intervention, but the daily use is not expended \\
	10-14 & A priest spell with cost less than 5 AET, chosen by the DM, takes effect \\
	14-19 & A priest spell with cost less than 5 AET, chosen by you, takes effect \\
	20+ & Any spell with cost less than 5 AET, chosen by you, takes effect \\
\end{DndTable}

Once you use this feature once, you cannot use it again until you complete a long rest. The number of uses per day increases as shown on the \nameref{tbl:priest} Table.

\subsection{Channel Divine Power}
Starting at 3rd level, you can channel divine power more directly, creating magical effects not possible through normal spells. Every priest gains the options to channel healing energy or to channel destructive energy (outlined below). Your Domain may grant additional options for this. Channeling divine power requires expending aether and is limited by your aether limit as if it was a spell, but cannot be countered or dispelled by non-legendary effects.

\subsubsection{Channel Healing Energy}
As an action, you expend at least 1 aether to radiate positive energy. For every AET spent, all creatures other than demons, undead, or constructs within 10 feet of you regain 1d6 hit points. Constructs are unaffected by this ability; demons and undead must make a Constitution saving throw against your spell save DC, taking 1d6 radiant damage per aether spent on a failed save or half as much on a success.

\subsubsection{Channel Destructive Energy}
As an action, you expend at least 1 aether to radiate destructive energy. For every AET spent, all creatures other than demons or undead within 10 feet of you must make a Constitution saving throw against your spell save DC, taking 1d6 radiant damage per aether spent on a failed save or half as much on a success. Demons and undead are healed for 1d6 hit points per aether spent.

\subsection{Ability Score Improvement}

When you reach 4th level, and again at 8th, 12th, 16th, and 19th level, you can increase one ability score of your choice by 2, or you can increase two ability scores of your choice by 1. As normal, you can't increase an ability score above 20 using this feature.

\subsection{Divine Overflow}

Starting at 5th level, the energy you expend on your spells and other magical effects overflows, allowing you to create additional effects. Every priest can use the Castigation overflow effect; your Domain grants you an additional option. Once you use this feature once, you cannot use it again until you finish a short or long rest.

\subsubsection{Castigation}
When you expend AET to heal one or more creatures, you can cause a creature you can see within 60 feet of you to take radiant damage equal to your level.

\subsection{Improved Divine Channel}
Starting at 7th level, when you use your Divine Channel ability, you add your Wisdom modifier to the damage or healing done.

\subsection{Improved Miracles}
Starting at 7th level, the outcomes of your miracle uses have improved. Use the table below instead of the previous one.

\begin{DndTable}[header=Miracle Outcomes]{XX}
	\textbf{Check Result} & \textbf{Outcome} \\
	Less than 5 & No intervention \\
	5-9 & No intervention, but the daily use is not expended \\
	10-14 & A priest spell with cost less than 8 AET, chosen by the DM, takes effect \\
	14-19 & A priest spell with cost less than 8 AET, chosen by you, takes effect \\
	20+ & Any spell with cost less than 8 AET, chosen by you, takes effect \\
\end{DndTable}

\subsection{Radiating Overflow}
Starting at 10th level, when you use your Divine Overflow ability, you can affect a number of creatures equal to half the ather expended instead of one. All creatures must be within 60 feet of you.

\subsection{Legendary Effects}
At 11th, 13th, and 15th levels, you can learn one Legendary effect tagged with Divine or Generic that is also tagged as Lesser. 

At 17th and 18th levels, you can learn one Legendary effect tagged with Divine or Generic whether it is tagged Lesser or Greater.

You can use each Legendary effect once per long rest, and your saving throw DC for these effects is your spell save DC. When you learn a new legendary effect, you can also swap out one legendary effect you know for a different one.

\subsection{Supreme Miracles}
At level 20, you can perform greater miracles. Use the table below to determine the outcome of your miracles. You can only gain the benefit of rolling a 25 on the check once per day; any other times that result comes up, treat it as a 24.

\begin{DndTable}[header=Miracle Outcomes]{XX}
	\textbf{Check Result} & \textbf{Outcome} \\
	Less than 5 & No intervention \\
	5-9 & No intervention, but the daily use is not expended \\
	10-14 & Any priest spell, chosen by the DM, takes effect \\
	14-19 & Any priest spell, chosen by you, takes effect \\
	20-24 & Any spell, chosen by you, takes effect \\
	25+ & Any legendary effect, chosen by you, takes effect.
\end{DndTable}

\subsection{Priest Domains}

\subsection{Life Domain}

The Life domain focuses on the vibrant positive energy—one of the fundamental forces of the universe—that sustains all life. The gods of life promote vitality and health through healing the sick and wounded, caring for those in need, and driving away the forces of death and undeath. Almost any non-evil deity can claim influence over this domain, particularly agricultural deities, sun gods , gods of healing or endurance, and gods of home and community.

\begin{DndTable}[header=Life Domain Spells]{XX}
 Priest Level & Spells                              \\ 
 1st          & bless, cure wounds                   \\
 3rd          & lesser restoration, spiritual weapon \\
 5th          & beacon of hope, revivify             \\
 7th          & death ward, guardian of faith        \\
 9th          & mass cure wounds, raise dead         \\
\end{DndTable}

\subsection{Bonus Proficiency}

When you choose this domain at 1st level, you gain proficiency with heavy armor.

\subsection{Disciple of Life}

Also starting at 1st level, your healing spells are more effective. Whenever you use a spell of 1st level or higher to restore hit points to a creature, the creature regains additional hit points equal to 2 + the spell's level.

\subsection{Channel Divinity: Preserve Life}

Starting at 2nd level, you can use your Channel Divinity to heal the badly injured.

As an action, you present your holy symbol and evoke healing energy that can restore a number of hit points equal to five times your priest level. Choose any creatures within 30 feet of you, and divide those hit points among them. This feature can restore a creature to no more than half of its hit point maximum. You can't use this feature on an undead or a construct.

\subsection{Blessed Healer}

Beginning at 6th level, the healing spells you cast on others heal you as well. When you cast a spell of 1st level or higher that restores hit points to a creature other than you, you regain hit points equal to 2 + the spell's level.

\subsection{Divine Strike}

At 8th level, you gain the ability to infuse your weapon strikes with divine energy. Once on each of your turns when you hit a creature with a weapon attack, you can cause the attack to deal an extra 1d8 radiant damage to the target. When you reach 14th level, the extra damage increases to 2d8.

\subsection{Supreme Healing}

Starting at 17th level, when you would normally roll one or more dice to restore hit points with a spell, you instead use the highest number possible for each die. For example, instead of restoring 2d6 hit points to a creature, you restore 12.