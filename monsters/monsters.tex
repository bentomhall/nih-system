\chapter{Appendix D: Monsters}\label{ch:monsters}

A monster's statistics, sometimes referred to as its \textbf{stat block}, provide the essential information that you need to run the monster.

\begin{figure}
\begin{DndComment}{Modifying Monsters}
    Despite the versatile collection of monsters in this book, you might be at a loss when it comes to finding the perfect creature for part of an adventure. Feel free to tweak an existing creature to make it into something more useful for you, perhaps by borrowing a trait or two from a different monster or by using a \textbf{variant} or \textbf{template}, such as the ones in this book. Keep in mind that modifying a monster, including when you apply a template to it, might change its challenge ratings.
\end{DndComment}
\end{figure}

\subsection{Size}

A monster can be Tiny, Small, Medium, Large, Huge, or Gargantuan. The Size Categories table shows how much space a creature of a particular size controls in combat. See the Core Rules for more information on creature size and space.

\begin{figure}
\begin{DndTable}[header=Size Categories]{lXr}
    \textbf{Size}       & \textbf{Space}                  & \textbf{Examples}    \\        
    Tiny       & 2½ by 2½ ft.           & Imp, sprite   \\      
    Small      & 5 by 5 ft.             & Giant rat, goblin \\  
    Medium     & 5 by 5 ft.             & Orc, werewolf       \\
    Large      & 10 by 10 ft.           & Hippogriff, ogre    \\
    Huge       & 15 by 15 ft.           & Fire giant, treant  \\
    Gargantuan & 20 by 20 ft. or larger & Kraken, purple worm \\
\end{DndTable}
\end{figure}

\subsection{Type}

A monster's type speaks to its fundamental nature. Certain spells, magic items, class features, and other effects in the game interact in special ways with creatures of a particular type. For example, an \textit{arrow of dragon slaying} deals extra damage not only to dragons but also other creatures of the dragon type, such as dragon turtles and wyverns.

The game includes the following monster types, which have no rules of their own.

\subparagraph*{Aberrations} are utterly alien beings. Many of them have innate magical abilities drawn from the creature's alien mind rather than the mystical forces of the world. The quintessential aberrations are the twisted comiedai, mkhulu, and Awakened corruptions.

\subparagraph*{Beasts} are non-humanoid creatures that are a natural part of the fantasy ecology. Some of them have magical powers, but most are unintelligent and lack any society or language. Beasts include all varieties of ordinary animals, dinosaurs, and giant versions of animals.

\subparagraph*{Celestials} are creatures of the Astral who are acting on behalf of an Ascendant or deity. They might be employed as messengers or agents in the mortal realm and throughout the planes.

\subparagraph*{Constructs} are made, not born. Some are programmed by their creators to follow a simple set of instructions, while others are imbued with sentience and capable of independent thought. Golems are the iconic constructs.

\subparagraph*{Dragons} are large reptilian creatures of ancient origin and tremendous power. True dragons, including the good metallic dragons and the evil chromatic dragons, are highly intelligent and have innate magic. Also in this category are creatures distantly related to true dragons, but less powerful, less intelligent, and less magical, such as wyverns and pseudo-dragons.

\subparagraph*{Elementals} are creatures native to the elemental planes. Some creatures of this type are little more than animate masses of their respective elements, including the creatures simply called elementals. Others have biological forms infused with elemental energy. The races of genies, including djinn and efreet, form the most important civilizations on the elemental planes. Other elemental creatures include azers, invisible stalkers, and water weirds.

\subparagraph*{Fey} are magical creatures closely tied to the forces of nature, called the kami. They dwell in twilight groves and misty forests. In some worlds, they are closely tied to the Plane of Shadow, also called the Plane of Faerie. Fey are most often obsessed with something mortal, called their "aspect". Fey include dryads, pixies, and satyrs.

\subparagraph*{Fiends} are creatures of either the Astral or Abyssal, summoned into the Mortal in various horrifying forms. A few are the servants of deities, but many more labor under the leadership of arch-devils and demon princes. Evil priests and mages sometimes summon fiends to the material world to do their bidding. Fiends include demons, devils, hell hounds, and other lesser creatures.

\subparagraph*{Giants} tower over humans and their kind. They are human-like in shape, though some have multiple heads (ettins) or deformities (fomorians). The six varieties of true giant are hill giants, stone giants, frost giants, fire giants, cloud giants, and storm giants. Besides these, creatures such as ogres and trolls are giants.

\subparagraph*{Humanoids} are the main peoples of a fantasy gaming world, both civilized and savage, including humans and a tremendous variety of other species. They have language and culture, few if any innate magical abilities (though most humanoids can learn spellcasting), and a bipedal form. The most common humanoid races are the ones most suitable as player characters: humans, dwarves, elves, and halflings. Almost as numerous are the races of goblinoids (goblins, hobgoblins, and bugbears), orcs, lizardfolk, and kobolds.

\subparagraph*{Monstrosities} are monsters in the strictest sense—frightening creatures that are not ordinary, not truly natural, and almost never benign. Some are the results of magical experimentation gone awry and others are the product of terrible curses. They defy categorization, and in some sense serve as a catch-all category for creatures that don't fit into any other type.

\subparagraph*{Oozes} are gelatinous creatures that rarely have a fixed shape. They are mostly subterranean, dwelling in caves and dungeons and feeding on refuse, carrion, or creatures unlucky enough to get in their way. Black puddings and gelatinous cubes are among the most recognizable oozes.

\subparagraph*{Plants} in this context are vegetable creatures, not ordinary flora. Most of them are ambulatory, and some are carnivorous. The quintessential plants are the shambling mound and the treant. Fungal creatures such as the gas spore and the myconid also fall into this category.

\subparagraph*{Undead} are once-living creatures brought to a horrifying state of undeath through the practice of necromantic magic or some unholy curse, generally involving the entropy spirits called jotnar. Undead include walking corpses, such as vampires and zombies, as well as bodiless spirits, such as ghosts and specters.

\subsubsection{Tags}

A monster might have one or more tags appended to its type, in parentheses. For example, an orc has the \textit{humanoid (orc)} type. The parenthetical tags provide additional categorization for certain creatures. The tags have no rules of their own, but something in the game, such as a magic item, might refer to them. For instance, a spear that is especially effective at fighting demons would work against any monster that has the demon tag.

\subsection{Attitude}

Monsters have an attitude listed that describes how a generic member of that type of monster will act by default. Specific individuals may act differently, but this serves as a guide to role play for the DM. Example attitudes:

\subparagraph*{Craven} A craven creature is easy to intimidate into fleeing (if possible) or surrendering (if flight is too risky). They tend to attack in groups and run when casualties start to mount or if their leader is killed.

\subparagraph*{Honorable} These creatures will generally honor bargains, parole, and challenges to duels. Even evil creatures can be honorable. Most honorable creatures can be reasoned with.

\subparagraph*{Individualistic} Every monster for themselves. These tend to live alone or, if the are encountered in groups, not really care about the other members of their groups. Killing one doesn't affect a different one, but an individual might flee if wounded even if it shatters the group's chances.

\subparagraph*{Single-minded} Oozes, zombies, and some other creatures are single-minded. This means that they generally cannot be reasoned with, distracted, confused, or intimidate. They will not flee or surrender. Generally they will pursue the closest target in combat; if given other orders they will do those to the exclusion of all else, including personal safety.

\subsection{Armor Class}

A monster that wears armor or carries a shield usually has an Armor Class (AC) that takes its armor, shield, and Dexterity into account. Otherwise, a monster's AC is based on its Dexterity modifier and natural armor, if any. If a monster has natural armor, wears armor, or carries a shield, this is noted in parentheses after its AC value.

\subsection{Hit Points}

A monster usually dies or is destroyed when it drops to 0 hit points. For more on hit points, see \nameref{ch:character-creation}.

A monster's hit points are presented both as a die expression and as an average number. For example, a monster with 2d8 hit points has 9 hit points on average (2 \texttimes 4½).

A monster's size determines the die used to calculate its hit points, as shown in the Hit Dice by Size table.

\subsection{Hit Dice by Size}

\begin{DndTable}[header=Hit Dice by Size]{XXX}
    \textbf{Monster Size} & \textbf{Hit Die} & \textbf{Average HP per Die} \\
    Tiny         & d4      & 2½                 \\
    Small        & d6      & 3½                 \\
    Medium       & d8      & 4½                 \\
    Large        & d10     & 5½                 \\
    Huge         & d12     & 6½                 \\
    Gargantuan   & d20     & 10½    
\end{DndTable}
                
A monster's Constitution modifier also affects the number of hit points it has. Its Constitution modifier is multiplied by the number of Hit Dice it possesses, and the result is added to its hit points. For example, if a monster has a Constitution of 12 (+1 modifier) and 2d8 Hit Dice, it has 2d8 + 2 hit points (average 11).

\subsection{Speed}

A monster's speed tells you how far it can move on its turn. For more information on speed, see \nameref{ch:order-of-combat}.

All creatures have a walking speed, simply called the monster's speed. Creatures that have no form of ground-based locomotion have a walking speed of 0 feet.

Some creatures have one or more of the following additional movement modes.

\subsubsection{Burrow}

A monster that has a burrowing speed can use that speed to move through sand, earth, mud, or ice. A monster can't burrow through solid rock unless it has a special trait that allows it to do so. Burrowing out of reach provokes opportunity attacks.

\subsubsection{Climb}

A monster that has a climbing speed can use all or part of its movement to move on vertical surfaces. The monster doesn't need to spend extra movement to climb.

\subsubsection{Fly}

A monster that has a flying speed can use all or part of its movement to fly. Some monsters have the ability to \textbf{hover}, which makes them hard to knock out of the air (as explained in the rules on flying in the core rules). Such a monster stops hovering when it dies.

\subsubsection{Swim}

A monster that has a swimming speed doesn't need to spend extra movement to swim.

\subsection{Ability Scores and Saving Throws}

Every monster has six ability scores (Strength, Dexterity, Constitution, Intelligence, Wisdom, and Charisma). If a creature has a bonus to a particular type of saving throw, that will be noted in parentheses after the modifier.

\subsection{Additional Bonuses}

Monsters do not have a proficiency bonus like player characters. Instead, they have an arbitrary additional bonus that affects selected ability checks, attack rolls, saving throw DCs, and saving throw modifiers. These are generally in the range of +0 --- +6 like proficiency bonuses, but are applied to make the numbers come out right while also representing the agility, grace, and training of the monster in question. Clumsy monsters or those who are not generally predatory frequently have no additional bonus beyond their raw ability score to attack rolls, for instance, while highly skilled combatants might get a similar bonus to a player character's proficiency, and many monsters may be in between the two. As a rule of thumb, use the offensive rating as the "character level" when determining the maximum additional bonus to apply.
              
\subsection{Skills}

The Skills entry is reserved for monsters that are exceptionally competent in one or more skills. For example, a monster that is very perceptive and stealthy might have bonuses to Wisdom (Perception) and Dexterity (Stealth) checks.

A skill bonus is usually the sum of a monster's relevant ability modifier and its additional bonus, if any. Other modifiers might apply. For instance, a monster might have a larger-than-expected bonus (usually double its additional bonus) to account for its heightened expertise.

\subsection{Vulnerabilities, Resistances, and Immunities}

Some creatures have vulnerability, resistance, or immunity to certain types of damage. In addition, some creatures are immune to certain conditions.

\subsection{Senses}

The Senses entry notes a monster's passive Wisdom (Perception) score, as well as any special senses the monster might have. Special senses are described below.

\subsubsection{Blindsight}

A monster with blindsight can perceive its surroundings without relying on sight, within a specific radius. For all purposes this counts as being able to see within that radius.

Creatures without eyes, such as grimlocks and gray oozes, typically have this special sense, as do creatures with echolocation or heightened senses, such as bats and true dragons.

If a monster is naturally blind, it has a parenthetical note to this effect, indicating that the radius of its blindsight defines the maximum range of its perception.

\subsubsection{Darkvision}

A monster with darkvision can see in the dark within a specific radius. The monster can see in dim light within the radius as if it were bright light, and in darkness as if it were dim light. The monster can't discern color in darkness, only shades of gray. Many creatures that live underground have this special sense.

\subsection{Tremorsense}

A monster with tremorsense can detect and pinpoint the origin of vibrations within a specific radius, provided that the monster and the source of the vibrations are in contact with the same ground or substance. Tremorsense can't be used to detect flying or incorporeal creatures. Many burrowing creatures, such as ankhegs and umber hulks, have this special sense. Tremorsense does not count as sight.

\subsection{Truesight}

A monster with truesight can, out to a specific range, see in normal and magical darkness, see invisible creatures and objects, automatically detect visual illusions and succeed on saving throws against them, and perceive the original form of a shapechanger or a creature that is transformed by magic. Furthermore, the monster can see into the Ethereal Plane within the same range.

\subparagraph*{Armor, Weapon, and Tool Proficiencies}

Assume that a creature is proficient with its armor and tools. If you swap them out, you decide whether the creature is proficient with its new equipment. Except in rare cases, monsters generally prefer to use weapons and armor suited for them.

For example, a hill giant typically wears hide armor and wields a greatclub. You could equip a hill giant with chain mail and a greataxe instead, and assume the giant is proficient with both, one or the other, or neither.

See the \nameref{ch:equipment} for rules on using armor without proficiency.

Monsters conditionally apply their additional bonus to attack rolls, as described above. This same bonus should apply to all attacks if it applies to any.

\subsection{Languages}

The languages that a monster can speak are listed in alphabetical order. Sometimes a monster can understand a language but can't speak it, and this is noted in its entry. A “—” indicates that a creature neither speaks nor understands any language.

\subsubsection{Telepathy}

Telepathy is a magical ability that allows a monster to communicate mentally with another creature within a specified range. The contacted creature doesn't need to share a language with the monster to communicate in this way with it, but it must be able to understand at least one language. A creature without telepathy can receive and respond to telepathic messages but can't initiate or terminate a telepathic conversation.

A telepathic monster doesn't need to see a contacted creature and can end the telepathic contact at any time. The contact is broken as soon as the two creatures are no longer within range of each other or if the telepathic monster contacts a different creature within range. A telepathic monster can initiate or terminate a telepathic conversation without using an action, but while the monster is incapacitated, it can't initiate telepathic contact, and any current contact is terminated.

A creature in any location where magic doesn't function can't send or receive telepathic messages.

\subsection{Challenge}

A monster's \textbf*{challenge ratings} tells you how great a threat the monster is. Each monster has two ratings, offensive and defensive, separated by a slash, written as "offensive/defensive". Generally, the overall rating is the geometric mean of the two (the square root of the product of the values), but this is really only used for ordering monsters within their groups and provides little relevant information.

\subparagraph*{Offensive Rating (OR)} The offensive rating of a monster tells you how likely it is for that monster to drop a full-health character to 0 HP in a single round.

\begin{itemize}
	\item[] \textbf{OR > 2\texttimes Level} The monster is very likely to drop a weak (ie d6 HD with +0 or +1 CON) character to zero from full in a single round even if they use defensive measures, and is capable of dropping a medium-health (ie d8 HD with +1 CON or d6 HD with +3 or more CON) character to zero in a single turn if they roll well or no defensive measures are taken. It is likely that this monster can kill a weak character through massive damage or by having enough attacks left over after downing the character to kill them outright.
	\item[] \textbf{OR > Level} The monster poses a substantial threat and is capable of dropping a weak (ie d6 HD with +0 or +1 CON) character to zero from full in a single round if they crit or roll well and significant defensive measures are not taken. It is unlikely, but possible, that a good round with critical hits can outright kill a weak character.
	\item[] \textbf{OR $\approx$ Level} If the monster hits with all its abilities and deals average damage and no defensive measures are taken, it will come very close to downing a weak character or will just barely down them. It is unlikely that the monster can outright kill a character in a single round, even with critical hits.
	\item[] \textbf{OR < Level} The monster has to crit and/or roll very well on its attacks to down a full-health weak character in one turn. It is highly unlikely to be able to outright kill a character in one turn.
	\item[] \textbf{OR < Level / 2} The monster poses little direct offensive threat to even a weak character except in groups. Be careful of synergies, however, where one monster buffs or enables another monster to go well beyond its normal nature.
\end{itemize}

\subparagraph*{Defensive Rating (DR)} The defensive rating of a monster tells you how likely it is for that monster to survive long enough under direct fire from the party's damage dealers in order to do use its cool abilities at least once (generally considered about 3 rounds). This assumes "baseline" characters without substantial combat-effective magic items or particularly targeting vulnerabilities. It's a rough "static training dummy" measure and should be taken with a grain of salt.

\begin{itemize}
	\item[] \textbf{DR > Level} The monster is likely to survive at least 3 rounds of direct fire from a baseline party.
	\item[] \textbf{DR $\approx$ Level} The monster will probably die in 2-3 rounds of direct fire from a baseline party, or survive double that if not focused down.
	\item[] \textbf{DR < Level} The monster will likely die in 1-2 rounds of direct fire from the party. If the gap is substantial, it is unlikely that the monster will even get to act if it is attacked first. Such monsters need bodyguards or need ways of ambushing the party to have any influence on combat.
\end{itemize}

\begin{figure}
	\begin{DndComment}{What is a baseline party?}
		A baseline party makes the following assumptions.
		\begin{itemize}
			\item 4 player characters without combat-effective allies, summoned creatures, or pets.
			\item Standard array
			\item Characters built competently, but not particularly optimized for damage. Their highest ability scores are in their class primary attributes, their secondary attributes are not neglected, and they're using equipment that suits their class abilities. ASIs are spent to improve the primary ability score until cap.
			\item Generally two of them are specialized damage dealers, with the other two playing any combination of support, healing, defensive, skills, or control.
			\item None of them have magic items that substantially improve their combat capabilities.
		\end{itemize}

		An example of such a party might be
		\begin{itemize}
			\item A sword-and-shield-using Defender Armsman
			\item A dual-wielding rogue
			\item A fire-loving Awakened Arcanist
			\item A Life-domain priest
		\end{itemize}
	\end{DndComment}
\end{figure}

\subsection{Special Traits}

Special traits (which appear after a monster's challenge rating but before any actions or reactions) are characteristics that are likely to be relevant in a combat encounter and that require some explanation.

\subsubsection{Spellcasting}

A monster with the Spellcasting special trait has the ability to cast selected spells, which it uses to cast its spells (as explained in \nameref{ch:spellcasting}).

The monster has a list of spells known, which may be drawn from multiple class lists.

Each spell says how many times it can be cast per day.

Specific monsters have innate spellcasting which means that they do require components to cast their spells; they can still be perceived to be casting a spell.

\begin{figure}
\begin{DndComment}{Spell selection}
    Casters generally only have a limited selection of spells in their stat blocks, those focused on combat use. Generally they have a big concentration spell and one or more lower priority or defensive spells. Their primary spell damage comes from their spell attacks, which count as cantrips. 
\end{DndComment}
\end{figure} 

\subsection{Actions}

When a monster takes its action, it can choose from the options in the Actions section of its stat block or use one of the actions available to all creatures, such as the Dash or Hide action, as described in \nameref{ch:order-of-combat}.

\subsubsection{Melee and Ranged Attacks}

The most common actions that a monster will take in combat are melee and ranged attacks. These can be spell attacks or weapon attacks, where the “weapon” might be a manufactured item or a natural weapon, such as a claw or tail spike. For more information on different kinds of attacks, see \nameref{ch:order-of-combat}.

\subparagraph*{Creature vs. Target} The target of a melee or ranged attack is usually either one creature or one target, the difference being that a “target” can be a creature or an object.

\subparagraph*{Hit} Any damage dealt or other effects that occur as a result of an attack hitting a target are described after the “\textit{Hit}” notation. You have the option of taking average damage or rolling the damage; for this reason, both the average damage and the die expression are presented.

\subparagraph*{Miss} If an attack has an effect that occurs on a miss, that information is presented after the “\textit{Miss:}” notation.

\subsubsection{Spell Attacks}

Some monsters have spell attacks. These count as cantrips for features for that care about spells. Generally, they do not require components and are specialized for that creature. 

\subsection{Multiattack}

A creature that can make multiple attacks on its turn has the Multiattack action. A creature can't use Multiattack when making an opportunity attack, which must be a single melee attack.

\subsection{Ammunition}

A monster carries enough ammunition to make its ranged attacks. You can assume that a monster has 5 pieces of ammunition for a thrown weapon attack, and 10 pieces of ammunition for a projectile weapon such as a bow or crossbow.

\section{Reactions}

If a monster can do something special with its reaction, that information is contained here. If a creature has no special reaction, this section is absent.

All creatures with a melee attack in their stat block can make opportunity attacks. Monsters cannot take the special actions outlined in \nameref{sec:stamina-and-aether}. If they have similar capabilities, those will appear as explicit actions in their stat blocks. 

\section{Limited Usage}

Some special abilities have restrictions on the number of times they can be used.

\subparagraph*{X/Day} The notation “X/Day” means a special ability can be used X number of times and that a monster must finish a long rest to regain expended uses. For example, “1/Day” means a special ability can be used once and that the monster must finish a long rest to use it again.

\subparagraph*{Recharge X-Y} The notation “Recharge X-Y” means a monster can use a special ability once and that the ability then has a random chance of recharging during each subsequent round of combat. At the start of each of the monster's turns, roll a d6. If the roll is one of the numbers in the recharge notation, the monster regains the use of the special ability. The ability also recharges when the monster finishes a short or long rest.

For example, “Recharge 5-6” means a monster can use the special ability once. Then, at the start of the monster's turn, it regains the use of that ability if it rolls a 5 or 6 on a d6.

\subparagraph*{Recharge after a Short or Long Rest} This notation means that a monster can use a special ability once and then must finish a short or long rest to use it again.

\textbf*{Grapple Rules for Monsters}

Many monsters have special attacks that allow them to quickly grapple prey. When a monster hits with such an attack, it doesn't need to make an additional ability check to determine whether the grapple succeeds, unless the attack says otherwise.

A creature grappled by the monster can use its action to try to escape. To do so, it must succeed on a Strength (Athletics) or Dexterity (Acrobatics) check against the escape DC in the monster's stat block. If no escape DC is given, assume the DC is 10 + the monster's Strength (Athletics) modifier.

\subsection{Equipment}

A stat block rarely refers to equipment, other than armor or weapons used by a monster. A creature that customarily wears clothes, such as a humanoid, is assumed to be dressed appropriately.

You can equip monsters with additional gear and trinkets however you like, and you decide how much of a monster's equipment is recoverable after the creature is slain and whether any of that equipment is still usable. A battered suit of armor made for a monster is rarely usable by someone else, for instance.

If a spellcasting monster needs material components to cast its spells, assume that it has the material components it needs to cast the spells in its stat block.

\section{Legendary Creatures}

A legendary creature can do things that ordinary creatures can't. It can take special actions outside its turn, and it might exert magical influence for miles around.

If a creature assumes the form of a legendary creature, such as through a spell, it doesn't gain that form's legendary actions, lair actions, or regional effects.

\subsection{Legendary Actions}

A legendary creature can take a certain number of special actions—called legendary actions—outside its turn. Only one legendary action option can be used at a time and only at the end of another creature's turn. A creature regains its spent legendary actions at the start of its turn. It can forgo using them, and it can't use them while incapacitated or otherwise unable to take actions. If surprised, it can't use them until after its first turn in the combat.

\subsection{A Legendary Creature's Lair}

A legendary creature might have a section describing its lair and the special effects it can create while there, either by act of will or simply by being present. Such a section applies only to a legendary creature that spends a great deal of time in its lair.

\subsubsection{Lair Actions}

If a legendary creature has lair actions, it can use them to harness the ambient magic in its lair. On initiative count 20 (losing all initiative ties), it can use one of its lair action options. It can't do so while incapacitated or otherwise unable to take actions. If surprised, it can't use one until after its first turn in the combat.

\subsubsection{Regional Effects}

The mere presence of a legendary creature can have strange and wondrous effects on its environment, as noted in this section. Regional effects end abruptly or dissipate over time when the legendary creature dies.

\section{Customizing Monsters}

There are many easy ways to customize the NPCs presented here for your home campaign.

\subparagraph*{Lineage and Cultural Traits} You can add lineage and cultural traits to an NPC (generally humanoids and giants, especially). Adding racial traits to an NPC doesn't alter its challenge rating. For more on lineage traits, see \nameref{ch:lineages}; for more on cultural traits, see \nameref{sec:cultures}.

\subparagraph*{Spell Swaps}. One way to customize an NPC spellcaster is to replace one or more of its spells. You can substitute any spell on the NPC's spell list with a different spell of the same cost from any list. Swapping spells in this manner can an NPC's challenge ratings.

\begin{figure}
	\begin{DndComment}{Non-damaging spells and abilities and challenge ratings}
		Damaging spells and abilities are fairly straightforward to apply to challenge rating. If they do more damage than the monster's other damaging actions, include them in the best-3-rounds model. Done. Non-damaging ones are more difficult. Some suggestions include:
		\begin{itemize}
			\item If a spell or ability does not deal damage, imposes no conditions on an enemy, and does not serve a defensive purpose (by healing, reducing damage, buffing the creature or its allies, etc.), then it likely has no effect on either challenge rating. For example, giving a monster access to the \nameref{spell:light} cantrip doesn't significantly change the combat difficulty.
			\item If a spell imposes negative conditions on an enemy, find a similar-targeted damaging spell with the same cost (or a monster ability from a similar-threat monster) and include it in the damage calculations. Conditions that dominate or completely disable an opponent (such as a banshee's wail) are particularly strong and will generally increase offensive rating significantly.
			\item If a spell or ability serves a defensive purpose, and especially if it does not require a full action, increase the monster's Defensive Rating by one. More may be warranted for particularly strong effects or those that cannot be overcome except by higher-level foes.
			\item It a spell has a long duration (ie > 10 minutes) and does not require concentration, assume that it has pre-cast that spell unless the monster is surprised. Include such things as \nameref{spell:mage-armor} directly in the stat block as a baseline.
        \end{itemize}
	\end{DndComment}
\end{figure}

\subparagraph*{Armor and Weapon Swaps} You can upgrade or downgrade an NPC's armor, or add or switch weapons. Adjustments to Armor Class and damage can change an NPC's challenge rating.

\subparagraph*{Magic Items}. The more powerful an NPC, the more likely it has one or more magic items in its possession. An archmage, for example, might have a magic staff or wand, as well as one or more potions and scrolls. Giving an NPC a potent damage-dealing magic item could alter its offensive challenge rating.

\section{Monster Math}
\subsection{Calculating Offensive Rating}
Offensive Rating is determined by the following process:
\begin{enumerate}
	\item Determine the most damaging three rounds the monster can do. If it has a recharge ability, assume that can be used once in those three rounds, or twice if it is Recharge (4-6) or more. Assume all attacks hit and deal average damage. Assume that abilities that hit an area (such as cones, lines, or circles), hit two creatures. If an ability requires a saving throw, assume that the affected targets fail. This is a worst-reasonable-case scenario, and should be treated as such. Include Legendary actions and damaging reactions, auras, etc.
	\item Calculate the average damage per round over those three rounds.
	\item Compare to the \nameref{tbl:offensive-rating} table and find the matching rating. Note the attack bonus and saving throw modifier listed there.
	\item Apply any listed adjustments. The most common is that if the creature's attack bonus is two or more points higher than the attack bonus for the listed row, the overall offensive rating should be adjusted up by one step for every two points higher the attack bonus is. Similarly, if the attack bonus is lower than the listed value by two or more, decrease the offensive rating by one step for every two points of difference. If most of the monster's damage comes from abilities that require saving throws, use that bonus instead of the attack bonus.
	\item Play test. Averages don't always tell the full story. If the monster has few attacks that each deal substantial damage, it will be more spiky (more liable to surprise-kill a character OR to fall flat and do nothing) than one that makes many smaller attacks. Higher number of dice do, however, also tend toward the average. 1d20 has a much broader distribution (being entirely flat) than 3d6+2, although the latter also has a higher floor (5 vs 1) and average (12.5 vs 10.5). The chances of rolling 20 damage on 1d20 is 1 in 20 (5\%); on 3d6+2 it's only 1 in 216 (0.46\%).
\end{enumerate}

The following table summarizes the relationship between Offensive Rating and damage output.

\begin{figure}
	\begin{DndTable}{llllll}
		\textbf{OR} & \textbf{Max Additional Bonus} & \textbf{Damage Range} & \textbf{Attack} & \textbf{DC} \\
		0 & 2 & 0-1 & +2 & 10 \\
		1/8 2 & & 2-3 & +2 & 10 \\
		1/4 & 2& 4-5 & +3 & 11 \\
		1/2 & 2& 6-7 & +3 & 11 \\
		1	& 2& 8-13 & +4 & 12 \\
		2	& 2& 14-19 & +4 & 12 \\
		3	& 2& 20-25 & +5 & 12 \\
		4 & 2 & 26-31 & +5 & 12 \\
		5 & 3 & 32-37 & +6 & 13 \\
		6 & 3 & 38-43 & +6 & 13 \\
		7 & 3 & 44-49 & +6 & 13 \\
		8 & 3 & 50-55 & +6 & 13 \\
		9 & 4 & 56-61 & +7 & 14 \\
		10 & 4 & 62-67 & +7 & 14 \\
		11 & 4 & 68-73 & +7 & 14 \\
		12 & 4& 74-79 & +7 & 14 \\
		13 & 5& 80-85 & +8 & 15 \\
		14 & 5& 86-91 & +8 & 15 \\
		15 & 5& 92-97 & +8 & 15 \\
		16 & 5& 98-103 & +8 & 15 \\
		17 & 6& 104-109 & +9 & 16 \\
		18 & 6& 110-115 & +9 & 16 \\
		19 & 6& 116-121 & +9 & 16 \\
		20 & 6& 122-127 & +9 & 16 \\
	\end{DndTable}
	\caption*{Offensive Rating Summary}
	\label{tbl:offensive-rating}
\end{figure}

\begin{figure}
	\begin{DndComment}
		Those who have played Dungeons and Dragons in its more recent incarnations may notice that the numbers on the rating tables are different--higher damage output at a given rating, but lower accuracy. This is intentional--it is much harder to get flat bonuses to AC without spending resources. The target accuracy is 65\% against a "normal" foe (neither defense focused nor glass cannon). That corresponds to a "normal" AC at level 20 of 16, which is where a mage-armored arcanist with maximum Dexterity would end up (at the weaker end), or below a plate-armored armsman (without a shield). It's totally normal for a character to have ACs ranging from 15-20 at least before they use resources. However, baseline D\&D had "soft" monsters, especially as defensive measures ramped up in later play. Which meant it required higher CRs to threaten people. Which caused undesirable results. The intent here is that combat should pose a threat even if the OR < level, as long as there are multiple monsters in play. Being forced to use your active defensive tools to absorb, avoid, or redirect incoming damage is totally normal and expected. You should not passively be able to ignore enemies whose OR is greater than half your level. And even ones below that can pose a threat in sufficient numbers.
	\end{DndComment}
\end{figure}

\subparagraph*{Modifiers to Offensive Rating}
The following are some considerations for adjusting offensive rating. These are not intended to be exhaustive, nor are they intended to be rules. As with everything homebrew, play-testing with the actual party involved is the final determiner.
\begin{enumerate}
	\item If the monster has an easy source of advantage on attacks such as Pack Tactics (advantage if an ally is within 5 feet of the target), increase their offensive rating by 1 after adjusting for accuracy.
	\item If the monster has a significant ability (especially one that deals substantial damage like a breath weapon or one that would automatically take out an opponent on a failed save such as a banshee's wail) that requires a saving throw your party is particularly weak at (such as a party with no one proficient in INT saves facing a mkhulu), increase the offensive rating by two.
	\item If the monster has a damage transfer ability, increase the actual damage by 50\% before calculating the offensive rating.
	\item Summoning abilities depend on the mechanics--if the monster is expected to start combat with its allies summoned or the allies do not disappear when the monster is defeated, treat them as separate combatants. If they require in-combat actions to summon and/or disappear when the summoner is defeated, increase the summoner's offensive rating by two.
\end{enumerate}

\subsection{Calculating Defensive Rating}

Defensive rating is based on the idea that a monster of Defensive Rating X should live for approximately 3 rounds against a baseline party of level X characters actively fighting it. This allows it the substantial chance of getting off its "Big Cool Abilities" at least once before being soundly defeated (the assumption is, after all, that the heroes win way more fights than they lose). As such, the primary factor in Defensive Rating is hit points. The underlying assumption is that monsters will generally be focus-fired, or at least need to survive being focus-fired by the entire party for some time.

\begin{figure}
	\begin{DndComment}{Expected PC Damage and accuracy}
		Not every action taken by player characters will be aimed at dealing damage. As a result, the system \textbf{does not} expect everyone to be fully-focused and optimized for damage dealing. Monsters are developed, and the defensive rating chart is built around the following principles:
		\begin{itemize}
			\item Player characters will generally hit between 60 and 70\% of the time (including critical hits). Deviations from this are very noticeable on the player side. Very few options other than advantage and disadvantage exist to modify attack rolls--this is on purpose. Monsters who have AC more than 1 away from the expected value from the charts should have clear narrative signals that they are either really easy or really hard to hit. For example, a knight wearing plate armor and wielding a shield signals that they're going to be a hard target, while an ooze with limited mobility is a signal of an easy-to-damage monster.
			\item Player characters will generally land between 50 and 80\% of the saving throw-based abilities they use. This one has much broader range because there are 6 ability scores. Generally, few monsters have really poor WIS saves, while DEX tends to be one of the lower saves. CON tends to be the hands-down best saving throw. The others are mixed--STR tends to be ok to great, INT is all over the place, and CHA is high for fiends and dragons but generally poor otherwise.
			\item After accounting for accuracy, a not-very-optimized \textit{party} is expected to deal on average 3x the damage of a similar-level rogue with a shortbow who gets sneak attack each round and hits (including critical hits) 70\% of the time. This rogue-based reference point is referred to as Rogue Equivalent Damage (RED). The variance in party damage is expected to be roughly between 2.5 RED and ~4-6 RED, depending on resource usage, targets, etc. This results in monsters generally surviving around 3 rounds, except at level 1 where a DR 1 monster survives roughly 2 rounds. Note that a damage-focused character normally does a baseline of ~1.2-1.5 RED (with spikes up to 2+ RED when burning resources) and a spell-caster using cantrips ends up around ~0.5 RED.
			\item Note from the table that there's an uptick in expected lifetime against a baseline party around defensive rating 12. That's because these monsters tend to be closer to "boss" monsters than normal scum. The expectation is that most significant encounters at high levels may have one "boss" monster and a bunch of smaller minions OR a larger swarm of smaller minions. A purely minion swarm at level 20 might look like 4-6 DR 10-11 monsters or 8-10 DR 6-8 monsters, while a "boss" encounter might be a single DR 17-20 and a 3-5 DR 6 or so monsters. Bounded accuracy ensures that these lower-DR monsters will still be a challenge.
		\end{itemize}
	\end{DndComment}
\end{figure}

To calculate Defensive Rating, follow the steps below:
\begin{itemize}
	\item calculate the base hit points of the monster, or pick an appropriate number from the \nameref{tbl:defensive-rating} table for the rating you approximately wish to end up with.
	\item Adjust the base hit points for regeneration and substantial defensive measures (such as damage transfer abilities) based on the considerations below. This produces the \textit{effective hit points} of the monster. 
	\item Compare the monster's AC to that on the line that matches the effective hit point total. If the listed AC is 2 or more points higher than the monster's AC, either change the AC or adjust the defensive CR down by one step for every 2 points difference. Similarly, if the monster's AC is two or more points higher than the table value, you can adjust the monster's AC to fit the table or adjust the defensive CR up by one step for every 2 points of difference.
	\item Play test. This is especially important if the monster has special abilities whose effect on their longevity is situational or isn't immediately apparent.
\end{itemize}

Once you've settled on a final hit point total, you can back-calculate the number of hit dice if you want, as long as you've defined the monster's ability scores. Generally, larger creatures have larger hit dice (but fewer of them), with Tiny creatures having d4 HD by default and Gargantuan creatures having d20 HD. The number of HD is given by $hit point total/(HD average + Constitution modifier)$, where the HD average is half of the maximum value on the die, plus 0.5 (e.g. a d4 has an average of 2.5). This will usually not come out cleanly--adjust the HP and HD until you're happy. As long as the total doesn't change to a different DR band, no other changes result.

Saving throw bonuses generally does not play into Defensive Rating much unless it has high saving throw modifiers in multiple of Constitution, Dexterity, and Wisdom. In that case, adjust defensive CR up by one if your party substantially relies on saving-throw based effects.

\begin{figure}
	\begin{DndTable}{llllll}
		\textbf{DR} & \textbf{Hit Points} & \textbf{AC} \\
		0 & 1-6 & 11 \\
		1/8 & 7-13 & 12 \\
		1/4 & 14-20 & 12 \\
		1/2 & 21-35 & 12 \\
		1	& 36-45 & 13 \\
		2   & 46-60 & 13 \\
		3	& 61-75 & 13 \\
		4   & 76-90 & 14 \\
		5   & 91-105 & 15 \\
		6   & 106-120 & 15 \\
		7   & 121-135 & 15 \\
		8   & 136-150 & 16 \\
		9   & 151-165 & 17 \\
		10  & 166-180 & 17 \\
		11  & 181-195 & 17 \\
		12  & 196-220 & 17 \\
		13  & 221-245 & 17 \\
		14  & 246-270 & 18 \\
		15  & 271-295 & 18 \\
		16  & 296-320 & 18 \\
		17  & 321-345 & 19 \\
		18  & 346-370 & 19 \\
		19  & 371-395 & 19 \\
		20  & 396-420 & 19 \\
	\end{DndTable}
	\caption*{Defensive Rating Summary}
	\label{tbl:defensive-rating}
\end{figure}

\begin{figure}
	\begin{DndTable}{ll}
		\textbf{Level} & \textbf{Party Baseline DPR} \\
		1 & 22 \\
		2 & 22 \\
		3 & 30 \\
		4 & 32 \\
		5 & 47 \\
		6 & 47 \\
		7 & 56 \\
		8 & 58 \\
		9 & 67 \\
		10 & 67 \\
		11 & 88 \\
		12 & 88 \\
		13 & 98 \\
		14 & 98 \\
		15 & 109 \\
		16 & 109 \\
		17 & 119 \\
		18 & 119 \\
		19 & 130 \\
		20 & 130
	\end{DndTable}
	\caption*{Party Expected DPR}
\end{figure}


\onecolumn
\import{./}{aberrations.tex}
\import{./}{beasts.tex}
\import{./}{celestials.tex}
\import{./}{constructs.tex}
\import{./}{dragons.tex}
\import{./}{elementals.tex}
\import{./}{fey.tex}
\import{./}{fiends.tex}
\import{./}{giants.tex}
\import{./}{humanoids.tex}
\import{./}{monstrosities.tex}
\import{./}{oozes.tex}
\import{./}{plants.tex}
\import{./}{undead.tex}
\twocolumn
