\chapter{Using Ability Scores}
\label{ch:using-ability-scores}
Six abilities provide a quick description of every creature's physical and mental characteristics:

\begin{itemize}
\item \textbf{Strength}, measuring physical power
\item \textbf{Dexterity}, measuring agility
\item \textbf{Constitution}, measuring endurance
\item \textbf{Intelligence}, measuring reasoning and memory
\item \textbf{Wisdom}, measuring perception and insight
\item \textbf{Charisma}, measuring force of personality
\end{itemize}

Is a character muscle-bound and insightful? Brilliant and charming? Nimble and hardy? Ability scores define these qualities—a creature's assets as well as weaknesses.

The three main rolls of the game—the ability check, the saving throw, and the attack roll—rely on the six ability scores. The book's introduction describes the basic rule behind these rolls: roll a d20, add an ability modifier derived from one of the six ability scores, and compare the total to a target number.

\section{Ability Scores and Modifiers}

Each of a creature's abilities has a score, a number that defines the magnitude of that ability. An ability score is not just a measure of innate capabilities, but also encompasses a creature's training and competence in activities related to that ability. In essence, an ability score is a measure of aptitude in a given approach to life. Those with high Strength are good at things that require forceful, direct physical approaches. Those with high Dexterity are nimble, quick to react, and good at things requiring a deft, subtle, or precise physical approach. Those with high Constitution are tough and hearty, good at things that benefit from endurance and resilience. Those with high Intelligence are quick of thought and knowledgeable, able to put the pieces together quickly and accurately. Those with high Wisdom are perceptive and "in tune" with events around them; they are good at understanding people and animals as well as more esoteric things like spirits. Those with high Charisma are forceful of personality, able to take charge and have people listen to them; they also often have the magical force of will to compel the universe to bend to them, even if only slightly.

For player characters, the score can range between -5 and +5, with 0 representing the human average in that area. Some powerful monsters such as giants, dragons, and beings of the planes might have ability scores above +5; no ability score can be greater than +10.

Ability scores (also called "modifiers") are added to ability checks, saving throws, attack rolls, and weapon damage rolls.

\section{Advantage and Disadvantage} \label{sec:advantage-and-disadvantage}

Sometimes a special ability or spell tells you that you have advantage or disadvantage on an ability check, a saving throw, or an attack roll. When that happens, you roll a second d20 when you make the roll. Use the higher of the two rolls if you have advantage, and use the lower roll if you have disadvantage. For example, if you have disadvantage and roll a 17 and a 5, you use the 5. If you instead have advantage and roll those numbers, you use the 17.

If multiple situations affect a roll and each one grants advantage or imposes disadvantage on it, you don't roll more than one additional d20. If two favorable situations grant advantage, for example, you still roll only one additional d20.

If circumstances cause a roll to have both advantage and disadvantage, you are considered to have neither of them, and you roll one d20. This is true even if multiple circumstances impose disadvantage and only one grants advantage or vice versa. In such a situation, you have neither advantage nor disadvantage.

When you have advantage or disadvantage and something in the game, such as the halfling's Lucky trait, lets you reroll the d20, you can reroll only one of the dice. You choose which one. For example, if a halfling has advantage or disadvantage on an ability check and rolls a 1 and a 13, the halfling could use the Lucky trait to reroll the 1.

You usually gain advantage or disadvantage through the use of special abilities, actions, or spells. Inspiration can also give a character advantage. The

GM can also decide that circumstances influence a roll in one direction or the other and grant advantage or impose disadvantage as a result.

\section{Proficiency Bonus}

Characters have a proficiency bonus determined by level. Monsters also have this bonus, which is incorporated in their stat blocks. The bonus is used in the rules on ability checks, saving throws, and attack rolls.

Your proficiency bonus can't be added to a single die roll or other number more than once. For example, if two different rules say you can add your proficiency bonus to a Wisdom saving throw, you nevertheless add the bonus only once when you make the save.

Occasionally, your proficiency bonus might be multiplied or divided (doubled or halved, for example) before you apply it. For example, the rogue's Expertise feature doubles the proficiency bonus for certain ability checks. If a circumstance suggests that your proficiency bonus applies more than once to the same roll, you still add it only once and multiply or divide it only once unless the feature specifically says otherwise (such as the Exert and Focus special actions).

By the same token, if a feature or effect allows you to multiply your proficiency bonus when making an ability check that wouldn't normally benefit from your proficiency bonus, you still don't add the bonus to the check. For that check your proficiency bonus is 0, given the fact that multiplying 0 by any number is still 0. For instance, if you lack proficiency in the History skill, you gain no benefit from a feature that lets you double your proficiency bonus when you make Intelligence (History) checks.

In general, you don't multiply your proficiency bonus for attack rolls or saving throws. If a feature or effect allows you to do so, these same rules apply.

\section{Stamina (STA) and Aether (AET)}\label{sec:stamina-and-aether}
Player characters have reserves of extraordinary stamina and magic (called aether). They can expend these reserves to accomplish feats beyond their norm. The size of the pools depends on your class–physically-oriented classes tend to have (and use) more stamina while spell-casting classes tend to have (and use) more aether. Classes that do both tend to have more evenly balanced pools. Every character has access to a few common uses; classes, lineages, and feats may grant extra ways.

\textbf{Deflect\label{action:deflect}}:  When you are targeted by an attack while you are wearing armor or are affected by mage armor, you can use your reaction and spend 2 STA to add your proficiency bonus to your AC against that attack. You must use this reaction before you know the outcome of the attack, but if you take this action and the attack misses, you can immediately make either a melee attack against them with a weapon or shield you are wielding or attempt to Shove them as part of the reaction. If you attack them with a shield, it counts as a melee weapon with a 1d4 damage die for that attack. If you do not have a weapon or shield in hand, you cannot make this special attack.

\textbf{Exert}: By spending 1 STA when you make an ability check that uses Strength, Dexterity, or Constitution, you can add your proficiency bonus to the check even if you are already adding your proficiency bonus or a multiple of your proficiency bonus to that check.

\textbf{Focus}: By spending 1 AET when you make an ability check that uses Intelligence, Wisdom, or Charisma, you can add your proficiency bonus to the check even if you are already adding your proficiency bonus or a multiple of your proficiency bonus to that check.

\section{Ability Checks}

An ability check tests a character's or monster's innate talent and training in an effort to overcome a challenge. The GM calls for an ability check when a character or monster attempts an action (other than an attack) that has a chance of failure. When the outcome is uncertain, the dice determine the results.

For every ability check, the GM decides which of the six abilities is relevant to the task at hand and the difficulty of the task, represented by a Difficulty Class.

The more difficult a task, the higher its DC. The Typical Difficulty Classes table shows the most common DCs. As a note, the names are calibrated around someone with a +4 bonus in the required ability score (possibly including proficiency).

\begin{DndTable}[header=Typical Difficulty Classes]{XX}
	\textbf{Task Difficulty} & \textbf{DC} \\
	Very easy & 5 \\
	Easy & 10 \\
	Medium & 15 \\
	Hard & 20 \\
	Very hard & 25 \\
	Nearly impossible & 30 \\
\end{DndTable}

To make an ability check, roll a d20 and add the relevant ability modifier. As with other d20 rolls, apply bonuses and penalties, and compare the total to the DC. If the total equals or exceeds the DC, the ability check is a success—the creature overcomes the challenge at hand. Otherwise, it's a failure, which means the character or monster makes no progress toward the objective or makes progress combined with a setback determined by the GM.

\subsection{Contests}

Sometimes one character's or monster's efforts are directly opposed to another's. This can occur when both of them are trying to do the same thing and only one can succeed, such as attempting to snatch up a magic ring that has fallen on the floor. This situation also applies when one of them is trying to prevent the other one from accomplishing a goal— for example, when a monster tries to force open a door that an adventurer is holding closed. In situations like these, the outcome is determined by a special form of ability check, called a contest.

Both participants in a contest make ability checks appropriate to their efforts. They apply all appropriate bonuses and penalties, but instead of comparing the total to a DC, they compare the totals of their two checks. The participant with the higher check total wins the contest. That character or monster either succeeds at the action or prevents the other one from succeeding.

If the contest results in a tie, the situation remains the same as it was before the contest. Thus, one contestant might win the contest by default. If two characters tie in a contest to snatch a ring off the floor, neither character grabs it. In a contest between a monster trying to open a door and an adventurer trying to keep the door closed, a tie means that the door remains shut.

\subsection{Skills}

Each ability covers a broad range of capabilities, including skills that a character or a monster can be proficient in. A skill represents a specific aspect of an ability score, and an individual's proficiency in a skill demonstrates a focus on that aspect. (A character's starting skill proficiencies are determined at character creation, and a monster's skill proficiencies appear in the monster's stat block.)

For example, a Dexterity check might reflect a character's attempt to pull off an acrobatic stunt, to palm an object, or to stay hidden. Each of these aspects of Dexterity has an associated skill: Acrobatics, Sleight of Hand, and Stealth, respectively. So a character who has proficiency in the Stealth skill is particularly good at Dexterity checks related to sneaking and hiding.

The skills related to each ability score are shown in the following list. (No skills are related to Constitution.) See an ability's description in the later sections of this section for examples of how to use a skill associated with an ability.

\subsubsection{Strength}

\begin{itemize}
\item Athletics
\end{itemize}

\subsubsection{Dexterity}

\begin{itemize}
\item Acrobatics
\item Sleight of Hand
\item Stealth
\end{itemize}

\subsubsection{Intelligence}

\begin{itemize}
\item Arcana
\item History
\item Investigation
\item Nature
\item Religion
\end{itemize}

\subsubsection{Wisdom}

\begin{itemize}
\item Animal Handling
\item Insight
\item Medicine
\item Perception
\item Survival
\end{itemize}

\subsubsection{Charisma}

\begin{itemize}
\item Deception
\item Intimidation
\item Performance
\item Persuasion
\end{itemize}

Sometimes, the GM might ask for an ability check using a specific skill—for example, “Make a Wisdom (Perception) check.” At other times, a player might ask the GM if proficiency in a particular skill applies to a check. In either case, proficiency in a skill means an individual can add his or her proficiency bonus to ability checks that involve that skill. Without proficiency in the skill, the individual makes a normal ability check.

For example, if a character attempts to climb up a dangerous cliff, the GM might ask for a Strength (Athletics) check. If the character is proficient in Athletics, the character's proficiency bonus is added to the Strength check. If the character lacks that proficiency, he or she just makes a Strength check.

\subsubsection{Variant: Skills with Different Abilities}

Normally, your proficiency in a skill applies only to a specific kind of ability check. Proficiency in Athletics, for example, usually applies to Strength checks. In some situations, though, your proficiency might reasonably apply to a different kind of check. In such cases, the GM might ask for a check using an unusual combination of ability and skill, or you might ask your GM if you can apply a proficiency to a different check. For example, if you have to swim from an offshore island to the mainland, your GM might call for a Constitution check to see if you have the stamina to make it that far. In this case, your GM might allow you to apply your proficiency in Athletics and ask for a Constitution (Athletics) check. So if you're proficient in Athletics, you apply your proficiency bonus to the Constitution check just as you would normally do for a Strength (Athletics) check. Similarly, when your half-orc warden uses a display of raw strength to intimidate an enemy, your GM might ask for a Strength (Intimidation) check, even though Intimidation is normally associated with Charisma.

\subsection{Passive Checks}

A passive check is a special kind of ability check that doesn't involve any die rolls. Such a check can represent the average result for a task done repeatedly, such as searching for secret doors over and over again, or can be used when the GM wants to secretly determine whether the characters succeed at something without rolling dice, such as noticing a hidden monster.

Here's how to determine a character's total for a passive check:

\textbf{10 + all modifiers that normally apply to the check}

If the character has advantage on the check, add 5. For disadvantage, subtract 5. The game refers to a passive check total as a \textbf{score}.

For example, if a 1st-level character has a Wisdom of 15 and proficiency in Perception, he or she has a passive Wisdom (Perception) score of 14.

The rules on hiding in the “Dexterity” section below rely on passive checks, as do the exploration rules.

\subsection{Working Together}

Sometimes two or more characters team up to attempt a task. The character who's leading the effort—or the one with the highest ability modifier—can make an ability check with advantage, reflecting the help provided by the other characters. In combat, this requires the Help action.

A character can only provide help if the task is one that he or she could attempt alone. For example, trying to open a lock requires proficiency with thieves' tools, so a character who lacks that proficiency can't help another character in that task. Moreover, a character can help only when two or more individuals working together would actually be productive. Some tasks, such as threading a needle, are no easier with help.

\subsubsection{Group Checks}

When a number of individuals are trying to accomplish something as a group, the GM might ask for a group ability check. In such a situation, the characters who are skilled at a particular task help cover those who aren't.

To make a group ability check, everyone in the group makes the ability check. If at least half the group succeeds, the whole group succeeds. Otherwise, the group fails.

Group checks don't come up very often, and they're most useful when all the characters succeed or fail as a group. For example, when adventurers are navigating a swamp, the GM might call for a group Wisdom (Survival) check to see if the characters can avoid the quicksand, sinkholes, and other natural hazards of the environment. If at least half the group succeeds, the successful characters are able to guide their companions out of danger. Otherwise, the group stumbles into one of these hazards.

\section{Using Each Ability}

Every task that a character or monster might attempt in the game is covered by one of the six abilities. This section explains in more detail what those abilities mean and the ways they are used in the game.

\subsection{Strength}

Strength measures bodily power, athletic training, and the extent to which you can exert raw physical force.

\subsubsection{Strength Checks}

A Strength check can model any attempt to lift, push, pull, or break something, to force your body through a space, or to otherwise apply brute force to a situation. The Athletics skill reflects aptitude in certain kinds of Strength checks.

\subparagraph*{Athletics} Your Strength (Athletics) check covers difficult situations you encounter while climbing, jumping, or swimming. Examples include the following activities:

\begin{itemize}
\item You attempt to climb a sheer or slippery cliff, avoid hazards while scaling a wall, or cling to a surface while something is trying to knock you off.
\item You try to jump an unusually long distance or pull off a stunt midjump.
\item You struggle to swim or stay afloat in treacherous currents, storm-tossed waves, or areas of thick seaweed. Or another creature tries to push or pull you underwater or otherwise interfere with your swimming.
\end{itemize}

\subparagraph*{Other Strength Checks} The GM might also call for a Strength check when you try to accomplish tasks like the following:

\begin{itemize}
\item Force open a stuck, locked, or barred door
\item Break free of bonds
\item Push through a tunnel that is too small
\item Hang on to a wagon while being dragged behind it
\item Tip over a statue
\item Keep a boulder from rolling
\end{itemize}

\subsubsection{Attack Rolls and Damage}

You add your Strength modifier to your attack roll and your damage roll when attacking with a melee weapon such as a mace, a battleaxe, or a javelin. You use melee weapons to make melee attacks in hand* to-hand combat, and some of them can be thrown to make a ranged attack.

\subsubsection{Lifting and Carrying}

Your Strength score determines the amount of weight you can bear. The following terms define what you can lift or carry.

\subparagraph*{Carrying Capacity} Your carrying capacity is 150 + your Strength score multiplied by 30. This is the weight (in pounds) that you can carry, which is high enough that most characters don't usually have to worry about it.

\subparagraph*{Push, Drag, or Lift} You can push, drag, or lift a weight in pounds up to twice your carrying capacity (or 300 lbs + 60 times your Strength score). While pushing or dragging weight in excess of your carrying capacity, your speed drops to 5 feet.

\subparagraph*{Size and Strength} Larger creatures can bear more weight, whereas Tiny creatures can carry less. For each size category above Medium, double the creature's carrying capacity and the amount it can push, drag, or lift. For a Tiny creature, halve these weights.

\subsection{Dexterity}

Dexterity measures agility, reflexes, and balance.

\subsubsection{Dexterity Checks}

A Dexterity check can model any attempt to move nimbly, quickly, or quietly, or to keep from falling on tricky footing. The Acrobatics, Sleight of Hand, and Stealth skills reflect aptitude in certain kinds of Dexterity checks.

\subparagraph*{Acrobatics} Your Dexterity (Acrobatics) check covers your attempt to stay on your feet in a tricky situation, such as when you're trying to run across a sheet of ice, balance on a tightrope, or stay upright on a rocking ship's deck. The GM might also call for a Dexterity (Acrobatics) check to see if you can perform acrobatic stunts, including dives, rolls, somersaults, and flips.

\subparagraph*{Sleight of Hand} Whenever you attempt an act of legerdemain or manual trickery, such as planting something on someone else or concealing an object on your person, make a Dexterity (Sleight of Hand) check. The GM might also call for a Dexterity (Sleight of Hand) check to determine whether you can lift a coin purse off another person or slip something out of another person's pocket.

\subparagraph*{Stealth} Make a Dexterity (Stealth) check when you attempt to conceal yourself from enemies, slink past guards, slip away without being noticed, or sneak up on someone without being seen or heard.

\subparagraph*{Other Dexterity Checks} The GM might call for a Dexterity check when you try to accomplish tasks like the following:

\begin{itemize}
\item Control a heavily laden cart on a steep descent
\item Steer a chariot around a tight turn
\item Pick a lock
\item Disable a trap
\item Securely tie up a prisoner
\item Wriggle free of bonds
\item Play a stringed instrument
\item Craft a small or detailed object
\end{itemize}

\subsubsection{Attack Rolls and Damage}

You add your Dexterity modifier to your attack roll and your damage roll when attacking with a ranged weapon, such as a sling or a longbow. You can also add your Dexterity modifier to your attack roll and your damage roll when attacking with a melee weapon that has the finesse property, such as a dagger or a rapier.

\subsubsection{Armor Class}

Depending on the armor you wear, you might add some or all of your Dexterity modifier to your Armor Class.

\subsubsection{Initiative} \label{sec:initiative}

At the beginning of every combat, you roll initiative by making a Dexterity check. Initiative determines the order of creatures' turns in combat.

\begin{figure}[htb]
    \begin{DndSidebar}{Hiding}
The DM decides when circumstances are appropriate for hiding. When you try to hide, make a Dexterity (Stealth) check. Until you are discovered or you stop hiding, that check's total is contested by the Wisdom (Perception) check of any creature that actively searches for signs of your presence.

You can't hide from a creature unless you have \textbf{total concealment} from that creature, and you give away your position if you make noise, such as shouting a warning or knocking over a vase.

An invisible creature can always try to hide. Signs of its passage might still be noticed, and it does have to stay quiet.

In combat, most creatures stay alert for signs of danger all around, so if you come out of hiding and approach a creature, it usually sees you. However, under certain circumstances, the GM might allow you to stay hidden as you approach a creature that is distracted, allowing you to gain advantage on an attack roll before you are seen. Once a creature has been seen, most creatures (other than near-mindless ones such as oozes) will remember that they exist even if the creature successfully hides again. Thus, once you've been seen or the alarm has been raised, alert guards will not be \nameref{condition:surprised} by you even if you successfully hide. This alarm can fade if you spend enough time out of sight---the exact details are up to the GM.

\subparagraph*{Passive Perception.} When you hide, there's a chance someone will notice you even if they aren't searching. To determine whether such a creature notices you, the DM compares your Dexterity (Stealth) check with that creature's passive Wisdom (Perception) score, which equals 10 + the creature's Wisdom modifier, as well as any other bonuses or penalties. If the creature has advantage, add 5. For disadvantage, subtract 5. For example, if a 1st-level character (with a proficiency bonus of +2) has a Wisdom of +2 modifier and proficiency in Perception, he or she has a passive Wisdom (Perception) of 14.

\subparagraph*{What Can You See?} One of the main factors in determining whether you can find a hidden creature or object is how well you can see in an area, which might be \textbf{lightly} or \textbf{heavily obscured}, as explained in chapter 8, “Adventuring.”
\end{DndSidebar}
\end{figure}
\subsection{Constitution}

Constitution measures health, stamina, and vital force.

\subsubsection{Constitution Checks}

Constitution checks are uncommon, and no skills apply to Constitution checks, because the endurance this ability represents is largely passive rather than involving a specific effort on the part of a character or monster. A Constitution check can model your attempt to push beyond normal limits, however.

The GM might call for a Constitution check when you try to accomplish tasks like the following:

\begin{itemize}
\item Hold your breath
\item March or labor for hours without rest
\item Go without sleep
\item Survive without food or water
\item Quaff an entire stein of ale in one go
\end{itemize}

\subsubsection{Hit Points}

Your Constitution modifier contributes to your hit points. Typically, you add your Constitution modifier to each Hit Die you roll for your hit points.

If your Constitution modifier changes, your hit point maximum changes as well, as though you had the new modifier from 1st level. For example, if you raise your Constitution score when you reach 4th level and your Constitution modifier increases from +1 to +2, you adjust your hit point maximum as though the modifier had always been +2. So you add 3 hit points for your first three levels, and then roll your hit points for 4th level using your new modifier. Or if you're 7th level and some effect lowers your Constitution score so as to reduce your Constitution modifier by 1, your hit point maximum is reduced by 7.

\subsection{Intelligence}

Intelligence measures mental acuity, accuracy of recall, and the ability to reason.

\subsubsection{Intelligence Checks}

An Intelligence check comes into play when you need to draw on logic, education, memory, or deductive reasoning. The Arcana, History, Investigation, Nature, and Religion skills reflect aptitude in certain kinds of Intelligence checks.

\subparagraph*{Arcana} Your Intelligence (Arcana) check measures your ability to recall lore about spells, magic items, eldritch symbols, magical traditions, the planes of existence, and the inhabitants of those planes.

\subparagraph*{History} Your Intelligence (History) check measures your ability to recall lore about historical events, legendary people, ancient kingdoms, past disputes, recent wars, and lost civilizations.

\subparagraph*{Investigation} When you look around for clues and make deductions based on those clues, you make an Intelligence (Investigation) check. You might deduce the location of a hidden object, discern from the appearance of a wound what kind of weapon dealt it, or determine the weakest point in a tunnel that could cause it to collapse. Poring through ancient scrolls in search of a hidden fragment of knowledge might also call for an Intelligence (Investigation) check.

\subparagraph*{Nature} Your Intelligence (Nature) check measures your ability to recall lore about terrain, plants and animals, the weather, and natural cycles.

\subparagraph*{Religion} Your Intelligence (Religion) check measures your ability to recall lore about deities, rites and prayers, religious hierarchies, holy symbols, and the practices of secret cults.

\subparagraph*{Other Intelligence Checks} The GM might call for an Intelligence check when you try to accomplish tasks like the following:

\begin{itemize}
\item Communicate with a creature without using words
\item Estimate the value of a precious item
\item Pull together a disguise to pass as a city guard
\item Forge a document
\item Recall lore about a craft or trade
\item Win a game of skill
\end{itemize}

\subsubsection{Spellcasting Ability}

Arcanists use Intelligence as their spellcasting ability, which helps determine the saving throw DCs of spells they cast.

\subsection{Wisdom}

Wisdom reflects how attuned you are to the world around you and represents perceptiveness and intuition.

\subsubsection{Wisdom Checks}

A Wisdom check might reflect an effort to read body language, understand someone's feelings, notice things about the environment, or care for an injured person. The Animal Handling, Insight, Medicine, Perception, and Survival skills reflect aptitude in certain kinds of Wisdom checks.

\subparagraph*{Animal Handling} When there is any question whether you can calm down a domesticated animal, keep a mount from getting spooked, or intuit an animal's intentions, the GM might call for a Wisdom (Animal Handling) check. You also make a Wisdom (Animal Handling) check to control your mount when you attempt a risky maneuver.

\subparagraph*{Insight} Your Wisdom (Insight) check decides whether you can determine the true intentions of a creature, such as when searching out a lie or predicting someone's next move. Doing so involves gleaning clues from body language, speech habits, and changes in mannerisms.

\subparagraph*{Medicine} A Wisdom (Medicine) check lets you try to stabilize a dying companion or diagnose an illness.

\subparagraph*{Perception} Your Wisdom (Perception) check lets you spot, hear, or otherwise detect the presence of something. It measures your general awareness of your surroundings and the keenness of your senses. For example, you might try to hear a conversation through a closed door, eavesdrop under an open window, or hear monsters moving stealthily in the forest. Or you might try to spot things that are obscured or easy to miss, whether they are orcs lying in ambush on a road, thugs hiding in the shadows of an alley, or candlelight under a closed secret door.

\subparagraph*{Survival} The GM might ask you to make a

Wisdom (Survival) check to follow tracks, hunt wild game, guide your group through frozen wastelands, identify signs that owlbears live nearby, predict the weather, or avoid quicksand and other natural hazards.

\subparagraph*{Other Wisdom Checks} The GM might call for a

Wisdom check when you try to accomplish tasks like the following:

\begin{itemize}
    \item Get a gut feeling about what course of action to follow
    \item Discern whether a seemingly dead or living creature is undead
\end{itemize}

\subsubsection{Spellcasting Ability}

Priests, rangers and shamans use Wisdom as their spellcasting ability, which helps determine the saving throw DCs of spells they cast.

\subsection{Charisma}

Charisma measures your ability to interact effectively with others. It includes such factors as confidence and eloquence, and it can represent a charming or commanding personality.

\subsubsection{Charisma Checks}

A Charisma check might arise when you try to influence or entertain others, when you try to make an impression or tell a convincing lie, or when you are navigating a tricky social situation. The Deception, Intimidation, Performance, and Persuasion skills reflect aptitude in certain kinds of Charisma checks.

\subparagraph*{Deception} Your Charisma (Deception) check determines whether you can convincingly hide the truth, either verbally or through your actions. This deception can encompass everything from misleading others through ambiguity to telling outright lies. Typical situations include trying to fast* talk a guard, con a merchant, earn money through gambling, pass yourself off in a disguise, dull someone's suspicions with false assurances, or maintain a straight face while telling a blatant lie.

\subparagraph*{Intimidation} When you attempt to influence someone through overt threats, hostile actions, and physical violence, the GM might ask you to make a Charisma (Intimidation) check. Examples include trying to pry information out of a prisoner, convincing street thugs to back down from a confrontation, or using the edge of a broken bottle to convince a sneering vizier to reconsider a decision.

\subparagraph*{Performance} Your Charisma (Performance) check determines how well you can delight or influence an audience with music, dance, acting, storytelling, or some other form of entertainment.

\subparagraph*{Persuasion} When you attempt to influence someone or a group of people with tact, social graces, or good nature, the GM might ask you to make a Charisma (Persuasion) check. Typically, you use persuasion when acting in good faith, to foster friendships, make cordial requests, or exhibit proper etiquette. Examples of persuading others include convincing a chamberlain to let your party see the king, negotiating peace between warring tribes, or inspiring a crowd of townsfolk.

\subparagraph*{Other Charisma Checks} The GM might call for a Charisma check when you try to accomplish tasks like the following:

\begin{itemize}
    \item Find the best person to talk to for news, rumors, and gossip
    \item Blend into a crowd to get the sense of key topics of conversation
\end{itemize}

\subsubsection{Spellcasting Ability}

Oathbound, spellblades, and warlocks use Charisma as their spellcasting ability, which helps determine the saving throw DCs of spells they cast.

\section{Saving Throws}
A saving throw\textemdash also called a save\textemdash represents an attempt to resist a spell, a trap, a poison, a disease, or a similar threat. You don't normally decide to make a saving throw; you are forced to make one because your character or monster is at risk of harm.

To make a saving throw, roll a d20 and add the appropriate ability modifier. For example, you use your Dexterity modifier for a Dexterity saving throw.

A saving throw can be modified by a situational bonus or penalty and can be affected by advantage and disadvantage, as determined by the GM.

Each class gives proficiency in at least two saving throws. The arcanist, for example, is proficient in Intelligence saves. As with skill proficiencies, proficiency in a saving throw lets a character add his or her proficiency bonus to saving throws made using a particular ability score. Some monsters have saving throw proficiencies as well.

The Difficulty Class for a saving throw is determined by the effect that causes it. For example, the DC for a saving throw allowed by a spell is determined by the caster's spellcasting ability and proficiency bonus.

The result of a successful or failed saving throw is also detailed in the effect that allows the save. Usually, a successful save means that a creature suffers no harm, or reduced harm, from an effect.

\section{Time}

In situations where keeping track of the passage of time is important, the GM determines the time a task requires. The GM might use a different time scale depending on the context of the situation at hand. In a dungeon environment, the adventurers' movement happens on a scale of \textbf{minutes}. It takes them about a minute to creep down a long hallway, another minute to check for traps on the door at the end of the hall, and a good ten minutes to search the chamber beyond for anything interesting or valuable.

In a city or wilderness, a scale of \textbf{hours} is often more appropriate. Adventurers eager to reach the lonely tower at the heart of the forest hurry across those fifteen miles in just under four hours' time.

For long journeys, a scale of \textbf{days} works best.

Following the road from Rauviz to Crisial City, the adventurers spend four uneventful days before a goblin ambush interrupts their journey.

In combat and other fast-paced situations, the game relies on \textbf{rounds}, a 6-second span of time.

\section{Movement}

Swimming across a rushing river, sneaking down a dungeon corridor, scaling a treacherous mountain slope—all sorts of movement play a key role in fantasy gaming adventures.

The GM can summarize the adventurers' movement without calculating exact distances or travel times: “You travel through the forest and find the dungeon entrance late in the evening of the third day.” Even in a dungeon, particularly a large dungeon or a cave network, the GM can summarize movement between encounters: “After killing the guardian at the entrance to the ancient dwarven stronghold, you consult your map, which leads you through miles of echoing corridors to a chasm bridged by a narrow stone arch.”

Sometimes it's important, though, to know how long it takes to get from one spot to another, whether the answer is in days, hours, or minutes. The rules for determining travel time depend on two factors: the speed and travel pace of the creatures moving and the terrain they're moving over.

\subsection{Speed}

Every character and monster has a speed, which is the distance in feet that the character or monster can walk in 1 round. This number assumes short bursts of energetic movement in the midst of a life* threatening situation.

The following rules determine how far a character or monster can move in a minute, an hour, or a day.

\subsubsection{Travel Pace}

While traveling, a group of adventurers can move at a normal, fast, or slow pace, as shown on the Travel Pace table. The table states how far the party can move in a period of time and whether the pace has any effect. A fast pace makes characters less perceptive, while a slow pace makes it possible to sneak around and to search an area more carefully.

\subparagraph*{Forced March} The Travel Pace table assumes that characters travel for 8 hours in day. They can push on beyond that limit, at the risk of exhaustion.

For each additional hour of travel beyond 8 hours, the characters cover the distance shown in the Hour column for their pace, and each character must make a Constitution saving throw at the end of the hour. The DC is 10 + 1 for each hour past 8 hours. On a failed saving throw, a character suffers one level of exhaustion (see appendix A).

\subparagraph*{Mounts and Vehicles} For short spans of time (up to an hour), many animals move much faster than humanoids. A mounted character can ride at a gallop for about an hour, covering twice the usual distance for a fast pace. If fresh mounts are available every 8 to 10 miles, characters can cover larger distances at this pace, but this is very rare except in densely populated areas.

Characters in wagons, carriages, or other land vehicles choose a pace as normal. Characters in a waterborne vessel are limited to the speed of the vessel, and they don't suffer penalties for a fast pace or gain benefits from a slow pace. Depending on the vessel and the size of the crew, ships might be able to travel for up to 24 hours per day.

Certain special mounts, such as a pegasus or griffon, or special vehicles, such as a \textit{carpet of flying}, allow you to travel more swiftly.

\begin{figure}
    \begin{DndTable}[header=Travel Pace]{XXXXX}
        \textbf{Pace} & \textbf{Distance per: Minute} & \textbf{Hour} & \textbf{Day} & \textbf{Effect} \\
        Fast & 400 feet & 4 miles & 32 miles & $-$5 penalty to passive Wisdom (Perception) scores \\
        Normal & 300 feet & 3 miles & 24 miles & -- \\
        Slow & 200 feet & 2 miles & 16 miles & Able to use stealth \\
    \end{DndTable}
\end{figure}

\subsubsection{Difficult Terrain}

The travel speeds given in the Travel Pace table assume relatively simple terrain: roads, open plains, or clear dungeon corridors. But adventurers often face dense forests, deep swamps, rubble-filled ruins, steep mountains, and ice-covered ground—all considered difficult terrain.

You move at half speed in difficult terrain---moving 1 foot in difficult terrain costs 2 feet of speed---so you can cover only half the normal distance in a minute, an hour, or a day.

\subsection{Special Types of Movement}

Movement through dangerous dungeons or wilderness areas often involves more than simply walking. Adventurers might have to climb, crawl, swim, or jump to get where they need to go.

\subsubsection{Climbing, Swimming, and Crawling}

While climbing or swimming, each foot of movement costs 1 extra foot (2 extra feet in difficult terrain), unless a creature has a climbing or swimming speed. At the GM's option, climbing an exceptionally slippery vertical surface or one with very few handholds requires a successful Strength (Athletics) check (usually DC 10, but higher for exceptional surfaces). Similarly, gaining any distance in rough water might require a successful Strength (Athletics) check.

\subsubsection{Jumping}

Your Strength determines how far you can jump. To jump further than you can automatically (as described below), make a Strength (Athletics) check. If your total is 10 or higher, you can jump 5 additional feet, with 1 additional foot for every point higher than 10 you rolled. On a 10 or lower, you do not lose any distance but just cannot jump further than your automatic distance.

\subparagraph*{Long Jump} When you make a long jump, you cover a number of feet up to 10 + twice your Strength score if you move at least 10 feet on foot immediately before the jump. When you make a standing long jump, you can leap only half that distance. Either way, each foot you clear on the jump costs a foot of movement.

This rule assumes that the height of your jump doesn't matter, such as a jump across a stream or chasm. At your GM's option, you must succeed on a DC 10 Strength (Athletics) check to clear a low obstacle (no taller than a quarter of the jump's distance), such as a hedge or low wall. Otherwise, you hit it.

When you land in difficult terrain, you must succeed on a DC 10 Dexterity (Acrobatics) check to land on your feet. Otherwise, you land prone.

\subparagraph*{High Jump} When you make a high jump, you leap into the air a number of feet equal to 3 + your Strength modifier if you move at least 10 feet on foot immediately before the jump. When you make a standing high jump, you can jump only half that distance. Either way, each foot you clear on the jump costs a foot of movement. In some circumstances, your GM might allow you to make a Strength (Athletics) check to jump higher than you normally can.

You can extend your arms half your height above yourself during the jump. Thus, you can reach above you a distance equal to the height of the jump plus 1½ times your height.

\section{The Environment}

By its nature, adventuring involves delving into places that are dark, dangerous, and full of mysteries to be explored. The rules in this section cover some of the most important ways in which adventurers interact with the environment in such places.

\subsection{Falling}

A fall from a great height is one of the most common hazards facing an adventurer. At the end of a fall, a creature takes 1d6 bludgeoning damage for every 10 feet it fell, to a maximum of 20d6. The creature lands prone, unless it avoids taking damage from the fall. Creatures that are not \nameref{condition:incapacitated} can generally fall a distance equal to their automatic standing high-jump distance without taking damage or twice that if they willingly fall prone when they land.

\subsection{Suffocating}

A creature can hold its breath for a number of minutes equal to 1 + its Constitution modifier (minimum of 30 seconds).

When a creature runs out of breath or is choking, it can survive for a number of rounds equal to its Constitution modifier (minimum of 1 round). At the start of its next turn, it drops to 0 hit points and is dying, and it can't regain hit points or be stabilized until it can breathe again.

For example, a creature with a Constitution of +2 can hold its breath for 3 minutes. If it starts suffocating, it has 2 rounds to reach air before it drops to 0 hit points.

\subsection{Vision and Light}

The most fundamental tasks of adventuring—--noticing danger, finding hidden objects, hitting an enemy in combat, and targeting a spell, to name just a few—--rely heavily on a character's ability to see. Darkness and other effects that obscure vision can prove a significant hindrance.

A given area might be lightly or heavily obscured. In a \textbf{lightly obscured} area, such as dim light, patchy fog, or moderate foliage, creatures have disadvantage on Wisdom (Perception) checks that rely on sight.

A \textbf{heavily obscured} area—--such as opaque fog, or dense foliage—--blocks vision entirely. A creature effectively suffers from the \nameref{condition:blinded} condition when trying to see something in that area or on the other side of the area.

The presence or absence of light in an environment creates three categories of illumination: bright, dim, and darkness.

\textbf{Bright illumination} lets most creatures see normally.

Even gloomy days provide bright light, as do torches, lanterns, fires, and other sources of illumination within a specific radius.

\textbf{Dim illumination}, also called shadows, creates a lightly obscured area. An area of dim light is usually a boundary between a source of bright light, such as a torch, and surrounding darkness. The soft light of twilight and dawn also counts as dim light. A particularly brilliant full moon might bathe the land in dim light.

\textbf{Darkness} creates an area where normal vision is impossible. It is as if it is heavily obscured, but one can see an illuminated object across any reasonable distance of darkness (whereas heavy obscuration blocks vision through the affected area).

Characters face darkness outdoors at night (even most moonlit nights), within the confines of an unlit dungeon or a subterranean vault, or in an area of magical darkness.

\begin{figure}[htb]
\begin{DndComment}{Visibility Ranges}
Generally, creatures with human-like vision can see brightly-illuminated creatures and objects from a substantial distance (assuming nothing blocks their sight). A rule of thumb is that creatures and objects are visible and recognizable at the following distances based on their size, assuming normal contrast with the background (ie not camouflaged):
\begin{DndTable}[header=Common visibility ranges]{Xll}
	\textbf{Size} & \textbf{Visible at...} \\
	Objects smaller than 6 in x 6 in & 25 ft  \\
	Objects smaller than 1 ft x 1 ft & 50 ft \\
	Tiny creatures & 100 ft \\
	Small creatures or objects & 400 ft \\
	Medium creatures or objects & 1/3 mile \\
	Large creatures or objects & 1 mile \\
	Huge+ creatures or objects & 2 miles  \\
	Mountains, the sun, etc & any (horizon limited) \\
\end{DndTable}

Dim illumination or low-contrast backgrounds reduce the distance by at least half, stacking. So a small creature, dimly illuminated and against a low-contrast background or camouflaged, might only be visible out to 100 ft.

As a rule of thumb, a candle flame in the darkness can be seen (obstructions allowing) for at least 1.5 miles. Larger fires can be seen from further away, limited mostly by the horizon (about 4 miles if standing at ground level).

Adventurers can see the significant terrain features of the land about 2 miles from where they are. If the party is high up or the features are tall, they might be visible from as much as 10-15 miles. Or less if the area is heavily forested (as little as a few dozen feet in dense undergrowth).
\end{DndComment}
\end{figure}

\subsubsection{Blindsight}

A creature with blindsight can perceive its surroundings without relying on sight, within a specific radius. Creatures without eyes, such as oozes, and creatures with echolocation or heightened senses, such as bats and true dragons, have this sense. This counts as being able to "see" the target for spells and other abilities.

\subsubsection{Darkvision}

Many creatures in fantasy gaming worlds, especially those that dwell underground, have darkvision. Within a specified range, a creature with darkvision can see in darkness as if the darkness were dim light, so areas of darkness are only lightly obscured as far as that creature is concerned. However, the creature can't discern color in darkness, only shades of gray.

\subsubsection{Truesight}

A creature with truesight can, out to a specific range, see in normal and magical darkness, see invisible creatures and objects, automatically detect visual illusions and succeed on saving throws against them, and perceives the original form of a shapechanger or a creature that is transformed by magic. Furthermore, the creature can see into the Ethereal Plane.

\subsection{Other Senses}
Most creatures rely primarily on sight, but are still capable of hearing and smelling other creatures, as well as perceiving them by touch. Some fantastic creatures have additional forms of perception, such as the ability to detect magic. The ranges at which this is possible are much more sharply limited than those of sight, but also vary wildly.

\subsubsection{Hearing}
For most humanoid (and similar) creatures, hearing is best at detecting \textit{presence and direction}, but not \textit{location} of creatures, and gives very little information about what they are doing. Background noise also plays a significant role in preventing hearing. As a rule of thumb, a Tiny or larger creature's movement produces enough noise to be audible out to at least 30 feet under normal conditions. If you cannot see the creature but can hear it, interactions that require pinpointing them are at disadvantage. If you can neither see nor hear the creature, they are hidden and you must guess their location (automatically failing any attempt to interact if you guess wrong). To prevent being heard within the normal hearing range, you need to have succeeded at a Hide attempt. 

\begin{figure}[!ht]
\begin{DndTable}[header=Normal Hearing Range]{Xl}
	\textbf{Noise} & \textbf{Audible Distance (ft)} \\
	Whispering or slow movement & 30 \\
	Conversation or normal movement & 60 \\
	Combat & 100 \\
	Shouting, unamplified concerts, loud spells & 300 \\
	Amplified concerts, explosions, thunder & 1 mile \\
\end{DndTable}
\end{figure}

Substantial background noise (noise at least one step higher) generally moves the distance down one step. So fighting during a riot would be audible out to 60 ft (instead of 100 ft). If this would reduce the audible distance below 30 ft, cut the distance in half instead. Creatures with advanced hearing such as dogs, cats, bats, etc. can generally hear and locate sounds at double the distance or more.

\begin{figure}
\begin{DndComment}{Commentary on hearing}
	This is fairly generous for those doing the perceiving. Adventurers and the foes they face are exceptional. If you're trying to get around normal commoners, trying to adjudicate such things too precisely can be more trouble than its worth and you should generally just use a Dexterity(Stealth) check instead.

	In indoor/underground conditions, sounds can generally be heard from further away but become muffled/indistinct sooner. So it's a tradeoff not considered here. If the echoes are particularly strong, increase the range by a step. If it's an absorbent area or wide open area, decrease them by a step.
\end{DndComment}
\end{figure}

\subsubsection{Smell}

For humanoid creatures, smell is even worse than hearing. You can detect, but not localize, exceptionally stinky creatures from a large distance as long as you're downwind. If you're upwind, you cannot detect anything. Smell is generally most useful for tracking creatures or detecting the presence of creatures (after which you use other senses).

Some creatures with exceptional olifactory capabilities (like bloodhounds) are capable of tracking even low-scent targets a very long distance.

\subsubsection{Magical Senses}
In a fantasy world, most creatures are at least somewhat sensitive to the manipulation of aether, especially in the form of spells. As a general rule, all creatures can recognize spellcasting occurring within 60 feet of them as long as it has one or more components. Hiding spellcasting is generally not possible unless you can remove all components.

\subsection{Food and Water}

Characters who don't eat or drink suffer the effects of \nameref{condition:exhaustion}. Exhaustion caused by lack of food or water can't be removed until the character eats and drinks the full required amount.

\subsubsection{Food}

A character needs one pound of food per day and can make food last longer by subsisting on half rations. Eating half a pound of food in a day counts as half a day without food. Large creatures such as horses require 15 lbs of food per day, although they can subsist on forage and only two pounds of grain or other concentrated foods per day.

A character can go without food for a number of days equal to 3 + his or her Constitution modifier (minimum 1). At the end of each day beyond that limit, a character automatically suffers one level of exhaustion.

A normal day of eating resets the count of days without food to zero.

\subsubsection{Water}

A character needs one gallon of water per day, or two gallons per day if the weather is hot. A character who drinks only half that much water must succeed on a DC 15 Constitution saving throw or suffer one level of exhaustion at the end of the day. A character with access to even less water automatically suffers one level of exhaustion at the end of the day. Large creatures such as horses used as mounts need 10 gallons per day, 20 if the weather is hot or the pace is fast. 

If the character already has one or more levels of exhaustion, the character takes two levels in either case.

\subsection{Interacting with Objects}

A character's interaction with objects in an environment is often simple to resolve in the game. The player tells the GM that his or her character is doing something, such as moving a lever, and the GM describes what, if anything, happens.

For example, a character might decide to pull a lever, which might, in turn, raise a portcullis, cause a room to flood with water, or open a secret door in a nearby wall. If the lever is rusted in position, though, a character might need to force it. In such a situation, the GM might call for a Strength check to see whether the character can wrench the lever into place. The GM sets the DC for any such check based on the difficulty of the task.

Characters can also damage objects with their weapons and spells. Objects are immune to poison and psychic damage, but otherwise they can be affected by physical and magical attacks much like creatures can. The GM determines an object's Armor Class and hit points, and might decide that certain objects have resistance or immunity to certain kinds of attacks. (It's hard to cut a rope with a club, for example.) Objects always fail Strength and Dexterity saving throws, and they are immune to effects that require other saves. When an object drops to 0 hit points, it breaks.

A character can also attempt a Strength check to break an object. The GM sets the DC for any such check.

\section{Resting}

Heroic though they might be, adventurers can't spend every hour of the day in the thick of exploration, social interaction, and combat. They need rest—time to sleep and eat, tend their wounds, refresh their minds and spirits for spellcasting, and brace themselves for further adventure.

Adventurers can take short rests in the midst of an adventuring day and a long rest to end the day.

\subsection{Short Rest}

A short rest is a period of downtime, at least 1 hour long, during which a character does nothing more strenuous than eating, drinking, reading, and tending to wounds. Any spellcasting interrupts a short rest.

A character can spend one or more Hit Dice at the end of a short rest, up to the character's maximum number of Hit Dice, which is equal to the character's level. For each Hit Die spent in this way, the player rolls the die and adds the character's Constitution modifier to it. The character regains hit points equal to the total. The player can decide to spend an additional Hit Die after each roll. A character regains some spent Hit Dice upon finishing a long rest, as explained below.

Additionally, characters recover their expended Stamina (see \nameref{sec:stamina-and-aether}) when they finish a short rest.

\subsection{Long Rest}

A long rest is a period of extended downtime, at least 8 hours long, during which a character sleeps or performs light activity: reading, talking, eating, or standing watch for no more than 2 hours. If the rest is interrupted by a period of strenuous activity—--at least 1 hour of walking or any fighting, casting spells, or similar adventuring activity—--the characters must begin the rest again to gain any benefit from it.

At the end of a long rest, a character regains all lost hit points and aether (see \nameref{sec:stamina-and-aether}). The character also regains spent Hit Dice, up to a number of dice equal to half of the character's total number of them (minimum of one die). For example, if a character has eight Hit Dice, he or she can regain four spent Hit Dice upon finishing a long rest.

A character can't benefit from more than one long rest in a 24-hour period, and a character must have at least 1 hit point at the start of the rest to gain its benefits. Creatures cannot progress personal projects during long rests.

\section{Between Adventures}

Between trips to dungeons and battles against ancient evils, adventurers need time to rest, recuperate, and prepare for their next adventure. Many adventurers also use this time to perform other tasks, such as crafting arms and armor, performing research, or spending their hard-earned gold.

In some cases, the passage of time is something that occurs with little fanfare or description. When starting a new adventure, the GM might simply declare that a certain amount of time has passed and allow you to describe in general terms what your character has been doing. At other times, the GM might want to keep track of just how much time is passing as events beyond your perception stay in motion.

\subsection{Lifestyle Expenses}

Between adventures, you choose a particular quality of life and pay the cost of maintaining that lifestyle.

Living a particular lifestyle doesn't have a huge effect on your character, but your lifestyle can affect the way other individuals and groups react to you. For example, when you lead an aristocratic lifestyle, it might be easier for you to influence the nobles of the city than if you live in poverty.

\subsection{Downtime Activities}

Between adventures, the GM might ask you what your character is doing during his or her downtime. Periods of downtime can vary in duration, but each downtime activity requires a certain number of days to complete before you gain any benefit, and at least 8 hours of each day must be spent on the downtime activity for the day to count. The days do not need to be consecutive. If you have more than the minimum amount of days to spend, you can keep doing the same thing for a longer period of time, or switch to a new downtime activity.

Downtime activities other than the ones presented below are possible. If you want your character to spend his or her downtime performing an activity not covered here, discuss it with your GM.

\subsubsection{Crafting}

You can craft nonmagical objects, including adventuring equipment and works of art. You must be proficient with tools related to the object you are trying to create (typically artisan's tools). You might also need access to special materials or locations necessary to create it. For example, someone proficient with smith's tools needs a forge in order to craft a sword or suit of armor. A travelling anvil and kit suffices for repairs or small modifications along the road.

For every day of downtime you spend crafting, you can craft one or more items with a total market value not exceeding 50 gp, and you must expend raw materials worth half the total market value. If something you want to craft has a market value greater than 50 gp, you make progress every day in 50-gp increments until you reach the market value of the item. For example, a suit of plate armor (market value 1,500 gp) takes 30 days to craft by yourself.

Multiple characters can combine their efforts toward the crafting of a single item, provided that the characters all have proficiency with the requisite tools and are working together in the same place. Each character contributes 5 gp worth of effort for every day spent helping to craft the item. For example, three characters with the requisite tool proficiency and the proper facilities can craft a suit of plate armor in 10 days, at a total cost of 750 gp.

While crafting, you can maintain a modest lifestyle without having to pay 1 gp per day, or a comfortable lifestyle at half the normal cost.

\subsubsection{Practicing a Profession}

You can work between adventures, allowing you to maintain a modest lifestyle without having to pay 1 gp per day. This benefit lasts as long you continue to practice your profession.

If you are a member of an organization that can provide gainful employment, such as a temple or a thieves' guild, you earn enough to support a comfortable lifestyle instead.

If you have proficiency in the Performance skill and put your performance skill to use during your downtime, you earn enough to support a wealthy lifestyle instead.

\subsubsection{Recuperating}

You can use downtime between adventures to recover from a debilitating injury, disease, or poison.

After three days of downtime spent recuperating, you can make a DC 15 Constitution saving throw. On a successful save, you can choose one of the following results:
\begin{itemize}
    \item End one effect on you that prevents you from regaining hit points.
    \item For the next 24 hours, gain advantage on saving throws against one disease or poison currently affecting you.
\end{itemize}
\subsubsection{Researching}

The time between adventures is a great chance to perform research, gaining insight into mysteries that have unfurled over the course of the campaign. Research can include poring over dusty tomes and crumbling scrolls in a library or buying drinks for the locals to pry rumors and gossip from their lips.

When you begin your research, the GM determines whether the information is available, how many days of downtime it will take to find it, and whether there are any restrictions on your research (such as needing to seek out a specific individual, tome, or location). The GM might also require you to make one or more ability checks, such as an Intelligence (Investigation) check to find clues pointing toward the information you seek, or a Charisma (Persuasion) check to secure someone's aid. Once those conditions are met, you learn the information if it is available.

For each day of research, you must spend 1 gp to cover your expenses. This cost is in addition to your normal lifestyle expenses.

\subsubsection{Training}

You can spend time between adventures learning a new language, training with a set of tools, or exchanging a skill trick you know for a different one you qualify for. Your GM might allow additional training options.

First, you must find an instructor willing to teach you. The GM determines how long it takes, and whether one or more ability checks are required.

The training lasts for 100 days minus 8 days per point of Intelligence modifier and costs 1 gp per day. After you spend the requisite amount of time and money, you learn the new language, gain proficiency with the new tool, or exchange the skill tricks.

\chapter{The Order of Combat}\label{ch:order-of-combat}
A typical combat encounter is a clash between two sides, a flurry of weapon swings, feints, parries, footwork, and spellcasting. The game organizes the chaos of combat into a cycle of rounds and turns. A \textbf{round} represents about 6 seconds in the game world. During a round, each participant in a battle takes a \textbf{turn}. The order of turns is determined at the beginning of a combat encounter, when everyone rolls initiative. Once everyone has taken a turn, the fight continues to the next round if neither side has defeated the other.

\begin{figure}[htb]
    \begin{DndComment}{Combat Step by Step\label{cmt:combat-step-by-step}}
        \begin{enumerate}
            \item \textbf{Determine surprise.} The GM determines whether anyone involved in the combat encounter is surprised.
            \item \textbf{Establish positions.} The GM decides where all the characters and monsters are located. Given the adventurers' marching order or their stated positions in the room or other location, the GM figures out where the adversaries are--how far away and in what direction.
            \item \textbf{Roll initiative.} Everyone involved in the combat encounter rolls initiative, determining the order of combatants' turns.
            \item \textbf{Take turns.} Each participant in the battle takes a turn in initiative order.
            \item \textbf{Begin the next round.} When everyone involved in the combat has had a turn, the round ends. Repeat step 4 until the fighting stops.
        \end{enumerate}
    \end{DndComment}
\end{figure}

\subsection{Surprise}

A band of adventurers sneaks up on a bandit camp, springing from the trees to attack them. A gelatinous cube glides down a dungeon passage, unnoticed by the adventurers until the cube engulfs one of them. In these situations, one side of the battle gains surprise over the other.

The GM determines who might be surprised, based on their understanding of the situation. If neither side tries to be stealthy, they automatically notice each other. If there is remaining uncertainty about who, if anyone, is surprised, the GM compares the Dexterity (Stealth) checks of anyone hiding with the passive Wisdom (Perception) score of each creature on the opposing side. Any character or monster that doesn't notice a threat is \nameref{condition:surprised} at the start of the encounter. They roll initiative and take their turn as normal. The condition automatically ends at the end of the surprised person's first turn. A member of a group can be surprised even if the other members aren't. For an example of this, see Appendix C (\nameref{example:surprise})

Surprise generally requires that all attackers are \nameref{condition:hidden} from the target. In exceptional circumstances, the GM might allow surprise in cases where the outbreak of hostilities is completely unexpected (such as an assassin pulling a hidden dagger at a peaceful ball). In social situations like the one mentioned, the GM might call for a Charisma (Stealth) check (in addition to any Dexterity (Sleight of Hand) checks necessary to conceal the weapon in the first place). Surprise is rare if both sides are openly wearing weapons or bearing spell foci and are interacting.

\subsection{Initiative}

Initiative determines the order of turns during combat. When combat starts, every participant makes a Dexterity check to determine their place in the initiative order. The GM makes one roll for an entire group of identical creatures, so each member of the group acts at the same time.

The GM ranks the combatants in order from the one with the highest Dexterity check total to the one with the lowest. This is the order (called the initiative order) in which they act during each round. The initiative order remains the same from round to round.

If a tie occurs, the GM decides the order among tied GM-controlled creatures, and the players decide the order among their tied characters. The GM can decide the order if the tie is between a monster and a player character. Optionally, the GM can have the tied characters and monsters each roll a d20 to determine the order, highest roll going first.

\subsection{Your Turn}

On your turn, you can \textbf{move} a distance up to your speed and \textbf{take one action}. You decide whether to move first or take your action first. Your speed— sometimes called your walking speed—is noted on your character sheet.

The most common actions you can take are described in the \nameref{sec:actions-in-combat} section later in this chapter. Many class features and other abilities provide additional options for your action.

The \nameref{sec:movement-and-position} section later in this chapter gives the rules for your move.

You can forgo moving, taking an action, or doing anything at all on your turn. If you can't decide what to do on your turn, consider taking Guard or Ready action, as described in \nameref{sec:actions-in-combat}.

\subsubsection{Bonus Actions}

Various class features, spells, and other abilities let you take an additional action on your turn called a bonus action. The Cunning Action feature, for example, allows a rogue to take a bonus action. You can take a bonus action only when a special ability, spell, or other feature of the game states that you can do something as a bonus action. You otherwise don't have a bonus action to take.

You can take only one bonus action on your turn, so you must choose which bonus action to use when you have more than one available.

You choose when to take a bonus action during your turn, unless the bonus action's timing is specified, and anything that deprives you of your ability to take actions also prevents you from taking a bonus action. You cannot interrupt the resolution of an action with a bonus action \textit{except} the Attack action (when you can make multiple attacks with that action)--in that case, you can take a bonus action between attacks just as you can move between attacks. If a bonus action requires you to take the Attack action, you must make at least one attack with that action before you can use the triggered bonus action.

\begin{figure}
    \begin{DndComment}{On action timing}
        Generally, NIH is more strict about actions being atomic than D\&D. Unless specifically indicated in the text of the ability, you must resolve an action completely before using another action. This includes reactions--you cannot take reactions during your actions unless specifically allowed in the ability that grants the reaction. \textbf{Note} movement is not an action--it can always be broken up however you wish in and around actions, bonus actions, and reactions.
    \end{DndComment}
\end{figure}

\subsubsection{Other Activity on Your Turn}

Your turn can include a variety of flourishes that require neither your action nor your move.

You can communicate however you are able, through brief utterances and gestures, as you take your turn. Communicating complex content or attempting to persuade, intimidate, or deceive an enemy into doing something takes your action.

You can also interact with one object or feature of the environment for free, during either your move or your action. For example, you could open a door during your move as you stride toward a foe, or you could draw your weapon as part of the same action you use to attack.

If you want to interact with a second object, you need to use your action. Some magic items and other special objects always require an action to use, as stated in their descriptions.

The GM might require you to use an action for any of these activities when it needs special care or when it presents an unusual obstacle. For instance, the GM could reasonably expect you to use an action to open a stuck door or turn a crank to lower a drawbridge.

\subsection{Reactions}

Certain special abilities, spells, and situations allow you to take a special action called a reaction. A reaction is an instant response to a trigger of some kind, which can occur on your turn or on someone else's. The opportunity attack, described later in this chapter, is the most common type of reaction.

When you take a reaction, you can't take another one until the start of your next turn. If the reaction interrupts another creature's turn, that creature can continue its turn right after the reaction. Reactions do not interrupt actions unless they specifically say that they do, but they can interrupt movement.

\section{Movement and Position}
\label{sec:movement-and-position}

In combat, characters and monsters are in constant motion, often using movement and position to gain the upper hand.

At the beginning of your turn, you gain available movement equal to your speed. You can use as much or as little of this movement as you like on your turn, following the rules here.

Your movement can include jumping, climbing, and swimming. These different modes of movement can be combined with walking, or they can constitute your entire move. If you have multiple speeds, each one has a separate pool of movement. However you're moving, you deduct the distance of each part of your move from your available movement in \textbf{all} mode (regardless of the mode you're actually using). When an available amount of movement in a mode reaches zero, you can no longer move using that mode that turn.

Effects such as the \nameref{spell:slow} spell that reduce your speed reduce \textit{all} speeds you may have proportionately. For example, if you have a 30 foot walking speed and a 60 foot flying speed and are affected by \nameref{spell:slow}, you start your turn with only 15 feet of walking movement and 30 feet of flying movement available. If an effect changes your speed during your turn, recalculate your total available pool of movement and subtract the distance you have already moved from each of the (now altered) pools. This may stop further movement that turn. For example, in the case above if you were afflicted by \nameref{spell:slow} after you had walked for 20 feet, your new pool of walking movement (15 feet) is already consumed and you have no walking movement left. You can still fly for 10 feet, since 60 ft, halved, is 30 feet, which is greater than the 20 you've already moved that turn.

\subsection{Breaking Up Your Move}

You can break up your movement on your turn, using some of your speed before and after your action. For example, if you have a speed of 30 feet, you can move 10 feet, take your action, and then move 20 feet.

\subsubsection{Moving between Attacks}

If you take an action that includes more than one weapon attack, you can break up your movement even further by moving between those attacks. For example, a armsman who can make two attacks with the Extra Attack feature and who has a speed of 25 feet could move 10 feet, make an attack, move 15 feet, and then attack again.

\subsubsection{Using Different Speeds}

If you have more than one speed, such as your walking speed and a flying speed, you can switch back and forth between your speeds during your move. Remember that any movement subtracts from all pools individually--if a movement mode has no movement available, you cannot use that mode for the remainder of that turn.

For example, if you have a 30 foot walking speed and a 60 foot flying speed (whether due to wings, a class ability, or an effect such as the \nameref{inc:fly} incantation), you start your turn with two pools of movement: 30 feet walking and 60 feet flying. If you fly for 20 feet, you now have 10 feet (30 - 20) of walking movement available and 40 feet (60 - 20) of flying movement available. If you then land and walk 10 feet, you now cannot walk further (since you have 0 feet of walking movement left) but could then fly for another 30 feet (because 60 ft - 20 ft flying - 10 ft walking - 30 ft flying = 0 remaining).

\subsection{Difficult Terrain}

Combat rarely takes place in bare rooms or on featureless plains. Boulder-strewn caverns, briar-choked forests, treacherous staircases\textemdash the setting of a typical fight contains difficult terrain.

Every foot of movement in difficult terrain costs 1 extra foot. This rule is true even if multiple things in a space count as difficult terrain.

Low furniture, rubble, undergrowth, steep stairs, snow, and shallow bogs are examples of difficult terrain. The space of another creature, whether hostile or not, also counts as difficult terrain.

\subsection{Being Prone}

Combatants often find themselves lying on the ground, either because they are knocked down or because they throw themselves down. In the game, they have the \nameref{condition:prone} condition.

You can \textbf{drop prone} without using any of your speed. \textbf{Standing up} takes more effort; doing so costs an amount of movement equal to half your speed. For example, if your speed is 30 feet, you must spend 15 feet of movement to stand up. You can't stand up if you don't have enough movement left or if your speed is 0.

To move while prone, you must \textbf{crawl} or use magic such as teleportation. Every foot of movement while crawling costs 1 extra foot. Crawling 1 foot in difficult terrain, therefore, costs 3 feet of movement.

\section{Interacting with Objects Around You}

Here are a few examples of the sorts of thing you can do in tandem with your movement and action:

\begin{itemize}
\item draw or sheathe a sword
\item open or close a door
\item withdraw a potion from your backpack
\item pick up a dropped axe
\item take a bauble from a table
\item remove a ring from your finger
\item stuff some food into your mouth
\item plant a banner in the ground
\item fish a few coins from your belt pouch
\item drink all the ale in a flagon
\item throw a lever or a switch
\item pull a torch from a sconce
\item take a book from a shelf you can reach
\item extinguish a small flame
\item don a mask
\item pull the hood of your cloak up and over your head
\item put your ear to a door
\item kick a small stone or a dropped item in your own space or an adjacent one.
\item turn a key in a lock
\item tap the floor with a 10-foot pole
\item hand an item to another character
\end{itemize}

\subsection{Moving Around Other Creatures}

You can move through a nonhostile creature's space. In contrast, you can move through a hostile creature's space only if the creature is at least two sizes larger or smaller than you. Remember that another creature's space is difficult terrain for you.

Whether a creature is a friend or an enemy, you can't willingly end your move in its space unless it is two sizes larger than you or you are using it as a mount, including interrupting your movement to take an action, reaction, or bonus action.

If you leave a hostile creature's reach during your move, you provoke an opportunity attack, as explained later in the chapter.

\subsection{Flying Movement}

Flying creatures enjoy many benefits of mobility, but they must also deal with the danger of falling. If a flying creature is knocked prone, has its speed reduced to 0, or is otherwise deprived of the ability to move, the creature falls, unless it has the ability to hover or it is being held aloft by magic, such as by the \nameref{inc:fly} incantation.

\subsection{Creature Size}

Each creature controls a different amount of space in combat. The Size Categories table shows how much space a creature of a particular size controls in combat. Objects sometimes use the same size categories.

\begin{figure}
\begin{DndTable}[header=Size Categories]{lX}
    Size & Space \\
    Tiny & 2.5 by 2.5 ft. or smaller \\
    Small & 5 by 5 ft. \\
    Medium & 5 by 5 ft. \\
    Large & 10 by 10 ft. \\
    Huge & 15 by 15 ft. \\
    Gargantuan & 20 by 20 ft. or larger \\
\end{DndTable}
\end{figure}

\subsubsection{Space}

A creature's space is the area in feet that it effectively controls in combat, not an expression of its physical dimensions. A typical Medium creature isn't 5 feet wide, for example, but it does control a space that wide. If a Medium hobgoblin stands in a 5-foot-wide doorway, other creatures can't get through unless the hobgoblin lets them.

A creature's space also reflects the area it needs to fight effectively. For that reason, there's a limit to the number of creatures that can surround another creature in combat. Assuming Medium combatants, eight creatures can fit in a 5-foot radius around another one.

Because larger creatures take up more space, fewer of them can surround a creature. If four Large creatures crowd around a Medium or smaller one, there's little room for anyone else. In contrast, as many as twenty Medium creatures can surround a Gargantuan one.

\subsubsection{Squeezing into a Smaller Space}

A creature can squeeze through a space that is large enough for a creature one size smaller than it. Thus, a Large creature can squeeze through a passage that's only 5 feet wide. While squeezing through a space, a creature must spend 1 extra foot for every foot it moves there, and it has disadvantage on attack rolls and Dexterity saving throws. Attack rolls against the creature have advantage while it's in the smaller space.

Creatures can squeeze through spaces two sizes smaller than themselves under extreme circumstances. However, while doing so they cannot make attack rolls, cast spells with somatic or material components, fail all Dexterity saving throws, and cannot move more than 5 feet per round.

\subsection{Effects that trigger on movement}
Some spells, abilities, or environmental effects have clauses such as "when the target moves" or "when a creature enters the area for the first time on a turn", etc. These trigger regardless of whether the movement was voluntary or not. Being shoved, pulled, or otherwise forced to move without using your action, reaction, or movement will trigger the effect.

If the condition is "if the target \textit{voluntarily} moves...", this requires the same trigger as Opportunity Attacks--the target must use their action, reaction, or movement to do so.

\section{Actions in Combat}\label{sec:actions-in-combat}

When you take your action on your turn, you can take one of the actions presented here, an action you gained from your class or a special feature, or an action that you improvise. Many monsters have action options of their own in their stat blocks.

When you describe an action not detailed elsewhere in the rules, the GM tells you whether that action is possible and what kind of roll you need to make, if any, to determine success or failure.

\subsection{Attack}

The most common action to take in combat is the Attack action, whether you are swinging a sword, firing an arrow from a bow, or brawling with your fists.

With this action, you make one melee or ranged weapon attack. See the “Making an Attack” section for the rules that govern attacks.

Certain features, such as the Extra Attack feature of the armsman, allow you to make more than one attack with this action.

\subsection{Cast a Spell}

Spellcasters such as arcanists and priests, as well as many monsters, have access to spells and can use them to great effect in combat. Each spell has a casting time, which specifies whether the caster must use an action, a reaction, minutes, or even hours to cast the spell. Casting a spell is, therefore, not necessarily an action. Most spells do have a casting time of 1 action, so a spellcaster often uses his or her action in combat to cast such a spell.

\subsection{Dash}

When you take the Dash action, you gain extra movement for the current turn. The increase equals your speed, after applying any modifiers. With a speed of 30 feet, for example, you can move up to 60 feet on your turn if you dash.

Any increase or decrease to your speed changes this additional movement by the same amount. If your speed of 30 feet is reduced to 15 feet, for instance, you can move up to 30 feet this turn if you dash.

\subsection{Disengage}

If you take the Disengage action, your movement doesn't provoke opportunity attacks for the rest of the turn.

\subsection{Guard}

When you take the Guard action, you focus entirely on avoiding attacks. Until the start of your next turn, any attack roll made against you has disadvantage if you can see the attacker, and you make Dexterity saving throws with advantage. You lose this benefit if you are \nameref{condition:incapacitated} or if your speed drops to 0.

\subsection{Help}

You can lend your aid to another creature in the completion of a task. When you take the Help action, the creature you aid gains advantage on the next ability check it makes to perform the task you are helping with, provided that it makes the check before the start of your next turn. The GM may rule that certain actions cannot receive help.

Alternatively, you can aid a friendly creature in attacking a creature within 5 feet of you. You feint, distract the target, or in some other way team up to make your ally's attack more effective. If your ally attacks the target before your next turn, the first attack roll is made with advantage. Creatures that cannot attack cannot Help with attacks.

\subsection{Hide}

When you take the Hide action, you make a Dexterity (Stealth) check in an attempt to hide, following the rules for hiding. If you succeed, you gain certain benefits, as described in the \nameref{subsec:unseen-attackers} section later in this chapter. Unless you take the Hide action and succeed, enemies in combat are generally aware of your location (to within 5 feet) as long as they could conceivably perceive you (via hearing, sight, smell, or other forms of perception).

\subsection{Ready}

Sometimes you want to get the jump on a foe or wait for a particular circumstance before you act. To do so, you can take the Ready action on your turn, which lets you act using your reaction before the start of your next turn.

First, you decide what perceivable circumstance will trigger your reaction. Then, you choose the action you will take in response to that trigger, or you choose to move up to your speed in response to it. Examples include “If the cultist steps on the trapdoor, I'll pull the lever that opens it,” and “If the goblin steps next to me, I move away.”

When the trigger occurs, you can either take your reaction right after the action or movement containing the trigger finishes or ignore the trigger. Remember that you can take only one reaction per round. Reactions happen after the action that triggered it completes; each 5 feet of movement is considered a separate "action" for this purpose. The ready action cannot interrupt an action, regardless of how the triggers are are phrased. If you ready an action for when a goblin attacks you, the attack is completely resolved before your Readied action triggers. If you ignore the trigger and it happens again before you take your next turn, you can decide to use your readied action at that time. If it does not reoccur, your action is lost.

When you ready a spell, you cast it as normal but hold its energy, which you release with your reaction when the trigger occurs. To be readied, a spell must have a casting time of 1 action, and holding onto the spell's magic requires concentration. If your concentration is broken, the spell dissipates without taking effect. For example, if you are concentrating on the \nameref{spell:web} spell and ready \nameref{spell:magic-missile}, your \nameref{spell:web} spell ends, and if you take damage before you release \nameref{spell:magic-missile} with your reaction, your concentration might be broken.

\subsection{Search}

When you take the Search action, you devote your attention to finding something. Depending on the nature of your search, the GM might have you make a Wisdom (Perception) check or an Intelligence (Investigation) check.

\subsection{Use an Object}

You normally interact with an object while doing something else, such as when you draw a sword as part of an attack. When an object requires your action for its use, you take the Use an Object action. This action is also useful when you want to interact with more than one object on your turn.

\section{Making an Attack}

Whether you're striking with a melee weapon, firing a weapon at range, or making an attack roll as part of a spell, an attack has a simple structure. The game does not model changes in modifiers and conditionals during an attack--whatever was true at when the attack was declared is true throughout unless specifically overriden by another ability.

\begin{enumerate}
\item \textbf{Choose a target.} Pick a target within your attack's range: a creature, an object, or a location. The target cannot have total cover from you.
\item \textbf{Determine modifiers.} The GM determines whether the target has cover and whether you have advantage or disadvantage against the target. In addition, spells, special abilities, and other effects can apply penalties or bonuses to your attack roll.
\item \textbf{Resolve the attack.} You make the attack roll. On a hit, you roll damage, unless the particular attack has rules that specify otherwise. Some attacks cause special effects in addition to or instead of damage.
\end{enumerate}

If there's ever any question whether something you're doing counts as an attack, the rule is simple: if you're making an attack roll, you're making an attack. And vice versa--attacks require attack rolls unless the ability specifically describes itself as an attack and calls for some other resolution method (such as the Grapple or Shove special attacks below).

\subsection{Attack Rolls}

When you make an attack, your attack roll determines whether the attack hits or misses. To make an attack roll, roll a d20 and add the appropriate modifiers. If the total of the roll plus modifiers equals or exceeds the target's Armor Class (AC), the attack hits. The AC of a character is determined at character creation and by modifications during play, whereas the AC of a monster is in its stat block.

\subsubsection{Modifiers to the Roll}

When a character makes an attack roll, the two most common modifiers to the roll are an ability modifier and the character's proficiency bonus. When a monster makes an attack roll, it uses whatever modifier is provided in its stat block.

\subparagraph*{Ability Modifier} The ability modifier used for a melee weapon attack is Strength, and the ability modifier used for a ranged weapon attack is Dexterity. Weapons that have the finesse or thrown property break this rule.

Some spells and magical abilities also require an attack roll. The ability modifier used for a spell attack depends on the spellcasting ability of the spellcaster.

\subparagraph*{Proficiency Bonus} You add your proficiency bonus to your attack roll when you attack using a weapon with which you have proficiency, as well as when you attack with a spell.

\subsubsection{Rolling 1 or 20}

Sometimes fate blesses or curses a combatant, causing the novice to hit and the veteran to miss.

If the d20 roll for an attack is a 20, the attack hits regardless of any modifiers or the target's AC. This is called a critical hit, which is explained later in this chapter. Features that allow scoring critical hits on other d20 results also cause such rolls to be automatic hits.

If the d20 roll for an attack is a 1, the attack misses regardless of any modifiers or the target's AC. No other penalty is assessed.

\subsection{Unseen Attackers and Targets}\label{subsec:unseen-attackers}

Combatants often try to escape their foes' notice by hiding, casting the invisibility spell, or lurking in darkness.

When you attack a target that you can't see, you have disadvantage on the attack roll. This is true whether you're guessing the target's location or you're targeting a creature you can hear but not see. If the target isn't in the location you targeted, you automatically miss, but the GM typically just says that the attack missed, not whether you guessed the target's location correctly. As a general rule, characters are assumed to be able to locate other creatures within 30 ft of them via sound, smell, or other senses even if they can't see them and do not need to guess their locations. A DM may rule otherwise in particular environments.

When a creature can't see you, you have advantage on attack rolls against it. If you are hidden—both unseen and unheard—when you make an attack, you give away your location when the attack hits or misses.

\subsection{Ranged Attacks}

When you make a ranged attack, you fire a bow or a crossbow, hurl a handaxe, or otherwise send projectiles to strike a foe at a distance. A monster might shoot spines from its tail. Many spells also involve making a ranged attack.

If the target of a ranged attack is totally concealed but does not have total cover (such by as if they are heavily obscured by dense fog), the attack is made with disadvantage. This overrides the \nameref{subsec:unseen-attackers} rules.

\subsubsection{Range}

You can make ranged attacks only against targets within a specified range.

If a ranged attack, such as one made with a spell, has a single range, you can't attack a target beyond this range.

Some ranged attacks, such as those made with a longbow or a shortbow, have two ranges. The smaller number is the normal range, and the larger number is the long range. Your attack roll has disadvantage when your target is beyond normal range, and you can't attack a target beyond the long range.

\subsubsection{Ranged Attacks in Close Combat}

Aiming a ranged attack is more difficult when a foe is next to you. When you make a ranged attack with a weapon, a spell, or some other means, you provoke opportunity attacks from hostile creatures who can see you and are within their reach of you.

\subsection{Melee Attacks}

Used in hand-to-hand combat, a melee attack allows you to attack a foe within your reach. A melee attack typically uses a handheld weapon such as a sword, a warhammer, or an axe. A typical monster makes a melee attack when it strikes with its claws, horns, teeth, tentacles, or other body part. A few spells also involve making a melee attack.

Most creatures have a 5-foot \textbf{reach} and can thus attack targets within 5 feet of them when making a melee attack. Certain creatures (typically those larger than Medium) have melee attacks with a greater reach than 5 feet, as noted in their descriptions.

Instead of using a weapon to make a melee weapon attack, you can use an \textbf{unarmed strike}: a punch, kick, head-butt, or similar forceful blow (none of which count as manufactured weapons). On a hit, an unarmed strike deals bludgeoning damage equal to 1 + your Strength modifier. You are proficient with your unarmed strikes.

\subsubsection{Opportunity Attacks}\label{sec:opportunity-attacks}

In a fight, everyone is constantly watching for a chance to strike an enemy who is fleeing or passing by. Such a strike is called an opportunity attack.

You can make an opportunity attack when a hostile creature that you can see moves out of your reach. To make the opportunity attack, you use your reaction to make one melee attack against the provoking creature. The attack occurs right before the creature leaves your reach.

You can avoid provoking an opportunity attack by taking the Disengage action. You also don't provoke an opportunity attack when you teleport or when someone or something moves you without using your movement, action, or reaction. For example, you don't provoke an opportunity attack if an explosion hurls you out of a foe's reach or if gravity causes you to fall past an enemy.

Additionally, making ranged attacks while in the reach of an opponent who can see you provokes an opportunity attack, as does casting a spell with one or more component whose range is anything other than Self (including cones and auras) or Touch, even if it does not require an attack roll or a saving throw.

\subsubsection{Two-Weapon Fighting}

When you take the Attack action and attack with a light melee weapon that you're holding in one hand, you can make an additional attack with a different light melee weapon that you're holding in the other hand. You don't add your ability modifier to the damage of the additional attack, unless that modifier is negative. Only one such additional attack can be made per action, regardless of how many other attacks you can make during that action.

If either weapon has the thrown property, you can throw the weapon, instead of making a melee attack with it.

\subsubsection{Grappling}

When you want to grab a creature or wrestle with it, you can use the Attack action to make a special melee attack, a grapple. If you're able to make multiple attacks with the Attack action, this attack replaces one of them.

The target of your grapple must be no more than one size larger than you and must be within your reach. Using at least one free hand, you try to seize the target by making a grapple check instead of an attack roll: a Strength (Athletics) check contested by the target's Strength (Athletics) check. If you succeed, you subject the target to the \nameref{condition:grappled} condition. The condition specifies the things that end it, and you can release the target whenever you like (no action required).

\subparagraph*{Escaping a Grapple} A grappled creature can use its action to escape. To do so, it must succeed on a Strength (Athletics) or Dexterity (Acrobatics) check contested by your Strength
(Athletics) check.

\subparagraph*{Moving a Grappled Creature} When you move, you can drag or carry the grappled creature with you, but you must spend 2 feet of movement for every foot you move, unless the creature is two or more sizes smaller than you. This stacks with difficult terrain. If the target is also being grappled by another creature and this movement would disrupt the grapple by moving the target out of their reach, treat it as an attempt to escape the grapple by the creature doing the movement. The mover and the other grappler make opposed checks as described. If the mover wins, the other person's grapple breaks. If the other grappler wins, no movement occurs.

If you try to stay in place and rotate the grappled creature around you, you must spend 2 feet of movement per foot moved by the grappled creature.

\subsection{Contests in Combat}

Battle often involves pitting your prowess against that of your foe. Such a challenge is represented by a contest. This section includes the most common contests that require an action in combat: grappling and shoving a creature. The GM can use these contests as models for improvising others.

\subsubsection{Shoving a Creature}

Using the Attack action, you can make a special melee attack to shove a creature, either to knock it prone or push it away from you. If you're able to make multiple attacks with the Attack action, this attack replaces one of them.

The target must be no more than one size larger than you and must be within your reach. Instead of making an attack roll, you make a Strength (Athletics) check contested by the target's Strength (Athletics) or Dexterity (Acrobatics) check (the target chooses the ability to use). If you win the contest, you either knock the target prone or push it 5 feet away from you. The distance you can push a creature increases by 5 feet for every 5 points your check beat theirs. Shoving a creature that is grappling someone and succeeding breaks the grapple if the shoved creature is pushed outside of their reach on the grappled creature.

Targets can choose to fail this context intentionally. If a friendly target is shoved while being grappled by a hostile creature, the hostile creature makes the opposed check instead of the friendly target.

Successfully shoving a creature that is concentrating on a spell or other effect forces them to make a DC 10 Constitution saving throw or lose concentration on the effect.

\section{Cover}

Walls, trees, creatures, and other obstacles can provide cover during combat, making a target more difficult to harm. A target can benefit from cover only when an attack or other effect originates on the opposite side of the cover.

There are three degrees of cover. If a target is behind multiple sources of cover, only the most protective degree of cover applies; the degrees aren't added together. For example, if a target is behind a creature that gives half cover and a tree trunk that gives three-quarters cover, the target has three-quarters cover.

A target with \textbf{half cover} has a +2 bonus to AC and Dexterity saving throws. A target has half cover if a solid obstacle blocks at least half of its body. The obstacle might be a low wall, a large piece of furniture, a narrow tree trunk, or a creature, whether that creature is an enemy or a friend.

A target with \textbf{three-quarters cover} has a +5 bonus to AC and Dexterity saving throws. A target has three-quarters cover if about three-quarters of it is covered by a solid obstacle. The obstacle might be a portcullis, an arrow slit, or a thick tree trunk. Creatures two sizes larger or more than the target provide three-quarters cover instead of half cover. 

A target with \textbf{total cover} can't be targeted directly by an attack or a spell, although some spells can reach such a target by including it in an area of effect. A target has total cover if it is completely behind a solid obstacle. Note that total cover and total concealment are different--a target has total cover when it stands behind a \nameref{spell:wall-of-force}, but as the wall is transparent, it has no concealment and cannot hide. And vice versa--a target concealed by heavy fog or by unilluminated darkness has no cover but is totally concealed and can attempt to hide.

\section{Damage and Healing}

Injury and the risk of death are constant companions of those who explore fantasy gaming worlds. The thrust of a sword, a well-placed arrow, or a blast of flame from a \nameref{spell:fireball} spell all have the potential to damage, or even kill, the hardiest of creatures.

\subsection{Hit Points}

Hit points represent the strength of the soul of the creature and its ability to suffer damage and keep fighting normally. Creatures with more hit points are more difficult to kill. Those with fewer hit points are more fragile.

A creature's current hit points (usually just called hit points) can be any number from the creature's hit point maximum down to 0. This number changes frequently as a creature takes damage or receives healing.

Whenever a creature takes damage, that damage is subtracted from its hit points. The loss of hit points has no effect on a creature's capabilities until the creature drops to 0 hit points. Generally, creatures show little signs of damage until their hit points reach half of their maximum, a state known as \nameref{condition:bloodied}. Any wounds they sustain as long as they have more than 0 hit points are superficial; the body's abilty to keep them going is unimpaired.

\begin{figure}
    \begin{DndComment}{Yes, hit points are meat}
        Creatures in Quartus are usually triune--they have a "soul" (aka spark), a "spirit" (aka nimbus), and a body. The Spark is the self; the body is the interface with the physical world, and the Nimbus is the interface with the immaterial, as well as between the Spark and the body. The body contains both deep reserves of aether (the stuff of creation, out of which all things material or not are made), represented by the creature's Hit Dice, and more available reserves (Hit Points), both of which it can use to repair damage. This also explains why a \nameref{spell:fireball} doesn't burn up your gear but can burn unattended objects--your nimbus enfolds all of your possessions and shields them from harm...at some cost to your aether reserves.

        When a creature is \nameref{condition:bloodied}, their autonomic body systems have pulled back their efforts, only healing the deep, potentially impairing or fatal injuries and leaving the superficial ones alone (and similarly for your gear). When you reach 0 hit points, your body can no longer heal those impairing injuries and it begins trying to grab aether from anywhere it can (aka Death Saves). Most people in the setting suffer lingering injuries such as lost limbs, broken bones, limps, etc. when they are reduced to zero hit points even if they are stabilized. PCs are special for game reasons--they don't suffer those injuries. This is an explicit departure from simulation and those can be added back in if so desired. I, personally, don't find them fun to play with, so I don't do them.
    \end{DndComment}
\end{figure}

\subsection{Damage Rolls}

Each weapon, spell, and harmful monster ability specifies the damage it deals. You roll the damage die or dice, add any modifiers, and apply the damage to your target. Magic weapons, special abilities, and other factors can grant a bonus to damage. With a penalty, it is possible to deal 0 damage, but never negative damage.

When attacking with a \textbf{weapon}, you add your ability modifier—the same modifier used for the attack roll—to the damage. A \textbf{spell} tells you which dice to roll for damage and whether to add any modifiers.

If a spell or other effect deals damage to \textbf{more than one target} at the same time, roll the damage once for all of them. For example, when an arcanist casts \nameref{spell:fireball} or a priest casts \nameref{spell:flame-strike}, the spell's damage is rolled once for all creatures caught in the blast.

Spells or effects that deal damage on a different creature's turn, such as \nameref{spell:flaming-sphere} or \nameref{spell:spirit-guardians}, roll their damage once per round at the beginning of the caster's turn. Anyone affected by the spell until the beginning of the caster's next turn takes that amount of damage (modified separately by any saving throws or resistances).

Abilities such as poison that trigger on a hit with a weapon attack still take effect even if the damage from the weapon is mitigated (via resistance, temporary hit points, or the like) to zero.

\subsubsection{Critical Hits} \label{sec:critical-hits}

When you score a critical hit with a weapon, you get to roll extra dice for the attack's damage against the target. Roll all of the attack's damage dice twice and add them together. Then add any relevant modifiers as normal. To speed up play, you can roll all the damage dice at once.

For example, if you score a critical hit with a dagger, roll 2d4 for the damage, rather than 1d4, and then add your relevant ability modifier. If the attack involves other damage dice, such as from the rogue's Sneak Attack feature, you roll those dice twice as well.

If an attack has additional effects that require a saving throw to take full effect (such as the poisoned from a Giant Spider's Bite attack), those additional damage dice are not doubled on a critical hit.

\subsubsection{Damage Types}

Different attacks, damaging spells, and other harmful effects deal different types of damage. Damage types have no rules of their own, but other rules, such as damage resistance, rely on the types.

The damage types follow, with examples to help a GM assign a damage type to a new effect.

\subparagraph*{Acid} The corrosive spray of a black dragon's breath and the dissolving enzymes secreted by a black pudding deal acid damage.

\subparagraph*{Bludgeoning} Blunt force attacks—hammers, constriction, unarmed blows, and the like—deal bludgeoning damage, as does falling.

\subparagraph*{Cold} The infernal chill radiating from an ice devil's spear and the frigid blast of a white dragon's breath deal cold damage.

\subparagraph*{Fire} Red dragons breathe fire, and many spells conjure flames to deal fire damage.

\subparagraph*{Lightning} A \nameref{spell:lightning-bolt} spell and a blue dragon's breath deal lightning damage.

\subparagraph*{Necrotic} Necrotic damage, dealt by certain undead and a spell such as \nameref{spell:grave-touch}, withers matter and even the soul.

\subparagraph*{Piercing} Puncturing and impaling attacks, including spears and monsters' bites, deal piercing damage.

\subparagraph*{Poison} Venomous stings and the toxic gas of a green dragon's breath deal poison damage.

\subparagraph*{Psychic} Mental abilities such as a mind flayer's psionic blast deal psychic damage.

\subparagraph*{Radiant} Radiant damage, dealt by a priest's \nameref{spell:flame-strike} spell or an angel's smiting weapon, sears the flesh like fire and overloads the spirit with power.

\subparagraph*{Slashing} Swords, axes, and monsters' claws deal slashing damage.

\subparagraph*{Thunder} A concussive burst of sound, such as the effect of the \nameref{spell:thunderwave} spell, deals thunder damage. Thunder damage is only notably loud if the spell or ability says it is. Otherwise, it's normally lost in the general sounds of combat.

\begin{figure}[htb]
    \begin{DndComment}{What happened to force damage?}
        I've removed force damage as a type. There are constructs of force, but any damage they deal (or \nameref{spell:magic-missile}, for example), deal an appropriate physical damage type.
    \end{DndComment}
\end{figure}

\subsection{Damage Resistance and Vulnerability}

Some creatures and objects are exceedingly difficult or unusually easy to hurt with certain types of damage.

If a creature or an object has \textbf{resistance} to a damage type, damage of that type is halved against it. If a creature or an object has \textbf{vulnerability} to a damage type, damage dice of that type are maximized when applying to that creature.

Resistance and then vulnerability are applied after all other modifiers to damage. For example, a creature has resistance to bludgeoning damage and is hit by an attack that deals 25 bludgeoning damage. The creature is also within a magical aura that reduces all damage by 5. The 25 damage is first reduced by 5 and then halved, so the creature takes 10 damage.

Multiple instances of resistance or vulnerability that affect the same damage type count as only one instance. For example, if a creature has resistance to fire damage as well as resistance to all nonmagical damage, the damage of a nonmagical fire is reduced by half against the creature, not reduced by three-quarters.

\subsubsection{Variant: Resistance, vulnerability, and multiple damage types}
Dealing with resistance and vulnerability when a damage source has multiple damage types and the target is only resistant or vulnerable to some of them can be a pain. As a variant to the above rules, a GM may decide to do the following in that case:
\begin{enumerate}
    \item calculate the total damage, disregarding resistance and vulnerability, but including immunity.
    \item count up how many of the relevant damage types the target is vulnerable or resistant to.
    \item For every relevant source of resistance, cut the damage by 25\%, to a maximum reduction of 50\%.
    \item For every relevant source of vulnerability, increase the damage by 25\%, to a maximum increase of 100\%.
\end{enumerate}

\subsection{Healing}

Unless it results in death, damage isn't permanent. Even death is reversible through powerful magic. Rest can restore a creature's hit points, and magical methods such as a \nameref{spell:cure-wounds} spell or a \textit{potion of healing} can remove damage in an instant.

When a creature receives healing of any kind, hit points regained are added to its current hit points. A creature's hit points can't exceed its hit point maximum, so any hit points regained in excess of this number are lost. For example, a shaman grants a ranger 8 hit points of healing. If the ranger has 14 current hit points and has a hit point maximum of 20, the ranger regains 6 hit points from the shaman, not 8.

A creature that has died can't regain hit points until magic such as the \nameref{spell:revivify} spell has restored it to life.

\subsection{Dropping to 0 Hit Points}

When you drop to 0 hit points, you either die outright or fall unconscious, as explained in the following sections.

\subsubsection{Instant Death}

Massive damage can kill you instantly. When damage reduces you to 0 hit points and there is damage remaining, you die if the remaining damage equals or exceeds your hit point maximum.

For example, a priest with a maximum of 12 hit points currently has 6 hit points. If she takes 18 damage from an attack, she is reduced to 0 hit points, but 12 damage remains. Because the remaining damage equals her hit point maximum, the priest dies.

\subsubsection{Falling Unconscious}

If damage reduces you to 0 hit points and fails to kill you, you fall \nameref{condition:unconscious}. This unconsciousness ends if you regain any hit points.

\begin{figure}
	\begin{DndComment}{Variant: Heroic Resilience}
		In particularly heroic games, the GM might allow PCs that are reduced to 0 hit points and are not killed outright to expend 1 Stamina per turn to become \nameref{condition:incapacitated} and \nameref{condition:prone} instead of \nameref{condition:unconscious}. This allows them to move (although slower) and speak, but not take actions, reactions, or bonus actions. Using this variant rule, you still make \nameref{sec:death-saves} as normal.
	\end{DndComment}
\end{figure}

\subsubsection{Death Saving Throws} \label{sec:death-saves}

Whenever you start your turn with 0 hit points, you must make a special saving throw, called a death saving throw, to determine whether you creep closer to death or hang onto life. Unlike other saving throws, this one isn't tied to any ability score. You are in the hands of fate now, aided only by spells and features that improve your chances of succeeding on a saving throw.

Roll a d20. If the roll is 10 or higher, you succeed. Otherwise, you fail. A success or failure has no effect by itself. On your third success, you become stable (see below). On your third failure, you die. The successes and failures don't need to be consecutive; keep track of both until you collect three of a kind. The number of both is reset to zero when you regain any hit points or become stable.

\subparagraph*{Rolling 1 or 20} When you make a death saving throw and roll a 1 on the d20, it counts as two failures. If you roll a 20 on the d20, you regain 1 hit point.

\subparagraph*{Damage at 0 Hit Points} If you take any damage while you have 0 hit points, you suffer a death saving throw failure. If the damage is from a critical hit, you suffer two failures instead. If the damage equals or exceeds your hit point maximum, you suffer instant death.

\subsubsection{Stabilizing a Creature}

The best way to save a creature with 0 hit points is to heal it. If healing is unavailable, the creature can at least be stabilized so that it isn't killed by a failed death saving throw.

You can use your action to administer first aid to an unconscious creature and attempt to stabilize it, which requires a successful DC 10 Wisdom (Medicine) check.

A \textbf{stable} creature doesn't make death saving throws, even though it has 0 hit points, but it does remain unconscious. The creature stops being stable, and must start making death saving throws again, if it takes any damage. A stable creature that isn't healed regains 1 hit point after 1d4 hours.

\subsubsection{Monsters and Death}

Most GMs have a monster die the instant it drops to 0 hit points, rather than having it fall unconscious and make death saving throws.

Mighty villains and special nonplayer characters are common exceptions; the GM might have them fall unconscious and follow the same rules as player characters.

\subsection{Knocking a Creature Out}

Sometimes an attacker wants to incapacitate a foe, rather than deal a killing blow. When an attacker reduces a creature to 0 hit points with a melee weapon attack, the attacker can knock the creature out. The attacker can make this choice the instant the damage is dealt. The creature falls unconscious and is stable. It might sustain a lingering injury as a result unless attended medically before it becomes conscious. Ranged weapon attacks can knock a creature out, but the creature is always severely wounded.

\subsection{Temporary Hit Points}

Some spells and special abilities confer temporary hit points to a creature. Temporary hit points aren't actual hit points; they are a buffer against damage, a pool of hit points that protect you from injury.

When you have temporary hit points and take damage, the temporary hit points are lost first, and any leftover damage carries over to your normal hit points. For example, if you have 5 temporary hit points and take 7 damage, you lose the temporary hit points and then take 2 damage.

Because temporary hit points are separate from your actual hit points, they can exceed your hit point maximum. A character can, therefore, be at full hit points and receive temporary hit points.

Healing can't restore temporary hit points, and they can't be added together. If you have temporary hit points and receive more of them, you decide whether to keep the ones you have or to gain the new ones. For example, if a spell grants you 12 temporary hit points when you already have 10, you can have 12 or 10, not 22.

If you have 0 hit points, receiving temporary hit points doesn't restore you to consciousness or stabilize you. They can still absorb damage directed at you while you're in that state (preventing additional failed death saving throws if the damage taken is reduced to zero), but only true healing can save you.

Unless a feature that grants you temporary hit points has a duration, they last until they're depleted or you finish a long rest.

\section{Mounted Combat}

A knight charging into battle on a warhorse, an arcanist casting spells from the back of a griffon, or a priest soaring through the sky on a pegasus all enjoy the benefits of speed and mobility that a mount can provide.

A willing creature that is at least one size larger than you and that has an appropriate anatomy can serve as a mount, using the following rules.

\subsection{Mounting and Dismounting}

Once during your move, you can mount a creature that is within 5 feet of you or dismount. Doing so costs an amount of movement equal to half your base walking speed. For example, if your speed is 30 feet, you must spend 15 feet of movement to mount a horse. Therefore, you can't mount it if you don't have 15 feet of movement left or if your speed is 0.

If an effect moves your mount against its will while you're on it, you must succeed on a DC 10 Dexterity saving throw or fall off the mount, landing prone in a space within 5 feet of it. If you're knocked prone while mounted, you must make the same saving throw.

If your mount is knocked prone, you can use your reaction to dismount it as it falls and land on your feet. Otherwise, you are dismounted and fall prone in a space within 5 feet it.

\subsection{Controlling a Mount}

While you're mounted, you have two options. You can either control the mount or allow it to act independently. Intelligent creatures, such as dragons, usually act independently. As a general rule, if the player is in control and deciding what the mount does, it's a controlled mount. If the GM is controlling the mount without player input, it's an independent mount.

You can control a mount only if it has been trained to accept a rider. Domesticated horses, donkeys, and similar creatures are assumed to have such training. The initiative of a controlled mount changes to match yours when you mount it and acts on your turn. It moves as you direct it, and it has only three action options: Dash, Disengage, and Dodge. A controlled mount can move and act even on the turn that you mount it. If you dismount, it cannot be mounted by anyone else until the beginning of your next turn.

An independent mount retains its place in the initiative order. Bearing a rider puts no restrictions on the actions the mount can take, and it moves and acts as it wishes. It might flee from combat, rush to attack and devour a badly injured foe, or otherwise act against your wishes.

In either case, if the mount provokes an opportunity attack while you're on it, the attacker can target you or the mount.

\section{Underwater Combat}

When adventurers pursue sahuagin back to their undersea homes, fight off sharks in an ancient shipwreck, or find themselves in a flooded dungeon room, they must fight in a challenging environment. Underwater the following rules apply.

When making a \textbf{melee weapon attack}, a creature that doesn't have a swimming speed (either natural or granted by magic) has disadvantage on the attack roll unless the weapon is a dagger, javelin, shortsword, spear, or trident.

A \textbf{ranged weapon attack} automatically misses a target beyond the weapon's normal range. Even against a target within normal range, the attack roll has disadvantage unless the weapon is a crossbow, a net, or a weapon that is thrown like a javelin (including a spear, trident, or dart).

Creatures and objects that are fully immersed in water have resistance to fire damage.

\section{Objects}

When characters need to saw through ropes, shatter a window, or smash a vampire's coffin, the only hard and fast rule is this: given enough time and the right tools, characters can destroy any destructible object. Use common sense when determining a character's success at damaging an object. Can a armsman cut through a section of a stone wall with a sword? No, the sword is likely to break before the wall does.

For the purpose of these rules, an object is a discrete, inanimate item like a window, door, sword, book, table, chair, or stone, not a building or a vehicle that is composed of many other objects.

\subsection{Statistics for Objects}

When time is a factor, you can assign an Armor Class and hit points to a destructible object. You can also give it immunities, resistances, and vulnerabilities to specific types of damage.

\subparagraph*{Armor Class} An object's Armor Class is a measure of how difficult it is to deal damage to the object when striking it (because the object has no chance of dodging out of the way). The Object Armor Class table provides suggested AC values for various substances.

\begin{figure}[htb]
\begin{DndTable}[header=Object Armor Class]{Xl}
    \textbf{Substance} & \textbf{AC} \\
    Cloth, paper, rope  & 11 \\
    Crystal, glass, ice & 13 \\
    Wood, bone          & 15 \\
    Stone or force      & 17 \\
    Iron, steel         & 19 \\
    Mithral             & 21 \\
    Adamantine          & 23 \\
\end{DndTable}
\end{figure}

\subparagraph*{Hit Points} An object's hit points measure how much damage it can take before losing its structural integrity. Resilient objects have more hit points than fragile ones. Large objects also tend to have more hit points than small ones, unless breaking a small part of the object is just as effective as breaking the whole thing. The Object Hit Points table provides suggested hit points for fragile and resilient objects that are Large or smaller.

\begin{figure}
\begin{DndTable}[header=Object Hit Points]{Xll}
        \textbf{Size} & \textbf{Fragile} & \textbf{Resilient} \\
        Tiny (bottle, lock) & 2 (1d4) & 5 (2d4) \\
        Small (chest, lute) & 3 (1d6) & 10 (3d6) \\
        Medium (barrel, chandelier) & 4 (1d8) & 18 (4d8) \\
        Large (cart, 10-ft.-by-10-ft. window) & 5 (1d10) & 27 (5d10) \\
\end{DndTable}
\end{figure}

\subparagraph*{Huge and Gargantuan Objects} Normal weapons are of little use against many Huge and Gargantuan objects, such as a colossal statue, towering column of stone, or massive boulder. That said, one torch can burn a Huge tapestry, and an \nameref{inc:earthquake} incantation can reduce a colossus to rubble. You can track a Huge or Gargantuan object's hit points if you like, or you can simply decide how long the object can withstand whatever weapon or force is acting against it. If you track hit points for the object, divide it into Large or smaller sections, and track each section's hit points separately. Destroying one of those sections could ruin the entire object. For example, a Gargantuan statue of a human might topple over when one of its Large legs is reduced to 0 hit points.

\subparagraph*{Objects and Damage Types} Objects are immune to poison and psychic damage. You might decide that some damage types are more effective against a particular object or substance than others. For example, bludgeoning damage works well for smashing things but not for cutting through rope or leather. Paper or cloth objects might be vulnerable to fire and lightning damage. A pick can chip away stone but can't effectively cut down a tree. As always, use your best judgment.

\subparagraph*{Damage Threshold} Big objects such as castle walls or particularly tough objects such as constructs of magical force often have extra resilience represented by a damage threshold. An object with a damage threshold has immunity to all damage unless it takes an amount of damage from a single attack or effect equal to or greater than its damage threshold, in which case it takes damage as normal. Any damage that fails to meet or exceed the object's damage threshold is considered superficial and doesn't reduce the object's hit points.
