\chapter{Appendix C: Worked Examples}\label{ch:examples}
This chapter contains examples of how certain parts of play are intended to work. They should be taken as (hopefully) illuminating examples only, not definitive statements of how play must go or in any way restrictive.

Throughout this chapter, Gary is always the GM, and players are
\begin{itemize}
	\item Alice, playing Alakabeth, a female dwarven Life priest of Melara.
	\item Beth, playing Barkor, a male silver dragonborn Defender armsman.
	\item Charlie, plaing Charleze, a female halfling rogue.
	\item Dave, playing Delenor, a male gwerin book-mage arcanist.
\end{itemize}

\section{Character Creation}\label{examples:character-creation}
\textbf{Situation:} It's session 0, and the group is sitting around the table coming up with their characters. Gary has said that the campaign will be starting in Crisial City, and will at least begin with chasing down rumors of a necromancer in the hills to the south.

\textbf{Alice:} Well, I like playing supporting characters. \textit{Looks through the class summaries.} Oh, hey, a priest looks nice for what I want. I'll pick that.

\textbf{Gary:} Ok, what lineage were you thinking? And priests are particularly tied to individual Ascendants, usually gods. Do any of the ones in \nameref{ch:the-world} look interesting?

\textbf{Alice:} Hmm...\textit{leafs through the descriptions}...I like the idea of being a dwarf. But not a traditional one. So my lineage will be Dwarf (surfacer, for reference), and my culture will be...Cosmopolitan. I think I'll draw power from Melara, Lady of Mercy. So that's +1 Constitution and +1 Wisdom for my ability scores, with darkvision, stone's endurance, dwarven resilience, coin-counter, and babyl-dweller. I'll take yonwach, the gwerin language, as my free pick from my culture.

\textbf{Alice:} As for background---I think Alakabeth (that's going to be her name) grew up in an orphanage run by some Melaran clergy here in Crisial and became chosen after serving in the shrine. So I'm going to pick the Shrine-keeper background to reflect that. That gives me a few things...\textit{notes them down}.

\textbf{Alice:} As for personality---let's see. She's not a particularly forceful person, she prefers to stand in the back and be unnoticed. Very devout, doesn't like Melara being defamed. Or any of the gods, really. Kinda cautious. So her ideal is Faith: I will be the best example of my religion for all to see. Her bond is the sisterhood and church that raised her--she'll protect it and obey if they ask her to do things. Her flaw is her lack of forcefulness--she's hesitant to engage in arguments or combat even when that's best. Why is she adventuring? Well...maybe she had a dream that she believes was a call from Melara to go fight the undead (since Melara \textit{hates} the undead)? Yeah, that will work.

\textbf{Alice:} Ok, since I want to stand back and support, that means I'll pick Life as my Domain. Gary, we're using standard array, right? That means I should put my highest score, a +3 (after the culture contribution) into Wisdom, a +2 into Dexterity, the +1 (+1 from being a dwarf, so +2) into Constitution. The others...well...she's bookish, so I'll put the other +1 into Intelligence. Not very good with people, but not horrible. So the +0 into Charisma. She's not a strong person, so the -1 goes into Strength. For cantrips, I'll take...\nameref{spell:guidance}, \nameref{spell:grave-touch}, and \nameref{spell:light}. I'll pick spells each day, but say...\nameref{spell:bless} and \nameref{spell:cure-wounds} for starters.

\textbf{Alice:} As for proficiencies: She gets Insight, Religion, and Persuasion from culture + background. She'll pick History and Medicine from being a priest. Tools: She's got wood-carvers tools from her background, and she'll take the hand drum and dice set from the culture. Languages---common, of course. Everyone gets that. But then yonwach (from culture) and lucian from her background. Yeah, she doesn't know dwarven. She wasn't raised among dwarves.

\textbf{Alice:} For equipment, I'll take the leather armor and a mace, with a priest's pack and a light crossbow. My holy symbol is a snowflake emblem around my neck.

\textbf{Alice:} With a +2 Constitution, that gives me 10 HP. Leather with a +2 Dexterity gives me an AC of 13. +3 Wisdom and +2 proficiency means my spellcasting DC is 8 + 3 + 2 = 13. My spell attack modifier is 3 + 2 = 5. I've got only one hit die, so 1d8. I've got 1 Stamina, 4 Aether, and an aether limit of 2. Ok, I think that's enough for me to start play.

\textit{The rest of the party makes their characters.}

\section{Surprise}\label{example:surprise}
\textbf{Gary:} You're moving through the woods that are known to contain the nefarious Red Eagle bandit group. How do you want to do this?

\textbf{Charlie:} Let's try to not be seen. We're moving slowly and stealthily. I'll scout ahead. 

\textbf{Group:} \textit{Nods in agreement}

\textbf{Gary:} Ok, everyone roll a Dexterity (Stealth) check.

\textbf{Group:} \textit{rolls}.

\textbf{Alice:} \textit{Thinks: I rolled a 12, and Alakabeth has a +1 Dexerity modifier and no proficiency.} Alakabeth got a 13.

\textbf{Beth:} \textit{Thinks: Barkor is in heavy armor, so he has disadvantage. The dice were a 4 and a 20. Darn. He does have proficiency and a +2 modifier, so that's ok but not great.} Barkor got an 8, at disadvantage.

\textbf{Charlie:} \textit{Thinks: Charleze has expertise in Stealth and a +4 modifier. I rolled a 16, so I'm probably hidden}. Charleze got a 24.

\textbf{Dave:} \textit{Thinks: No proficiency, but Delenor has a cloak of elvenkind, so he has advantage and a +2 modifier. Rolled a 2 and an 18. Great!} Delenor got a 20, non-natural, at advantage.

\textbf{Gary:} You move through the woods with fair grace for an hour or so, taking care not to step on any twigs. Barkor, you're not so successful as your armor kinda glitters in the sun--you scare some rabbits and birds. Soon you see campfire smoke rising through the trees and hear the Red Eagle bandits going about their daily routine. Charleze, since you're scouting ahead, you catch sight of them first. They do have a watch, but it seems somewhat apathetic. Three bandits in the open, with tents for about 10 total. How do you want to handle this?

\textbf{Charlie:} I motion the group to silence. Let's ambush the guards. \textit{The group agrees.}

\textbf{Gary:} Ok, \textit{draws out a battle map with the position of the guards marked, as well as trees.} Place yourselves behind cover from the guards. Looks like one of the guards (\textit{indicates on the map}) was more alert than most. The others have disadvantage on their Passive Perception due to distraction. \textit{Checks the bandits' Passive Perception: all the bandits normally have 10. With disadvantage, that makes 2 of them have a result of 5 and the other has a result of 10.} The two inattentive ones are \nameref{condition:surprised}; the alert one notices Barkor trying to hide and is not surprised. Everyone, roll initiative. \textit{The players and the GM all roll initiative, resulting in an order of Charleze, Delenor, the bandits, Barkor, then Alakabeth}.

\textbf{Charlie:} Charleze shoots her crossbow at the alert guard, at advantage because I'm hidden from it. \textit{Rolls, hits, doesn't deal enough damage to kill the bandit because of all 1s on the damage dice. Drat.}

\textbf{Dave:} Delenor shoots a \nameref{spell:produce-flame} at the only bandit in range, one of the surprised ones. \textit{Rolls at advantage because hidden, hits, kills the bandit.}

\textbf{Gary:} Ok, the alert bandit goes. He yells to the camp, and you hear movement and shouts of alarm. He then draws his light crossbow and fires at the target he saw first, Barkor. \textit{Rolls, misses.} He misses. The other bandits do nothing, but recover from their surprise at the end of their turn.

\textbf{Beth:} Well, since Barkor isn't hidden, he's going to charge the alert bandit. I'll move my speed closer and throw a javelin. \textit{Rolls normally, hits, kills the bandit.}

\textbf{Alice:} Alakabeth casts \nameref{spell:bless} on herself, Barkor, and Charleze.

\textit{The rest of the combat plays out as the now-alerted bandits pour out of the camp...}