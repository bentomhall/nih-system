\section{Downtime Rules}
This section contains optional rules a GM can use to enhance downtime---that is, time spent not adventuring. Many campaigns will not have significant downtime, and so these rules won't apply. Others may have substantial amounts (months or more), and so a way to spend time and treasure to further character development or aid their next adventure is welcome. And some players love crafting items. The base rules already include rules for crafting and training, but they're very threadbare. These are intended to add a bit more choice and meat to those processes, while also including ways that characters not interested in training or crafting can contribute to the group.

\subsection{What is Downtime?}
Downtime is any period one day or longer spent \textit{not} adventuring. Out of character, downtime activities are designed to cover those times when a GM might say "You'll have three weeks before ... what are you doing for that time?" Some features may allow limited progress on downtime activities (usually training and crafting) when traveling between adventure locations, but usually your downtime activities are on pause during those times.

Downtime is \textit{supplemental} activity. The focus of the game is on adventuring. It's not a game about building a crafting empire or running a business---those may happen in the background but aren't the focus of your on-camera, in-session time. As a result, downtime activities are necessarily abstracted and generalized, with the emphasis being on quick resolution and balancing simplicity with making interesting decisions and getting the whole party involved.

\begin{figure}[htb]
\begin{DndComment}{Disallowing use of currency}
  As a GM, it's your responsibility to make sure that uses of downtime currency or even downtime activities make sense, whether those might be. For example, Favors gained in a big city might not translate well to a tiny town that has been isolated for generations. There may be no way that a character could have discovered a certain fact (in keeping with the narrative). A goblin tribe might not use money, so Trading might not make sense there. Etc. When you make this determination, the attempted use of the currency fails and the currency is \textbf{not} expended. But generally, it's best to be generous. If you can imagine any possible scenario where it makes sense, let them do it. Don't look for ways it may not make sense---those are always too easy to find.
\end{DndComment} 
\end{figure}

\subsection{Downtime Currencies}
These downtime rules involve three different "currencies"---abstracted "points" that can be accumulated, exchanged, or spent to gain benefits in other areas. There is also a pair of trackers (one for crafting and one for training) that accumulate until you finish whatever training or crafting project you may be working on, called Progress. These currencies are Money (gold pieces, silver pieces, etc., your normal coins gained by selling treasures, completing quests, and finding monster hoards), Favors, and Discoveries.

\subsubsection{Money}
Most downtime activities require money in some form, if for nothing else but paying living expenses. The cost of downtime activities is measured in gold pieces (gp). The only reliable way to make gold during downtime is engaging in the Trading activity, which might represent anything from working jobs to buying and selling goods for profit. Money is also the only downtime currency that you can earn \textit{outside} of downtime, and it's the only reliable way to buy things during adventuring. You can't generally pay for a new sword with a Favor, although you might get a better deal or find a specialized crafter via one.

Outside of downtime, as long as you're in a location with shops, you can generally buy anything in the \smartref{ch:equipment}{Equipment} tables within reason at the listed prices. Small towns may not have specialized crafters---you're unlikely to find magic or masterwork items outside of a capital city, for example. But general spell components, mundane equipment, vehicles, and tools? Those are commonplace. No need to haggle---just mark off the currency and add the item (telling your GM you're doing so, of course). Basic potions of healing count as mundane equipment.

\subsubsection{Favors}
Favors represent transactional interpersonal connections. Think of a favor as a single use "hey, you owe me one" item. A letter of recommendation or introduction, a name to drop to convince someone to help, a pointer to an underground market, a debt you can call in with someone. Use it and its gone. You gain Favors mostly by engaging in the Socializing activity, although Trading has a small chance of granting Favors (in addition to or instead of its usual effects). You can only store favors up to your proficiency bonus. During downtime, you can use a Favor (for yourself or someone else) to ensure at least one success on a Crafting, Trading, or Training activity. 

Outside of downtime, you can spend Favors to establish "hey, I know a guy" or "I can get us in there". The GM may disallow this use (such as trying to use a Favor to access the lich's lair), but generally you can use Favors to set up friendly (or at least indifferent) relations. For example, you might use a Favor to gain access/acceptance at the tavern housing the local criminal organization's clearinghouse.

\subsubsection{Discoveries}
Discoveries represent facts or plans or other small things you've uncovered during the Research activity. Occasionally you might make a Discovery while crafting as well. These are never world-shaking things, but they might represent a bit of knowledge about a weather pattern, a monster habitat, some historical fact, a genealogy, or something similar. When you make the discovery, you don't specify exactly what it is---instead, it stands for something unspecified that you may have discovered. When you use it, that's when it gets fixed in place. Each Discovery is single-use, and you can only store Discoveries up to your proficiency bonus. During downtime, you can use a Discovery (for yourself or someone else) to ensure at least one success on a Trading or Training activity.

Outside of downtime, you can spend a Discovery to gain extra knowledge from your GM about a specific situation you're facing (subject to his approval). Effectively, spending a Discovery is the equivalent of making an Intelligence check (using Arcana, History, or Nature) or sometimes a Wisdom (Insight) check and getting the maximum result. It won't get you, for example, the code to bypass the Secret Trap on the Secret Chest containing the Secret Plot Token of Doom, but it might get you the answer to a riddle trap or the name of the king who started the war.

\subsubsection{Progress}
Crafting and Training act a bit differently than the other activities. Instead of generating (or consuming) currencies to produce other (abstract, unspecified) currencies, they involve specifying exactly what kind of thing you're trying to craft or learn how to do (as described in their section). The outcome of the activity is 0 or more units of Progress towards that specific goal. Progress is not fungible--you can't trade Progress in learning how to speak Elvish for Progress crafting a suit of plate armor (or vice versa). Or even trade Progress in Elvish for Progress in Perception.

See each activity's section for the necessary progress to reach each kind of goal.

\subsection{Activities}
The general rules for any activity are the same---at the end of each period of downtime, make one downtime check per workweek. Roll a number of d6s equal to the related ability score for that activity. For each die that rolls a 5 or 6, you accumulate one success (point of Progress, one Favor, one Discovery, or one unit of Money as described). If there is a required skill or tool proficiency and you have expertise in that skill or tool, you can reroll a number of failures up to half your proficiency bonus.

Each activity has one or more expended currency and may have one associated currency. When you make a downtime check and have no successes, you or someone in your party may use one unit of the expended currency to convert one failure into a success. Whenever you make a downtime check and two or more dice show the same value (ie doubles), you earn one point of the associated currency up to a maximum of one per check.

Thus, if you roll 4d6 on Crafting check and the results are 3, 3, 4, 4, you had no successes and can spend a Favor (the associated currency) to gain one. But since you had at least one doubled result (the 3s, and 4s), you gain one Discovery (the associated currency for Crafting).

\subsubsection{Crafting}
\subparagraph*{Produced Currency} Crafting Progress.
\subparagraph*{Expended Currency} Favors. You can call in favors from other crafters to help out, guaranteeing at least one success on a downtime check.
\subparagraph*{Associated Currency} Discoveries. Crafters will sometimes discover new secrets about their craft they can parlay for other favors or knowledge.

Crafting requires a few things---
\begin{itemize}
\item \textbf{Proficiency} You must have proficiency in the relevant tool. For wondrous items that don't fit into any regular proficiency, use Arcana as the relevant proficiency. For magic items or masterwork equipment that list a required minimum proficiency, your proficiency bonus must be at least that high to craft the item successfully.
\item \textbf{Proper Space and Equipment} You need proper workspace to craft efficiently. You can craft on the road as long as you have a set of tools, but your progress is reduced. Generally, this means you need access to a forge (for metalwork), a loom or spinning wheel (for textiles), enough space and fuel to run your alchemy equipment, etc.
\item \textbf{Appropriate materials} For magic items with specific formulae, you need the listed special materials to begin crafting; they are consumed when you finish.
\item \textbf{Formula} For magic items, you need a formula. Some are known and listed in the item entry, others need to be discovered.
\end{itemize}

\begin{DndTable}[header=Crafting Abilities]{llX}
  \textbf{Proficiency} & \textbf{Ability} & \textbf{Example Items} \\
  Alchemist's supplies & INT & Alchemist fire, potions, oils \\
  Arcana & INT & Miscellaneous magical items, scrolls, spell stones. \\
  Brewer's supplies & WIS & Alchemist fire, alcohol, tisanes \& infusions \\
  Calligrapher's supplies & DEX & Books, documents, scrolls \\ 
  Carpenter's tools & STR & Wooden buildings, furniture, ships (including repairs) \\
  Cartographer's tools & INT & Maps \\
  Cobbler's tools & DEX & Footwear \\
  Cook's utensils & WIS & Food and drink. \\
  Glassblower's tools & DEX & Glass items (flasks, bottles, windows, etc) \\
  Jeweler's tools & DEX & Rings, necklaces, crystalline items. Items of precious metals. \\
  Leatherworker's tools & STR & Armor, bags, straps, general leather goods \\
  Mason's tools & STR & Stonework, including carving \\
  Painter's supplies & DEX & Painted art \\
  Potter's tools & DEX & Pottery \\
  Smith's tools & STR & Metalwork of iron, steel, etc. Weapons, armor. \\
  Tinker's tools & DEX & Gears, springs, mechanisms. \\
  Weaver's tools & STR & Textiles. Includes sewing. \\
  Woodcarver's tools & DEX & Staffs, wands, rods, arrows, bolts. Decorative carving of wood.
\end{DndTable}

Each time you gain Progress, you should check to see if you've accumulated enough to complete your goal, as shown on the tables below. If you have, you can expend any necessary money or special materials (for crafting) and then your goal is accomplished---you've crafted the item or gained the noted proficiency. Consumables (such as potions or scrolls) cost half the listed progress and money, but cannot be made Masterwork. Very Rare and Legendary items cannot generally be crafted under normal conditions. Individual items specify their crafting cost---for mundane items it is half the listed sale price. For otherwise mundane items made from special materials, the special material specifies an additional cost above the regular cost to make the mundane item, and you must specifically provide the requisite amount of special material.

If a mundane item costs less than 100 gp, you can make as many as would be paid for by 100 gp (rounded down) per Progress gained, up to a maximum of 8. Ammunition is priced per bundle (20 for most, 50 for blowgun needles).

Two or more characters with relevant tool proficiencies can combine their efforts on a single piece (assuming there's enough space, as decided by the GM). They each roll separately and contribute any Progress toward the same goal.

\begin{figure}
\begin{DndComment}{GP Cost}
  For custom mundane items not mentioned, the actual gp cost is going to require negotiation, although the tables of loot items can be a useful starting point. Crafting an intricate jewel-encrusted figurine is going to cost more than whittling a small carving.

  The crafting downtime activity is designed to be used for larger things, rather than trinkets or extremely low-value items without direct adventuring use. For those, it's often best to allow them to be crafted narratively as part of regular activity. That is, simply say "Ok, you make those things".
\end{DndComment}
\end{figure}

\begin{DndTable}[header=Crafting]{lcc}
  \textbf{Tier} & \textbf{Progress} & \textbf{Masterwork}  \\
  Mundane & 1 per 100 gp & 1 per 25 gp \\
  Journeyman (Common) & 2 & 8 \\
  Adventurer (Uncommon) & 20 & 40 \\
  Hero (Rare) & 100 & 400 \\
\end{DndTable}

\subsubsection{Research}
\subparagraph*{Produced Currency} Discoveries.
\subparagraph*{Expended Currency} Money. Research costs money. You can expend 100 gp to guarantee a successful workweek of Research.
\subparagraph*{Associated Currency} Favors. Some secrets are valuable to other people more than yourself, and can be bartered for access.

Research requires research materials. Either a laboratory or a source of information (such as a library) and an appropriate proficiency (Arcana, History, Nature, or Religion). The associated ability score for research is Intelligence.

\subsubsection{Socializing}
\subparagraph*{Produced Currency} Favors.
\subparagraph*{Expended Currency} Discoveries or Money. Lifestyles must be maintained, parties thrown, and bribes (in cash or information) must be spread around. You can expend 200 gp or one Discovery to guarantee a successful workweek of Socializing. 
\subparagraph*{Associated Currency} None.

Socializing requires a social scene, but it can occur in humble roadside taverns or grand palaces. Generally, you need proficiency in one of the following to gain benefits from socialization:
\begin{itemize}
  \item Brewer's supplies, cook's utensils, or painter's supplies
  \item a gaming set
  \item a musical instrument
  \item a forgery kit (to forge an invitation)
  \item or Performance
  \item a background appropriate to the situation (Aristocrat for high society, merchant for trade talks, etc.)
\end{itemize}

The associated ability score for socializing is Charisma.

\subsubsection{Trading}
\subparagraph*{Produced Currency} Money.
\subparagraph*{Expended Currency} Discoveries or Favors. 
\subparagraph*{Associated Currency} Discoveries or Favors.

Trading is the most flexible activity and the only way to reliably make money from downtime. You might be selling more specific treasures (including magic items), working a short-term job, or engaging in other mercantile business.

Trading requires proficiency in one of the following:
\begin{itemize}
  \item an artisan's tool
  \item herbalism kit
  \item navigator's tools
  \item a vehicle
  \item Animal Handling, Insight, or Persuasion.
\end{itemize}

\subsubsection{Training}
\subparagraph*{Produced Currency} Training Progress.
\subparagraph*{Expended Currency} Discoveries or Favors. Lost techniques, training manuals, and similar knowledge can accelerate your training, as can access to a teacher.
\subparagraph*{Associated Currency} None.

You can only train in a single thing at a time. Training can increase your proficiency in a specified area. For skills and tools, you can gain expertise (double proficiency) if you are already proficient by paying the Advanced cost. If you train for an armor proficiency, it must be of the lightest type you aren't already proficient with. For example, an Arcanist can only train for a form of light armor. You gain the Armored Caster feature for that one specific type of armor or shield (as well as any you already have it for). For languages, common languages use the Regular cost and rare languages (such as abyssal) use the Advanced cost. For weapons, gaining proficiency in a Simple weapon uses the regular cost and a Martial weapon uses the Advanced cost.

\begin{DndTable}[header=Training]{llcc}
  \textbf{Proficiency} & \textbf{Regular} & \textbf{Advanced} \\
  Armor (specific type) & 8 & --- \\
  Language & 5 & 10 \\
  Shields & 5 & --- \\
  Skill &10 & 40 \\
  Tool &5 & 20 \\
  Weapon &3 & 6 \\
\end{DndTable}

Consult the list below to determine what ability score is relevant:
\begin{enumerate}
  \item[] Armor: STR for heavy or medium armor, DEX for light armor
  \item[] Language: INT
  \item[] Shields: STR
  \item[] Skill: The default ability score for that skill
  \item[] Tool: Use the one listed on the Crafting Abilities table above
  \item[] Weapon: Use the main attack ability (STR for melee, DEX for ranged, STR or DEX for finesse).
\end{enumerate}

\begin{figure}
  \begin{DndComment}{On the Road}
    Limited amounts of crafting and training can be done on the road, subject to GM approval. To gain the benefit of a day's work, you need to spend at least 8 hours \textit{other} than during a long or short rest working on the goal in question. Due to the focus switching and interruptions, as well as the limited space and tools, you gain progress at half the normal rate---it takes 16 days of travel-based work to equal a single workweek. Some features may modify this rate. You cannot expend any downtime currency on crafting or training during an adventure.

    The only exception is brewing potions of healing, which progresses normally.
  \end{DndComment}
\end{figure}