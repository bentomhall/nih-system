\chapter{Introduction}
\section{Core Principles of NIH}

\begin{DndSidebar}[float=b]{What is Magic?}
    That word is used throughout these documents, and deserves a little more reflection. "Magic", as used here, is all those things that separate the fictional world in which the game takes place from the real world in which the players act. Spells? Magic. Dragons? Magic. Heroic mortals breaking "normal" constraints? Magic. \textbf{It's magic all the way down.} Unlike Dungeons and Dragons, I attempt to be more precise in my use of this term. So abilities that counter \textit{specific forms} of magic will be written precisely.
\end{DndSidebar}

\subsection{Core Assumptions}
The core assumptions are that:
\begin{enumerate}
\item \textbf{The world is thoroughly magical.} Magic is in and through everything and everyone. There is no mundane (in the sense of "bound by all the principles of real world"), at least as far as adventurers and adventuring goes.
\item \textbf{Adventuring is normalized.} Adventurers are a known social "group", even if a disreputable one.
\item \textbf{The world responds to mortal efforts.} In these sorts of worlds, training really hard physically \textit{is} a form of access to magical power just as much as learning magic gestures and words in a book. But both of these plateau quite hard for most people.
\item \textbf{The world has its own logic, not real-world science.} Players should be able to rely on \textit{surface} similarity between the real world and the fictional one. Things will fall when dropped. Water flows downhill. Paper burns and fire hurts. Fire turns water to steam. If it looks like a cow, walks like a cow, and moos like a cow, it's probably a cow. But the \textit{reasons} why these happen are not at all guaranteed to be the same. The further and deeper you get into modern scientific understanding of the world, the less this will apply to the fictional world. A fire spell will burn flammable things...but not because of exothermic oxidation reactions. There might not even be oxygen, and fires might burn just fine even in a "vacuum"...if such a thing even exists. Atoms, molecules, cells, DNA--these sorts of things are not at all guaranteed to exist.
\item \textbf{PCs are among the few that break the normal (soft) limits.} Most people in the world are relatively low power, and will never grow much further. PCs are not bounded in the same way.
\item \textbf{World-ending threats are rare, but problems are common.} Most threats a party will face don't have whole-world-changing consequences. But they do change the local world.
\item \textbf{PCs change the world...but not by pushing buttons.} It's expected that the outcome of the PCs adventure will be changes to the status quo. The world \textit{should} react to their actions. But the PCs don't have powers that allow them to directly do that on the large scale. Large-scale changes happen because of the relationships the PCs form with others, the movements they support, and the people they affect.
\item \textbf{PCs are usually the underdogs.} Either due to numbers (the enemy has an army) or due to individual power. PCs generally win against significant foes not by overwhelming the opponents with bigger numbers but by teamwork, guile, good strategy, finding the opponents' weak spots, building alliances with others, etc.
\item \textbf{Accuracy is bounded.} What does this mean? It means that attack bonuses, armor class, and saving throw and ability check DCs and modifiers are not assumed to grow significantly (relative to the d20's effect) with level. They may change and grow, but it should be hard if not impossible to "move off the d20" permanently in most cases. Monsters that hit PCs at level 1 should still be able to hit some appreciable fraction of the time against level 20 PCs, even if that fraction is smaller. Unsaveable saving throws shouldn't usually happen unless the PC or monster has a strongly negative modifier. Neither should "unmissable" saving throws unless class features or magic is involved, and then rarely. PCs and monsters scale mostly in three ways--(a) having more health to absorb hits and stay standing, (b) dealing more damage (usually via more attacks rather than bigger single attacks, but this varies), and (c) having abilities that give different ways to approach the problem entirely (horizontal growth). As a result of this, magic items no longer give +1 (etc) to AC, saving throw DCs, or attack rolls. Such passive stacking bonuses are rare if they exist at all.   
\end{enumerate}

\subsection{How to Play}
NIH requires one player to assume the role of Game Master (hereafter GM), while the remaining players (usually 2-4 in number) create and control individual characters, called Player Characters (PCs) or "the party". It's assumed that the PCs will work together cooperatively--the basic unit of the game is the party, not the individual. The GM's role is several-fold:
\begin{enumerate}
    \item He voices and decides the actions for all the non-Player Character characters (NPCs, for short). If the party is fighting someone, the GM makes decisions for that antagonist. This doesn't mean the GM is antagonistic toward the \textit{players} or is trying to kill the PCs, but the characters he or she controls most certainly may be antagonists.
    \item The GM is the voice of the setting and the narrative. Nothing happens in-game until he narrates it, and he is the eyes through which the players experience the world and the story they are collectively creating. Which makes it imperative that the GM is not biased toward or against any of the players and should, when speaking as the "voice of the world", never lie to the players. NPCs may lie and try to deceive, the GM as the GM should not. Of course, when illusions and compulsions are in play, what a PC sees or experiences may not be real. But what is described should be what the PC experiences.
    \item The GM is the rules engine for the game. These rules are inputs and guidelines, not mandates. It's the GM's role to decide how, when, and even if the rules apply and to adjust on the fly. Many GMs rely on group consensus for rule modifications and rulings, but at the end of the day, the GM is the final decider. If a player says that his PC acts a certain way, the GM is responsible for deciding how to execute that action and what success or failure looks like, as well as narrating the result. 
    \item Often, the GM is responsible for coming up with the world and/or the set of events surrounding the PCs. Unlike a video game, there may not be a "main quest" that the PCs must follow, but the GM is the one placing things in the world for them to find and interact with.
\end{enumerate}

\subsubsection{The Basic Game Loop}
The most basic, most generic pattern of play is as follows:
\begin{itemize}
    \item The GM describes a scenario, including what the PCs see, hear, smell, etc.
    \item He or she then asks either a single player or the group "What do you do"? (or something similar).
    \item That player or group of players then describes what action or actions they want their PC to attempt. This description may be vague ("I attack the giant with my sword!") or specific ("Gerrold lunges forward, slashing upward at the giant's leg with his sword"), but must always provide
    \begin{enumerate}
        \item An indication of what the player wants to have happen (a goal)
        \item An indication of how the player character is achieving that goal (a method)
        \item And any pertinent facts, such as abilities being used, amounts of bribes being offered, etc. that may change the resolution of the action.
    \end{enumerate}
    \item The GM, often after discussing details with the player, decides how the action will be resolved. This often involves some sort of die roll for randomization purposes. These rules are full of resolution mechanics such as attack rolls, ability checks, saving throws, etc. Many actions don't need any explicit resolution method--they just succeed. It is not expected that you have to roll to tie your shoes in the morning. Actions that have little opposition (chance of failure), are a core fictional competency of the character in question (the sailor can climb masts in calm weather), or where failure doesn't have meaningful consequences that change the situation (picking a practice lock in the safety of your home) rarely, if ever, call for active resolution.
    \item Once the action is resolved, the GM narrates the changes to the situation and the loop continues.
\end{itemize}

Many times, multiple actions can be resolved simultaneously and the GM may ask multiple players for their actions and decide how to order their resolutions. The first person to speak doesn't necessarily go first--that depends on the entire situation.

At times when exact sequencing is important (such as combat), the GM may call for initiative checks (see \nameref{sec:initiative}). When in initiative order, players take actions from highest initiative downward. In the game world, they're all acting within the same short period of time--one complete pass through the initiative order, called a \textbf{round} represents about 6 seconds of game time, but they are sequenced for ease of play.

\subsection{Rolling the Dice}
The roll of dice (pun intended) in NIH is to provide a way of resolving actions when the outcome is uncertain and the outcomes are interesting. The dice never dictate the result but shouldn't be called on if one of the outcomes isn't acceptable. NIH calls for four major categories of dice rolls: attack rolls, saving throws, ability checks, and damage rolls.

Attack rolls, saving throws, and ability checks all follow the same format. When the GM asks for one of them in response to a proposed course of action, roll one twenty-sided die (1d20, for short) and add any modifiers (usually an ability score and possibly your proficiency bonus) and tell the GM the result. If the total is equal to or higher than the pre-set target number of the check, the roll is a success; generally this means you either succeed at what you're trying to do or suffer no or less of a consequence from what someone else is trying to do. In the rare case where both parties make an opposed check (both rolling and adding modifiers), ties go to the defender (and generall nothing happens).

\subparagraph*{Advantage and Disadvantage} Some abilities and situations make a result more probable (giving you an advantage) or improbable (giving you a disadvantage). These will generally be referred to as "making a roll at advantage/disadvantage". See \nameref{sec:advantage-and-disadvantage} for the details--in short, advantage and disadvantage cancel out to a normal roll; if you have either, roll the d20 twice instead of once and keep the higher (for advantage) or lower (for disadvantage) result, adding modifiers as normal.

\subparagraph*{Attack rolls} Attack rolls use the target's armor class as a target number--if you roll equal to or higher than the target's armor class (including modifiers), you hit the target and generally roll damage. If lower, you miss and no damage is rolled.

Attack rolls can be either melee or ranged and either weapon or spell attacks. Melee attacks generally have a short range (5' or 10' for weapons with reach, although some monsters have larger reach) and can be made with enemies adjacent; ranged attacks have longer range and are dangerous to do while enemies are next to you. Weapon attacks default to either Strength (for melee) or Dexterity (for ranged), although the Finesse property and the Thrown property allows options. Spell attacks specify what ability score to add. You add your proficiency bonus if you are proficient with the weapon; you are always proficient with spell attacks you can make.

A roll of 1 on the d20 is always a miss (and nothing more), called a "critical miss". A roll of 20 on the d20 is always a hit, called a "critical hit". See \nameref{sec:critical-hits} for more details.

\subparagraph*{Saving Throws} You make saving throws (or saves) to resist or evade traps, spells, and many monster abilities. Saving throws are the defensive version of attacks. Each saving throw is linked to a particular ability score and you add that ability score to the result of the d20. You add proficiency if you have it in that kind of saving throw. Generally, Strength saves are made to avoid being knocked down or held in place, Dexterity saving throws are most common to avoid or reduce the damage from abilities affecting an area, Constitution saving throws are most common against disease, poison, and cold, as well as many environmental effects. Intelligence saving throws are mostly against psychic attacks and illusions, where figuring out that the influence isn't real removes the threat. Wisdom saving throws are made to discern external influences from your own--most spells that try to affect your mind, perceptions, or actions call for Wisdom saves. Charisma saving throws are to hold to your identity and place in the world--spells and abilities that change your shape or force you to other planes call for Charisma saving throws.

Each class is proficient in two of the six saving throws by default--usually one of Dexterity, Constitution, or Wisdom, and one of Strength, Intelligence, and Charisma. Some classes, such as the brawler, get extended saving throw proficiencies later on.

\subparagraph*{Ability Checks} Ability checks are the catch-all--if you need to make a check to determine success or failure and it isn't one of the other types, it's probably an ability check. Ability checks all involve an ability decided by the GM and may possibly include proficiency, which may come from one or more skills, tools, languages, or even weapons or armor. If you have proficiency that applies from multiple sources, you can make the check at advantage.

\subparagraph*{Damage Rolls} When an ability calls for a damage roll, it will specify how many of what size of dice to roll and what, if anything, to add to the roll. This entire total (total of the numbers on the dice plus any modifiers) is the damage roll. You can never roll less than zero damage, but modifiers may reduce the damage to zero.

\subparagraph*{Inspiration} Sometimes, the dice just aren't in your favor after you've done everything you can. The party, as befits heros, has a pool of luck they can call on, called Inspiration. Each party has a maximum number of Inspiration Points equal to the number of player characters. You start the session with this pool full, and when someone \textit{else} needs a boost, you can grant them inspiration by spending an Inspiration Point. Whether your character says some inspiring words or a plea to the heavens is up to you, but it's most moving when you do \textit{something} in-character. When you do so, they can choose to reroll one attack roll, saving throw, or ability check they had just made and take either result. If the original roll had advantage or disadvantage, so does the new roll--it completely replaces the original. \textbf{Remember, you cannot give yourself inspiration}, although the character whose player is giving you inspiration does not have to be conscious, capable of taking actions, or even alive to do so (as you're inspired by their memory).
