\section{Monsters (N)}\label{sec:monsters-n}
\subsection{Naga, Guardian}
\begin{DndMonster}[width=\textwidth + 8pt]{Guardian Naga}
\begin{multicols}{2}
\DndMonsterType{Large monstrosity}
\DndMonsterBasics[armor-class={18 (natural armor)}, hit-points={127 (15d10 + 45)}, speed={40 ft.}]
\MonsterStats{+4}{+4}{+3}{+3}{+4}{+4}
\DndMonsterDetails[saving-throws={Dex +8, Con +7, Int +7, Wis +8, Cha +8}, skills={}, damage-immunities={poison}, damage-resistances={}, damage-vulnerabilities={}, condition-immunities={charmed, poisoned}, senses={darkvision 60 ft., passive Perception 14}, languages={Celestial, Common}, challenge={10 (5,900 XP)}]
\DndMonsterAction{Rejuvenation} If it dies, the naga returns to life in 1d6 days and regains all its hit points. Only a \textit{wish} spell can prevent this trait from functioning.

\DndMonsterAction{Spellcasting} The naga is an 11th-level spellcaster. Its spellcasting ability is Wisdom (spell save DC 16, +8 to hit with spell attacks), and it needs only verbal components to cast its spells. It has the following priest spells prepared:\\nCantrips (at will): \textit{mending}, \textit{sacred flame}, \textit{thaumaturgy}\\n1st level (4 slots): \textit{command}, \textit{cure wounds}, \textit{shield of faith}\\n2nd level (3 slots) : \textit{calm emotions}, \textit{hold person}\\n3rd level (3 slots): \textit{bestow curse}, \textit{clairvoyance}\\n4th level (3 slots): \textit{banishment}, \textit{freedom of movement}\\n5th level (2 slots): \textit{flame strike}, \textit{geas}\\n6th level (1 slot): \textit{true seeing}

\DndMonsterSection{Actions}
\DndMonsterAttack[
	name=Bite,
	distance=melee,
	type=weapon,
	mod=+8,
	reach=10,
	dmg=\DndDice{1d8 + 4},
	dmg-type=piercing,
	extra={, and the target must make a DC 15 Constitution saving throw, taking 45 (10d8) poison damage on a failed save, or half as much damage on a successful one.}
]
\DndMonsterAttack[
	name=Spit Poison,
	distance=ranged,
	type=weapon,
	mod=+8,
	range=15/30,
	dmg=\DndDice{10d8},
	dmg-type=poison,
	extra={ on a failed save, or half as much damage on a successful one.}
]
\end{multicols}
\end{DndMonster}
\subsection{Spirit Naga}
\begin{DndMonster}[width=\textwidth + 8pt]{Spirit Naga}
\begin{multicols}{2}
\DndMonsterType{Large monstrosity}
\DndMonsterBasics[armor-class={15 (natural armor)}, hit-points={75 (10d10 + 20)}, speed={40 ft.}]
\MonsterStats{+4}{+3}{+2}{+3}{+2}{+3}
\DndMonsterDetails[saving-throws={}, skills={}, damage-immunities={poison}, damage-resistances={}, damage-vulnerabilities={}, condition-immunities={charmed, poisoned}, senses={darkvision 60 ft., passive Perception 12}, languages={Abyssal, Common}, challenge={8 (3,900 XP)}]
\DndMonsterAction{Rejuvenation} If it dies, the naga returns to life in 1d6 days and regains all its hit points. Only a \textit{wish} spell can prevent this trait from functioning.

\DndMonsterAction{Spellcasting} The naga is a 10th-level spellcaster. Its spellcasting ability is Intelligence (spell save DC 14, +6 to hit with spell attacks), and it needs only verbal components to cast its spells. It has the following arcanist spells prepared:\\nCantrips (at will): \textit{mage hand}, \textit{minor illusion}, \textit{ray of frost}\\n1st level (4 slots): \textit{charm person}, \textit{detect magic}, \textit{sleep}\\n2nd level (3 slots): \textit{detect thoughts}, \textit{hold person}\\n3rd level (3 slots): \textit{lightning bolt}, \textit{water breathing}\\n4th level (3 slots): \textit{blight}, \textit{dimension door}\\n5th level (2 slots): \textit{dominate person}

\DndMonsterSection{Actions}
\DndMonsterAttack[
	name=Bite,
	distance=melee,
	type=weapon,
	mod=+7,
	reach=10,
	dmg=\DndDice{1d6 + 4},
	dmg-type=piercing,
	extra={, and the target must make a DC 13 Constitution saving throw, taking 31 (7d8) poison damage on a failed save, or half as much damage on a successful one.}
]
\end{multicols}
\end{DndMonster}


\FloatBarrier
\section{Monsters (O)}\label{sec:monsters-o}
\subsection{Ogre}


\subsection{Ooze, Grey}
\begin{DndMonster}[width=\textwidth + 8pt]{Gray Ooze}
\begin{multicols}{2}
\DndMonsterType{Medium ooze}
\DndMonsterBasics[armor-class={8}, hit-points={22 (3d8 + 9)}, speed={10 ft., climb 10 ft.}]
\MonsterStats{+1}{-2}{+3}{-5}{-2}{-4}
\DndMonsterDetails[saving-throws={}, skills={Stealth +2}, damage-immunities={}, damage-resistances={acid, cold, fire}, damage-vulnerabilities={}, condition-immunities={blinded, charmed, deafened, exhaustion, frightened, prone}, senses={blindsight 60 ft. (blind beyond this radius), passive Perception 8}, languages={—}, challenge={1/2 (100 XP)}]
\DndMonsterAction{Amorphous} The ooze can move through a space as narrow as 1 inch wide without squeezing.

\DndMonsterAction{Corrode Metal} Any nonmagical weapon made of metal that hits the ooze corrodes. After dealing damage, the weapon takes a permanent and cumulative -1 penalty to damage rolls. If its penalty drops to -5, the weapon is destroyed. Nonmagical ammunition made of metal that hits the ooze is destroyed after dealing damage.\\nThe ooze can eat through 2-inch-thick, nonmagical metal in 1 round.

\DndMonsterAction{False Appearance} While the ooze remains motionless, it is indistinguishable from an oily pool or wet rock.

\DndMonsterSection{Actions}
\DndMonsterAttack[
	name=Pseudopod,
	distance=melee,
	type=weapon,
	mod=+3,
	reach=5,
	dmg=\DndDice{1d6 + 1},
	dmg-type=bludgeoning,
	extra={ plus 7 (2d6) acid damage, and if the target is wearing nonmagical metal armor, its armor is partly corroded and takes a permanent and cumulative -1 penalty to the AC it offers. The armor is destroyed if the penalty reduces its AC to 10.}
]
\end{multicols}
\end{DndMonster}

\subsection{Orc}
\begin{DndMonster}[width=\textwidth + 8pt]{Orc}
\begin{multicols}{2}
\DndMonsterType{Medium humanoid (orc)}
\DndMonsterBasics[armor-class={13 (hide armor)}, hit-points={15 (2d8 + 6)}, speed={30 ft.}]
\MonsterStats{+3}{+1}{+3}{-2}{+0}{+0}
\DndMonsterDetails[saving-throws={}, skills={Intimidation +2}, damage-immunities={}, damage-resistances={}, damage-vulnerabilities={}, condition-immunities={}, senses={darkvision 60 ft., passive Perception 10}, languages={Common, Orc}, challenge={1/2 (100 XP)}]
\DndMonsterAction{Aggressive} As a bonus action, the orc can move up to its speed toward a hostile creature that it can see.

\DndMonsterSection{Actions}
\DndMonsterAttack[
	name=Greataxe,
	distance=melee,
	type=weapon,
	mod=+5,
	reach=5,
	dmg=\DndDice{1d12 + 3},
	dmg-type=slashing
]
\DndMonsterAttack[
	name=Javelin,
	distance=both,
	type=weapon,
	mod=+5,
	reach=5,
	dmg=\DndDice{1d6 + 3},
	dmg-type=piercing
]
\end{multicols}
\end{DndMonster}

\subsection{Otyugh}
\begin{DndMonster}[width=\textwidth + 8pt]{Otyugh}
\begin{multicols}{2}
\DndMonsterType{Large monstrosity}
\DndMonsterBasics[armor-class={14 (natural armor)}, hit-points={114 (12d10 + 48)}, speed={30 ft.}]
\MonsterStats{+3}{+0}{+4}{-2}{+1}{-2}
\DndMonsterDetails[saving-throws={Con +7}, skills={}, damage-immunities={}, damage-resistances={}, damage-vulnerabilities={}, condition-immunities={}, senses={darkvision 120 ft., passive Perception 11}, languages={Otyugh}, challenge={5 (1,800 XP)}]
\DndMonsterAction{Limited Telepathy} The otyugh can magically transmit simple messages and images to any creature within 120 feet of it that can understand a language. This form of telepathy doesn't allow the receiving creature to telepathically respond.

\DndMonsterSection{Actions}
\DndMonsterAction{Multiattack} The otyugh makes three attacks: one with its bite and two with its tentacles.
\DndMonsterAttack[
	name=Bite,
	distance=melee,
	type=weapon,
	mod=+6,
	reach=5,
	dmg=\DndDice{2d8 + 3},
	dmg-type=piercing,
	extra={. If the target is a creature, it must succeed on a DC 15 Constitution saving throw against disease or become poisoned until the disease is cured. Every 24 hours that elapse, the target must repeat the saving throw, reducing its hit point maximum by 5 (1d10) on a failure. The disease is cured on a success. The target dies if the disease reduces its hit point maximum to 0. This reduction to the target's hit point maximum lasts until the disease is cured.}
]
\DndMonsterAttack[
	name=Tentacle,
	distance=melee,
	type=weapon,
	mod=+6,
	reach=10,
	dmg=\DndDice{1d8 + 3},
	dmg-type=bludgeoning,
	extra={ plus 4 (1d8) piercing damage. If the target is Medium or smaller, it is grappled (escape DC 13) and restrained until the grapple ends. The otyugh has two tentacles, each of which can grapple one target.}
]
\DndMonsterAction{Slam}
The otyugh slams creatures grappled by it into each other or a solid surface. Each creature must succeed on a DC 14 Constitution saving throw or take 10 (2d6 + 3) bludgeoning damage and be stunned until the end of the otyugh's next turn. On a successful save, the target takes half the bludgeoning damage and isn't stunned.
\end{multicols}
\end{DndMonster}
\subsection{Owlbear}
\begin{DndMonster}[width=\textwidth + 8pt]{Owlbear}
\begin{multicols}{2}
\DndMonsterType{Large monstrosity}
\DndMonsterBasics[armor-class={13 (natural armor)}, hit-points={59 (7d10 + 21)}, speed={40 ft.}]
\MonsterStats{+5}{+1}{+3}{-4}{+1}{-2}
\DndMonsterDetails[saving-throws={}, skills={Perception +3}, damage-immunities={}, damage-resistances={}, damage-vulnerabilities={}, condition-immunities={}, senses={darkvision 60 ft., passive Perception 13}, languages={—}, challenge={3 (700 XP)}]
\DndMonsterAction{Keen Sight and Smell} The owlbear has advantage on Wisdom (Perception) checks that rely on sight or smell.

\DndMonsterSection{Actions}
\DndMonsterAction{Multiattack} The owlbear makes two attacks: one with its beak and one with its claws.
\DndMonsterAttack[
	name=Beak,
	distance=melee,
	type=weapon,
	mod=+7,
	reach=5,
	dmg=\DndDice{1d10 + 5},
	dmg-type=piercing
]
\DndMonsterAttack[
	name=Claws,
	distance=melee,
	type=weapon,
	mod=+7,
	reach=5,
	dmg=\DndDice{2d8 + 5},
	dmg-type=slashing
]
\end{multicols}
\end{DndMonster}

\FloatBarrier
\section{Monsters (P)}\label{sec:monsters-p}

\subsection{Pudding, Black}
\begin{DndMonster}[width=\textwidth + 8pt]{Black Pudding}
\begin{multicols}{2}
\DndMonsterType{Large ooze}
\DndMonsterBasics[armor-class={7}, hit-points={85 (10d10 + 30)}, speed={20 ft., climb 20 ft.}]
\MonsterStats{+3}{-3}{+3}{-5}{-2}{-5}
\DndMonsterDetails[saving-throws={}, skills={}, damage-immunities={acid, cold, lightning, slashing}, damage-resistances={}, damage-vulnerabilities={}, condition-immunities={blinded, charmed, deafened, exhaustion, frightened, prone}, senses={blindsight 60 ft. (blind beyond this radius), passive Perception 8}, languages={—}, challenge={4 (1,100 XP)}]
\DndMonsterAction{Amorphous} The pudding can move through a space as narrow as 1 inch wide without squeezing.

\DndMonsterAction{Corrosive Form} A creature that touches the pudding or hits it with a melee attack while within 5 feet of it takes 4 (1d8) acid damage. Any nonmagical weapon made of metal or wood that hits the pudding corrodes. After dealing damage, the weapon takes a permanent and cumulative -1 penalty to damage rolls. If its penalty drops to -5, the weapon is destroyed. Nonmagical ammunition made of metal or wood that hits the pudding is destroyed after dealing damage.\\nThe pudding can eat through 2-inch-thick, nonmagical wood or metal in 1 round.

\DndMonsterAction{Spider Climb} The pudding can climb difficult surfaces, including upside down on ceilings, without needing to make an ability check.

\DndMonsterSection{Actions}
\DndMonsterAttack[
	name=Pseudopod,
	distance=melee,
	type=weapon,
	mod=+5,
	reach=5,
	dmg=\DndDice{1d6 + 3},
	dmg-type=bludgeoning,
	extra={ plus 18 (4d8) acid damage. In addition, nonmagical armor worn by the target is partly dissolved and takes a permanent and cumulative -1 penalty to the AC it offers. The armor is destroyed if the penalty reduces its AC to 10.}
]
\DndMonsterAction{Split}
When a pudding that is Medium or larger is subjected to lightning or slashing damage, it splits into two new puddings if it has at least 10 hit points. Each new pudding has hit points equal to half the original pudding's, rounded down. New puddings are one size smaller than the original pudding.
\end{multicols}
\end{DndMonster}



\subsection{Purple Worm}
\begin{DndMonster}[width=\textwidth + 8pt]{Purple Worm}
\begin{multicols}{2}
\DndMonsterType{Gargantuan monstrosity}
\DndMonsterBasics[armor-class={18 (natural armor)}, hit-points={247 (15d20 + 90)}, speed={50 ft., burrow 30 ft.}]
\MonsterStats{+9}{-2}{+6}{-5}{-1}{-3}
\DndMonsterDetails[saving-throws={Con +11, Wis +4}, skills={}, damage-immunities={}, damage-resistances={}, damage-vulnerabilities={}, condition-immunities={}, senses={blindsight 30 ft., tremorsense 60 ft., passive Perception 9}, languages={—}, challenge={15 (13,000 XP)}]
\DndMonsterAction{Tunneler} The worm can burrow through solid rock at half its burrow speed and leaves a 10-foot-diameter tunnel in its wake.

\DndMonsterSection{Actions}
\DndMonsterAction{Multiattack} The worm makes two attacks: one with its bite and one with its stinger.
\DndMonsterAttack[
	name=Bite,
	distance=melee,
	type=weapon,
	mod=+9,
	reach=10,
	dmg=\DndDice{3d8 + 9},
	dmg-type=piercing,
	extra={. If the target is a Large or smaller creature, it must succeed on a DC 19 Dexterity saving throw or be swallowed by the worm. A swallowed creature is blinded and restrained, it has total cover against attacks and other effects outside the worm, and it takes 21 (6d6) acid damage at the start of each of the worm's turns.\\nIf the worm takes 30 damage or more on a single turn from a creature inside it, the worm must succeed on a DC 21 Constitution saving throw at the end of that turn or regurgitate all swallowed creatures, which fall prone in a space within 10 feet of the worm. If the worm dies, a swallowed creature is no longer restrained by it and can escape from the corpse by using 20 feet of movement, exiting prone.}
]
\DndMonsterAttack[
	name=Tail Stinger,
	distance=melee,
	type=weapon,
	mod=+9,
	reach=10,
	dmg=\DndDice{3d6 + 9},
	dmg-type=piercing,
	extra={, and the target must make a DC 19 Constitution saving throw, taking 42 (12d6) poison damage on a failed save, or half as much damage on a successful one.}
]
\end{multicols}
\end{DndMonster}

\FloatBarrier
\section{Monsters (R)} \label{sec:monsters-r}
\subsection{Remorhaz}
\begin{DndMonster}[width=\textwidth + 8pt]{Remorhaz}
\begin{multicols}{2}
\DndMonsterType{Huge monstrosity}
\DndMonsterBasics[armor-class={17 (natural armor)}, hit-points={195 (17d12 + 85)}, speed={30 ft., burrow 20 ft.}]
\MonsterStats{+7}{+1}{+5}{-3}{+0}{-3}
\DndMonsterDetails[saving-throws={}, skills={}, damage-immunities={cold, fire}, damage-resistances={}, damage-vulnerabilities={}, condition-immunities={}, senses={darkvision 60 ft., tremorsense 60 ft., passive Perception 10}, languages={—}, challenge={11 (7,200 XP)}]
\DndMonsterAction{Heated Body} A creature that touches the remorhaz or hits it with a melee attack while within 5 feet of it takes 10 (3d6) fire damage.

\DndMonsterSection{Actions}
\DndMonsterAttack[
	name=Bite,
	distance=melee,
	type=weapon,
	mod=+11,
	reach=10,
	dmg=\DndDice{6d10 + 7},
	dmg-type=piercing,
	extra={ plus 10 (3d6) fire damage. If the target is a creature, it is grappled (escape DC 17). Until this grapple ends, the target is restrained, and the remorhaz can't bite another target.}
]
\DndMonsterAction{Swallow}
The remorhaz makes one bite attack against a Medium or smaller creature it is grappling. If the attack hits, that creature takes the bite's damage and is swallowed, and the grapple ends. While swallowed, the creature is blinded and restrained, it has total cover against attacks and other effects outside the remorhaz, and it takes 21 (6d6) acid damage at the start of each of the remorhaz's turns.\\nIf the remorhaz takes 30 damage or more on a single turn from a creature inside it, the remorhaz must succeed on a DC 15 Constitution saving throw at the end of that turn or regurgitate all swallowed creatures, which fall prone in a space within 10 feet of the remorhaz. If the remorhaz dies, a swallowed creature is no longer restrained by it and can escape from the corpse using 15 feet of movement, exiting prone.
\end{multicols}
\end{DndMonster}

\subsection{Roc}
\begin{DndMonster}[width=\textwidth + 8pt]{Roc}
\begin{multicols}{2}
\DndMonsterType{Gargantuan monstrosity}
\DndMonsterBasics[armor-class={15 (natural armor)}, hit-points={248 (16d20 + 80)}, speed={20 ft., fly 120 ft.}]
\MonsterStats{+9}{+0}{+5}{-4}{+0}{-1}
\DndMonsterDetails[saving-throws={Dex +4, Con +9, Wis +4, Cha +3}, skills={Perception +4}, damage-immunities={}, damage-resistances={}, damage-vulnerabilities={}, condition-immunities={}, senses={passive Perception 14}, languages={—}, challenge={11 (7,200 XP)}]
\DndMonsterAction{Keen Sight} The roc has advantage on Wisdom (Perception) checks that rely on sight.

\DndMonsterSection{Actions}
\DndMonsterAction{Multiattack} The roc makes two attacks: one with its beak and one with its talons.
\DndMonsterAttack[
	name=Beak,
	distance=melee,
	type=weapon,
	mod=+13,
	reach=10,
	dmg=\DndDice{4d8 + 9},
	dmg-type=piercing
]
\DndMonsterAttack[
	name=Talons,
	distance=melee,
	type=weapon,
	mod=+13,
	reach=5,
	dmg=\DndDice{4d6 + 9},
	dmg-type=slashing,
	extra={, and the target is grappled (escape DC 19). Until this grapple ends, the target is restrained, and the roc can't use its talons on another target.}
]
\end{multicols}
\end{DndMonster}

\subsection{Roper}
\begin{DndMonster}[width=\textwidth + 8pt]{Roper}
\begin{multicols}{2}
\DndMonsterType{Large monstrosity}
\DndMonsterBasics[armor-class={20 (natural armor)}, hit-points={93 (11d10 + 33)}, speed={10 ft., climb 10 ft.}]
\MonsterStats{+4}{-1}{+3}{-2}{+3}{-2}
\DndMonsterDetails[saving-throws={}, skills={Perception +6, Stealth +5}, damage-immunities={}, damage-resistances={}, damage-vulnerabilities={}, condition-immunities={}, senses={darkvision 60 ft., passive Perception 16}, languages={—}, challenge={5 (1,800 XP)}]
\DndMonsterAction{False Appearance} While the roper remains motionless, it is indistinguishable from a normal cave formation, such as a stalagmite.

\DndMonsterAction{Grasping Tendrils} The roper can have up to six tendrils at a time. Each tendril can be attacked (AC 20; 10 hit points; immunity to poison and psychic damage). Destroying a tendril deals no damage to the roper, which can extrude a replacement tendril on its next turn. A tendril can also be broken if a creature takes an action and succeeds on a DC 15 Strength check against it.

\DndMonsterAction{Spider Climb} The roper can climb difficult surfaces, including upside down on ceilings, without needing to make an ability check.

\DndMonsterSection{Actions}
\DndMonsterAction{Multiattack} The roper makes four attacks with its tendrils, uses Reel, and makes one attack with its bite.
\DndMonsterAttack[
	name=Bite,
	distance=melee,
	type=weapon,
	mod=+7,
	reach=5,
	dmg=\DndDice{4d8 + 4},
	dmg-type=piercing
]
\DndMonsterAttack[
	name=Tendril,
	distance=melee,
	type=weapon,
	mod=+7,
	reach=50,
	dmg=\DndDice{1d8+4},
	dmg-type=bludgeoning,
	extra={. The target is grappled (escape DC 15). Until the grapple ends, the target is restrained and has disadvantage on Strength checks and Strength saving throws, and the roper can't use the same tendril on another target.}
]
\DndMonsterAction{Reel}
The roper pulls each creature grappled by it up to 25 feet straight toward it.
\end{multicols}
\end{DndMonster}

\subsection{Rust Monster}
\begin{DndMonster}[width=\textwidth + 8pt]{Rust Monster}
\begin{multicols}{2}
\DndMonsterType{Medium monstrosity}
\DndMonsterBasics[armor-class={14 (natural armor)}, hit-points={27 (5d8 + 5)}, speed={40 ft.}]
\MonsterStats{+1}{+1}{+1}{-4}{+1}{-2}
\DndMonsterDetails[saving-throws={}, skills={}, damage-immunities={}, damage-resistances={}, damage-vulnerabilities={}, condition-immunities={}, senses={darkvision 60 ft., passive Perception 11}, languages={—}, challenge={1/2 (100 XP)}]
\DndMonsterAction{Iron Scent} The rust monster can pinpoint, by scent, the location of ferrous metal within 30 feet of it.

\DndMonsterAction{Rust Metal} Any nonmagical weapon made of metal that hits the rust monster corrodes. After dealing damage, the weapon takes a permanent and cumulative -1 penalty to damage rolls. If its penalty drops to -5, the weapon is destroyed. Nonmagical ammunition made of metal that hits the rust monster is destroyed after dealing damage.

\DndMonsterSection{Actions}
\DndMonsterAttack[
	name=Bite,
	distance=melee,
	type=weapon,
	mod=+3,
	reach=5,
	dmg=\DndDice{1d8 + 1},
	dmg-type=piercing
]
\DndMonsterAction{Antennae}
The rust monster corrodes a nonmagical ferrous metal object it can see within 5 feet of it. If the object isn't being worn or carried, the touch destroys a 1-foot cube of it. If the object is being worn or carried by a creature, the creature can make a DC 11 Dexterity saving throw to avoid the rust monster's touch.\\nIf the object touched is either metal armor or a metal shield being worn or carried, its takes a permanent and cumulative -1 penalty to the AC it offers. Armor reduced to an AC of 10 or a shield that drops to a +0 bonus is destroyed. If the object touched is a held metal weapon, it rusts as described in the Rust Metal trait.
\end{multicols}
\end{DndMonster}

\FloatBarrier
\section{Monsters (S)} \label{sec:monsters-s}

\subsection{Shadow}
\begin{DndMonster}[width=\textwidth + 8pt]{Shadow}
\begin{multicols}{2}
\DndMonsterType{Medium undead}
\DndMonsterBasics[armor-class={12}, hit-points={16 (3d8 + 3)}, speed={40 ft.}]
\MonsterStats{-2}{+2}{+1}{-2}{+0}{-1}
\DndMonsterDetails[saving-throws={}, skills={Stealth +4 (+6 in dim light or darkness)}, damage-immunities={necrotic, poison}, damage-resistances={acid, cold, fire, lightning, thunder; bludgeoning, piercing, and slashing from nonmagical attacks}, damage-vulnerabilities={radiant}, condition-immunities={exhaustion, frightened, grappled, paralyzed, petrified, poisoned, prone, restrained}, senses={darkvision 60 ft., passive Perception 10}, languages={—}, challenge={1/2 (100 XP)}]
\DndMonsterAction{Amorphous} The shadow can move through a space as narrow as 1 inch wide without squeezing.

\DndMonsterAction{Shadow Stealth} While in dim light or darkness, the shadow can take the Hide action as a bonus action.

\DndMonsterAction{Sunlight Weakness} While in sunlight, the shadow has disadvantage on attack rolls, ability checks, and saving throws.

\DndMonsterSection{Actions}
\DndMonsterAttack[
	name=Strength Drain,
	distance=melee,
	type=weapon,
	mod=+4,
	reach=5,
	dmg=\DndDice{2d6 + 2},
	dmg-type=necrotic,
	extra={, and the target's Strength score is reduced by 1d4. The target dies if this reduces its Strength to 0. Otherwise, the reduction lasts until the target finishes a short or long rest.\\nIf a humanoid dies from this attack, a new shadow rises from the corpse 1d4 hours later.}
]
\end{multicols}
\end{DndMonster}

\subsection{Shambling Mound}
\begin{DndMonster}[width=\textwidth + 8pt]{Shambling Mound}
\begin{multicols}{2}
\DndMonsterType{Large plant}
\DndMonsterBasics[armor-class={15 (natural armor)}, hit-points={136 (16d10 + 48)}, speed={20 ft., swim 20 ft.}]
\MonsterStats{+4}{-1}{+3}{-3}{+0}{-3}
\DndMonsterDetails[saving-throws={}, skills={Stealth +2}, damage-immunities={lightning}, damage-resistances={cold, fire}, damage-vulnerabilities={}, condition-immunities={blinded, deafened, exhaustion}, senses={blindsight 60 ft. (blind beyond this radius), passive Perception 10}, languages={—}, challenge={5 (1,800 XP)}]
\DndMonsterAction{Lightning Absorption} Whenever the shambling mound is subjected to lightning damage, it takes no damage and regains a number of hit points equal to the lightning damage dealt.

\DndMonsterSection{Actions}
\DndMonsterAction{Multiattack} The shambling mound makes two slam attacks. If both attacks hit a Medium or smaller target, the target is grappled (escape DC 14), and the shambling mound uses its Engulf on it.
\DndMonsterAttack[
	name=Slam,
	distance=melee,
	type=weapon,
	mod=+7,
	reach=5,
	dmg=\DndDice{2d8 + 4},
	dmg-type=bludgeoning
]
\DndMonsterAction{Engulf}
The shambling mound engulfs a Medium or smaller creature grappled by it. The engulfed target is blinded, restrained, and unable to breathe, and it must succeed on a DC 14 Constitution saving throw at the start of each of the mound's turns or take 13 (2d8 + 4) bludgeoning damage. If the mound moves, the engulfed target moves with it. The mound can have only one creature engulfed at a time.
\end{multicols}
\end{DndMonster}

\subsection{Skeleton}
\begin{DndMonster}[width=\textwidth + 8pt]{Skeleton}
\begin{multicols}{2}
\DndMonsterType{Medium undead}
\DndMonsterBasics[armor-class={13 (armor scraps)}, hit-points={13 (2d8 + 4)}, speed={30 ft.}]
\MonsterStats{+0}{+2}{+2}{-2}{-1}{-3}
\DndMonsterDetails[saving-throws={}, skills={}, damage-immunities={poison}, damage-resistances={}, damage-vulnerabilities={bludgeoning}, condition-immunities={exhaustion, poisoned}, senses={darkvision 60 ft., passive Perception 9}, languages={understands all languages it knew in life but can't speak}, challenge={1/4 (50 XP)}]
\DndMonsterSection{Actions}
\DndMonsterAttack[
	name=Shortsword,
	distance=melee,
	type=weapon,
	mod=+4,
	reach=5,
	dmg=\DndDice{1d6 + 2},
	dmg-type=piercing
]
\DndMonsterAttack[
	name=Shortbow,
	distance=ranged,
	type=weapon,
	mod=+4,
	range=80/320,
	dmg=\DndDice{1d6 + 2},
	dmg-type=piercing
]
\end{multicols}
\end{DndMonster}

\subsection{Skeleton, Minotaur}
\begin{DndMonster}[width=\textwidth + 8pt]{Minotaur Skeleton}
\begin{multicols}{2}
\DndMonsterType{Large undead}
\DndMonsterBasics[armor-class={12 (natural armor)}, hit-points={67 (9d10 + 18)}, speed={40 ft.}]
\MonsterStats{+4}{+0}{+2}{-2}{-1}{-3}
\DndMonsterDetails[saving-throws={}, skills={}, damage-immunities={poison}, damage-resistances={}, damage-vulnerabilities={bludgeoning}, condition-immunities={exhaustion, poisoned}, senses={darkvision 60 ft., passive Perception 9}, languages={understands Abyssal but can't speak}, challenge={2 (450 XP)}]
\DndMonsterAction{Charge} If the skeleton moves at least 10 feet straight toward a target and then hits it with a gore attack on the same turn, the target takes an extra 9 (2d8) piercing damage. If the target is a creature, it must succeed on a DC 14 Strength saving throw or be pushed up to 10 feet away and knocked prone.

\DndMonsterSection{Actions}
\DndMonsterAttack[
	name=Greataxe,
	distance=melee,
	type=weapon,
	mod=+6,
	reach=5,
	dmg=\DndDice{2d12 + 4},
	dmg-type=slashing
]
\DndMonsterAttack[
	name=Gore,
	distance=melee,
	type=weapon,
	mod=+6,
	reach=5,
	dmg=\DndDice{2d8 + 4},
	dmg-type=piercing
]
\end{multicols}
\end{DndMonster}

\subsection{Skeleton, Warhorse}
\begin{DndMonster}[width=\textwidth + 8pt]{Warhorse Skeleton}
\begin{multicols}{2}
\DndMonsterType{Large undead}
\DndMonsterBasics[armor-class={13 (barding scraps)}, hit-points={22 (3d10 + 6)}, speed={60 ft.}]
\MonsterStats{+4}{+1}{+2}{-4}{-1}{-3}
\DndMonsterDetails[saving-throws={}, skills={}, damage-immunities={poison}, damage-resistances={}, damage-vulnerabilities={bludgeoning}, condition-immunities={exhaustion, poisoned}, senses={darkvision 60 ft., passive Perception 9}, languages={—}, challenge={1/2 (100 XP)}]
\DndMonsterSection{Actions}
\DndMonsterAttack[
	name=Hooves,
	distance=melee,
	type=weapon,
	mod=+6,
	reach=5,
	dmg=\DndDice{2d6 + 4},
	dmg-type=bludgeoning
]
\end{multicols}
\end{DndMonster}

\subsection{Specter}
\begin{DndMonster}[width=\textwidth + 8pt]{Specter}
\begin{multicols}{2}
\DndMonsterType{Medium undead}
\DndMonsterBasics[armor-class={12}, hit-points={22 (5d8)}, speed={0 ft., fly 50 ft. (hover)}]
\MonsterStats{-5}{+2}{+0}{+0}{+0}{+0}
\DndMonsterDetails[saving-throws={}, skills={}, damage-immunities={necrotic, poison}, damage-resistances={acid, cold, fire, lightning, thunder; bludgeoning, piercing, and slashing from nonmagical attacks}, damage-vulnerabilities={}, condition-immunities={charmed, exhaustion, grappled, paralyzed, petrified, poisoned, prone, restrained, unconscious}, senses={darkvision 60 ft., passive Perception 10}, languages={understands all languages it knew in life but can't speak}, challenge={1 (200 XP)}]
\DndMonsterAction{Incorporeal Movement} The specter can move through other creatures and objects as if they were difficult terrain. It takes 5 (1d10) force damage if it ends its turn inside an object.

\DndMonsterAction{Sunlight Sensitivity} While in sunlight, the specter has disadvantage on attack rolls, as well as on Wisdom (Perception) checks that rely on sight.

\DndMonsterSection{Actions}
\DndMonsterAttack[
	name=Touch,
	distance=melee,
	type=spell,
	mod=+4,
	reach=5,
	dmg=\DndDice{3d6},
	dmg-type=necrotic,
	extra={. The target must succeed on a DC 10 Constitution saving throw or its hit point maximum is reduced by an amount equal to the damage taken. This reduction lasts until the creature finishes a long rest. The target dies if this effect reduces its hit point maximum to 0.}
]
\end{multicols}
\end{DndMonster}

\subsection{Stirge}
\begin{DndMonster}[width=\textwidth + 8pt]{Stirge}
\begin{multicols}{2}
\DndMonsterType{Tiny beast}
\DndMonsterBasics[armor-class={14 (natural armor)}, hit-points={2 (1d4)}, speed={10 ft., fly 40 ft.}]
\MonsterStats{-3}{+3}{+0}{-4}{-1}{-2}
\DndMonsterDetails[saving-throws={}, skills={}, damage-immunities={}, damage-resistances={}, damage-vulnerabilities={}, condition-immunities={}, senses={darkvision 60 ft., passive Perception 9}, languages={—}, challenge={1/8 (25 XP)}]
\DndMonsterSection{Actions}
\DndMonsterAttack[
	name=Blood Drain,
	distance=melee,
	type=weapon,
	mod=+5,
	reach=5,
	dmg=\DndDice{1d4 + 3},
	dmg-type=piercing,
	extra={, and the stirge attaches to the target. While attached, the stirge doesn't attack. Instead, at the start of each of the stirge's turns, the target loses 5 (1d4 + 3) hit points due to blood loss.\\nThe stirge can detach itself by spending 5 feet of its movement. It does so after it drains 10 hit points of blood from the target or the target dies. A creature, including the target, can use its action to detach the stirge.}
]
\end{multicols}
\end{DndMonster}


\FloatBarrier
\section{Monsters (T)} \label{sec:monsters-t}
\subsection{Tarrasque}
\begin{DndMonster}[width=\textwidth + 8pt]{Tarrasque}
\begin{multicols}{2}
\DndMonsterType{Gargantuan monstrosity (titan)}
\DndMonsterBasics[armor-class={25 (natural armor)}, hit-points={676 (33d20 + 330)}, speed={40 ft.}]
\MonsterStats{+10}{+0}{+10}{-4}{+0}{+0}
\DndMonsterDetails[saving-throws={Int +5, Wis +9, Cha +9}, skills={}, damage-immunities={fire, poison; bludgeoning, piercing, and slashing from nonmagical attacks}, damage-resistances={}, damage-vulnerabilities={}, condition-immunities={charmed, frightened, paralyzed, poisoned}, senses={blindsight 120 ft., passive Perception 10}, languages={—}, challenge={30 (155,000 XP)}]
\DndMonsterAction{Legendary Resistance (3/Day)} If the tarrasque fails a saving throw, it can choose to succeed instead.

\DndMonsterAction{Magic Resistance} The tarrasque has advantage on saving throws against spells and other magical effects.

\DndMonsterAction{Reflective Carapace} Any time the tarrasque is targeted by a \textit{magic missile} spell, a line spell, or a spell that requires a ranged attack roll, roll a d6. On a 1 to 5, the tarrasque is unaffected. On a 6, the tarrasque is unaffected, and the effect is reflected back at the caster as though it originated from the tarrasque, turning the caster into the target.

\DndMonsterAction{Siege Monster} The tarrasque deals double damage to objects and structures.

\DndMonsterSection{Actions}
\DndMonsterAction{Multiattack} The tarrasque can use its Frightful Presence. It then makes five attacks: one with its bite, two with its claws, one with its horns, and one with its tail. It can use its Swallow instead of its bite.
\DndMonsterAttack[
	name=Bite,
	distance=melee,
	type=weapon,
	mod=+19,
	reach=10,
	dmg=\DndDice{4d12 + 10},
	dmg-type=piercing,
	extra={. If the target is a creature, it is grappled (escape DC 20). Until this grapple ends, the target is restrained, and the tarrasque can't bite another target.}
]
\DndMonsterAttack[
	name=Claw,
	distance=melee,
	type=weapon,
	mod=+19,
	reach=15,
	dmg=\DndDice{4d8 + 10},
	dmg-type=slashing
]
\DndMonsterAttack[
	name=Horns,
	distance=melee,
	type=weapon,
	mod=+19,
	reach=10,
	dmg=\DndDice{4d10 + 10},
	dmg-type=piercing
]
\DndMonsterAttack[
	name=Tail,
	distance=melee,
	type=weapon,
	mod=+19,
	reach=20,
	dmg=\DndDice{4d6 + 10},
	dmg-type=bludgeoning,
	extra={. If the target is a creature, it must succeed on a DC 20 Strength saving throw or be knocked prone.}
]
\DndMonsterAction{Frightful Presence}
Each creature of the tarrasque's choice within 120 feet of it and aware of it must succeed on a DC 17 Wisdom saving throw or become frightened for 1 minute. A creature can repeat the saving throw at the end of each of its turns, with disadvantage if the tarrasque is within line of sight, ending the effect on itself on a success. If a creature's saving throw is successful or the effect ends for it, the creature is immune to the tarrasque's Frightful Presence for the next 24 hours.
\DndMonsterAction{Swallow}
The tarrasque makes one bite attack against a Large or smaller creature it is grappling. If the attack hits, the target takes the bite's damage, the target is swallowed, and the grapple ends. While swallowed, the creature is blinded and restrained, it has total cover against attacks and other effects outside the tarrasque, and it takes 56 (16d6) acid damage at the start of each of the tarrasque's turns.\\nIf the tarrasque takes 60 damage or more on a single turn from a creature inside it, the tarrasque must succeed on a DC 20 Constitution saving throw at the end of that turn or regurgitate all swallowed creatures, which fall prone in a space within 10 feet of the tarrasque. If the tarrasque dies, a swallowed creature is no longer restrained by it and can escape from the corpse by using 30 feet of movement, exiting prone.

\DndMonsterSection{Legendary Actions}
The Tarrasque can take 3 legendary actions, choosing from the options below. Only one legendary action option can be used at a time and only at the end of another creature's turn. The Tarrasque regains spent legendary actions at the start of its turn.
\begin{DndMonsterLegendaryActions}
\DndMonsterLegendaryAction{Attack}{The tarrasque makes one claw attack or tail attack.}
\DndMonsterLegendaryAction{Move}{The tarrasque moves up to half its speed.}
\DndMonsterLegendaryAction{Chomp (Costs 2 Actions)}{The tarrasque makes one bite attack or uses its Swallow.}
\end{DndMonsterLegendaryActions}
\end{multicols}
\end{DndMonster}

\subsection{Treant}
\begin{DndMonster}[width=\textwidth + 8pt]{Treant}
\begin{multicols}{2}
\DndMonsterType{Huge plant}
\DndMonsterBasics[armor-class={16 (natural armor)}, hit-points={138 (12d12 + 60)}, speed={30 ft.}]
\MonsterStats{+6}{-1}{+5}{+1}{+3}{+1}
\DndMonsterDetails[saving-throws={}, skills={}, damage-immunities={}, damage-resistances={bludgeoning, piercing}, damage-vulnerabilities={fire}, condition-immunities={}, senses={passive Perception 13}, languages={Common, Druidic, Elvish, Sylvan}, challenge={9 (5,000 XP)}]
\DndMonsterAction{False Appearance} While the treant remains motionless, it is indistinguishable from a normal tree.

\DndMonsterAction{Siege Monster} The treant deals double damage to objects and structures.

\DndMonsterSection{Actions}
\DndMonsterAction{Multiattack} The treant makes two slam attacks.
\DndMonsterAttack[
	name=Slam,
	distance=melee,
	type=weapon,
	mod=+10,
	reach=5,
	dmg=\DndDice{3d6 + 6},
	dmg-type=bludgeoning
]
\DndMonsterAttack[
	name=Rock,
	distance=ranged,
	type=weapon,
	mod=+10,
	range=60/180,
	dmg=\DndDice{4d10 + 6},
	dmg-type=bludgeoning
]
The treant magically animates one or two trees it can see within 60 feet of it. These trees have the same statistics as a treant, except they have Intelligence and Charisma scores of 1, they can't speak, and they have only the Slam action option. An animated tree acts as an ally of the treant. The tree remains animate for 1 day or until it dies; until the treant dies or is more than 120 feet from the tree; or until the treant takes a bonus action to turn it back into an inanimate tree. The tree then takes root if possible.
\end{multicols}
\end{DndMonster}

\subsection{Troll}

\FloatBarrier
\section{Monsters (U)} \label{sec:monsters-u}

\FloatBarrier
\section{Monsters (V)} \label{sec:monsters-v}
\subsection{Vampire}
\begin{DndMonster}[width=\textwidth + 8pt]{Vampire}
\begin{multicols}{2}
\DndMonsterType{Medium undead (shapechanger)}
\DndMonsterBasics[armor-class={16 (natural armor)}, hit-points={144 (17d8 + 68)}, speed={30 ft.}]
\MonsterStats{+4}{+4}{+4}{+3}{+2}{+4}
\DndMonsterDetails[saving-throws={Dex +9, Wis +7, Cha +9}, skills={Perception +7, Stealth +9}, damage-immunities={}, damage-resistances={necrotic; bludgeoning, piercing, and slashing from nonmagical attacks}, damage-vulnerabilities={}, condition-immunities={}, senses={darkvision 120 ft., passive Perception 17}, languages={the languages it knew in life}, challenge={13 (10,000 XP)}]
\DndMonsterAction{Shapechanger} If the vampire isn't in sunlight or running water, it can use its action to polymorph into a Tiny bat or a Medium cloud of mist, or back into its true form.\\nWhile in bat form, the vampire can't speak, its walking speed is 5 feet, and it has a flying speed of 30 feet. Its statistics, other than its size and speed, are unchanged. Anything it is wearing transforms with it, but nothing it is carrying does. It reverts to its true form if it dies.\\nWhile in mist form, the vampire can't take any actions, speak, or manipulate objects. It is weightless, has a flying speed of 20 feet, can hover, and can enter a hostile creature's space and stop there. In addition, if air can pass through a space, the mist can do so without squeezing, and it can't pass through water. It has advantage on Strength, Dexterity, and Constitution saving throws, and it is immune to all nonmagical damage, except the damage it takes from sunlight.

\DndMonsterAction{Legendary Resistance (3/Day)} If the vampire fails a saving throw, it can choose to succeed instead.

\DndMonsterAction{Misty Escape} When it drops to 0 hit points outside its resting place, the vampire transforms into a cloud of mist (as in the Shapechanger trait) instead of falling unconscious, provided that it isn't in sunlight or running water. If it can't transform, it is destroyed.\\nWhile it has 0 hit points in mist form, it can't revert to its vampire form, and it must reach its resting place within 2 hours or be destroyed. Once in its resting place, it reverts to its vampire form. It is then paralyzed until it regains at least 1 hit point. After spending 1 hour in its resting place with 0 hit points, it regains 1 hit point.

\DndMonsterAction{Regeneration} The vampire regains 20 hit points at the start of its turn if it has at least 1 hit point and isn't in sunlight or running water. If the vampire takes radiant damage or damage from holy water, this trait doesn't function at the start of the vampire's next turn.

\DndMonsterAction{Spider Climb} The vampire can climb difficult surfaces, including upside down on ceilings, without needing to make an ability check.

\DndMonsterAction{Vampire Weaknesses} The vampire has the following flaws:\\n\textit{Forbiddance.} The vampire can't enter a residence without an invitation from one of the occupants.\\n\textit{Harmed by Running Water.} The vampire takes 20 acid damage if it ends its turn in running water.\\n\textit{Stake to the Heart.} If a piercing weapon made of wood is driven into the vampire's heart while the vampire is incapacitated in its resting place, the vampire is paralyzed until the stake is removed.\\n\textit{Sunlight Hypersensitivity.} The vampire takes 20 radiant damage when it starts its turn in sunlight. While in sunlight, it has disadvantage on attack rolls and ability checks.

\DndMonsterSection{Actions}
The vampire makes two attacks, only one of which can be a bite attack.
\DndMonsterAttack[
	name=Unarmed Strike (Vampire Form Only),
	distance=melee,
	type=weapon,
	mod=+9,
	reach=5,
	dmg=\DndDice{1d8 + 4},
	dmg-type=bludgeoning,
	extra={. Instead of dealing damage, the vampire can grapple the target (escape DC 18).}
]
\DndMonsterAttack[
	name=Bite (Bat or Vampire Form Only),
	distance=melee,
	type=weapon,
	mod=+9,
	reach=5,
	dmg=\DndDice{1d6 + 4},
	dmg-type=piercing,
	extra={ plus 10 (3d6) necrotic damage. The target's hit point maximum is reduced by an amount equal to the necrotic damage taken, and the vampire regains hit points equal to that amount. The reduction lasts until the target finishes a long rest. The target dies if this effect reduces its hit point maximum to 0. A humanoid slain in this way and then buried in the ground rises the following night as a vampire spawn under the vampire's control.}
]
\DndMonsterAction{Charm}
The vampire targets one humanoid it can see within 30 feet of it. If the target can see the vampire, the target must succeed on a DC 17 Wisdom saving throw against this magic or be charmed by the vampire. The charmed target regards the vampire as a trusted friend to be heeded and protected. Although the target isn't under the vampire's control, it takes the vampire's requests or actions in the most favorable way it can, and it is a willing target for the vampire's bite attack.\\nEach time the vampire or the vampire's companions do anything harmful to the target, it can repeat the saving throw, ending the effect on itself on a success. Otherwise, the effect lasts 24 hours or until the vampire is destroyed, is on a different plane of existence than the target, or takes a bonus action to end the effect.

\DndMonsterAction(Call Vermin)
The vampire magically calls 2d4 swarms of bats or rats, provided that the sun isn't up. While outdoors, the vampire can call 3d6 wolves instead. The called creatures arrive in 1d4 rounds, acting as allies of the vampire and obeying its spoken commands. The beasts remain for 1 hour, until the vampire dies, or until the vampire dismisses them as a bonus action.

\DndMonsterSection{Legendary Actions}
The Vampire can take 3 legendary actions, choosing from the options below. Only one legendary action option can be used at a time and only at the end of another creature's turn. The Vampire regains spent legendary actions at the start of its turn.
\begin{DndMonsterLegendaryActions}
\DndMonsterLegendaryAction{Move}{The vampire moves up to its speed without provoking opportunity attacks.}
\DndMonsterLegendaryAction{Unarmed Strike}{The vampire makes one unarmed strike.}
\DndMonsterLegendaryAction{Bite (Costs 2 Actions)}{The vampire makes one bite attack.}
\end{DndMonsterLegendaryActions}
\end{multicols}
\end{DndMonster}

\subsection{Vampire Spawn}
\begin{DndMonster}[width=\textwidth + 8pt]{Vampire Spawn}
\begin{multicols}{2}
\DndMonsterType{Medium undead}
\DndMonsterBasics[armor-class={15 (natural armor)}, hit-points={82 (11d8 + 33)}, speed={30 ft.}]
\MonsterStats{+3}{+3}{+3}{+0}{+0}{+1}
\DndMonsterDetails[saving-throws={Dex +6, Wis +3}, skills={Perception +3, Stealth +6}, damage-immunities={}, damage-resistances={necrotic; bludgeoning, piercing, and slashing from nonmagical attacks}, damage-vulnerabilities={}, condition-immunities={}, senses={darkvision 60 ft., passive Perception 13}, languages={the languages it knew in life}, challenge={5 (1,800 XP)}]
\DndMonsterAction{Regeneration} The vampire regains 10 hit points at the start of its turn if it has at least 1 hit point and isn't in sunlight or running water. If the vampire takes radiant damage or damage from holy water, this trait doesn't function at the start of the vampire's next turn.

\DndMonsterAction{Spider Climb} The vampire can climb difficult surfaces, including upside down on ceilings, without needing to make an ability check.

\DndMonsterAction{Vampire Weaknesses} The vampire has the following flaws:\\n\textit{Forbiddance.} The vampire can't enter a residence without an invitation from one of the occupants.\\n\textit{Harmed by Running Water.} The vampire takes 20 acid damage when it ends its turn in running water.\\n\textit{Stake to the Heart.} The vampire is destroyed if a piercing weapon made of wood is driven into its heart while it is incapacitated in its resting place.\\n\textit{Sunlight Hypersensitivity.} The vampire takes 20 radiant damage when it starts its turn in sunlight. While in sunlight, it has disadvantage on attack rolls and ability checks.

\DndMonsterSection{Actions}
\DndMonsterAction{Multiattack} The vampire makes two attacks, only one of which can be a bite attack.
\DndMonsterAttack[
	name=Claws,
	distance=melee,
	type=weapon,
	mod=+6,
	reach=5,
	dmg=\DndDice{2d4 + 3},
	dmg-type=slashing,
	extra={. Instead of dealing damage, the vampire can grapple the target (escape DC 13).}
]
\DndMonsterAttack[
	name=Bite,
	distance=melee,
	type=weapon,
	mod=+6,
	reach=5,
	dmg=\DndDice{1d6 + 3},
	dmg-type=piercing,
	extra={ plus 7 (2d6) necrotic damage. The target's hit point maximum is reduced by an amount equal to the necrotic damage taken, and the vampire regains hit points equal to that amount. The reduction lasts until the target finishes a long rest. The target dies if this effect reduces its hit point maximum to 0.}
]
\end{multicols}
\end{DndMonster}
\FloatBarrier
\section{Monsters (W)} \label{sec:monsters-w}


\subsection{Wight}
\begin{DndMonster}[width=\textwidth + 8pt]{Wight}
\begin{multicols}{2}
\DndMonsterType{Medium undead}
\DndMonsterBasics[armor-class={14 (studded leather)}, hit-points={45 (6d8 + 18)}, speed={30 ft.}]
\MonsterStats{+2}{+2}{+3}{+0}{+1}{+2}
\DndMonsterDetails[saving-throws={}, skills={Perception +3, Stealth +4}, damage-immunities={poison}, damage-resistances={necrotic; bludgeoning, piercing, and slashing from nonmagical attacks that aren't silvered}, damage-vulnerabilities={}, condition-immunities={exhaustion, poisoned}, senses={darkvision 60 ft., passive Perception 13}, languages={the languages it knew in life}, challenge={3 (700 XP)}]
\DndMonsterAction{Sunlight Sensitivity} While in sunlight, the wight has disadvantage on attack rolls, as well as on Wisdom (Perception) checks that rely on sight.

\DndMonsterSection{Actions}
\DndMonsterAction{Multiattack} The wight makes two longsword attacks or two longbow attacks. It can use its Life Drain in place of one longsword attack.
\DndMonsterAttack[
	name=Life Drain,
	distance=melee,
	type=weapon,
	mod=+4,
	reach=5,
	dmg=\DndDice{1d6 + 2},
	dmg-type=necrotic,
	extra={. The target must succeed on a DC 13 Constitution saving throw or its hit point maximum is reduced by an amount equal to the damage taken. This reduction lasts until the target finishes a long rest. The target dies if this effect reduces its hit point maximum to 0.\\nA humanoid slain by this attack rises 24 hours later as a zombie under the wight's control, unless the humanoid is restored to life or its body is destroyed. The wight can have no more than twelve zombies under its control at one time.}
]
\DndMonsterAttack[
	name=Longsword,
	distance=melee,
	type=weapon,
	mod=+4,
	reach=5,
	dmg=\DndDice{1d8 + 2},
	dmg-type=slashing,
	extra={, or 7 (1d10 + 2) slashing damage if used with two hands.}
]
\DndMonsterAttack[
	name=Longbow,
	distance=ranged,
	type=weapon,
	mod=+4,
	range=150/600,
	dmg=\DndDice{1d8 + 2},
	dmg-type=piercing
]
\end{multicols}
\end{DndMonster}

\subsection{Will-o'-Wisp}
\begin{DndMonster}[width=\textwidth + 8pt]{Will-o'-Wisp}
\begin{multicols}{2}
\DndMonsterType{Tiny undead}
\DndMonsterBasics[armor-class={19}, hit-points={22 (9d4)}, speed={0 ft., fly 50 ft. (hover)}]
\MonsterStats{-5}{+9}{+0}{+1}{+2}{+0}
\DndMonsterDetails[saving-throws={}, skills={}, damage-immunities={lightning, poison}, damage-resistances={acid, cold, fire, necrotic, thunder; bludgeoning, piercing, and slashing from nonmagical attacks}, damage-vulnerabilities={}, condition-immunities={exhaustion, grappled, paralyzed, poisoned, prone, restrained, unconscious}, senses={darkvision 120 ft., passive Perception 12}, languages={the languages it knew in life}, challenge={2 (450 XP)}]
\DndMonsterAction{Consume Life} As a bonus action, the will-o'-wisp can target one creature it can see within 5 feet of it that has 0 hit points and is still alive. The target must succeed on a DC 10 Constitution saving throw against this magic or die. If the target dies, the will-o'-wisp regains 10 (3d6) hit points.

\DndMonsterAction{Ephemeral} The will-o'-wisp can't wear or carry anything.

\DndMonsterAction{Incorporeal Movement} The will-o'-wisp can move through other creatures and objects as if they were difficult terrain. It takes 5 (1d10) force damage if it ends its turn inside an object.

\DndMonsterAction{Variable Illumination} The will-o'-wisp sheds bright light in a 5- to 20-foot radius and dim light for an additional number of feet equal to the chosen radius. The will-o'-wisp can alter the radius as a bonus action.

\DndMonsterSection{Actions}
\DndMonsterAttack[
	name=Shock,
	mod=+4,
	dmg=\DndDice{2d8},
	dmg-type=lightning
]

\DndMonsterAction{Vanish}
The will-o'-wisp and its light magically become invisible until it attacks or uses its Consume Life, or until its concentration ends (as if concentrating on a spell).
\end{multicols}
\end{DndMonster}
\subsection{Wraith}
\begin{DndMonster}[width=\textwidth + 8pt]{Wraith}
\begin{multicols}{2}
\DndMonsterType{Medium undead}
\DndMonsterBasics[armor-class={13}, hit-points={67 (9d8 + 27)}, speed={0 ft., fly 60 ft. (hover)}]
\MonsterStats{-2}{+3}{+3}{+1}{+2}{+2}
\DndMonsterDetails[saving-throws={}, skills={}, damage-immunities={necrotic, poison}, damage-resistances={acid, cold, fire, lightning, thunder; bludgeoning, piercing, and slashing from nonmagical attacks that aren't silvered}, damage-vulnerabilities={}, condition-immunities={charmed, exhaustion, grappled, paralyzed, petrified, poisoned, prone, restrained}, senses={darkvision 60 ft., passive Perception 12}, languages={the languages it knew in life}, challenge={5 (1,800 XP)}]
\DndMonsterAction{Incorporeal Movement} The wraith can move through other creatures and objects as if they were difficult terrain. It takes 5 (1d10) force damage if it ends its turn inside an object.

\DndMonsterAction{Sunlight Sensitivity} While in sunlight, the wraith has disadvantage on attack rolls, as well as on Wisdom (Perception) checks that rely on sight.

\DndMonsterSection{Actions}
\DndMonsterAttack[
	name=Life Drain,
	distance=melee,
	type=weapon,
	mod=+6,
	reach=5,
	dmg=\DndDice{4d8 + 3},
	dmg-type=necrotic,
	extra={. The target must succeed on a DC 14 Constitution saving throw or its hit point maximum is reduced by an amount equal to the damage taken. This reduction lasts until the target finishes a long rest. The target dies if this effect reduces its hit point maximum to 0.}
]
\DndMonsterAction{Create Specter}
The wraith targets a humanoid within 10 feet of it that has been dead for no longer than 1 minute and died violently. The target's spirit rises as a specter in the space of its corpse or in the nearest unoccupied space. The specter is under the wraith's control. The wraith can have no more than seven specters under its control at one time.
\end{multicols}
\end{DndMonster}


\section{Monsters (X)}\label{sec:monsters-x}

\FloatBarrier
\section{Monsters (Z)}\label{sec:monsters-z}
\subsection{Zombie}
\begin{DndMonster}[width=\textwidth + 8pt]{Zombie}
\begin{multicols}{2}
\DndMonsterType{Medium undead}
\DndMonsterBasics[armor-class={8}, hit-points={22 (3d8 + 9)}, speed={20 ft.}]
\MonsterStats{+1}{-2}{+3}{-4}{-2}{-3}
\DndMonsterDetails[saving-throws={Wis +0}, skills={}, damage-immunities={poison}, damage-resistances={}, damage-vulnerabilities={}, condition-immunities={poisoned}, senses={darkvision 60 ft., passive Perception 8}, languages={understands the languages it knew in life but can't speak}, challenge={1/4 (50 XP)}]
\DndMonsterAction{Undead Fortitude} If damage reduces the zombie to 0 hit points, it must make a Constitution saving throw with a DC of 5 + the damage taken, unless the damage is radiant or from a critical hit. On a success, the zombie drops to 1 hit point instead.

\DndMonsterSection{Actions}
\DndMonsterAttack[
	name=Slam,
	distance=melee,
	type=weapon,
	mod=+3,
	reach=5,
	dmg=\DndDice{1d6 + 1},
	dmg-type=bludgeoning
]
\end{multicols}
\end{DndMonster}
\subsection{Ogre Zombie}
\begin{DndMonster}[width=\textwidth + 8pt]{Ogre Zombie}
\begin{multicols}{2}
\DndMonsterType{Large undead}
\DndMonsterBasics[armor-class={8}, hit-points={85 (9d10 + 36)}, speed={30 ft.}]
\MonsterStats{+4}{-2}{+4}{-4}{-2}{-3}
\DndMonsterDetails[saving-throws={Wis +0}, skills={}, damage-immunities={poison}, damage-resistances={}, damage-vulnerabilities={}, condition-immunities={poisoned}, senses={darkvision 60 ft., passive Perception 8}, languages={understands Common and Giant but can't speak}, challenge={2 (450 XP)}]
\DndMonsterAction{Undead Fortitude} If damage reduces the zombie to 0 hit points, it must make a Constitution saving throw with a DC of 5 + the damage taken, unless the damage is radiant or from a critical hit. On a success, the zombie drops to 1 hit point instead.

\DndMonsterSection{Actions}
\DndMonsterAttack[
	name=Morningstar,
	distance=melee,
	type=weapon,
	mod=+6,
	reach=5,
	dmg=\DndDice{2d8 + 4},
	dmg-type=bludgeoning
]
\end{multicols}
\end{DndMonster}
