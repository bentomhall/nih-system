\chapter{Appendix B: The World} \label{ch:the-world}
Throughout these rules, you will see references to Quartus and Noefra. Quartus is the main inhabited planet of the Dreams of Hope setting, and Noefra is the north-eastern continent, which serves as the default setting for this system. Other worlds can be used, and Quartus itself has other places to set stories. Dreams of Hope is a long-running, "living world" setting, where each adventuring party makes changes based on their actions, where PCs retire to become NPCs at the end of their adventures and future parties can interact with them. If you wish to ignore all of that, feel free. The world is yours.

A full description of the world will not fit in these margins, but can be found at the \href{https://wiki.admiralbenbo.org}[Dreams of Hope Wiki page]. Everything there is licensed CC-BY 4.0 unless specified otherwise. The current year is 252 AC (After Cataclysm); changes after about 250 AC are marked as such.

\section{Cosmology}
Dreams of Hope is divided into several planes of existence, all constrained to fit within a spherical shell about the same size as the Inner Solar System ($\approx$2 Astronomical Units in radius). There are four major planes and one aberrant plane, although the Elemental is further sub-divided, as is Shadow:
\begin{itemize}
	\item The Mortal plane is the foundation on which all the other planes rest. It is the source of all aether, and the home of most mortal beings who produce said aether.
	\item The Astral plane is the "heavens", the home of most of the gods, ascendants, and angels (as well as devils!), but it is \textit{not} the place of the afterlife. A luminous plane of drifting, super-earth-sized inhabited plates, its nature is hard to comprehend for most mortals.
	\item The Elemental plane is composed of 12 sub-planes forming a radial "pie" shape, 3 for each of the base elements. They are fixed in space, and as the planets of the Mortal orbit through their influence, they cause the seasons. In order from the beginning of spring, those planes are
	\begin{itemize}
		\item Clay, being Earth + Water with Earth dominant.
		\item Stone, being pure Earth.
		\item Coal, being Earth + Fire with Earth dominant.
		\item Lava, being Fire + Earth with Fire dominant.
		\item Flames, being pure Fire.
		\item Lightning, being Fire + Air, with Fire dominant.
		\item Ash or Smoke, being Air + Fire, with Air dominant.
		\item Wind, being pure Air.
		\item Cloud, being Air + Water, with Air dominant.
		\item Ice, being Water + Air with Water dominant.
		\item Ocean, being pure Water.
		\item Mud, being Water + Earth with Water dominant.
	\end{itemize}
	\item Shadow, being the liminal plane that acts as the interface between the Mortal and all the other planes. It serves as the afterlife as well as the home of many of the fey and a hunting ground for demons. Its geography reflects the geography and especially areas of magical significance in the Mortal, although distorted as if through a funhouse mirror. It is composed of four sub-planes:
	\begin{itemize}
		\item Border Shadow, being an empty region where one can transition between planes easily. Other names for this include the Ethereal. Teleportation as well as many spells touch this plane. Movement is by thought, and the space itself reacts to your stray thoughts.
		\item Beastholm, being the place of quiet somnilence and rest. As well as depression and ennui. Home to fantastic animals and plants, it is somber but quite dangerous.
		\item Mirrorhaven, being the place of excitement, energy, illusion, and mania. Full of color and light, akin to a hallucinatory trip. Underneath its cheerful, almost cartoonish surface lurks many dangers.
		\item The Waste, being the result of abyssal corruption, cuts across the layers in some areas. A wasteland of cracked earth and demonic plants and animals, here beings of the Abyss hunt for souls to devour and brave souls (as well as devils from the Astral) hunt the demons in turn.
	\end{itemize}
	\item The Abyss is an aberration. A cyst, a wound in reality, it is not the same size as the rest of the universe. Instead, it is approximately the volume of Quartus and orbits in a complex cloverleaf, causing its influence to wax and wane unpredictably except to the savants. At its heart is the Oblivion Gate, the ever-hungry living black hole that sends fragments of itself, called \textbf{jotnar} to devour everything. These jotnar, when they infect a soul, convert it into a demon; thus the residents of the Abyss (by choice or otherwise) are demons. Others may journey there, but too-long residence is hazardous. Demons can only exist natively in the Wastes and in the Abyss, but can be summoned elsewhere...with the result that the area they are summoned in is contaminated by jotnar energies. Undead are also the result of jotnar infestation, this time of the dead bodies of mortals and the dying remains of mortals.
\end{itemize}

\section{The Mortal Plane}
The primary plane upon which all others rest is the Mortal plane, called so because it is home to most of the mortal souls. This is the normal plane of matter, energy, humans, planets, etc. It consists of the central star, Eua, and four planets named in order from the star outward:
\begin{enumerate}
	\item Eua, the star. Unlike Sol, this is \textit{not} a burning ball of gas undergoing fusion reactions. It's instead a great glowing crystal, radiating luminous- and fire-aspected aether throughout the plane. Visually, it is very similar to Sol, at least as seen from Quartus. Slightly smaller, but still yellow. It does not give off nearly as much heat--the thermal input for the planets comes from the influence of the elemental planes, not the star itself, which provides mostly light.
	\item Primus, the rapidly-shifting, elementally-dominated world bereft of much normal life, located about 0.25 AU from the star (less than the radius of Mercury in our solar system).
	\item Secundus and Tertius, the twin oppositional planets, both slightly smaller than Earth; Secundus is dominated by Earth and Air and consists of vast deserts of barren sand with floating islands of rock. Tertius is dominated by Water + Fire and is dominated by a world-girdling jungle of strange plants and animals. The two are very close and orbit their common barycenter--a sufficiently strong flier can pass from one to the other in a matter of hours. They orbit Eua at about 0.5 AU.
	\item Quartus, with its two (originally 3) moons. An Earthlike planet, it has zero orbital inclination and a perfectly circular orbit. Seasonal variation comes from elemental influence. It is the home of most of the intelligent mortal life in the setting, and is the primary setting for the game. For practical purposes, it can be thought of as Earthlike, except with two moons. It orbits at almost exactly 1 AU, with a year of 384 days of 24 hours (in the conventional reckoning) each. The moons are Quella, a large red moon with a period of 32 days, and Tekki, a smaller (but closer, so their visual size is similar) bone-white moon with a period of 8 days.
	\item The Crystal Sphere forms the boundary of the universe. A sphere of some unimaginably tough, transparent crystal located at 2 AU from Eua, it encompasses all of creation. Outside there is only the Dark Beyond and the memetic thought-forms that inhabit it. Those forms often leak through the boundary (rather the least of them do), where the angels fight to destroy them lest they infect and destroy all life.
\end{enumerate}

The stars are \textit{not} fixed--they are beacons used by angelic legions in their never-ending fight against the entities of the Dark Beyond.

Quartus has five continents, but the game defaults to the north-eastern continent of Noefra and more particularly the western half of that continent. Noefra is the most mixed as to the lineages of mortals, having representatives of most, if not all the lineages.

One key event in recent history is the Cataclysm, a time about 250 years ago when misuse of an artifact, combined with the invasion of a primordial entity of chaos, caused the elemental planes to shift, all the gods to die or become depowered, spells to stop working, and massive natural disasters to sweep the lands, killing roughly 70\% of the population of Noefra (and slightly less on other continents). All Noefran calendars use this date as the starting point for their enumeration of years.

\subsection{Western Noefra, ca. 250 AC}
The half-continent of Western Noefra contains several major geographical areas, each with their unique cultures and nations. From the cold boreal forests of the Fiach Woods to the unknown depths of the Blood-thirst Wildlands, from the high-mountain Nocthian Caldera to the rolling Sea of Grass and onto the Giant-spine mountains, there is much diversity. Politically, most of the nations belong to the Federated Nations Council (FNC), which keeps them mostly at peace with each other. Instead of armies, most nations depend on adventurers.

\subparagraph*{Lost Coast} The far western edge of the continent is the Lost Coast, stretching north-south from the Great Sea to the west to the mountain wall of the Nocthian Caldera on the eastern edge. Occupied (as far as anyone knows) by nomadic tribes, the story-book (literally) nation of Auringon, and the xenophobic dwarven city of Hammerhead, as well as rumors of a strange empire known as the Tlalocana to the south, this area has only recently been rediscovered in the wake of the Cataclysm. Auringon joined the FNC only in the summer of 251 AC after adventurers assited them through some troubles; the others are only known by reputation.

\subparagraph*{Nocthian Caldera and Byssia} Dominating the western quarter of the continent is the Nocthian Caldera, an ancient extinct (or so everyone hopes) supervolcano whose caldera stretches about 200 miles in diameter and whose floor is around 7,000 ft in elevation, with walls hitting 14,000 ft in places. Accessible through one real pass to the east, as well as a tunnel from the south and a broad slope to the north-west, it used to be the home of the Nocthians before they departed in the wake of the Cataclysm, as well as the dragonborn nation of Wyrmhold. After the nocthians departed, the Hungering Frost, a planar force from the Plane of Ice invaded and waged war against the dragonborn, pushing them out past Last Hope Pass, where they held (with help) for the next 90 years until the Frost disappeared almost as mysteriously as they arrived.

The Nocthians fled south, conquering the lands now known as Byssia on the edge of Gap-Tooth Bay and becoming the modern Night's Children. There they founded the kami-worshipping, decentralized nation of Byssia. Low on metals, they discovered innovative ways of using enchanted wood. There they live today, raising rice and fishing on the Bay; they've expanded back into the Caldera in search of their ancestral home, but that is going slowly.

\subparagraph*{Fiach Woods/Kotimaa} The entire northern stretch of the subcontinent is one dense mixed boreal forest, occupied mainly by tribal orcs and ihmisi. They live in and among the lost ruins of pre-Cataclysm civilizations, for the wood only grew in the wake of the Cataclysm. The only major city there is Godsfall to the east, part of the Duarchy of Kotimaa (as the eastern part of the woods is named).

\subparagraph*{Lake Coy'in and Surroundings} The entire center portion of the subcontinent is shaped somewhat like a bowl, and the lowest point is the inland sea known as Lake Coy'in. It is roughly 300 miles across in most directions and despite having no outlets, remains fresh water. Some speculate that a portal to the Plane of Ocean lies at its depths and maintains the nature of the lake, but that is unknown. Surrounding the lake are some of the most fertile and heavily occupied lands, at least to teh south and east. The north is the Fiatch Woods, the west is the Lupaus Plain, occupied by the Veteln'aya nomadic elves. They do not appreciate outsiders. To the east is the Sea of Grass, Rauviz, and the Duarchy of Kotimaa; to the south is Southshore and Crisial Kingdom.

\subparagraph*{Wyrmhold} Occupying the eastern flank of the Nocthian Caldera and out onto the Lupaus Plains is the dragonborn, orc, and goblinoid kingdom of Wyrmhold. Warlike, industrial and proud, they are only recently learning how to be at peace after four generations of existential war against the Hungering Frost. They've re-occupied the north-eastern portion of the Caldera. Their capital is at Lyodnoir [l-YOHD-noir], at the foot of Last Hope Pass.

\subparagraph*{Southshore} Only settled in the last 50 years, the plains and hills of Southshore (stretching south from Lake Coy'in to the Windwalker hills) are home to the progressive cosmopolitan nation called the Crisial Kingdom, ruled by an ancient aelvar adventurer who was transported through time to the modern age. To the far south is the Windwalker "confederacy", a group of semi-allied goblinoid tribes whose hobgoblins serve as the neutral heart of the FNC infrastructure; the administrative heart of the FNC is there at Fort Hope.

Much trade passes from Wyrmhold and Byssia through Crisial to parts east and vice versa--it is a natural chokepoint for travel and trade. Almost anyone can be found in Crisial City; many of the remaining gwerin have made their homes there. On the plains to the east rises the Crystal Spire, an ancient and mysterious tower that stretches into the heavens and is said to never be the same twice as one tries to climb it, full of monsters and doors to demiplanes.

\subparagraph*{Bloodthirst Wildlands} The wildlands have never been settled and civilized, even at the height of the western empire that dominated this subcontinent for nearly three millennia. The land itself is strange there, and since it was heavily corrupted by demonic machinations in more recent years (said corruption only stemmed and partially purged by adventurers in 205 AC), it has only become more strange. The only significant towns there are Freeport on the western edge (along shores of Gap-tooth Bay) and a few small villages of fang-kin on the far eastern flank near the Moon Sea.

The rest is inhabited by goblinoids, were-touched shifters, and many other strange creatures...or so anyone believes.

\subparagraph*{Sea of Grass} Wide, open, and nearly treeless, the Sea of Grass covers from the eastern edge of Lake Coy'in all the way to the Dreamwall Mountains surrouding the Sea of Dreams and the Outer Barrier Range to the far east that splits the continent in half. The most densely inhabited area, it is home to the wall-builders and halflings of the Duarchy, the cosmopolitan city-state of Rauviz and the fanatics of the Holy Kaelthian Republic. Significant cities include Rauviz, the trading capital of the FNC, Kaelthia, once the jewel of the survivors of the Cataclysm and now the home of an oppressive theocracy, and numberless smaller towns and villages.

\subparagraph*{Jungle of Fangs} South of the Sea of Grass, sandwitched between the Outer and Inner Barrier Range, is the Jungle of Fangs. Home to three nations who split from the old Stone Throne theocracy, these are the fang-kin and ophidians. The most caste- and status-oriented of the peoples, these jungles are lush with mystery and malice. The city of Asai'ka is dominated by a criminal organization; Kel'al'ar to the south is struggling to get out of caste-induced paralysis, and the northern city of Sha'slar is too beset with problems from the surrouding jungle to have much control.

To the south of the Jungle is the Moon Sea, dotted with islands and dominated by the piratical slavers of the Ship Folk.

\subparagraph*{Dreamwall and Sea of Dreams} The Cataclysm raised a line of hills into low mountains when it sank the former Flower Kingdom beneath the North Sea, creating the shallow and cold Sea of Dreams. Those mountains are now home to many peoples, all owing allegiance (if not much loyalty, being independent sort of folks) to the Duarchy of Kotimaa, one of whose capitals is at Baile Craan (the other at Godsfall in Kotimaa). The Duarchy is a mixed pseudo-aristocracy, with nobles who hold some power but checked by strong and independent yeomanry. Not much trade happens on the Sea of Dreams, as few ports are found along its shores. For now.

\subparagraph*{Uulan Confederacy and Giant-spine Mountains}
The eastern flank of the subcontinent is closed off by two great mountain ranges--the southern Outer Barrier Range and the north-south Giant-spine Mountains. In the former live the arch-traditionalist dwarves of the Uulan Confederacy, with their holds buried under those great mountains. In the latter can be found the iconoclastic nations of Shinevog (founded by dwarves who rejected the dead hand of Uulani traditionalism and now a research state) and Zhapai Karmap, which was founded by adventurers and is accepting to all...as a last resort place of refuge. If you can keep your nose out of trouble, they don't care who you were elsewhere. Also found there are the non-FNC nations of the Abandoned Clans--dwarves who were left behind when their ancestors fled south in the wake of the Cataclysm and who have been touched by various forces--and the jazuu nation of Tuura Adam.

\section{Common Ascendants}
Ascendants are the "gods" of the setting. Some of them are true gods, called the Congregation, and others are just powerful former-mortals who are worshipped as gods and grant some power to their followers.

\subsection{The Congregation, AKA True Gods}
There are 16 true gods, those who draw their power from the universe itself and have fixed domains (or areas of concern). They are shown on the Congregation table, with their holy symbols and a brief description. The true gods are limited in how much they can interact directly with mortals (to preserve free will) and mostly work through obscure omens, dreams, and servants.

\begin{figure*}
	\begin{DndTable}{lXll}
		\textbf{Name} &\textbf{Domains} & \textbf{Personality} & \textbf{Holy Symbol} \\
		Aerielara & Art, music, lust, sensuality. & Vain and beautiful, she is the patron of art, music, and sensuality. Hates ugliness. & A stylized harp \\
		The Hollow King &Assassination, Law, Cruelty, Death (Untimely) & Dour, cruel, but fair. A firm believer in law and order for the masses. Also believes that some need to stand in the shadows to enforce that law. Despises the demonic cults. & Crossed daggers, point downward. \\
		Kela Loran & Commerce, Luck, Gambling & Greedy. Calculating. Mercenary. Prone to cheating at dice or cards. Amoral. & Stacked coins. \\
		Korokonolkom & Mountains, elemental earth. Endurance. & Mostly quiet with volcanic temper if angered. Does not forget or forgive easily. Strong sense of honor. & Three mountains. \\
		Lae Loara & Hunters, wilderness, travellers. Nature. & Impatient with civilization. Kill or be killed. Unconcerned with social niceties. & A stylized tree. \\
		Lon-Ka & The forge, technology, smiths, crafters, and alchemists. Invention. & Stubborn and direct. Takes offense easily if criticized. Believes in freedom of knowledge and dislikes traditionalism. & A hammer. \\
		Melara & Timely death. Endings. Memory. Winter. Life. & Kind but strict. Believes in order and justice. Tends toward stasis. Dislikes (and is disliked by) the Hollow King. Both despise undead. Melara specifically hates necromancy bitterly. Once mother to Sakara and wife to Loran Hae. & A stylized snowflake. \\
		Peor-fala & The home, motherhood, peace, fire. & Gentle but can be fierce if pushed. The peacemaker of the Congregation. Many of her clergy swear an oath of non-violence. & A stylized flame. \\
		Pinwheel & Deception, practical jokes, chaos. & The trickster of the Congregation. Not given to evil acts, but also not to good acts. Hates tyranny and compulsion. & A domino mask. \\
		Roel Kor & Tyranny, control, order. Conquest. & Micromanages. Opposes Pinwheel strongly. Not particularly concerned about collateral damage. Generally not honorable. & A castle wall. \\
		Sarapha & Harvest, agriculture, festivals. Alcohol. Autumn. & Very new to the Congregation (since 251 AC after Loran Hae was deposed). Generous and peace-loving. She and Peor-fala get on well. & A sheaf of grain \\
		Tor Elan & The sun, honorable war, strength, work. Summer. & Stodgy. No sense of humor. Bitter enemy of the Red Fang and hostile were creatures in particular. & A shield emblazoned with the full sun. \\
		Sakara & Growth, beginnings, fertility. Spring. & Friend to all living things. Bubbly. Easily appeased. Smarter than she lets on. Once married to Tor Elan. & A cherry blossom or butterfly (both are used). \\
		Selesurala & Oceans and sailors, cold, storms. Natural disasters. & Changeable, with a temper. No pity for the weak. Not particularly subtle. Once the ruler of the plane of Water and a greatwrym. & A lightning bolt \\
		Yogg-Maggus & Arcane magic, knowledge, sages. & Neutral and dry. Rarely raises his voice. Works to preserve the ebb and flow of magical power throughout the planes. Hates the over-use of magic and punishes those who threaten the fabric of the planes. Promotes responsible sharing of knowledge (which puts him at odds with Lon-ka). & A two-fold spiral. \\
		Ytra & Justice, law, order, contracts & Absolutely no sense of humor or mercy. Does not participate in most of the schemes of the gods. Acts as the enforcer for rules that govern Ascendants. & An eight-pointed star with a dot in the center. \\
	\end{DndTable}
	\caption*{The Congregation}\label{tbl:true-gods}
\end{figure*}


\subsection{Other Commonly-Worshiped Ascendants}
Many other ascendants are worshipped in various regions across Noefra. The more common of them are listed below. These Ascendants take a much more direct role in their worship than the true gods, but are much weaker.
\begin{itemize}
	\item Loran Hae. Until the end of 250 AC, he filled Sarapha's place on the Congregation (the domain of Autumn, the harvest, and agriculture). He was maneuvered (in part by his fellow gods and in part by mortals) into breaching the highest rules binding the gods and intervening directly as an avatar in the Mortal. Whereupon he was roundly beaten in fair combat by said mortals in a trial broadcast (by Ytra) to all mortals and astral residents. He was thrust down from his position, but still commands many worshippers. He's taken this very badly and is scheming to bring Noefra under his heel and break it to his idea of "proper order." Still has a significant worship base in the Sea of Grass, especially the Holy Kaelthian Republic.
	\item Nocthis. Also called the Lady of the Moons, Lady of Mystery, or Lady of the Night. She was a full goddess before the Cataclysm and escaped the fate of the other gods by trickery, but was stuck in a semi-mortal body until 211 AC, when she regained her Ascendant state. Much weaker than before, she is worshipped in Byssia primarily. She favors women and mysteries---her church (the Church of Night Reborn) is composed of nested layers of cells, each of which only knows a few others. Dancing is a major form of worship.
	\item The Queen Ascendant. Once a sacrificial conduit between the mortals of the Jungle of Fangs and a demon prince, she was freed by adventurers in 206 AC and ascended shortly thereafter. She is the patron of the jungle kingdoms. Her religion is complex and caste-oriented, with much place put on status. Has a special veneration for snakes.
\end{itemize}

\subsection{Demon Princes}
There are five demons (dwellers in the Abyss) who are of special note and are called "Princes" by mortals. They have enough power to maintain their own domains in that lightless place and most of the other demons have pledged fealty to one of them (even if working to undermine their prince). Each has a particular area of interest where they are most likely to be found interferering, and by which they are worshipped.
\begin{itemize}
	\item Lloitira, the Lady of Pain. She is a beautiful but twisted creature who delights in torture, pain, and sexual perversion (including necrophilia, rape, and other such evils). She also delights in the corruption of the self-righteous and the destruction of social order, as well as vampirism (she being an ur-vampire herself). Her cult is especially active among the high-ranking members of society including rich merchants. It offers free indulgence in perversion and other darker urges. The symbol of the Pain Cult is a barbed-wire ring. Pain cultists strongly oppose the followers of the Red Fang.
	\item The Twisted. This being, in shape much like a kraken, was one of the four eldest children of the Echidra, the so-called Mother of Monsters. In the aftermath of the Dawn War, it turned on its brethren and mother and was instrumental in getting them imprisoned. Over the ages that followed, it served Leviathan, while also acting towards its own perverted ends. It daemonized after its deceptions were discovered near the end of the Interregnum, becoming the demon prince of corruption. It is the patron of all things distorted and mutated, of abominations, of disease. Its motivations are obscure and incomprehensible. The Twisted Cult has no central organization or symbol. It exists in cells of 3-8 members--cells do not know other cells. These cells grow in power by corrupting the land and living things and spreading disease. Eventually, they collapse. Whether they are destroyed from the outside or the inside varies, and the Twisted does not care. These cells are frequently at cross-purposes with each other--the Twisted believes in the survival of the fittest.
	\item Oro-laen, Black Lord. Proud and formal, Oro-laen is the most intellectual and the most approachable (for mortals) of the Outcast. He even keeps his promises; those contracting with him are advised to carefully review the contract before signing as he insists \textit{everyone} keep to the letter of the agreements. He is interested in the dark side of magic--necromancy, undeath (especially free-willed undeath and lichdom, as he is the ur-lich), black magic, blood sacrifice, etc. He emphasizes knowledge as a source of power over others. His cult (under the symbol of a white skull) recruits from the magically active (especially warlocks and sorcerers) and practices ritual magic including blood sacrifices and necromancy.
	\item The Red Fang. The father of the orcish blood rage, the Red Fang was once an orc who made a deal with a nameless demon for power. He overpowered it and ruled in blood and horror until overthrown by the First Heroes at the dawn of the Third Age.Now bestial, nearly mindless and brutal, the Red Fang is the patron of slaughter, savagery, and blood-lust. Most were-creatures and other shape-shifters pay at least token homage to the Fang, as do spree-killers and berserkers who glory in blood. Its cult is most active among the less civilized tribes (especially orcs). It urges its followers to rend, kill and devour. Its plots are direct and usually extremely violent.
	\item Seleleana, the Jester. Said to be a fragment of the Nameless who rebelled before the Dawn War, this entity is chaos manifest. Any semblance of coherence or consistency is abhorrent to it. It plots against everything and everyone, itself included. It glories in insanity, malicious tricks, false prophecies, and other deceptions. Its cult appears to act randomly and frequently intentionally self-sabotages \textit{within the cells}. Some observers are worried that this appearance of randomness is actually a cunning facade.
\end{itemize}


