\section{Rogue}
\label{cls:rogue}

Design Discussion: Less overtly criminal. Focuses on precision. Tons of skill tricks. Subclasses give overt magical abilities. SDCT 3/7/5/5
\subsection{Class Features}

As a rogue, you have the following class features.

\subsubsection{Hit Points}

\textbf{Hit Dice:} 1d8 per rogue level

\textbf{Hit Points at 1st Level:} 8 + your Constitution modifier

\textbf{Hit Points at Higher Levels:} 1d8 (or 5) + your Constitution modifier per rogue level after 1st

\subsubsection{Proficiencies}

\textbf{Armor:} Light armor

\textbf{Weapons:} Simple weapons, hand crossbows, longswords, rapiers, shortswords

\textbf{Tools:} Thieves' tools

\textbf{Saving Throws:} Dexterity, Intelligence

\textbf{Skills:} Choose four from Acrobatics, Athletics, Deception, Insight, Intimidation, Investigation, Perception, Performance, Persuasion, Sleight of Hand, and Stealth

\subsubsection{Equipment}

You start with the following equipment, in addition to the equipment granted by your background:
\begin{itemize}
	\item (\textit{a}) a rapier or (\textit{b}) a shortsword
	\item (\textit{a}) a shortbow and quiver of 20 arrows or (\textit{b}) a shortsword
	\item (\textit{a}) a burglar's pack, (\textit{b}) a dungeoneer's pack, or (\textit{c}) an explorer's pack
  \item (\textit{a}) Leather armor, two daggers, and thieves' tools
\end{itemize}

\onecolumn
\begin{DndTable}[header=The Rogue\label{tbl:rogue}]{XXXXXXXX}
	\textbf{Level} & \textbf{Proficiency Bonus} & \textbf{Sneak Attack} & \textbf{Features} & \textbf{Skill Tricks} & \textbf{Stamina} & \textbf{Aether} & \textbf{Aether Limit} \\
	1st & +2 & 1d6 & Expertise, Sneak Attack & -- & 1 + CON & 1 & 1 \\
	2nd & +2 & 1d6 & Cunning Action, Skill Tricks & 1 & 2 + CON & 1 & 1 \\
	3rd & +2 & 2d6 & Roguish Archetype & 2 & 3 + CON & 2 & 1 \\
	4th & +2 & 2d6 & Ability Score Improvement & 3 & 4 + CON & 2 & 1 \\
	5th & +3 & 3d6 & Uncanny Dodge & 3 & 5 + CON & 3 & 2 \\
	6th & +3 & 3d6 & Expertise, Roguish Archetype Feature & 4 & 6 + CON & 3 & 2 \\
	7th & +3 & 4d6 & Evasion, Improved Skill Tricks & 4 & 7 + CON & 4 & 2 \\
	8th & +3 & 4d6 & Ability Score Improvement & 5 & 8 + CON & 4 & 2 \\
	9th & +4 & 5d6 & Roguish Archetype Feature & 5 & 9 + CON & 5 & 2 \\
	10th & +4 & 5d6 & Ability Score Improvement & 6 & 10 + CON & 5 & 3 \\
	11th & +4 & 6d6 & Reliable Talent, Expert Skill Tricks & 6 & 11 + CON & 6 & 3 \\
	12th & +4 & 6d6 & Ability Score Improvement & 7 & 12 + CON & 6 & 3 \\
	13th & +5 & 7d6 & Roguish Archetype Feature & 7 & 13 + CON & 7 & 3 \\
	14th & +5 & 7d6 & Blindsense & 7 & 14 + CON & 7 & 3 \\
	15th & +5 & 8d6 & Slippery Mind, Master Skill Tricks & 7 & 15 + CON & 8 & 3 \\
	16th & +5 & 8d6 & Ability Score Improvement & 8 & 16 + CON & 8 & 3 \\
	17th & +6 & 9d6 & Roguish Archetype Feature & 8 & 17 + CON & 9 & 4 \\
	18th & +6 & 9d6 & Elusive & 8 & 18 + CON & 9 & 4 \\
	19th & +6 & 10d6 & Ability Score Improvement & 9 & 19 + CON & 10 & 4 \\
	20th & +6 & 11d6 & Stroke of Luck & 9 & 20 + CON & 10 & 4 \\
\end{DndTable}
\twocolumn

\subsubsection{Expertise}

At 1st level, choose two of your skill proficiencies, or one of your skill proficiencies and your proficiency with thieves' tools. Your proficiency bonus is doubled for any ability check you make that uses either of the chosen proficiencies.

At 6th level, you can choose two more of your proficiencies (in skills or with thieves' tools) to gain this benefit.

\subsubsection{Sneak Attack}

Beginning at 1st level, you know how to strike subtly and exploit a foe's distraction. Once per turn, you can deal an extra 1d6 damage to one creature you hit with an attack if you have advantage on the attack roll. The attack must use a finesse or a ranged weapon.

You don't need advantage on the attack roll if another enemy of the target is within 5 feet of it, that enemy isn't \nameref{condition:incapacitated}, and you don't have disadvantage on the attack roll.

The amount of the extra damage increases as you gain levels in this class, as shown in the Sneak Attack column of the Rogue table.

\subsubsection{Cunning Action}

Starting at 2nd level, your quick thinking and agility allow you to move and act quickly. You can take a bonus action on each of your turns in combat. This action can be used only to take the Dash, Disengage, or Hide action.

\subsubsection{Skill Tricks}

Starting at 2nd level, you've learned additional ways to employ your abilities. You learn one \nameref{sec:skill-tricks-basic} of your choice, even if you don't have proficiency in that skill. See \nameref{ch:skill-tricks} for more details and the rules governing skill tricks.

You gain additional Skill Tricks as shown in the Skill Tricks column of the \nameref{tbl:rogue} table. When you reach 7th level, you can learn \nameref{sec:skill-tricks-advanced}; at 11th level \nameref{sec:skill-tricks-expert}; and 17th level \nameref{sec:skill-tricks-master}. When you learn advanced, expert, or master skill tricks in this way, you do not have to meet any prerequisites.

When you gain access to a new Skill Trick, you can also swap any Skill Trick you know for a new one you could otherwise learn at that point.

\subsubsection{Roguish Archetype}

At 3rd level, you choose an archetype that you emulate in the exercise of your rogue abilities: , all detailed at the end of the class description. Your archetype choice grants you features at 3rd level and then again at 6th, 9th, 13th, and 17th level.

\subsubsection{Ability Score Improvement}

When you reach 4th level, and again at 8th, 10th, 12th, 16th, and 19th level, you can increase one ability score of your choice by 1. As normal, you can't increase an ability score above 20 using this feature.

You can also pick a Skill Trick (included in the skill tricks column of the \nameref{tbl:rogue} table) but you must meet the prerequisites for skill tricks learned in this way. See \nameref{ch:skill-tricks} for that list.

\subsubsection{Uncanny Dodge}

Starting at 5th level, when an attacker that you can see hits you with an attack, you can use your reaction to halve the attack's damage against you. You can use this ability even if you have used your reaction already by expending 2 STA.

\subsubsection{Evasion}

Beginning at 7th level, you can nimbly dodge out of the way of certain area effects, such as a red dragon's fiery breath or an *ice storm* spell. When you are subjected to an effect that allows you to make a Dexterity saving throw to take only half damage, you instead take no damage if you succeed on the saving throw, and only half damage if you fail.

\subsubsection{Reliable Talent}

By 11th level, you have refined your chosen skills until they approach perfection. Whenever you make an ability check that lets you add your proficiency bonus, you can treat a d20 roll of 9 or lower as a 10.

\subsubsection{Blindsense}

Starting at 14th level, if you are able to hear, you are aware of the location of any hidden or invisible creature within 10 feet of you.

\subsubsection{Slippery Mind}

By 15th level, you have acquired greater mental strength. You gain proficiency in Wisdom saving throws.

\subsubsection{Elusive}

Beginning at 18th level, you are so evasive that attackers rarely gain the upper hand against you. No attack roll has advantage against you while you aren't incapacitated.

\subsubsection{Stroke of Luck}

At 20th level, you have an uncanny knack for succeeding when you need to. If your attack misses a target within range, you can turn the miss into a hit. Alternatively, if you fail an ability check, you can treat the d20 roll as a 20.

Once you use this feature, you can't use it again until you finish a short or long rest.

\subsection{Roguish Archetypes}

Rogues have many features in common, including their emphasis on perfecting their skills, their precise and deadly approach to combat, and their increasingly quick reflexes. But different rogues steer those talents in varying directions, embodied by the rogue archetypes. Your choice of archetype is a reflection of your focus—not necessarily an indication of your chosen profession, but a description of your preferred techniques.

\subsubsection{Shadowdancer}

Shadowdancers are infiltration and covert work specialists. They've trained their souls to the degree that they can truly become one with the shadows, wrapping the substance of the Shadow plane around themselves and slipping between the cracks just as that liminal plane "slips between the cracks" of the other planes. This lends them both extraordinary grace in combat as well as enhanced stealth abilities.

\begin{DndTable}[header=Shadowdancer\label{tbl:shadowdancer}]{XX}
	\textbf{Rogue Level} & \textbf{Features} \\
	3 & Shadow Strike, Stalk the Shadows \\
	6 & Improved Uncanny Dodge \\
	9 & Hide in Plain Sight \\
	13 & Shrouded Nature \\
	17 & Ethereal Body \\
\end{DndTable}

\subsubsection{Shadow Strike}
Starting at 3rd level when you take this archetype, your strikes while in dimmed lighting are harder to stop. You have advantage on any attack made while you or your target are in any lighting condition other than sunlight.

Additionally, when you hit with a weapon attack and have advantage on the attack, you can spend 1 STA to add your proficiency bonus to the damage dealt. If you do so, the entire attack deals force damage.

\subsubsection{Stalk the Shadows}
Starting at 3rd level when you take this archetype, you can magically transport yourself between the shadows. Spend 1 AET and a bonus action to teleport to an area of shadow large enough to fit your body that you can see within 60 ft. You must be in an area of shadow large enough to fit your body already. For this ability, "shadow" includes any dimly-lit or unlit area as well as the shadows cast by objects and other obstructions. The vertical dimension of the shadow doesn't matter.

\subsubsection{Improved Uncanny Dodge}
Starting at 6th level, you can shunt incoming attacks partially into Shadow more effectively. When you use your Uncanny Dodge, the effect lasts until the end of the current turn instead of only for that attack.

\subsubsection{Hide in Plain Sight}
Starting at 9th level, you can wrap the stuff of Shadow around yourself. You can spend 1 AET to attempt to hide even when directly observed and/or in bright light. If you succeed at the attempt and do not reach total obscurement by the beginning of your next turn, you immediately become unhidden.

\subsubsection{Shrouded Nature}
Starting at 13th level, you have learned to shroud yourself in shadow even in bright light. As an action, you can expend 2 AET to create an aura of shadow around yourself for one hour. This magical aura leaves you lightly obscured and upgrades other sources of light obscurement to heavy obscurement. In addition, you always count as being in shadow for the purpose of Stalk the Shadows; neither your starting or ending point need to be shadowed by any external source.  

\subsubsection(Ethereal Body)
Starting at 17th level, you have learned to transition to the Border Shadow more easily than most. As a bonus action while you are affected by Shrouded Nature, you can go fully ethereal. While you are ethereal you cannot be seen by any creature on the Mortal plane that does not have truesight, but you can see the Mortal plane. You can pass through walls and most barriers other than ones that explicitly affect the Border Shadow. You cannot affect the Mortal while ethereal. You can exit the Border Shadow as a bonus action.

When you go ethereal, you can bring your gear, the objects you are carrying, and any unconscious creatures. You cannot bring a conscious creature with you, willing or not.