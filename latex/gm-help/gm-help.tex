\documentclass[10pt,twoside,openany,bg=none]{dndbook}

\usepackage[english]{babel}
\usepackage[utf8]{inputenc}
\usepackage{import}
\usepackage{mdframed}
\usepackage{caption}
\usepackage{hyperref}
\usepackage{tabularx}
\usepackage[section]{placeins}

\hfuzz=1pt

\extrafloats{1000}

\hypersetup {
    colorlinks = true,
    urlcolor = blue,
    linkcolor = blue,
    citecolor = red
}

\makeatletter
\newcommand{\smartnameref}[2]{%
    \@ifundefined{r@#1}%
      {#2}%
      {\nameref{#1}}%
}
\makeatother

\title{NIH GM Guidance}
\author{Ben Hall}
\date{}
\begin{document}
\maketitle
\section{How to use this document}
This is an auxiliary document for Game Masters (GMs) of the NIH TTRPG system. It's not needed to play the game, but is designed to answer questions GMs might have and to provide additional helps, including guidance on treasure distribution, encounter building, setting DCs for checks, etc. There is no expectation that players will or will not read it. It does not contain \textit{rules}---everything here is couched as \textit{guidance and suggestions}. The tone is intended to be much less formal than the main play document, as it's friendly advice from one long-running GM to another, as well as a frank discussion of how I, as the designer, was thinking the game would be run. Feel free to deviate from this guidance if your table needs something different. That's a core principle of NIH---no one can know better what your table needs than you do. You're the final authority on that subject. As a game designer, all I can do is point out pitfalls I've seen and solutions that have worked at other tables. Consider this document to be a toolkit.

\subsection{Core Principles}
NIH is built around a few core principles.
\begin{itemize}
    \item[] \textbf{The focus is the PCs.} This is the story of the player characters and their heroics (or heroic failures). They are the ones shaping the narrative by their actions and the consequences of those actions. Things that boil down to bookeeping or get in the way of that should be the first things to go when something has to be ditched. This also means that it's not \textit{your} story that's being told, although you're a critical part of it.
    \item[] \textbf{The party's fun outweighs all rules.} As the GM, you are the final backstop for the table's fun. Each player has a substantial responsibility to help themselves and their party have fun. But you're the last resort. Not the rules. "Sorry, I have to do this horribly unfun thing because the rules say so" is a \textit{dodge}, an attempt to shirk your responsibility to the party. Now there are certainly things that aren't immediately fun. For example, a PC dying. But maintaining those systems helps the party have fun long-term (most of the time) because a lack of consequences often robs the game of depth. So it's a tricky, table-by-table balance. But certainly, people on the internet or even these rules themselves are going to be worse at maintaining that balance than you will be. Because we can't see what you're dealing with and don't know the people involved.
    \item[] \textbf{You are a player too and so deserve to have fun.} A GM is a critical moving piece of the system. Very few things kill games faster than the GM getting burnt out or losing interest or feeling like they're just going through the motions. Work with your party if you start to feel that way. Talk out of character. Make whatever changes are necessary. Don't feel like you need to sacrifice yourself on the altar of their fun---come to a mutually beneficial solution or stop playing.
    \item[] \textbf{Player agency is important.} Agency is the feeling that your decisions matter. When players don't feel like they have agency, they tend to check out or start acting badly, trying to break things to regain a sense of agency (or as a form of protest). And this is generally bad. Player agency is maintained when players make informed decisions with actual consequences. It doesn't mean they have to have \textit{total} freedom to choose anything, and they certainly don't get to choose the consequences. But throughout the game, they have to be able to make choices that matter. If you're going to do a set piece that requires them to fail, get them to agree to it out of character first. Or do something differently. Both because otherwise you'll probably end up with a broken story (players are notorious for \textit{not} generally wanting to fail and doing all sorts of things to get out of it) and you'll hurt your relationship with the players if you try to force the issue.
    \item[] \textbf{The PCs are working together against the enemies, not each other.} NIH is not designed for player-vs-player action. It's designed for the party working as a (mostly) cohesive group against monstrous foes and challenges. But remember that many abilities do affect everyone. I generally add an out of character breakpoint when someone is going to, for example, hit another PC with a \textit{fireball} spell---I check with the threatened PC's player. If they're not ok, I ask the other player to choose a different target or different spell. PvP actions only resolve if all affected players are ok with it. Stealing from the group or getting the group in serious trouble (such as by attacking a king in his court) are, in my mind, pvp actions.
    \item[] \textbf{The monsters are the enemy, not you.} You will be called on to portray antagonists. That does \textit{not} mean you yourself should be antagonistic to the player or the PCs. That's not a fair fight---the GM automatically wins anything they choose to. And that's bad. You, acting as players' window to the world, shouldn't lie to the players. They may not be observing the real truth (due to failed rolls or illusions or whatever), but your job is to accurately portray what their characters experience. Similarly, you shouldn't act out of malice or desire to hurt the characters or the players. The monsters certainly want to hurt the characters, but that's their problem, and something the rules handle.
    \item[] \textbf{The monsters generally lose.} NIH is a game about heroes. And the basic assumption is that the heroes will win, eventually. Not all the time, but most fights should end up going their way. When they lose, it should be because they messed up, made bad choices, didn't do their research, etc. Not because you threw a "rocks fall, everyone dies" no-win scenario. And loss may or may not mean \textit{death}. An average party will face dozens if not hundreds of fights over their campaign. If they had an average fatality rate of even 5\%, the chances of having no deaths is under 75\% by the 6th fight and under 35\% by the 20th. 
    \item[] \textbf{Out of character problems need out of character solutions.} If someone is causing issues, talk to them. Seriously. And kindly. Not accusing, not angry. Solve out of character problems out of character. Trying to "punish" their character for the player being bad rarely, if ever, works. It usually just makes things worse for everyone via escalation. The game system isn't designed to help you solve people problems. No rules can stop a malicious actor. In the end, it comes down to only playing with people you trust.
    \item[] \textbf{NIH is a game about heroes, not villains.} This is a correlary to the principle above. NIH presumes that the player characters, in their role as protagonists, are trying to make the world generally a better place. Nothing in it is designed for evil parties. Sure, there is no alignment rules. Thus it falls to the party to decide where their collective lines are. This is why you won't see things like player necromancers (or in fact any player-facing way to create undead), and why player-facing charm and domination effects have been toned down. This is mostly a tone thing---the game won't actively \textit{break} if people are bad. But you'll get tonal dissonance. The intended setting is one where there are major threats, but where heroes working together can genuinely make the world a better place (even if only temporarily until the next threat). Monsters exist (including ones in humanoid shape), but the PCs shouldn't \textit{act} like monsters.
\end{itemize}

\subsection{Layout of this document}
This document is divided into several sections that each cover a specific chunk of running the game.
\begin{itemize}
    \item This current section discusses how to use this document.
    \item \nameref{sec:resolving-checks} discusses the role of dice in resolving checks (attack rolls, ability checks, or saving throws). It discusses different ways to set DCs, and what the DC values are intended to mean, as well as gives a few more examples as to when one proficiency might be more appropriate than others. It lightly touches on non-binary resolutions (aka degrees of success/failure) and when you'd want (or not want) to use those.
    \item \nameref{sec:encounter-design} discusses different models for designing both individual encounters or challenges as well as "adventures" (strings of connected challenges). It discusses particular gotchas commonly encountered, and gives some guidance on manipulating the difficulty to be what you want it to be.
    \item \nameref{sec:treasure} discusses the expectations for treasure distribution, as well as presenting ways to randomly generate "appropriate" treasure.
    \item \nameref{sec:worldbuilding} discusses the system's expectations around the world you play in. There is a default setting for NIH, but it's not binding.
    \item \nameref{sec:sample-adventure} presents a sample (short) adventure for level 1-3 characters, designed to quickly introduce people to the system, especially for new GMs. Although it takes place in Quartus, it is designed to be "low lore" and transplantable into any setting.
\end{itemize}

\section{Resolving Checks}\label{sec:resolving-checks}
The mechanical core of NIH is using a d20 to resolve \textit{uncertainty}. If it's already clear from the fiction what should happen, adding dice rolls is at best fluff and at worse confusing. Now, rolling dice is fun for many people. And I don't want to discount that. But in general, the safest guiding principle behind any attack roll, ability check, or saving throw is "is this a situation where both success and failure are plausible outcomes and, more importantly, \textit{interesting, acceptable} outcomes with interesting consequences that move the narrative along." If the answer to either of those questions is no, \textit{don't roll. Just narrate.} This takes some finesse on the GM's part. There are differences between the different kinds of checks, so let's look at those in more detail.

\subsection{Attack Rolls}
NIH is a game about heroes and villains worthy of heros. So most of the time, if the task is "can I damage that creature", there is meaningful uncertainty. Even a completely unaware enemy might flinch out of the way at a critical moment. Even if asleep. And even attacking an object under stress can fail. And with attack rolls, a natural 1 is a miss and a natural 20 is a hit, regardless of the AC. So generally, a player asking to attack something will be resolved with an attack roll.

There are exceptions. A party or NPC might decide to kill creatures that, in the fiction, pose them no threat and that they should be able to defeat without problem (such as a villain kicking a puppy). In this case, there's little reason to spend the extra time to roll the attack---we can just simply move along to the consequences of them hitting. Similarly, a character trying to break an enchanted adamantine wall with a wooden toothpick doesn't require an attack roll---he simply cannot do that. Simply narrate the failure of the attack. In cases like this, there is one situation where rolling might be useful---to decide \textit{how messy} the successful attack is or \texit{how bad} the failure is. That's covered in \nameref{subsec:degrees-of-success}.

\subsection{Saving Throws}
Nobody likes feeling powerless. It's a rare circumstance that a PC will not be able to make a saving throw against an effect---basically unless they are subjected to a condition that imposes automatic failures (such as \smartnameref{condition:paralyzed}{the Paralyzed condition} against Dexterity saving throws), they will be able to at least attempt the saving throw. And the consequences are almost always meaningful---death, damage, or a disabling condition. As a result, saving throws are almost always useful to roll and play out.

There are a few exceptions, discussed below. In all of them, the core principle is \textit{to keep the focus on the party and keep the narrative moving.} NIH is not a physics engine; it's a toolkit designed to help you bring narratives to life and figure out what happens to the party and their goals.
\begin{itemize}
    \item If there isn't anything threatening the party and they're not under time pressure and the condition naturally resolves given time. For example, a slippery slope where the only consequence is falling prone, but no other threat looms. Since all this does is slightly slow down the party, rolling might not be all that important. Just narrate how they slip around and have to move carefully and move on to more important things.
    \item If a damaging effect will do more than the monster's hit points even if halved, don't bother rolling saving throws for them. They can't survive even if they pass the saving throw. This is particularly common at higher levels facing off against a horde of small creatures such as bandits targeted by \textit{fireball}.
    \item If it's just set dressing and the party isn't directly involved. A dragon doing a drive-by flaming of a camp of guards and commoners just kills however many of them as is thematically appropriate. Important NPCs may get saves if narratively appropriate.
\end{itemize}

\subsection{Ability Checks}
Unlike attack rolls and saving throws, ability checks are rarely the result of a fixed ability and thus require more GM adjudication. The same basic principles (only roll when the outcome is uncertain, both results are acceptable both in the narrative and to the table's culture, and both results are interesting) apply, but there's the additional need to set a DC. At the end of the day, the exact value you decide for the DC isn't \textit{that} important, as long as it's reasonable. Don't worry too much about distinguishing a 16 from a 17---it's totally fine to only ever use one of four options (and the fourth very rarely): 10, 15, 20, and 25 (discussed below).

\subsubsection{Resolving Actions Without Fixed Features}
When there isn't a fixed character or monster feature involved in a player or monster's actions, there's a short process you'll have to go through to resolve the action and decide what happens next.
\begin{enumerate}
    \item Decide if any mechanical resolution is actually necessary. Use the core principles here:
    \begin{enumerate}
        \item Is the proposed action impossible? If so, and if the character would reasonably \textit{know} that (even if the player doesn't), I recommend informing the player of that fact and letting them do something else without wasting their action or resources. If the character wouldn't know, simply resolve the attempt as a failure. This should be used sparingly---for things that aren't logical or physical impossiblities (and are merely unlikely), it's often better to let them try and set a high DC. Note that if there is a range of outcomes for failure that could depend on how well you do at an otherwise impossible task, it might be better to use a non-binary check (described below).
        \item Is the proposed action unlikely to fail, given the narrative circumstances? Alternatively, would it be a breach of estabilshed characterization for that routine action to fail? For example, a professional chef will rarely, if ever, have a routine dish (one they cook frequently) just fail. And it would be weird if even 5\% of the time a trained sailor couldn't balance on the mast during normal weather conditions. In these cases, simply resolve the action as a success and move on. This one can be used liberally---players love feeling important. Especially if narrated that this is something most people would have to roll for, but this character is just that awesome. Similarly to above, if there are a range of success options that depend on how well you do (but no failure case), it might be better to use a non-binary check (in this case degrees of success, rather than degrees of failure).
        \item Are there actual consequences that matter and are interesting for both success and failure? This combines a few factors. If there isn't any factor preventing the person from trying again until they succeed and time \textit{doesn't} matter...simply let them succeed. If time \textit{does} matter, but there aren't any other meaningful consequences of failing other than having to try again, a simple ruling is to allow them to automatically succeed at the cost of taking 10 times as much time as normal (defaulting to taking 1 minute, or 10 rounds).
        \item Are you ok with the outcomes of both success and failure? If not, simply narrate whichever one you \textit{are} ok with or change the outcome to be something you can be ok with. In this case, we don't want to let the dice or the rules override our own knowledge of what is fun. \textbf{CAUTION:} Using this to force your preferred narrative denies players their agency and is one form of railroading. This lever is best only pulled when something critically un-fun (as judged by your party) would happen. For many parties, an example might be an ignominious, pointless death for something simple. Instead of a party member falling to their death on a failed Dexterity (Acrobatics) check to land safely from a risky jump (narratively slipping off the edge of a huge cliff), instead they might succeed at balancing (success on the check) at a cost---maybe they lose something from their pack, take an injury (mechanically implemented by a level of exhaustion, etc), or something like that. Similarly, if a party would get deadlocked on a quest if they failed an Intelligence (Investigation) check and not be able to make any more progress, don't have them make that check. They simply find the needed information. Find a different way of testing them.
        \item Is this something the person is doing routinely, rather than a focused effort? Use a passive check (10 + the relevant modifiers) instead of an active one. The most common use of this is Wisdom (Perception) to avoid being surprised.
    \end{enumerate}
    \item Once you've decided that a check is, after all, necessary, you need to decide \textit{what kind of check} is needed. This involves picking an ability score and possibly one or more sources of proficiency to add. There is no extra bonus for a character having proficiency from more than one source on a given check, generally.
    \item Now it's time to pick a DC.
    \item Now determine whether there are other circumstances, not rolled into the DC, that either give the PC a boost or a penalty (ie decide whether the check is at advantage or disadvantage). Examples include having someone else Help (granting advantage) or wearing certain types of armor while trying to sneak (disadvantage on the Dexterity (Stealth) check).
    \item After that, you resolve the check by having the player roll and comparing the result to the DC, as usual.
    \item Finish up by narrating the results and handling any mechanical consequences (lost hit points, death, consumed resources, etc.)
\end{enumerate}

Note that a natural 1 or a natural 20 have no meaning for ability checks \textit{other} than the worst or best possible result. If the DC ends up being set to 25 (a possible but very difficult task), a character with a +4 cannot succeed on the check. Conversely, if the DC ends up being set to 5 (a task that could plausibly fail but that is quite easy), a character with a +4 cannot fail (4 + 1 = 5) on the check. And that's normal.

Remember that a number of things are listed as not generally requiring checks at all, such as jumping, riding a normal mount, climbing, or swimming. As a rule of thumb, if it's something that a normal (normally abled, normally in-shape, but not specifically trained for the task) person can reliably do in real life, it shouldn't require a check from an adventurer. Adventurers are above the norm and shouldn't have to roll for mundane things.

\subsubsection{Ability Scores and Checks}
Ability checks only optionally include proficiency in a skill, tool, or something else---they're listed as [Ability Score]([List of possible proficiencies]) (e.g. Strength (Athletics) would be a Strength check that includes proficiency/expertise if the one making the check is proficient, while a bare Strength check doesn't include proficiency). That said, \textit{most} checks you make on a day-to-day basis will involve some source of proficiency, often from a skill. While skills are tied to an ability score by default (such as Athletics defaulting to Strength), it is totally fine to apply the proficiency to a check for a different ability score if that makes more sense. The canonical example is using Intimidation (normally a Charisma skill relying on verbal threats) with Strength---this might represent a \textit{wordless} intimidation check, such as by bending the target's weapon in half with your bare hands or casually hefting a huge boulder over their head. There are a few traps to avoid, however.
\begin{enumerate}
    \item Letting characters always use their focused ability score. For example, letting the rogue jump longer distances using Dexterity (Acrobatics) instead of Strength (Athletics). This cheapens the game by not having consequences for different choices. Some ability scores are used more than others---for example Dexterity is referred to more than Strength. By letting people do things with Dexterity that should be done with Strength, it provides incentives to use Dexterity and leave Strength out, which harms those classes that are focused on Strength.
    \item Being too rigid about what applies. The GM's role is to decide what checks are called for---players do not call for checks, they describe actions. On the flip side, you should be willing to negotiate with players if they want to, for example, substitute one source of proficiency for a different one. A tactic to use here is to change the narrated result depending on which proficiency they rely on. This is easiest for Intelligence-based "lore checks"---give them different information for an Intelligence (History) check than an Intelligence (Religion) check. On both, a success should give meaningful (in the context of the check) information, but the viewpoint may be different depending on what they're going for.
\end{enumerate}

\subsubsection{Deciding the DC}
The higher the DC, the harder the check. In general, DCs in NIH are context dependent, including who is making the check. This does not mean that the DC is higher for a character with a low ability score (or vice versa). But a noble-born character interacting with nobility may be the only one who \textit{can even possibly make the attempt}, even if they have a low Charisma. Or, if that particular character has annoyed the NPC in question, the DC to get their desired result may be higher. And DCs are generally not constant---it's really rare that the party attempts to perform the same task under the same exact circumstances multiple times. That said, it is generally bad play to adjust the DC depending on the modifier of the person attempting it---the rogue having a +17 to Dexterity (Sleight of Hand) checks shouldn't imply that all such checks should be much harder to "provide a challenge" to the rogue. It's totally ok for tasks to be trivial for one person and difficult for another---that's what class features are for.

Note that setting the DC comes \textit{after} you've decided that a check is necessary. Attempting to do an impossible task has a DC of SUCCESS NOT POSSIBLE, not any specific numeric DC. Similarly, attempting to do a task that has no risk or is otherwise an automatic success has a DC of FAILURE NOT POSSIBLE.

The system expects that DCs will be in a range between about 5 and 30, although most of the time, useful DCs are between 10 and 20. DCs below 10 usually indicate that a task should probably have been an automatic success \textit{unless} the consequences of failing it are so severe that even that small chance of failure is enough. And usually that is only even a possibility if someone in the party has a low or negative modifier. The three most common DCs will probably be 10, 15, and 20. Remember that DCs for tasks without proficiency are between 10 percentage points and 60 percentage points (accounting for possible expertise) more difficult than those without proficiency.

Generally, the system works best when the probability of success on most checks is approximately 50\% ($\plusminus$ 10-20\%). Too much higher and things like advantage provide less value. Too much lower and you're rolling in vain, which can feel disheartening. These other types still happen, but if it's happening regularly, you might want to reconsider how you're setting DCs.

The following discussion and tables assumes as a baseline three different "personas" to help give intuition on what the various DCs mean. One is a "commoner"---they have +0 to all ability scores and no proficiency, leading to a consistent +0 modifier. This represents both the unspecialized PC and the "common man" (although see below for using checks for NPCs). The next is someone with total modifier of +4, whether from proficiency, ability score, or a combination of both. This represents the character of "normal skill," as well as a task that does not involve proficiency being performed by someone specialized in that attribute. The third has a modifier of +11, the maximum possible without expertise or other magic. This represents a high-level specialized character.

\begin{DndTable}[DCs and Chance of Success]{Xccccc}
    Modifier & DC 5 & DC 10 & DC 15 & DC 20 & DC 25 \\
    +0 (Disadvantage) & 64 & 30 & 9 & 0.25 & 0 \\
    +0 (Normal) & 80 & 55 & 30 & 5 & 0 \\
    +0 (Advantage) & 96 & 79.75 & 51 & 9.75 & 0\\
    +4 (Disadvantage) & 100 & 56.25 & 25 & 6.25 & 0 \\
    +4 (Normal) & 100 & 75 & 50 & 25 & 0 \\
    +4 (Advantage) & 100 & 93.75 & 75 & 43.75 & 0 \\
    +11 (Disadvantage) & 100 & 100 & 72.25 & 36 & 12.25 \\
    +11 (Normal) & 100 & 100 & 85 & 60 & 35 \\
    +11 (Advantage) & 100 & 100 & 97.75 & 84 & 57.75
\end{DndTable}

Note that advantage and disadvantage don't make impossible things possible (or vice versa), they merely make possible things more likely/unlikely (respectively). Higher modifiers make a dramatic difference---someone with a +4 at disadvantage has about the same chance of success on a DC 10 check as someone at +0 rolling normally and only an approximately 5\% lower chance on a DC 15 or 20. Effectively, "disadvantage" is worth a variable amount between -4 and -5, except that something that cannot fail for a higher modifier isn't now failable with disadvantage. Similarly, advantage is worth a variable +4 or +5, \textit{except} that it doesn't actually change 0\% success chances to something doable.

Most people aren't sensitive to small percentage-point changes in probability. As a result, it's usually not worth trying to be more specific than about steps of 5. It is totally fine to pick from the following (all assuming that people are going to do things they're good at, so using the "normal skill" persona):
\begin{itemize}
    \item A very easy check that a commoner might fail but a normally-skilled person shouldn't has \textbf{DC 5}. Use these sparingly.
    \item An easy check that a normal-skill person should only fail rarely has \textbf{DC 10}.
    \item A check that a normal-skill person should fail about 50\% of the time has \textbf{DC 15}.
    \item A check that should be difficult for a normal-skill person (in the absence of advantage or magical help) has \textbf{DC 20}.
    \item A check that should require magical help to even be possible for a normal-skill person (but isn't fundamentally impossible given that assistance) has \textbf{DC 25}. Use these sparingly.
\end{itemize}

\subsection{Passive vs Active Checks}
Some of the time, the GM needs to determine whether an adventurer succeeds at something without the \textit{player} saying that they're particularly doing anything. Most often, this is in connection to one of two cases: routine actions or secret information. Routine actions are things the adventurer is just doing all the time, effectively standing orders. People are presumed to be normally watchful of their surroundings while not distracted or actively doing something that requires their focus. As a result, there's a large number of things they might be able to see. Instead of constantly rolling Wisdom (Perception) checks, you can summarize this by comparing the character's Passive Perception score (calculated as 10 + the character's Wisdom modifier + their proficiency if proficient in Perception) to the DC of the trap or item (or especially hiding creature).

Similarly, if there's a question about something that the player shouldn't necessarily know, where the act of asking for a check might give away information and change behavior (a type of metagaming), the DM can choose to use a passive check for a different ability score or proficiency source (calculated similarly to passive Perception).

Note that these are passive \textit{from the perspective of the player}, not from the perspective of the character. The character is actively looking around; if they're \textit{not} (because they're unconscious or busy doing something else), they don't get the benefit of any check and automatically fail any necessary check. But the player isn't actively making a check.

Other than passive Perception, these checks come up infrequently at most tables. Passive Perception is the exception because it's used to determine whether a creature is surprised when initiative begins.

\subsection{Non-binary Check Results}
Also called "degrees of success or failure", non-binary checks represent cases where the uncertainty isn't in whether the action succeeds or not, but \textit{how good or bad the results are}. For example, a PC asking a king to give up their throne (without something like a victorious army backing them up or similar force majure) is always a failure. But someone who is particularly charming \textit{might} avoid having the guards throw them and the party into jail or kill them and instead merely be ejected from court and/or laughed at. Here, the Charisma (Persuasion) check might have a DC of 10 to soften the consequences at all (anything below a 10 is "guards try to take them to jail") and for every 5 above they get slightly softened consequences. To be clear---no check result can end with them in control of the kingdom. At best, they might get the king to find them amusing and throw a few copper their direction (ie no signficant consequence). On the other side (degrees of success), many checks about "what do I know about \textit{topic}" will never result in "I don't know anything." Characters can be presumed to have heard rumors, legends, etc. about various monster types, for example. In this case, a higher result might result in either more information or more relevant information. A check result below 10 might be "trolls are very hungry all the time" (a true, but not very useful fact), while a 15 might be "trolls are known to fear fire and acid", while a result of 20 might be more like "trolls will regenerate from anything unless their wounds are cauterized with fire or acid. They are faster and smarter than they look."

I prefer to err on the side of more information rather than less, so I tend to be generous with checks made to gain knowledge. But each table can decide what level of mystery they're shooting for.

\subsection{Group Checks and Piling On}
One very common question is "who rolls this check?" Alongside that, a common temptation for parties is to do "pile in" checks, where one person attempts, rolls low, and then others go "can I roll as well"? Letting everyone roll, just like always letting the specialist roll, both result in a distinct decrease in difficulty. If each person has a 50\% chance of success, two people rolling is like rolling at advantage. Three makes it almost a guaranteed success.

Two tactics to tame this tendency are group checks and the Help action. If the task in question is something where more assistance isn't always better (such as a group of people debating a topic) but the party has time to work together and cover for each other, you can use a Group Check. Here, have everyone roll, and the check is a success if half the party or more succeeds. Examples include decoding inscriptions on a wall as part of solving a puzzle (the party can discuss and come to consensus) or helping each other scale obstacles that they'd need to roll for. Here, the party succeeds or fails as a group. Another path is to have the others Help the person actually doing the check, granting them advantage. Holding the light, handing them tools, etc.

In general, well-run checks are not repeatable because there are consequences that happen at the end of the check for both a success and a failure. The PC has already given it their best attempt---trying again with that method is futile. Either the circumstances must change or the party must change their approach.

\section{Encounter Design}\label{sec:encounter-design}
adsfads

\section{Treasure}\label{sec:treasure}
adsfads

\section{Worldbuilding}\label{sec:worldbuilding}
adsfadsf

\section{Sample Adventure}\label{sec:sample-adventure}
asdfasdf

\end{document}