\section{Shaman}

Design Discussion: Replaces the druid. No wild shape. UCT is manifest zones: basically placeable aoes. Heavily control-oriented. SDCT 5/4/7/4.

\subsection{Class Features}

As a shaman, you gain the following class features.

\subsection{Hit Points}

\textbf{Hit Dice:} 1d8 per shaman level

\textbf{Hit Points at 1st Level:} 8 + your Constitution modifier

\textbf{Hit Points at Higher Levels:} 1d8 (or 5) + your Constitution modifier per shaman level after 1st

\subsection{Proficiencies}

\textbf{Armor:} Light armor, medium armor, shields (shamans will not wear armor or use shields made of metal)

\textbf{Weapons:} Clubs, daggers, darts, javelins, maces, quarterstaffs, scimitars, sickles, slings, spears

\textbf{Tools:} Herbalism kit

\textbf{Saving Throws:} Intelligence, Wisdom

\textbf{Skills:} Choose two from Arcana, Animal Handling, Insight, Medicine, Nature, Perception, Religion, and Survival

\subsection{Equipment}

You start with the following equipment, in addition to the equipment granted by your background:
\begin{itemize}
\item (*a*) a wooden shield or (*b*) any simple weapon
\item (*a*) a scimitar or (*b*) any simple melee weapon
\item Leather armor, an explorer's pack, and a shamanic focus
\end{itemize}

\begin{DndTable}[header=The Shaman\label{tbl:shaman}]{XXXXXXXXXXXXX}
 Level & Proficiency Bonus & Features                               & Cantrips Known & Spells Known & Stamina & Aether & Aether Limit \\
 1st   & +2                & Spellcasting                           & 2              & 2 + WIS   & 1    & 4     & 2   \\
 2nd   & +2                & Manifest Zones                         & 2              & 3 + WIS   & 1    & 8     & 3   \\
 3rd   & +2                & Shaman Circle                          & 2              & 4 + WIS   & 2    & 12    & 4   \\
 4th   & +2                & Ability Score Improvement 				& 3    		     & 5 + WIS   & 3    & 16    & 5   \\
 5th   & +3                & -                                      & 3              & 6 + WIS   & 3    & 20    & 6   \\
 6th   & +3                & Shaman Circle feature                  & 3              & 7 + WIS   & 4    & 24    & 7   \\
 7th   & +3                & Dual Zones                             & 3              & 8 + WIS   & 4    & 28    & 8   \\
 8th   & +3                & Ability Score Improvement 				& 3              & 9 + WIS 	 & 5    & 32    & 9   \\
 9th   & +4                & -                                      & 3              & 10 + WIS  & 5    & 36    & 10   \\
 10th  & +4                & Shaman Circle feature                  & 4              & 11 + WIS  & 6    & 40    & 11   \\
 11th  & +4                & Enhanced Zones                         & 4              & 12 + WIS  & 6    & 44    & 12   \\
 12th  & +4                & Ability Score Improvement              & 4              & 13 + WIS  & 7    & 48    & 13   \\
 13th  & +5                & -                                      & 4              & 14 + WIS  & 7    & 52    & 13   \\
 14th  & +5                & Shaman Circle feature                  & 4              & 15 + WIS  & 8    & 56    & 14   \\
 15th  & +5                & -                                      & 4              & 16 + WIS  & 8    & 60    & 14   \\
 16th  & +5                & Ability Score Improvement              & 4              & 17 + WIS  & 9    & 64    & 15   \\
 17th  & +6                & -                                      & 4              & 18 + WIS  & 9    & 68   & 15   \\
 18th  & +6                & Timeless Body                       		& 4              & 19 + WIS  & 10   & 72   & 16   \\
 19th  & +6                & Ability Score Improvement              & 4              & 20 + WIS  & 10   & 76   & 16   \\
 20th  & +6                &                                        & 4              & 20 + WIS  & 10   & 80   & 17   \\
\end{DndTable}

\subsection{Spellcasting}

Drawing on the primal essence of nature itself, you can cast spells to shape that essence to your will.

\subsection{Cantrips}

At 1st level, you know two cantrips of your choice from the shaman spell list. You learn additional shaman cantrips of your choice at higher levels, as shown in the Cantrips Known column of the Shaman table.

\subsubsection{Preparing and Casting Spells}

\nameref{tbl:shaman} table shows how much aether (AET) you have to cast your spells and do other magical tasks. To cast a spell that requires aether, you must expend aether equal to its cost or greater. You regain all expended aether when you finish a long rest. It also shows your Aether Limit, which is the maximum aether you can expend on a single action.

You know a certain number of shaman spells, choosing from the shaman spell list. You can trade out any known spell for any other spell you can learn from that list when you finish a long rest. When you do so, choose a number of shaman spells equal to your Wisdom modifier + your shaman level (minimum of one spell). To prepare a spell you must be able to cast it without exceeding your Aether Limit.

\subsection{Spellcasting Ability}

Wisdom is your spellcasting ability for your shaman spells. The power of your spells comes from your devotion to your deity. You use your Wisdom whenever a shaman spell refers to your spellcasting ability. In addition, you use your Wisdom modifier when setting the saving throw DC for a shaman spell you cast and when making an attack roll with one.

\textbf{Spell save DC} = 8 + your proficiency bonus + your Wisdom modifier

\textbf{Spell attack modifier} = your proficiency bonus + your Wisdom modifier

\subsubsection{Ritual Casting}

You learn a common incantation (see \nameref{ch:incantations} for the list) of your choice. When you reach 5th level, you learn an uncommon incantation of your choice, and at 11th level you learn a rare incantation of your choice. You can cast any incantation you learned from this feature without needing a Ritual Scroll in hand.

\subsubsection{Spellcasting Focus}

You can use a shamanic focus (see chapter \ref{ch:equipment}, “Equipment”) as a spellcasting focus for your shaman spells.

\subsection{Manifest Zones}
Shamans are ties between the planes. Their magic allows them to bring influences of the other planes into the Mortal, creating areas called Manifest Zones. Starting at level 2, you've learned to manifest the energy of one of the planes into your present one. As an action on your turn, you can manifest a zone you know at a point you can see within 60 ft. It persists for one minute or until you manifest another zone and affects an area of 10 ft in radius. Choose one of the following as your first known zone.
\subsubsection{Manifest Zone: Air}
For the duration of the zone, allies who start their turn in the zone gain +15 ft to their speed and do not provoke opportunity attacks by moving out of the reach of enemies. Enemies who make ranged attacks against allies in the zone have disadvantage on the attack.
\subsubsection{Manifest Zone: Earth}
For the duration of the zone, allies who start their turn in the zone gain temporary hit points equal to twice your proficiency bonus. In addition, the zone is difficult terrain for your enemies.
\subsubsection{Manifest Zone: Fire}
For the duration of the zone, enemies who enter the zone for the first time on a turn or start their turn in the zone must make a Dexterity saving throw against your Spellcasting DC. On a failed save, they take 2d6 fire damage, or half as much on a success. The damage increases by 1d6 at 5th, 9th, 13th, and 17th level.
\subsubsection{Manifest Zone: Spirit}
For the duration of the zone, you can use your bonus action to heal one creature within the zone by 1d6 + your spellcasting modifier. The amount healed increases by 1d6 at 5th, 9th, 13th, and 17th level.
\subsubsection{Manifest Zone: Water}
For the duration of the zone, enemies that start their turn in the zone or enter it for the first time on a turn must make a Strength saving throw against your Spellcasting DC. On a failed save, their speed is reduced to zero until they spend an action to break themselves out. Enemies that succeed can only move at half speed within the zone unless they succeed on a Dexterity saving throw against the same DC. On a failed Dexterity saving throw, they fall prone.

\subsection{Shaman Circle}

At 3rd level, you choose to identify with a circle of shamans: Circle of the Spirit, Circle of the Elements, or Indwelling Circle, all of which are detailed at the end of the class description. Your choice grants you features at 3rd level and again at 6th, 10th, and 14th level.

\subsection{Ability Score Improvement}

When you reach 4th level, and again at 8th, 12th, 16th, and 19th level, you can increase one ability score of your choice by 1, up to a maximum of +5.

You can also pick a Skill Trick but you must meet the prerequisites for skill tricks learned in this way. See \nameref{ch:skill-tricks} for that list. You can swap out a known skill trick for another you can learn when you gain another skill trick.

\subsection{Timeless Body}

Starting at 18th level, the primal magic that you wield causes you to age more slowly. For every 10 years that pass, your body ages only 1 year.

\subsection{Shaman Circles}

\subsection{Circle of the Spirit}

The Circle of the Spirit acts as a bridge between man and the spirits of nature, as well as the spirits of the departed. Most frequently, shamans of the Spirit Circle are found as tribal advisors, priests of nature-focused communities, and the like.

\subsection{Bonus Cantrip}

When you choose this circle at 2nd level, you learn one additional shaman cantrip of your choice.

\subsection{Natural Recovery}

Starting at 3rd level, you can regain some of your magical energy by sitting in meditation and communing with nature. During a short rest, you can meditate and regain Aether points equal to half your shaman level, rounded up. Once you use this once, you can't do so again until you complete a long rest.

\subsection{Circle Spells}

Your mystical connection to the spirits of nature and man infuses you with the ability to cast certain spells. At 3rd, 5th, 7th, and 9th level you gain access to extra circle spells.

Once you gain access to a circle spell, you always have it prepared, and it doesn't count against the number of spells you can prepare each day. If you gain access to a spell that doesn't appear on the shaman spell list, the spell is nonetheless a shaman spell for you.

\begin{DndTable}[header=Arctic]{XX}
    Shaman Level & Circle Spells      \\              
    3rd         & hold person, detect thoughts \\         
    5th         & spirit guardians, mass healing word \\
    7th         & death ward, greater invisibility \\  
    9th         & reincarnate, dispel evil and good \\ 
\end{DndTable}

\subsection{Spirit's Advice}


\subsection{Nature's Ward}

When you reach 10th level, you can't be charmed or frightened by elementals or fey, and you are immune to poison and disease.

\subsection{Nature's Sanctuary}


\begin{DndComment}{Sacred Plants and Wood}

A shaman holds certain plants to be sacred, particularly alder, ash, birch, elder, hazel, holly, juniper, mistletoe, oak, rowan, willow, and yew. Shamans often use such plants as part of a spellcasting focus, incorporating lengths of oak or yew or sprigs of mistletoe.

 Similarly, a shaman uses such woods to make other objects, such as weapons and shields. Yew is associated with death and rebirth, so weapon handles for scimitars or sickles might be fashioned from it. Ash is associated with life and oak with strength. These woods make excellent hafts or whole weapons, such as clubs or quarterstaffs, as well as shields. Alder is associated with air, and it might be used for thrown weapons, such as darts or javelins.

 Shamans from regions that lack the plants described here have chosen other plants to take on similar uses. For instance, a shaman of a desert region might value the yucca tree and cactus plants.
\end{DndComment}

\begin{DndComment}{Shamans and the Gods}
 Some shamans venerate the forces of nature themselves, but most shamans are devoted to one of the many nature deities worshiped in the multiverse (the lists of gods in appendix B include many such deities). The worship of these deities is often considered a more ancient tradition than the faiths of clerics and urbanized peoples.
\end{DndComment}