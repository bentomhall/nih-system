\section{Warlock}

Design Discussion: This one's going to change a lot (not yet done). The goal is to move toward the 3e version, with most things tied up in (class feature) eldritch blast + blast shapes + essences. Spellcasting will come only via invocations that grant the ability to grab specific spells off of specific (full-caster) lists. Only class that will get a familiar. SDCT 3/5-7/5-7/5.

\subsection{Class Features}

As a warlock, you gain the following class features.

\subsection{Hit Points}

\textbf{Hit Dice:} 1d8 per warlock level

\textbf{Hit Points at 1st Level:} 8 + your Constitution modifier

\textbf{Hit Points at Higher Levels:} 1d8 (or 5) + your Constitution modifier per warlock level after 1st

\subsection{Proficiencies}

\textbf{Armor:} Light armor

\textbf{Weapons:} Simple weapons

\textbf{Tools:} None

\textbf{Saving Throws:} Wisdom, Charisma

\textbf{Skills:} Choose two skills from Arcana, Deception, History, Intimidation, Investigation, Nature, and Religion

\subsection{Equipment}

You start with the following equipment, in addition to the equipment granted by your background:
\begin{itemize}
\item (\textit{a}) a light crossbow and 20 bolts or (\textit{b}) any simple weapon
\item (\textit{a}) a component pouch or (\textit{b}) an arcane focus
\item (\textit{a}) a scholar's pack or (\textit{b}) a dungeoneer's pack
\item Leather armor, any simple weapon, and two daggers
\end{itemize}

\begin{DndTable}[header=The Warlock\label{tbl:warlock}]{XXXXXXXXXXXX}
 Level & Proficiency Bonus & Features              & Cantrips Known & Blast Shapes & Blast Effects & Invocations & Stamina & Aether & Aether Limit \\
 1st   & +2  & Otherworldly Patron, Eldritch Blast & 2  & ---  & ---  & --- & 1  & 3  & 2  \\
 2nd   & +2  & Eldritch Invocations, Blast Effects & 2  & ---  & 1    & 2   & 1  & 4  & 2  \\
 3rd   & +2  & Pact Boon                           & 2  & ---  & 1    & 3   & 2  & 7  & 3  \\
 4th   & +2  & Ability Score Improvement           & 3  & ---  & 2    & 3   & 2  & 9  & 3  \\
 5th   & +3  & Blast Shapes                        & 3  & 1    & 2    & 4   & 3  & 11 & 5  \\
 6th   & +3  & Otherworldly Patron feature         & 3  & 1    & 2    & 4   & 3  & 13 & 5  \\
 7th   & +3  & Greater Blast Effects               & 3  & 2    & 2    & 5   & 4  & 17 & 8  \\
 8th   & +3  & Ability Score Improvement           & 3  & 2    & 2    & 5   & 4  & 20 & 8  \\
 9th   & +4  & Greater Blast Shapes                & 3  & 3    & 2    & 6   & 5  & 26 & 12 \\
 10th  & +4  & Otherworldly Patron feature         & 4  & 3    & 2    & 6   & 5  & 30 & 12 \\
 11th  & +4  & Mystic Arcanum                      & 4  & 4    & 3    & 7   & 5  & 35 & 14 \\
 12th  & +4  & Ability Score Improvement           & 4  & 4    & 3    & 7   & 6  & 35 & 14 \\
 13th  & +5  & Mystic Arcanum                      & 4  & 5    & 3    & 8   & 6  & 40 & 16 \\
 14th  & +5  & Otherworldly Patron feature         & 4  & 5    & 3    & 8   & 6  & 40 & 16 \\
 15th  & +5  & Mystic Arcanum                      & 4  & 6    & 3    & 9   & 7  & 45 & 18 \\
 16th  & +5  & Ability Score Improvement           & 4  & 6    & 3    & 9   & 7  & 45 & 18 \\
 17th  & +6  & Mystic Arcanum                      & 4  & 7    & 4    & 10  & 7  & 50 & 20 \\
 18th  & +6  & -                                   & 4  & 7    & 4    & 10  & 8  & 50 & 20 \\
 19th  & +6  & Ability Score Improvement           & 4  & 8    & 4    & 11  & 8  & 55 & 22 \\
 20th  & +6  & Eldritch Master                     & 4  & 8    & 4    & 11  & 8  & 55 & 22 \\
\end{DndTable}

\subsection{Otherworldly Patron}

At 1st level, you have struck a bargain with an otherworldly being of your choice: the Archfey, the Fiend, or the Great Old One, each of which is detailed at the end of the class description. Your choice grants you features at 1st level and again at 6th, 10th, and 14th level.

\subsection{Cantrips}

You know two cantrips of your choice from any list. You learn additional warlock cantrips of your choice at higher levels, as shown in the Cantrips Known column of the Warlock table.

\subsubsection{Spellcasting Ability}

Charisma is your spellcasting ability for your warlock spells, so you use your Charisma whenever a spell refers to your spellcasting ability. In addition, you use your Charisma modifier when setting the saving throw DC for a warlock spell you cast and when making an attack roll with one.

\textbf{Spell save DC} = 8 + your proficiency bonus + your Charisma modifier

\textbf{Spell attack modifier} = your proficiency bonus + your Charisma modifier

\subsection{Eldritch Blast}

The contact with your patron has awoken you to a strange, otherworldly power. Unlike conventional spellcasters, you don't learn or cast regular spells naturally, other than cantrips. Instead, you primarily shape and throw raw aether, molding it into shapes and aspecting it in various ways. This is called an "eldritch blast".

At its most basic, as an action on your turn you can shoot a bolt of raw kinetic energy at a creature or object within 90 ft. of you. Make a spell attack roll. On a hit, the bolt deals 1d8 + your Charisma modifier bludgeoning damage to the target. This damage increases by 1d8 at level 5, 11, and 17. This counts as casting a cantrip.

\subsection{Blast Effects}

At 2nd level, you have learned to add additional effects to your \textit{eldritch blast} on hit by spending AET. These are called Blast Effects, and are detailed at the end of the class entry. You learn 1 blast effect and can apply it to your \textit{eldritch blast}. You learn additional blast effects as your level increases, as shown in the Blast Effects column of the \nameref{tbl:warlock} table. Whenever you learn a new blast effect, you can choose one of the blast effects you know and replace it with another blast effect you could learn at that level. You can only apply a single blast effect to each use of eldritch blast unless the effect says otherwise.

Starting at level 7, you can learn Blast Effects labeled as Greater and the cost of non-greater blast effects is reduced by 1 (to a minumum of 0). Expending AET on Blast Effects increases the effective cost of the eldritch blast for the purpose of overcoming resistances, immunities, and other effects that care about the aether cost of a spell.

\subsection{Eldritch Invocations}

In your study of occult lore, you have unearthed eldritch invocations, fragments of forbidden knowledge that imbue you with an abiding magical ability.

At 2nd level, you gain two eldritch invocations of your choice. Your invocation options are detailed at the end of the class description. When you gain certain warlock levels, you gain additional invocations of your choice, as shown in the Invocations Known column of the Warlock table.

Additionally, when you gain a level in this class, you can choose one of the invocations you know and replace it with another invocation that you could learn at that level.

\subsection{Pact Boon}

At 3rd level, your otherworldly patron bestows a gift upon you for your loyal service. You gain one of the following features of your choice.

\subsection{Pact of the Chain}

You learn the \textit{find familiar} spell and can cast it as a ritual. The spell doesn't count against your number of spells known.

When you cast the spell, you can choose one of the normal forms for your familiar or one of the following special forms: imp, pseudodragon, quasit, or sprite.

Additionally, when you take the Attack action, you can forgo one of your own attacks to allow your familiar to make one attack of its own with its reaction.

\subsection{Pact of the Blade}

You can use your \textit{eldritch blast} to create a magical weapon of solidified aether in your hand. It takes the form and statistics of any melee weapon. You have proficiency with this weapon even if you normally would not. You can use your Charisma as the ability modifier for attacks, but you add your Strength modifier (or Dexterity for finesse weapons if you choose) to the damage as usual. You can apply blast effects to it by expending AET as normal. The first time you hit with this weapon on a turn, the damage dealt is equal to your \textit{eldritch blast} damage instead of the normal weapon damage if this would be greater.

Your pact weapon disappears if it is more than 5 feet away from you for 1 minute or more. It also disappears if you use this feature again, if you dismiss the weapon (no action required), or if you die.

You can transform one magic weapon into your pact weapon by performing a special ritual while you hold the weapon. You perform the ritual over the course of 1 hour, which can be done during a short rest. You can then dismiss the weapon, shunting it into an extradimensional space, and it appears whenever you create your pact weapon thereafter. You can't affect an artifact or a sentient weapon in this way. The weapon ceases being your pact weapon if you die, if you perform the 1-hour ritual on a different weapon, or if you use a 1-hour ritual to break your bond to it. The weapon appears at your feet if it is in the extradimensional space when the bond breaks.

\subsection{Pact of the Tome}

Your patron gives you a grimoire called a Book of Shadows. When you gain this feature, choose three cantrips from any class's spell list (the three needn't be from the same list). Choose one spell from any list that costs less than 3 AET. While the book is on your person, you can cast those cantrips at will and cast the spell by expending AET. They don't count against your number of cantrips known. Regardless of what list they came from, Charisma is your spellcasting modifier for these spells. When you gain a level, you can replace the known spell with another that costs less than your AET limit.

If you lose your Book of Shadows, you can perform a 1-hour ceremony to receive a replacement from your patron. This ceremony can be performed during a short or long rest, and it destroys the previous book. The book turns to ash when you die.

\subsection{Ability Score Improvement}

When you reach 4th level, and again at 8th, 12th, 16th, and 19th level, you can increase one ability score of your choice by 2, or you can increase two ability scores of your choice by 1. As normal, you can't increase an ability score above 20 using this feature.

\subsection{Blast Shapes}

When you reach 5th level, you learn to modify the shape of your \textit{eldritch blast} in various ways. You learn one Blast Shape (detailed below) and can apply it by spending the indicated amount of AET. You must choose the shape when you use your eldritch blast feature and before you know whether it hits or not. You learn additional blast shapes as your level increases, as shown in the Blast Shapes column of the \nameref{tbl:warlock} table. Whenever you learn a new blast effect, you can choose one of the blast shapes you know and replace it with another blast shapes you could learn at that level.

Starting at level 9, you can learn Greater Blast Shapes, and the cost of non-greater blast shapes is reduced by 1 (to a minimum of 0). Expending AET on Blast Shapes increases the effective cost of the eldritch blast for the purpose of overcoming resistances, immunities, and other effects that care about the aether cost of a spell.

\subsection{Mystic Arcanum}

At 11th level, your patron bestows upon you a magical secret called an arcanum. Choose one legendary effect from the legendary list as this arcanum.

You can cast your arcanum spell once without expending aether. You must finish a long rest before you can do so again.

At higher levels, you gain more warlock spells of your choice that can be cast in this way: one at 13th level, one at 15th level, and one at 17th level. You regain all uses of your Mystic Arcanum when you finish a long rest.

\subsection{Eldritch Master}

At 20th level, you can draw on your inner reserve of mystical power while entreating your patron to regain expended aether. You can spend 1 minute entreating your patron for aid to regain all your expended aether. Once you use this feature, you must complete a long rest before using it again.

\subsection{Eldritch Invocations}

If an eldritch invocation has prerequisites, you must meet them to learn it. You can learn the invocation at the same time that you meet its prerequisites. A level prerequisite refers to your level in this class.

\subsubsection{Armor of Shadows}

You can cast \textit{mage armor} on yourself at will, without expending aether or material components.

\subsubsection{Ascendant Step}

\textit{Prerequisite: 9th level}

You can cast \textit{levitate} on yourself at will, without expending aether or material components or requiring concentration.

\subsubsection{Beast Speech}

You can cast speak with beasts at will, as if you were under the effects of the Voice the Voiceless (beast) incantation.

\subsubsection{Beguiling Influence}

You gain proficiency in the Deception and Persuasion skills and gain one skill trick of your choice that relies on one of those skills.

\subsubsection{Bewitching Whispers}

\textit{Prerequisite: 7th level}

You can cast \textit{compulsion} once without using aether. You can't do so again until you finish a long rest.

\subsubsection{Book of Ancient Secrets}

\textit{Prerequisite: Pact of the Tome feature}

You learn two common incantations (see \nameref{ch:incantations} for details) of your choice and can perform them without needing a Ritual Scroll in hand.

Special: you can take this invocation more than once, learning a new incantation each time. If you take it when you are 5th level or above, you can learn an uncommon incantation instead. At 9th level or above you can learn a rare incantation.

\subsubsection{Chains of Carceri}

\textit{Prerequisite: 15th level, Pact of the Chain feature}

You can cast \textit{hold monster} at will—targeting a celestial, fiend, or elemental—without expending aether or material components. You must finish a long rest before you can use this invocation on the same creature again.

\subsubsection{Devil's Sight}

You can see normally in darkness, both magical and nonmagical, to a distance of 120 feet.

\subsubsection{Dreadful Word}

\textit{Prerequisite: 7th level}

You can cast \textit{confusion} once without using aether. You can't do so again until you finish a long rest.

\subsubsection{Eldritch Sight}

You can cast \textit{detect magic} at will, without expending aether.

\subsubsection{Eyes of the Rune Keeper}

You can read all writing.

\subsubsection{Fiendish Vigor}

You can cast \textit{false life} on yourself at will, without expending aether or material components. When you reach 5th level, it acts as if you spent 3 AET on it. At 9th level, you get the benefit of casting it with 5 AET.

\subsubsection{Gaze of Two Minds}

You can use your action to touch a willing humanoid and perceive through its senses until the end of your next turn. As long as the creature is on the same plane of existence as you, you can use your action on subsequent turns to maintain this connection, extending the duration until the end of your next turn. While perceiving through the other creature's senses, you benefit from any special senses possessed by that creature, and you are blinded and deafened to your own surroundings.

\subsubsection{Lifedrinker}

\textit{Prerequisite: 12th level, Pact of the Blade feature}

When you hit a creature with your pact weapon, the creature takes extra necrotic damage equal to your Charisma modifier (minimum 1).

\subsubsection{Mask of Many Faces}

You can cast \textit{disguise self} at will, without expending aether.

\subsubsection{Master of Myriad Forms}

\textit{Prerequisite: 9th level}

You can cast \textit{alter self} at will, without expending aether.

\subsubsection{Minions of Chaos}

\textit{Prerequisite: 9th level}

You can cast \textit{conjure elemental} once without using aether. You can't do so again until you finish a long rest.

\subsubsection{Mire the Mind}

\textit{Prerequisite: 5th level}

You can cast \textit{slow} once without using aether. You can't do so again until you finish a long rest.

\subsubsection{Misty Visions}

You can cast \textit{silent image} at will, without expending aether or material components.

\subsubsection{One with Shadows}

\textit{Prerequisite: 5th level}

When you are in an area of dim light or darkness, you can use your action to become invisible until you move or take an action or a reaction.

\subsubsection{Otherworldly Leap}

\textit{Prerequisite: 9th level}

You can cast \textit{jump} on yourself at will, without expending aether or material components.

\subsubsection{Stolen Knowledge}

You learn one spell from the shaman, priest, or arcanist spell list. You must be able to cast it without breaking your aether limit. You can cast any spells you know this way using aether as normal; your casting ability is Charisma.

Special: you can select this invocation more than once. Each time, pick a different spell. Each time you gain an invocation choice, you can also switch one spell you know via this invocation.

\subsubsection{Sculptor of Flesh}

\textit{Prerequisite: 7th level}

You can cast \textit{polymorph} once without using aether. You can't do so again until you finish a long rest.

\subsubsection{Sign of Ill Omen}

\textit{Prerequisite: 5th level}

You can cast \textit{bestow curse} once without using aether. You can't do so again until you finish a long rest.

\subsubsection{Thief of Five Fates}

You can cast \textit{bane} once without using aether. You can't do so again until you finish a long rest.

\subsubsection{Thirsting Blade}

\textit{Prerequisite: 5th level, Pact of the Blade feature}

You can attack with your pact weapon twice, instead of once, whenever you take the Attack action on your turn. Additional hits after the first only deal the weapon's normal damage instead of the eldritch blast damage.

\subsubsection{Visions of Distant Realms}

\textit{Prerequisite: 15th level}

You can cast \textit{arcane eye} at will, without expending aether.

\subsubsection{Uncanny Skill}
\textit{Prerequisite: 4th level}

You learn a skill trick (see \nameref{ch:skill-tricks}) that you otherwise qualify for. Special: You can pick this invocation multiple times, each time learning a new skill trick.

\subsubsection{Voice of the Chain Master}

\textit{Prerequisite: Pact of the Chain feature}

You can communicate telepathically with your familiar and perceive through your familiar's senses as long as you are on the same plane of existence. Additionally, while perceiving through your familiar's senses, you can also speak through your familiar in your own voice, even if your familiar is normally incapable of speech.

\subsubsection{Whispers of the Grave}

\textit{Prerequisite: 5th level}

You learn the \textit{voice the voiceless} incantation and can cast it without a Ritual Scroll.

\subsubsection{Witch Sight}

\textit{Prerequisite: 15th level}

You can see the true form of any shapechanger or creature concealed by illusion or shape-changing magic while the creature is within 30 feet of you and within line of sight.

\subsection{Blast Effects}

Each blast effect has a cost listed after the name. Applying that blast effect requires expending the listed aether cost. If the cost is listed with a + sign, you can expend additional aether (up to your limit) to increase the effect.

\subsubsection{Beckoning Blast: 1+ AET}
Targets hit by your \textit{eldritch blast} are pulled 5 feet toward you for every aether spent. If they are more than two sizes larger than you are, they can makea Strength saving throw to halve the distance they are moved.

\subsubsection{Blinding Blast: 3 AET}
Targets of your \textit{eldritch blast} must make a Constitution saving throw or take the damage as necrotic damage and be blinded until the end of your next turn. On a successful saving throw, targets take half damage. This Constitution saving throw replaces the attack.

\subsubsection{Draining Blast: 5 AET, greater}
Targets hit your \textit{eldritch blast} take additional necrotic damage equal to your Charisma bonus and you regain hit points equal to the necrotic damage dealt. If this affects multiple creatures, each creature takes the damage but you only heal once.

\subsubsection{Elemental Blast: 1 AET}
When you finish a long or short rest, choose one damage type from the following list: acid, cold, fire, lightning. When you use your eldritch blast, you can choose to deal the chosen damage type instead of bludgeoning.

\subsubsection{Frightening Blast: 3 AET}
Targets of your \textit{eldritch blast} must make a Wisdom saving throw instead of you making an attack roll. On a failed save, they are frightened of you until the end of your next turn and take the full damage. On a successful save they take half damage as psychic damage and are not frightened.

\subsubsection{Hellfire Blast: 5 AET, greater}
Your eldritch blast deals fire damage. Targets hit by your \textit{eldritch blast} take additional fire damage equal to your Charisma bonus and the same amount again at the start of their next turn. This damage pierces resistance and immunity to fire damage.

\subsubsection{Repelling Blast: 1+ AET}
Targets hit by your \textit{eldritch blast} are pushed 5 feet away from you for every aether spent. If they are more than two sizes larger than you are, they can makea Strength saving throw to halve the distance they are moved.

\subsubsection{Clinging Lightning Blast: 5 AET, greater}
Your eldritch blast deals lightning damage and deals additional damage equal to your Charisma bonus, and requires a Dexterity saving throw instead of an attack roll. On a failed save, targets take the regular damage and are paralyzed until the end of your next turn. On a successful save, targets take half damage and are not paralyzed.

\subsection{Blast Shapes}
Blast shapes alter the form of the \textit{eldritch blast} and may alter it from a spell attack to a saving throw.

\subsubsection{Eldritch Arc: 3 AET}
When you use your \textit{eldritch blast}, you can instead choose to make it take the shape of a circular arc with a 10 ft radius centered on you. All creatures within a 20 ft radius of the chosen point must make a Dexterity saving throw. On a failed save, they take damage equal to the eldritch blast damage and are affected by any blast effects. On a success, they take half damage and are not affected by the non-damaging blast effects. Damaging blast effects apply the additional damage (halved on a success) to all targets in the area.

\subsubsection{Eldritch Cone: 4 AET, greater}
When you use your \textit{eldritch blast}, you can instead choose to make it take the shape of a 30 ft cone. All creatures in the area must make a Dexterity saving throw. On a failed save, they take damage equal to the eldritch blast damage and are affected by any blast effects. On a success, they take half damage and are not affected by the non-damaging blast effects. Damaging blast effects apply the additional damage (halved on a success) to all targets in the area.

\subsubsection{Eldritch Spear: 1 AET}
The range of your \textit{eldritch blast} doubles. Additionally, you do not have disadvantage on ranged attacks with it against prone targets or on ranged attacks with enemies within 5 ft of you.

\subsubsection{Eldritch Sphere: 6 AET, greater}
When you use your \textit{eldritch blast}, you can instead choose to make it take the shape of a spherical orb that erupts from a point of your choice within the spells' normal range. All creatures within a 20 ft radius of the chosen point must make a Dexterity saving throw. On a failed save, they take damage equal to the eldritch blast damage and are affected by any blast effects. On a success, they take half damage and are not affected by the non-damaging blast effects. Damaging blast effects apply the additional damage (halved on a success) to all targets in the area. Blast effects that push or pull the target use the center point of the effect as the reference point.

\subsubsection{Eldritch Claws: 1 AET}
Make an unarmed attack using Charisma as your weapon attack modifier. On a hit, apply your \textit{eldritch blast} damage. This counts as slashing damage from a magical weapon.

\subsubsection{Split Bolts: 2 AET}
When you use your \textit{eldritch blast}, you can choose to split the attack into a number of separate attacks equal to the number of damage dice. If you do, make separate attack rolls for each one. On a hit, each bolt deals 1d8 damage of the appropriate attacks. The additional damage from your Charisma modifier only applies to one of them. Additional damage from blast effects only affects a single target hit (of your choice); non-damaging blast effects affect all targets hit.

\subsection{Otherworldly Patrons}

The beings that serve as patrons for warlocks are mighty inhabitants of other planes of existence—not gods, but almost godlike in their power. Various patrons give their warlocks access to different powers and invocations, and expect significant favors in return.

Some patrons collect warlocks, doling out mystic knowledge relatively freely or boasting of their ability to bind mortals to their will. Other patrons bestow their power only grudgingly, and might make a pact with only one warlock. Warlocks who serve the same patron might view each other as allies, siblings, or rivals.

\begin{DndComment}{Breaking your pact}
	A patron's secrets, once given, cannot be withdrawn by an act of will or by losing favor with the patron. This separates them from a priest, whose patron can withdraw their support at any time.

	If a warlock falls out of favor with their patron, DMs may decide that the warlock cannot progress further unless they find a new patron willing to swear the same sort of pact. Alternatively, the warlock can take steps to regain favor. This should not happen lightly or arbitrarily--this should further the narrative and be decided between player and DM and not used as a punishment.
\end{DndComment}

%TODO: rewrite
\subsubsection{Pact of Destruction}

Patrons who proffer pacts of Destruction do so for many reasons, but all of them desire to see their enemies (which list may include all creation) crushed before them. Some do so out of a desire to see new systems grow in the wake of the consuming fire; others desire power or just want to see the world burn. Their servants are warriors, their goals are to reduce their enemies to ash.

\subsubsection{Expanded Spell List}

The Pact of Destruction grants some particular spells to its warlocks at specific points in your career. These spells count as warlock spells for you and you can cast them using aether.

\textbf{Destruction Spells (table)}
\begin{DndTable}[header=Destruction Spells\label{tbl:fiend-spells}]{XX}
 Warlock Level & Spells              \\
 3rd          & false life           \\
 5th          & flaming sphere 			 \\
 9th          & vampiric touch       \\
 13th         & wall of fire         \\
 17th         & dispel evil and good \\
\end{DndTable}

\subsection{Destroyer's Blessing}

Starting at 1st level, when you reduce a hostile creature to 0 hit points, you gain temporary hit points equal to your Charisma modifier + your warlock level (minimum of 1).

\subsection{Unravel}

Starting at 6th level, you can call on your patron to unravel the threads of aether. As a reaction when a creature you can see casts a spell or uses a magical ability that requires a saving throw, you can cause all targets of that spell or effect to make the saving throw at advantage and gain resistance to any damage caused by the effect. You can do so after seeing the initial roll but before any of the roll's effects occur.

Once you use this feature, you can't use it again until you finish a short or long rest.

\subsection{Resilience}

Starting at 10th level, you can choose one damage type when you finish a short or long rest. You gain resistance to that damage type until you choose a different one with this feature.

\subsection{Retribution}

Starting at 14th level, when you take damage from an attack or ability, you can force the caster to make a Constitution saving throw. On a failed save, the creature takes damage equal to being hit by two of your eldritch blasts. You can apply any single blast effect you know to this damage without expending aether; the creature counts as having failed any required saving throws. On a success, the creature takes half as much damage and is considered to have succeeded on any requisite saving throw (the damage is not halved again).

\begin{DndComment}{Your Pact Boon}
    Each Pact Boon option produces a special creature or an object that reflects your patron's nature.
    
    \subparagraph*{Pact of the Chain.} Your familiar is more cunning than a typical familiar. Its default form can be a reflection of your patron, with sprites and pseudodragons tied to the Archfey and imps and quasits tied to the Fiend. Because the Great Old One's nature is inscrutable, any familiar form is suitable for it.
    
    \subparagraph*{Pact of the Blade.} If your patron is the Archfey, your weapon might be a slender blade wrapped in leafy vines. If you serve the Fiend, your weapon could be an axe made of black metal and adorned with decorative flames. If your patron is the Great Old One, your weapon might be an ancient-looking spear, with a gemstone embedded in its head, carved to look like a terrible unblinking eye.
    
    \subparagraph*{Pact of the Tome.} Your Book of Shadows might be a fine, gilt-edged tome with spells of enchantment and illusion, gifted to you by the lordly Archfey. It could be a weighty tome bound in demon hide studded with iron, holding spells of conjuration and a wealth of forbidden lore about the sinister regions of the cosmos, a gift of the Fiend. Or it could be the tattered diary of a lunatic driven mad by contact with the Great Old One, holding scraps of spells that only your own burgeoning insanity allows you to understand and cast.    
\end{DndComment}
