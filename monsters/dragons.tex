\clearpage
\section{Dragons}

All dragons trace their descent from the wyrms of old, primal forces of destruction whose speech unmade matter, returning it to pure elemental force. True dragons retain the largest portion of that essence and are tied to elements; false dragons were either bred from wyrm essence or are the non-sapient descendents of corrupted dragons.

\subsection{Dragon, True}

All true dragons are born as chromatic, rainbow colored hatchlings. Before they leave the nest, they undergo the First Molt, which involves seeking out a form of elemental or planar energy and encysting themselves inside, becoming attuned to that elemental energy if they survive, awakening as wyrmlings. As they grow, they undergo a Second Molt, becoming young dragons when their hoard imperative awakens. This hoard imperative is the driving force for all maturing true dragons. It commands them to seek out and gather \textit{something}, which could be as simple as shiny coins or gems...or as esoteric as the smiles on people's faces when they hear a story. This hoard serves as the dragon's strength and weakness--it channels the aether of the land to the dragon, extending their lifespans and fueling their power (both physical and magical), but if it is lost or destroyed, the dragon experiences extreme trauma, often dying (but not before going berserk and destroying everything around it). Some few dragons can find a new hoard, but dragons protect their hoards with their lives. Once a young dragon collects enough of a hoard, it undergoes the Third Molt, becoming an adult. Adult dragons are mature and fertile, forming bonds with other dragons. Most dragons end their lifecycle here; a few have the added need to dominate other dragons. These eventually undergo a Fourth Molt, becoming the fearsome ancients, leaders of dragon flights. Many dragons have the ability to transform into mortal forms (although they retain their draconic power)

Dragons are intensely magical creatures--their very bones and body is steeped in elemental energy. As such, a true dragon's corpse is quite valuable to alchemists and crafters and mages of all kinds; the eggs of true dragons (which can last centuries in a "fossilized" state before hatching when exposed to the correct energies) have been used as arcane power sources. Both of these uses cause extremely negative from any true dragons who discover them.

Some dragons seek out and learn mortal magic--this effort leaves them visually changed for reasons that are not clear. Their scales become metallic. Generally, metallic dragons are more tolerant of humanoid mortals than their chromatic (non-metallic) kin; their hoards tend to reflect their mortal-focused nature. This does not make them less dangerous. For many metallics, manipulating mortals is a form of a game and they have little concern for the wellbeing of their play pieces except as sources of their hoard.

\begin{figure*}
	\begin{DndTable}[]{XllX}
		\textbf{Color} & \textbf{Damage Type} & \textbf{Saving throw} & \textbf{Special Feature} \\	
		Black &  Necrotic & Constitution & Spellcasting, Shapechange \\
		Blue & Lightning & DEX & Shapechange\\
		Brass & Thunder & CON & Spellcasting or Shapechange\\
		Bronze or Red& Fire & DEX & Spellcasting (Bronze) or shapechange (either) \\
		Copper & Acid & CON & Spellcasting or Shapechange\\
		Green & Acid & DEX & Shapechange OR Amphibious\\
		Gold & Radiant & DEX & Spellcasting, Shapechange\\
		Silver or White & Cold & CON Spellcasting (Silver) or Ice walk (either) or Shapechange (either)\\
	\end{DndTable}
	\caption{True Dragon Colors and Damage Types}
	\label{tbl:dragon-colors}
\end{figure*}

\begin{DndMonster}[float*=b,width=\textwidth + 8pt]{Dragon, Wyrmling}
	\begin{multicols}{2}
\DndMonsterType{Medium dragon}
\DndMonsterBasics[armor-class={17 (natural armor)}, hit-points={33 (6d8 + 6)}, speed={30 ft., fly 60 ft.}]
\MonsterStats{+2}{+0 (+2)}{+1 (+3)}{+0}{+0 (+2)}{+1 (+3)}
\DndMonsterDetails[saving-throws={}, skills={Perception +4, Stealth +4}, damage-immunities={Variable based on color}, damage-resistances={}, damage-vulnerabilities={}, condition-immunities={}, senses={blindsight 10 ft., darkvision 60 ft., passive Perception 14}, languages={Draconic}, challenge={2:2}]
\DndMonsterAction{Special Traits} Pick one of the following based on color:
\begin{itemize}
	\item[] Amphibious: The dragon can breath both air and water.
	\item[] Ice Walk. The dragon can move across and climb icy surfaces without needing to make an ability check. Additionally, difficult terrain composed of ice or snow doesn't cost it extra moment.
\end{itemize}

\DndMonsterAction{Damage Type} The dragon's bite attacks and breath weapon deal damage based on its color. The saving throw also depends on the color. See \nameref{tbl:dragon-colors}.

\DndMonsterAction{Blast Shape} The dragon's breath weapon targets an area chosen from the list below:
\begin{itemize}
	\item[] Ball: 10 ft radius sphere centered on a point within 60 ft
	\item[] Cone: 15 ft cone
	\item[] Line: 15 ft line that is 5 ft wide
\end{itemize}

\DndMonsterSection{Actions}
\DndMonsterAttack[
	name=Bite,
	distance=melee,
	type=weapon,
	mod=+4,
	reach=5,
	dmg=\DndDice{1d10 + 2},
	dmg-type=piercing,
	extra={ plus 2 (1d4) damage of the dragon's damage type.}
]

\DndMonsterAction{Breath Weapon (recharge 5-6)}
The dragon exhales energy in a burst (see Blast Shape). Each creature in that area must make a DC 12 saving throw (DEX or CON based on color), taking 22 (6d6) damage of the dragon's Damage Type on a failed save, or half as much damage on a successful one.
\subsection{Scaling}
To make a 3:3 version, add 2 hit dice (15 HP), increase the ability scores to +4/+1/+3/+1/+0/+2, which increases the bite's modifier to +6 and the damage to 1d10 + 4 + 1d4. Don't change the breath weapon.

To make a 4:4 version, add 6 hit dice (45 HP), increase the ability scores to +4/+1/+3/+1/+0/+3, which increases the bite's modifier to +6 and the damage to 1d10 + 4 + 1d4. The breath weapon goes up to DC 13 and gains 1d6 damage.
\end{multicols}
\end{DndMonster}

\subsection{Dragon, Young}
\begin{DndMonster}[float*=b, width=\textwidth + 8pt]{Dragon, Young}
\begin{multicols}{2}
\DndMonsterType{Large dragon}
\DndMonsterBasics[armor-class={18 (natural armor)}, hit-points={142 (15d10 + 45)}, speed={40 ft., fly 80 ft.}]
\MonsterStats{+5}{+1}{+3}{+2}{+1}{+3}
\DndMonsterDetails[saving-throws={Dex +4, Con +7, Wis +4, Cha +6}, skills={Perception +6, +one}, damage-immunities={Variable by color}, damage-resistances={}, damage-vulnerabilities={}, condition-immunities={}, senses={blindsight 30 ft., darkvision 120 ft., passive Perception 16}, languages={Common, Draconic}, challenge={8:9}]
\DndMonsterAction{Special Traits} Pick one of the following based on color:
\begin{itemize}
	\item[] Amphibious: The dragon can breath both air and water.
	\item[] Ice Walk. The dragon can move across and climb icy surfaces without needing to make an ability check. Additionally, difficult terrain composed of ice or snow doesn't cost it extra moment.
	\item[] Shapechange. The dragon can use its action to transform into a humanoid shape of Medium or smaller size. Its features other than its size do not change.
	\item[] Spellcasting. The dragon can cast spells. These generally should be defensive, control, or utility spells as the dragon has enough raw firepower. Its spellcasting DC is 15.
	\begin{itemize}
		\item[] Pick 3 spells that cost 2 AET. It can cast these 3x/day each.
		\item[] Pick 2 spells that cost 3-4 AET. It can cast these 2x/day each.
		\item[] Pick 1 spell that costs 5 AET. It can cast this once/day.
	\end{itemize}
\end{itemize}

\DndMonsterAction{Damage Type} The dragon's bite attacks and breath weapon deal damage based on its color. The saving throw also depends on the color. See \nameref{tbl:dragon-colors}.

\DndMonsterAction{Blast Shape} The dragon's breath weapon targets an area chosen from the list below:
\begin{itemize}
	\item[] Ball: 15 ft radius sphere centered on a point within 60 ft
	\item[] Cone: 30 ft cone
	\item[] Line: 30 ft line that is 5 ft wide
\end{itemize}

\DndMonsterSection{Actions}
\DndMonsterAction{Multiattack} The dragon makes three attacks: one with its bite and two with its claws.
\DndMonsterAttack[
	name=Bite,
	distance=melee,
	type=weapon,
	mod=+8,
	reach=10,
	dmg=\DndDice{2d10 + 5},
	dmg-type=piercing,
	extra={ plus 4 (1d8) damage of the dragon's Damage Type.}
]
\DndMonsterAttack[
	name=Claw,
	distance=melee,
	type=weapon,
	mod=+8,
	reach=5,
	dmg=\DndDice{2d6 + 5},
	dmg-type=slashing
]
\DndMonsterAction{Breath Weapon (recharge 5-6)}
The dragon exhales energy in a burst (see Blast Shape). Each creature in that area must make a DC 15 saving throw (DEX or CON based on color), taking 45 (10d8) damage of the dragon's Damage Type on a failed save, or half as much damage on a successful one.
\subsection{Scaling}
To increase the dragon's ratings to 9:10, increase its hit points by 15 (no extra hit dice but +2 CON), increase its STR, CON, and CHA by 1 each (increasing its attack modifier by 1 and physical damage modifiers by +1), its breath weapon by 9 (+2d8), and its save DC (spellcasting and breath weapon) by 1 (to 16). Its proficiency goes up by 1 as well.
To increase the dragon's CR to 10:11, increase its hit points by 21 (adding 2 hit dice), increase its STR, CON, and CHA by +2 and its DEX by 1 (increasing its attack modifier by 2 and physical damage modifiers by +2). Its breath weapon increases by 9 (+2d8) and its save DC increases by 2 (to 17). Its proficiency goes up by 1 as well.
\end{multicols}
\end{DndMonster}

\subsection{Dragon, Adult}
\begin{DndMonster}[float*=b,width=\textwidth + 8pt]{Dragon, Adult}
\begin{multicols}{2}
\DndMonsterType{Huge dragon}
\DndMonsterBasics[armor-class={19 (natural armor)}, hit-points={195 (17d12 + 85)}, speed={40 ft., fly 80 ft., swim 40 ft.}]
\MonsterStats{+6}{+1}{+5}{+3}{+2}{+3}
\DndMonsterDetails[saving-throws={Dex +6, Con +10, Wis +6, Cha +8}, skills={Perception +11, Stealth +7}, damage-immunities={Variable based on color}, damage-resistances={}, damage-vulnerabilities={}, condition-immunities={}, senses={blindsight 60 ft., darkvision 120 ft., passive Perception 21}, languages={Common, Draconic}, challenge={16:14}]
\DndMonsterAction{Special Traits} Pick one of the following based on color:
\begin{itemize}
	\item[] Amphibious: The dragon can breath both air and water.
	\item[] Ice Walk. The dragon can move across and climb icy surfaces without needing to make an ability check. Additionally, difficult terrain composed of ice or snow doesn't cost it extra moment.
	\item[] Shapechange. The dragon can use its action to transform into a humanoid shape of Medium or smaller size. Its features other than its size do not change.
	\item[] Spellcasting. The dragon can cast spells. These generally should be defensive, control, or utility spells, as the dragon has enough raw firepower. Its spellcasting DC is 16.
	\begin{itemize}
		\item[] Pick 3 spells that cost 4 AET or less. It can cast these 3x/day each. 
		\item[] Pick 2 spells that cost 7 AET or less. It can cast these 2x/day each.
		\item[] Pick 1 spell that costs 8 AET. It can cast this once/day.
	\end{itemize}
\end{itemize}

\DndMonsterAction{Damage Type} The dragon's bite attacks and breath weapon deal damage based on its color. The saving throw also depends on the color. See \nameref{tbl:dragon-colors}.

\DndMonsterAction{Blast Shape} The dragon's breath weapon targets an area chosen from the list below:
\begin{itemize}
	\item[] Ball: 20 ft radius sphere centered on a point within 90 ft\
	\item[] Cone: 60 ft cone\
	\item[] Line: 60 ft line that is 10 ft wide
\end{itemize}

\DndMonsterAction{Legendary Resistance (3/Day)} If the dragon fails a saving throw, it can choose to succeed instead.

\DndMonsterSection{Actions}
\DndMonsterAction{Multiattack} The dragon can use its Frightful Presence. It then makes three attacks: one with its bite and two with its claws.
\DndMonsterAttack[
	name=Bite,
	distance=melee,
	type=weapon,
	mod=+11,
	reach=10,
	dmg=\DndDice{2d10 + 6},
	dmg-type=piercing,
	extra={ plus 4 (1d8) damage of the dragon's Damage Type.}
]
\DndMonsterAttack[
	name=Claw,
	distance=melee,
	type=weapon,
	mod=+11,
	reach=5,
	dmg=\DndDice{2d6 + 6},
	dmg-type=slashing
]
\DndMonsterAttack[
	name=Tail,
	distance=melee,
	type=weapon,
	mod=+11,
	reach=15,
	dmg=\DndDice{2d8 + 6},
	dmg-type=bludgeoning
]
\DndMonsterAction{Frightful Presence}
Each creature of the dragon's choice that is within 120 feet of the dragon and aware of it must succeed on a DC 16 Wisdom saving throw or become frightened for 1 minute. A creature can repeat the saving throw at the end of each of its turns, ending the effect on itself on a success. If a creature's saving throw is successful or the effect ends for it, the creature is immune to the dragon's Frightful Presence for the next 24 hours.

\DndMonsterAction{Breath Weapon (recharge 5-6)}
The dragon exhales energy in a burst (see Blast Shape). Each creature in that area must make a DC 18 Dexterity saving throw, taking 54 (12d8) acid damage on a failed save, or half as much damage on a successful one.

\DndMonsterSection{Legendary Actions}
The Dragon, Adult can take 3 legendary actions, choosing from the options below. Only one legendary action option can be used at a time and only at the end of another creature's turn. The Dragon, Adult regains spent legendary actions at the start of its turn.
\begin{DndMonsterLegendaryActions}
\DndMonsterLegendaryAction{Detect}{The dragon makes a Wisdom (Perception) check.}
\DndMonsterLegendaryAction{Tail Attack}{The dragon makes a tail attack.}
\DndMonsterLegendaryAction{Wing Attack (Costs 2 Actions)}{The dragon beats its wings. Each creature within 10 feet of the dragon must succeed on a DC 19 Dexterity saving throw or take 13 (2d6 + 6) bludgeoning damage and be knocked prone. The dragon can then fly up to half its flying speed.}
\end{DndMonsterLegendaryActions}

\subsection{Scaling}
To increase the dragon's CR to 17/15, increase its hit points to 212, its ability scores to +7/+1/+6/+3/+2/+3 (increasing its attack bonus to +12 and its weapon damage output by 1), its breath weapon to 63 (14d8), and increase each of its saving throw DCs by 1.

To increase the dragon's CR to 18/16, increase its hit points to 225, its ability scores to +7/+1/+6/+3/+2/+4, its breath weapon to 67 (15d8), its frightful presence DC to 18, its breath weapon DC to 19, and its wing attack DC to 20.
\end{multicols}
\end{DndMonster}

\subsection{Dragon, Ancient}
\begin{DndMonster}[float*=b,width=\textwidth + 8pt]{Dragon, Ancient}
\begin{multicols}{2}
\DndMonsterType{Gargantuan dragon}
\DndMonsterBasics[armor-class={22 (natural armor)}, hit-points={350 (20d20 + 140)}, speed={40 ft., fly 80 ft.}]
\MonsterStats{+8}{+1}{+7}{+4}{+2}{+4}
\DndMonsterDetails[saving-throws={Dex +8, Con +14, Wis +9, Cha +11}, skills={Perception +16, Stealth +9}, damage-immunities={Variable based on color}, damage-resistances={}, damage-vulnerabilities={}, condition-immunities={}, senses={blindsight 60 ft., darkvision 120 ft., passive Perception 26}, languages={Common, Draconic}, challenge={20+:20+}]
\DndMonsterAction{Special Traits} Pick one of the following based on color:
\begin{itemize}
	\item[] Amphibious: The dragon can breath both air and water.
	\item[] Ice Walk. The dragon can move across and climb icy surfaces without needing to make an ability check. Additionally, difficult terrain composed of ice or snow doesn't cost it extra moment.
	\item[] Shapechange. The dragon can use its action to transform into a humanoid shape of Medium or smaller size. Its features other than its size do not change.
	\item[] Spellcasting. The dragon can cast spells. These generally should be defensive, control, or utility spells, as the dragon has enough raw firepower. Its spellcasting DC is 19.
	\begin{itemize}
	\item[]Pick 3 spells that cost 5 AET or less. It can cast these 3x/day each. 
	\item[]Pick 2 spells that cost 8 AET or less. It can cast these 2x/day each.
	\item[]Pick 1 spell that costs 14 AET or less. It can cast this once/day.
	\end{itemize}
\end{itemize}

\DndMonsterAction{Damage Type} The dragon's bite attacks and breath weapon deal damage based on its color. The saving throw also depends on the color. See \nameref{tbl:dragon-colors}.

\DndMonsterAction{Blast Shape} The dragon's breath weapon targets an area chosen from the list below:
\begin{itemize}
	\item[]Ball: 40 ft radius sphere centered on a point within 120 ft
	\item[]Cone: 90 ft cone
	\item[]Line: 90 ft line that is 10 ft wide
\end{itemize}

\DndMonsterAction{Legendary Resistance (3/Day)} If the dragon fails a saving throw, it can choose to succeed instead.

\DndMonsterSection{Actions}
\DndMonsterAction{Multiattack} The dragon can use its Frightful Presence. It then makes three attacks: one with its bite and two with its claws.
\DndMonsterAttack[
	name=Bite,
	distance=melee,
	type=weapon,
	mod=+15,
	reach=15,
	dmg=\DndDice{2d10 + 8},
	dmg-type=piercing,
	extra={ plus 9 (2d8) damage of the dragon's Damage Type.}
]
\DndMonsterAttack[
	name=Claw,
	distance=melee,
	type=weapon,
	mod=+15,
	reach=10,
	dmg=\DndDice{2d6 + 8},
	dmg-type=slashing
]
\DndMonsterAttack[
	name=Tail,
	distance=melee,
	type=weapon,
	mod=+15,
	reach=20,
	dmg=\DndDice{2d8 + 8},
	dmg-type=bludgeoning
]
\DndMonsterAction{Frightful Presence}
Each creature of the dragon's choice that is within 120 feet of the dragon and aware of it must succeed on a DC 19 Wisdom saving throw or become frightened for 1 minute. A creature can repeat the saving throw at the end of each of its turns, ending the effect on itself on a success. If a creature's saving throw is successful or the effect ends for it, the creature is immune to the dragon's Frightful Presence for the next 24 hours.
\DndMonsterAction{Breath Weapon (recharge 5-6)}
The dragon exhales energy in an area (see Blast Shape). Each creature in that area must make a DC 22 saving throw of the type indicated for the dragon's color, taking 71 (13d10) damage of the dragon's Damage Type on a failed save, or half as much damage on a successful one.

\DndMonsterSection{Legendary Actions}
An ancient dragon can take 3 legendary actions, choosing from the options below. Only one legendary action option can be used at a time and only at the end of another creature's turn. The Dragon, Ancient regains spent legendary actions at the start of its turn.
\begin{DndMonsterLegendaryActions}
\DndMonsterLegendaryAction{Detect}{The dragon makes a Wisdom (Perception) check.}
\DndMonsterLegendaryAction{Tail Attack}{The dragon makes a tail attack.}
\DndMonsterLegendaryAction{Wing Attack (Costs 2 Actions)}{The dragon beats its wings. Each creature within 15 feet of the dragon must succeed on a DC 23 Dexterity saving throw or take 15 (2d6 + 8) bludgeoning damage and be knocked prone. The dragon can then fly up to half its flying speed.}
\end{DndMonsterLegendaryActions}
\subsection{Scaling}
To increase the dragon's CR to 20$++$, increase its HP to 385 (22d20+154), its AC to 22, its breath weapon damage to 77 (14d10), and its saving throw DCs to 20 (Frightful Presence and spellcasting), 23 (Breath weapon), and 24 (Wing Attack).

To increase the dragon's CR to 20$+++$, increase its HP to 487 (25d20+225), its AC to 23, its ability scores to +10/+1/+9/+4/+2/+6 (increasing its attack bonus to +17 and the damage modifier of each physical attack to +10), its breath weapon damage to 88 (16d10), and its saving throw DCs to 21 (Frightful Presence and spellcasting), 24 (Breath weapon), and 25 (Wing Attack)/
\end{multicols}
\end{DndMonster}

\clearpage
\subsection{Dragons, False}

\subsection{Wyvern}
\begin{DndMonster}[width=\textwidth + 8pt]{Wyvern}
\DndMonsterType{Large dragon}
\DndMonsterBasics[armor-class={13 (natural armor)}, hit-points={110 (13d10 + 39)}, speed={20 ft., fly 80 ft.}]
\MonsterStats{+4}{+0}{+3}{-3}{+1}{-2}
\DndMonsterDetails[saving-throws={}, skills={Perception +4}, damage-immunities={}, damage-resistances={}, damage-vulnerabilities={}, condition-immunities={}, senses={darkvision 60 ft., passive Perception 14}, languages={—}, challenge={8:5}]
\DndMonsterSection{Actions}
\DndMonsterAction{Multiattack} The wyvern makes two attacks: one with its bite and one with its stinger. While flying, it can use its claws in place of one other attack.
\DndMonsterAttack[
	name=Bite,
	distance=melee,
	type=weapon,
	mod=+7,
	reach=10,
	dmg=\DndDice{2d6 + 4},
	dmg-type=piercing
]
\DndMonsterAttack[
	name=Claws,
	distance=melee,
	type=weapon,
	mod=+7,
	reach=5,
	dmg=\DndDice{2d8 + 4},
	dmg-type=slashing
]
\DndMonsterAttack[
	name=Stinger,
	distance=melee,
	type=weapon,
	mod=+7,
	reach=10,
	dmg=\DndDice{2d6 + 4},
	dmg-type=piercing,
	extra={. The target must make a DC 15 Constitution saving throw, taking 24 (7d6) poison damage on a failed save, or half as much damage on a successful one.}
]
\end{DndMonster}

\subsection{Dragon Turtle}
\begin{DndMonster}[float*=b,width=\textwidth + 8pt]{Dragon Turtle}
\begin{multicols}{2}
\DndMonsterType{Gargantuan dragon}
\DndMonsterBasics[armor-class={20 (natural armor)}, hit-points={341 (22d20 + 110)}, speed={20 ft., swim 40 ft.}]
\MonsterStats{+7}{+0 (+6)}{+5 (+11)}{+0}{+1 (+7)}{+1}
\DndMonsterDetails[saving-throws={}, skills={}, damage-immunities={}, damage-resistances={fire}, damage-vulnerabilities={}, condition-immunities={}, senses={darkvision 120 ft., passive Perception 11}, languages={Aquan, Draconic}, challenge={15:17}]
\DndMonsterAction{Amphibious} The dragon turtle can breathe air and water.

\DndMonsterSection{Actions}
\DndMonsterAction{Multiattack} The dragon turtle makes three attacks: one with its bite and two with its claws. It can make one tail attack in place of its two claw attacks.
\DndMonsterAttack[
	name=Bite,
	distance=melee,
	type=weapon,
	mod=+13,
	reach=15,
	dmg=\DndDice{4d12 + 7},
	dmg-type=piercing
]
\DndMonsterAttack[
	name=Claw,
	distance=melee,
	type=weapon,
	mod=+13,
	reach=10,
	dmg=\DndDice{3d8 + 7},
	dmg-type=slashing
]
\DndMonsterAttack[
	name=Tail,
	distance=melee,
	type=weapon,
	mod=+13,
	reach=15,
	dmg=\DndDice{4d12 + 7},
	dmg-type=bludgeoning,
	extra={. If the target is a creature, it must succeed on a DC 20 Strength saving throw or be pushed up to 10 feet away from the dragon turtle and knocked prone.}
]
\DndMonsterAction{Steam Breath (Recharge 5-6)}
The dragon turtle exhales scalding steam in a 60-foot cone. Each creature in that area must make a DC 18 Constitution saving throw, taking 52 (15d6) fire damage on a failed save, or half as much damage on a successful one. Being underwater doesn't grant resistance against this damage.
\end{multicols}
\end{DndMonster}