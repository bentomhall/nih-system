\section{Legendary Effects}
Legendary effects are magical abilities of great power. In some respects they are similar to spells, but one cannot learn and cast them normally using your personal aether. Instead, you must gain access via a class feature. Some creatures and magic items have the ability to create specific legendary effects. In general, each legendary effect can only be cast (brought into effect) once per day. The class feature will also define what saving throw DC is used if the effect calls for a saving throw.

The class feature that grants access will tell you what level you gain access as well as which effects you can choose from when learning a legendary effect.  For spellcasters, this is most commonly your spell save DC.

Each legendary effect listed below has one or more tags associated with it, listed after the name in the entry. Most of these tags, by themselves, have no effect. Other features may interact with them, however. An effect with the tag \textbf{Concentration} requires concentration as if concentrating on a spell. Every effect will either be tagged as \textbf{Lesser} or \textbf{Greater}. Generally, Lesser Legendary effects are suitable for learning by characters of levels 11-15 and Greater Legendary Effects are suitable for learning by characters of levels 16+. Some Greater effects are scaled up versions of Lesser ones; there are others that contain entirely new effects.

For abilities and effects that key off of a spell or ability's cost, Legendary Effects count as spells costing aether equal to 8 + your character level. Some legendary effects scale with your level as well--that is called out in the effect description.

\section{Legendary Effects A-Z}
\subsection{Abjure Aether Manipulation}
\subparagraph*{Tags} Generic, Greater, Concentration
\subparagraph*{Cast Time} 1 action

A 10-foot-radius invisible sphere of antimagic surrounds you. Within the sphere, spells can’t be cast, summoned creatures disappear, and even magic items become mundane. The sphere moves with you, centered on you until you lose concentration or 1 minute passes.

Spells and other magical effects (including anything that requires the expenditure of aether), except those created by an artifact or a deity, are suppressed in the sphere and can’t protrude into it. Aether expended to cast a suppressed spell is consumed. While an effect is suppressed, it doesn’t function, but the time it spends suppressed counts against its duration.

\subparagraph*{Targeted Effects} Spells and other magical effects, such as \textit{magic missile} and \textit{charm person}, that target a creature or an object in the sphere have no effect on that target.

\subparagraph*{Areas of Magic} The area of another spell or magical effect, such as \textit{fireball}, can’t extend into the sphere. If the sphere overlaps an area of magic, the part of the area that is covered by the sphere is suppressed. For example, the flames created by a \textit{wall of fire} are suppressed within the sphere, creating a gap in the wall if the overlap is large enough.

\subparagraph*{Spells} Any active spell or other magical effect on a creature or an object in the sphere is suppressed while the creature or object is in it.

\subparagraph*{Magic Items} The properties and powers of non-artifact magic items are suppressed in the sphere. For example, a \textit{+1 longsword} in the sphere functions as a nonmagical longsword.

A magic weapon’s properties and powers are suppressed if it is used against a target in the sphere or wielded by an attacker in the sphere. If a magic weapon or a piece of magic ammunition fully leaves the sphere (for example, if you fire a magic arrow or throw a magic spear at a target outside the sphere), the magic of the item ceases to be suppressed as soon as it exits.

\subparagraph*{Magical Travel} Teleportation and planar travel fail to work in the sphere, whether the sphere is the destination or the departure point for such magical travel. A portal to another location, world, or plane of existence, as well as an opening to an extradimensional space, temporarily closes while in the sphere.

\subparagraph*{Creatures and Objects} A creature or object summoned or created by non-instantaneous magic temporarily winks out of existence in the sphere. Such a creature instantly reappears once the space the creature occupied is no longer within the sphere. Creatures controlled but not animated or summoned by magic have the control suppressed while in the sphere.

\subparagraph*{Dispel Magic} Spells and magical effects such as \textit{dispel magic} have no effect on the sphere. Likewise, the spheres created by different \textit{antimagic field} spells don’t nullify each other.

\subsection{Animal Shapes}
\subparagraph*{Tags} Primal, Greater, Concentration
\subparagraph*{Casting Time} 1 action

Your magic turns others into beasts. Choose any number of willing creatures that you can see within 60 feet. You transform each target into the form of a Large or smaller beast with a challenge rating of 4 or lower. On subsequent turns, you can use your action to transform affected creatures into new forms.

The transformation lasts for up to 24 hours for each target, or until the target drops to 0 hit points or dies. You must concentrate on this effect as if on a spell. You can choose a different form for each target. A target’s game statistics are replaced by the statistics of the chosen beast, though the target retains its alignment and Intelligence, Wisdom, and Charisma scores. The target assumes the hit points of its new form, and when it reverts to its normal form, it returns to the number of hit points it had before it transformed. If it reverts as a result of dropping to 0 hit points, any excess damage carries over to its normal form. As long as the excess damage doesn’t reduce the creature’s normal form to 0 hit points, it isn’t knocked unconscious. The creature is limited in the actions it can perform by the nature of its new form, and it can’t speak or cast spells.

The target’s gear melds into the new form. The target can’t activate, wield, or otherwise benefit from any of its equipment.

\subsection{Arcane Sword}

\subparagraph*{Tags} Arcane, Lesser

As an action, you create a sword-shaped plane of force that hovers in your space. It lasts for the one minute.

When the sword appears, you make a melee spell attack against a target of your choice within 60 ft. On a hit, the target takes 3d10 force damage as the sword streaks out at the target. If the target struck is a creature not native to your current plane, it must make a Charisma saving throw. On a failed save, it is banished back to its home plane. If you hit a construct of magical force such as that produced by \textit{wall of force}, the target is instantly destroyed. Until the spell ends, you can use a bonus action on each of your turns to repeat this attack against the same target or a different one as long as they are within 60 ft of you.

\subsection{Circle of Death}
\subparagraph*{Tags} Arcane or Divine, Greater, Concentration
\subparagraph*{Cast Time} 1 action

A portal to the Abyss appears and necrotic energy washes out in a 20-foot-radius, 60' tall cylinder centered at a point within 150 feet. Each creature in that area when it opens must make a Constitution saving throw. A target takes 8d6 + your level necrotic damage on a failed save, or half as much damage on a successful one. Creatures entering the zone for the first time on a turn or starting their turn in the zone must make the same Constitution saving throw, taking the same damage on a failure. The zone lasts until you lose concentration or 1 minute has passed.

% \subsection{Clone}

% \textit{8th-level necromancy}

% **Casting Time:** 1 hour

% **Range:** Touch

% **Components:** V, S, M (a diamond worth at least 1,000 gp and at least 1 cubic inch of flesh of the creature that is to be cloned, which the spell consumes, and a vessel worth at least 2,000 gp that has a sealable lid and is large enough to hold a Medium creature, such as a huge urn, coffin, mud* filled cyst in the ground, or crystal container filled with salt water)

% **Duration:** Instantaneous

% This spell grows an inert duplicate of a living creature as a safeguard against death. This clone forms inside a sealed vessel and grows to full size and maturity after 120 days; you can also choose to have the clone be a younger version of the same creature. It remains inert and endures indefinitely, as long as its vessel remains undisturbed.

% At any time after the clone matures, if the original creature dies, its soul transfers to the clone, provided that the soul is free and willing to return.

% The clone is physically identical to the original and has the same personality, memories, and abilities, but none of the original’s equipment. The original creature’s physical remains, if they still exist, become inert and can’t thereafter be restored to life, since the creature’s soul is elsewhere.

\subsection{Call Divine Servant}
\subparagraph*{Tags} Divine, Lesser
\subparagraph*{Cast Time} 1 minute
\subparagraph*{Other Requirements} A favorable relationship with an ascended power.

By praying ritually and vocally for one minute, you summon a servant of your patron Ascendant. Choose a celestial of challenge rating 4 or lower or a fiend of challenge rating 4 or lower, which appears in an unoccupied space that you can see within 90 ft. The being disappears when it drops to 0 hit points or after one hour. You must concentrate on this as if on a spell.

The being is friendly to you and your companions for the duration. Roll initiative for the being, which has its own turns. It obeys any verbal commands that you issue to it (no action required by you), as long as they don’t violate its nature. If you don’t issue any commands to the being, it defends itself from hostile creatures but otherwise takes no actions.

The GM has the being’s statistics.

\subsection{Call Fey Ally}
\subparagraph*{Tags} Primal, Lesser
\subparagraph*{Cast Time} 1 minute
By vocally pleading with the spirits of nature, you summon a fey creature of challenge rating 6 or lower, or a fey spirit that takes the form of a beast of challenge rating 6 or lower. It appears in an unoccupied space that you can see within 90 ft. The fey creature disappears when it drops to 0 hit points or after one hour. You must concentrate on this as if on a spell.

The fey creature is friendly to you and your companions for the duration. Roll initiative for the creature, which has its own turns. It obeys any verbal commands that you issue to it (no action required by you), as long as they don’t violate its nature. If you don’t issue any commands to the fey creature, it defends itself from hostile creatures but otherwise takes no actions.

The GM has the fey creature’s statistics.

\subsection{Contingency}
\subparagraph*{Tags} Arcane, Greater
\subparagraph*{Cast Time} 10 minutes
\subparagraph*{Other Requirements} A jeweled statuette of yourself costing 1500 gp.

Choose a spell costing 15 or less AET that you can cast, that has a casting time of 1 action, and that can target you. You cast that spell—called the contingent spell—as part of creating the \textit{contingency} effect and must expend the aether to cast it, but the contingent spell doesn’t come into effect. Instead, it takes effect when a certain circumstance occurs. You describe that circumstance when you create the effect. For example, a \textit{contingency} cast with \textit{water breathing} might stipulate that \textit{water breathing} comes into effect when you are engulfed in water or a similar liquid.

The contingent spell takes effect immediately after the circumstance is met for the first time, whether or not you want it to, and then \textit{contingency} ends.

The contingent spell takes effect only on you, even if it can normally target others. You can use only one \textit{contingency} spell at a time. If you cast this spell again, the effect of another \textit{contingency} spell on you ends. Also, \textit{contingency} ends on you if its material component is ever not on your person.

\subsection{Control Weather}
\subparagraph*{Tags} Primal, Greater
\subparagraph*{Cast Time} 10 minutes

You take control of the weather within 5 miles of you for the 8 hours or until you lose concentration (as if on a spell). You must be outdoors to cast this spell. Moving to a place where you don’t have a clear path to the sky ends the spell early.

When you cast the spell, you change the current weather conditions, which are determined by the GM based on the climate and season. You can change precipitation, temperature, and wind. It takes 1d4 × 10 minutes for the new conditions to take effect. Once they do so, you can change the conditions again. When the spell ends, the weather gradually returns to normal.

When you change the weather conditions, find a current condition on the following tables and change it to a new one. When changing the wind, you can change its direction.

\begin{DndTable}[header=Precipitation]{XX}
	\textbf{Stage} & \textbf{Condition} \\
	1 & Clear \\
	2 & Light clouds \\
	3 & Overcast or ground fog \\
	4 & Rain, hail, or snow \\
	5 & Torrential rain, driving hail, or blizzard \\
\end{DndTable}

\begin{DndTable}[header=Temperature]{XX}
	\textbf{Stage} & \textbf{Condition} \\
1     & Unbearable heat \\
2     & Hot             \\
3     & Warm            \\
4     & Cool            \\
5     & Cold            \\
6     & Arctic cold     \\
\end{DndTable}

\begin{DndTable}[header=Wind]{XX}
\textbf{Stage} & \textbf{Condition}     \\
1     & Calm          \\
2     & Moderate wind \\
3     & Strong wind   \\
4     & Gale          \\
5     & Storm         \\
\end{DndTable}

% \subsection{Create Undead}

% \textit{6th-level necromancy}

% **Casting Time:** 1 minute

% **Range:** 10 feet

% **Components:** V, S, M (one clay pot filled with grave dirt, one clay pot filled with brackish water, and one 150 gp black onyx stone for each corpse)

% **Duration:** Instantaneous

% You can cast this spell only at night. Choose up to three corpses of Medium or Small humanoids within range. Each corpse becomes a ghoul under your control. (The GM has game statistics for these creatures.)

% As a bonus action on each of your turns, you can mentally command any creature you animated with this spell if the creature is within 120 feet of you (if you control multiple creatures, you can command any or all of them at the same time, issuing the same command to each one). You decide what action the creature will take and where it will move during its next turn, or you can issue a general command, such as to guard a particular chamber or corridor. If you issue no commands, the creature only defends itself against hostile creatures. Once given an order, the creature continues to follow it until its task is complete.

% The creature is under your control for 24 hours, after which it stops obeying any command you have given it. To maintain control of the creature for another 24 hours, you must cast this spell on the creature before the current 24-hour period ends. This use of the spell reasserts your control over up to three creatures you have animated with this spell, rather than animating new ones.

% **_At Higher Levels._** When you cast this spell using a 7th-level spell slot, you can animate or reassert control over four ghouls. When you cast this spell using an 8th-level spell slot, you can animate or reassert control over five ghouls or two ghasts or wights. When you cast this spell using a 9th-level spell slot, you can animate or reassert control over six ghouls, three ghasts or wights, or two mummies.

% \subsection{Delayed Blast Fireball}

% \textit{7th-level evocation}

% **Casting Time:** 1 action

% **Range:** 150 feet

% **Components:** V, S, M (a tiny ball of bat guano and sulfur)

% **Duration:** Concentration, up to 1 minute

% A beam of yellow light flashes from your pointing finger, then condenses to linger at a chosen point within range as a glowing bead for the duration. When the spell ends, either because your concentration is broken or because you decide to end it, the bead blossoms with a low roar into an explosion of flame that spreads around corners. Each creature in a 20-foot-radius sphere centered on that point must make a Dexterity saving throw. A creature takes fire damage equal to the total accumulated damage on a failed save, or half as much damage on a successful one.

% The spell’s base damage is 12d6. If at the end of your turn the bead has not yet detonated, the damage increases by 1d6.

% If the glowing bead is touched before the interval has expired, the creature touching it must make a Dexterity saving throw. On a failed save, the spell ends immediately, causing the bead to erupt in flame. On a successful save, the creature can throw the bead up to 40 feet. When it strikes a creature or a solid object, the spell ends, and the bead explodes.

% The fire damages objects in the area and ignites flammable objects that aren’t being worn or carried.

% **_At Higher Levels._** When you cast this spell using a spell slot of 8th level or higher, the base damage increases by 1d6 for each slot level above 7th.

\subsection{Disintegrate}
\subparagraph*{Tags} Arcane, Lesser
\subparagraph*{Casting Time}

A thin green ray springs from your pointing finger to a target that you can see within range. The target can be a creature, an object not being worn or carried by a creature, or a creation of magical force, such as the wall created by \textit{wall of force}.

A creature targeted by this effect must make a Dexterity saving throw. On a failed save, the target takes 10d6 + 4x your level force damage. If this damage reduces the target to 0 hit points, it is disintegrated.

A disintegrated creature and everything it is wearing and carrying, except magic items, are reduced to a pile of fine gray dust. The creature can be restored to life only by means of a \textit{true resurrection} effect.

This effect automatically disintegrates a Large or smaller nonmagical object or a creation of magical force. If the target is a Huge or larger object or creation of force, this spell disintegrates a 10-foot-cube portion of it. Unattended Large or smaller magic items struck by this effect are disintegrated unless they are legendary or artifacts. Magic items worn or carried by a creature cannot be individually targeted.

\subsection{Divine Word}
\subparagraph*{Tags} Divine, Lesser
\subparagraph*{Casting Time} 1 action

You utter a divine word, imbued with the power that shaped the world at the dawn of creation. Choose any number of creatures you can see within 30 feet. Each creature that can hear you must make a Charisma saving throw. On a failed save, a creature suffers an effect based on its current hit points:

* 50 hit points or fewer: deafened for 1 minute
* 40 hit points or fewer: deafened and blinded for 10 minutes
* 30 hit points or fewer: blinded, deafened, and stunned for 1 hour
* 20 hit points or fewer: killed instantly 

Regardless of its current hit points, a celestial, an elemental, or a fiend that fails its save is forced back to its plane of origin (if it isn’t there already) and can’t return to your current plane for 24 hours by any means other.

\subsection{Domination}
\subsection*{Tags} Any, Greater
\subsection*{Cast Time} 1 action

You attempt to beguile a creature that you can see within range. It must succeed on a Wisdom saving throw or be charmed by you for one hour or until you lose concentration. If you or creatures that are friendly to you are fighting it, it has advantage on the saving throw.

While the creature is charmed, you have a telepathic link with it as long as the two of you are on the same plane of existence. You can use this telepathic link to issue commands to the creature while you are conscious (no action required), which it does its best to obey. You can specify a simple and general course of action, such as “Attack that creature,” “Run over there,” or “Fetch that object.” If the creature completes the order and doesn’t receive further direction from you, it defends and preserves itself to the best of its ability.

You can use your action to take total and precise control of the target. Until the end of your next turn, the creature takes only the actions you choose, and doesn’t do anything that you don’t allow it to do. During this time, you can also cause the creature to use a reaction, but this requires you to use your own reaction as well.

Each time the target takes damage, it makes a new Wisdom saving throw against the spell. If the saving throw succeeds, the spell ends.

\subsection{Etherealness}
\subparagraph*{Tags} Arcane, Lesser
\subparagraph*{Cast Time} 1 action

You step into the Border Shadow, in the area where it overlaps with your current plane. You remain in the Border Shadow for the 8 hours or until you use your action to dismiss the effect. During this time, you can move in any direction. If you move up or down, every foot of movement costs an extra foot. You can see and hear the plane you originated from, but everything there looks gray, and you can’t see anything more than 60 feet away.

While on the  Border Shadow, you can only affect and be affected by other creatures on that plane. Creatures that aren’t on the Border Shadow can’t perceive you and can’t interact with you, unless a special ability or magic has given them the ability to do so.

You ignore all objects and effects that aren’t on the Border Shadow, allowing you to move through objects you perceive on the plane you originated from.

When the effect ends, you immediately return to the plane you originated from in the spot you currently occupy. If you occupy the same spot as a solid object or creature when this happens, you are immediately shunted to the nearest unoccupied space that you can occupy and take force damage equal to twice the number of feet you are moved.

This effect cannot be cast while you are on the Border Shadow.

% \subsection{Eyebite}

% _6th-level necromancy_

% **Casting Time:** 1 action

% **Range:** Self

% **Components:** V, S

% **Duration:** Concentration, up to 1 minute

% For the spell’s duration, your eyes become an inky void imbued with dread power. One creature of your choice within 60 feet of you that you can see must succeed on a Wisdom saving throw or be affected by one of the following effects of your choice for the duration. On each of your turns until the spell ends, you can use your action to target another creature but can’t target a creature again if it has succeeded on a saving throw against this casting of \textit{eyebite}.

% **_Asleep._** The target falls unconscious. It wakes up if it takes any damage or if another creature uses its action to shake the sleeper awake.

% **_Panicked._** The target is frightened of you. On each of its turns, the frightened creature must take the Dash action and move away from you by the safest and shortest available route, unless there is nowhere to move. If the target moves to a place at least 60 feet away from you where it can no longer see you, this effect ends.

% **_Sickened._** The target has disadvantage on attack rolls and ability checks. At the end of each of its turns, it can make another Wisdom saving throw. If it succeeds, the effect ends.

% \subsection{Feeblemind}

% _8th-level enchantment_

% **Casting Time:** 1 action

% **Range:** 150 feet

% **Components:** V, S, M (a handful of clay, crystal, glass, or mineral spheres)

% **Duration:** Instantaneous

% You blast the mind of a creature that you can see within range, attempting to shatter its intellect and personality. The target takes 4d6 psychic damage and must make an Intelligence saving throw.

% On a failed save, the creature’s Intelligence and Charisma scores become 1. The creature can’t cast spells, activate magic items, understand language, or communicate in any intelligible way. The creature can, however, identify its friends, follow them, and even protect them.

% At the end of every 30 days, the creature can repeat its saving throw against this spell. If it succeeds on its saving throw, the spell ends.

% The spell can also be ended by \textit{greater restoration}, \textit{heal}, or \textit{wish}.

\subsection{Rip Soul}
\subparagraph*{Tags} Arcane, Lesser
\subparagraph*{Cast Time} 1 action

You send negative energy coursing through a creature that you can see within range, causing it searing pain. The target must make a Constitution saving throw. It takes 7d8 + 4x your level necrotic damage on a failed save, or half as much damage on a successful one.

Living creatures who fail the saving throw against this effect gain 5 levels of \nameref{condition:exhaustion}.

% \subsection{Fire Storm}

% _7th-level evocation_

% **Casting Time:** 1 action

% **Range:** 150 feet

% **Components:** V, S

% **Duration:** Instantaneous

% A storm made up of sheets of roaring flame appears in a location you choose within range. The area of the storm consists of up to ten 10-foot cubes, which you can arrange as you wish. Each cube must have at least one face adjacent to the face of another cube. Each creature in the area must make a Dexterity saving throw. It takes 7d10 fire damage on a failed save, or half as much damage on a successful one.

% The fire damages objects in the area and ignites flammable objects that aren’t being worn or carried. If you choose, plant life in the area is unaffected by this spell.

\subsection{Flesh to Stone}
\subparagraph*{Tags} Primal, Lesser, Concentration
\subparagraph*{Cast Time} 1 action

You attempt to turn one creature that you can see within 60 feet into stone. If the target’s body is made of flesh, the creature must make a Constitution saving throw. On a failed save, it is restrained as its flesh begins to harden. On a successful save, the creature isn’t affected.

A creature restrained by this spell must make another Constitution saving throw at the end of each of its turns. If it successfully saves against this spell three times, the spell ends. If it fails its saves three times, it is turned to stone and subjected to the petrified condition for the duration. The successes and failures don’t need to be consecutive; keep track of both until the target collects three of a kind.

If the creature is physically broken while petrified, it suffers from similar deformities if it reverts to its original state.

If you maintain your concentration on this spell for one minute, the creature is turned to stone until the effect is removed.

% \subsection{Forcecage}

% _7th-level evocation_

% **Casting Time:** 1 action

% **Range:** 100 feet

% **Components:** V, S, M (ruby dust worth 1,500 gp)

% **Duration:** 1 hour

% An immobile, invisible, cube-shaped prison composed of magical force springs into existence around an area you choose within range. The prison can be a cage or a solid box, as you choose.

% A prison in the shape of a cage can be up to 20 feet on a side and is made from 1/2-inch diameter bars spaced 1/2 inch apart.

% A prison in the shape of a box can be up to 10 feet on a side, creating a solid barrier that prevents any matter from passing through it and blocking any spells cast into or out from the area.

% When you cast the spell, any creature that is completely inside the cage’s area is trapped. Creatures only partially within the area, or those too large to fit inside the area, are pushed away from the center of the area until they are completely outside the area.

% A creature inside the cage can’t leave it by nonmagical means. If the creature tries to use teleportation or interplanar travel to leave the cage, it must first make a Charisma saving throw. On a success, the creature can use that magic to exit the cage. On a failure, the creature can’t exit the cage and wastes the use of the spell or effect. The cage also extends into the  Border Shadow, blocking ethereal travel.

% This spell can’t be dispelled by \textit{dispel magic}.

% \subsection{Foresight}
% \subparagraph*{Tags} Generic, greater
% \subparagraph*{Cast Time}
% **Casting Time:** 1 minute

% **Range:** Touch

% **Components:** V, S, M (a hummingbird feather)

% **Duration:** 8 hours

% You touch a willing creature and bestow a limited ability to see into the immediate future. For the duration, the target can’t be surprised and has advantage on attack rolls, ability checks, and saving throws. Additionally, other creatures have disadvantage on attack rolls against the target for the duration.

% This spell immediately ends if you cast it again before its duration ends.

% \subsection{Freezing Sphere}

% _6th-level evocation_

% **Casting Time:** 1 action

% **Range:** 300 feet

% **Components:** V, S, M (a small crystal sphere)

% **Duration:** Instantaneous

% A frigid globe of cold energy streaks from your fingertips to a point of your choice within range, where it explodes in a 60-foot-radius sphere. Each creature within the area must make a Constitution saving throw. On a failed save, a creature takes 10d6 cold damage. On a successful save, it takes half as much damage.

% If the globe strikes a body of water or a liquid that is principally water (not including water-based creatures), it freezes the liquid to a depth of 6 inches over an area 30 feet square. This ice lasts for 1 minute. Creatures that were swimming on the surface of frozen water are trapped in the ice. A trapped creature can use an action to make a Strength check against your spell save DC to break free.

% You can refrain from firing the globe after completing the spell, if you wish. A small globe about the size of a sling stone, cool to the touch, appears in your hand. At any time, you or a creature you give the globe to can throw the globe (to a range of 40 feet) or hurl it with a sling (to the sling’s normal range). It shatters on impact, with the same effect as the normal casting of the spell. You can also set the globe down without shattering it. After 1 minute, if the globe hasn’t already shattered, it explodes.

% **_At Higher Levels._** When you cast this spell using a spell slot of 7th level or higher, the damage increases by 1d6 for each slot level above 6th.

\subsection{Glibness}
\subparagraph*{Tags} Generic, Greater
\subparagraph*{Cast Time} 1 action

When you make a Charisma check for the next hour, you can replace the number you roll with a 15. Additionally, no matter what you say, magic that would determine if you are telling the truth indicates that you are being truthful and magic that would compel you to tell the truth is ignored without notifying the caster of that interference.

\subsection{Globe of Invulnerability}
\subparagraph*{Tags} Arcane, Lesser, Concentration
\subparagraph*{Cast Time} 1 action

An immobile, faintly shimmering barrier springs into existence in a 10-foot radius around you and remains for one minute or until you lose concentration.

Any non-legendary spell cast from outside the barrier can’t affect creatures or objects within it. Such a spell can target creatures and objects within the barrier, but the spell has no effect on them. Similarly, the area within the barrier is excluded from the areas affected by such spells.

\subsection{Harm}
\subparagraph*{Tags} Arcane or Divine, Lesser
\subparagraph*{Cast Time} 1 action

You unleash a virulent curse on a creature that you can see within range. The target must make a Constitution saving throw. On a failed save, it is cursed and takes 14d6 + 2x your level necrotic damage, or half as much damage on a successful save. Cursed creatures cannot regain hit points by any means until they complete two consecutive long rests, after which time the curse fades.

\subsection{Heal}
\subparagraph*{Tags} Divine, Lesser
\subparagraph*{Cast Time} 1 action

Choose a creature that you can see within range. A surge of positive energy washes through the creature, causing it to regain 70 hit points or half their total hit points, whichever is greater. This spell also ends blindness, deafness, and any diseases affecting the target, as well as the curse caused by \textit{harm}. This spell has no effect on constructs or undead.

\subsection{Holy Aura}
\subparagraph*{Tags} Divine, Greater, concentration
\subparagraph*{Cast Time} 1 action

Divine light washes out from you and coalesces in a soft radiance in a 30-foot radius around you for one minute. Creatures of your choice in that radius when you cast this effect shed dim light in a 5-foot radius and have advantage on all saving throws, and other creatures have disadvantage on attack rolls against them until the spell ends. In addition, when a fiend or an undead hits an affected creature with a melee attack, the aura flashes with brilliant light. The attacker must succeed on a Constitution saving throw or be blinded until the spell ends.

% \subsection{Imprisonment}

% _9th-level abjuration_

% **Casting Time:** 1 minute

% **Range:** 30 feet

% **Components:** V, S, M (a vellum depiction or a carved statuette in the likeness of the target, and a special component that varies according to the version of the spell you choose, worth at least 500 gp per Hit Die of the target)

% **Duration:** Until dispelled

% You create a magical restraint to hold a creature that you can see within range. The target must succeed on a Wisdom saving throw or be bound by the spell; if it succeeds, it is immune to this spell if you cast it again. While affected by this spell, the creature doesn’t need to breathe, eat, or drink, and it doesn’t age. Divination spells can’t locate or perceive the target.

% When you cast the spell, you choose one of the following forms of imprisonment.

% **_Burial._** The target is entombed far beneath the earth in a sphere of magical force that is just large enough to contain the target. Nothing can pass through the sphere, nor can any creature teleport or use planar travel to get into or out of it.

% The special component for this version of the spell is a small mithral orb.

% **_Chaining._** Heavy chains, firmly rooted in the ground, hold the target in place. The target is restrained until the spell ends, and it can’t move or be moved by any means until then.

% The special component for this version of the spell is a fine chain of precious metal.

% **_Hedged Prison._** The spell transports the target into a tiny demiplane that is warded against teleportation and planar travel. The demiplane can be a labyrinth, a cage, a tower, or any similar confined structure or area of your choice.

% The special component for this version of the spell is a miniature representation of the prison made from jade.

% **_Minimus Containment._** The target shrinks to a height of 1 inch and is imprisoned inside a gemstone or similar object. Light can pass through the gemstone normally (allowing the target to see out and other creatures to see in), but nothing else can pass through, even by means of teleportation or planar travel. The gemstone can’t be cut or broken while the spell remains in effect.

% The special component for this version of the spell is a large, transparent gemstone, such as a corundum, diamond, or ruby.

% **_Slumber._** The target falls asleep and can’t be awoken. The special component for this version of the spell consists of rare soporific herbs.

% **_Ending the Spell._** During the casting of the spell, in any of its versions, you can specify a condition that will cause the spell to end and release the target. The condition can be as specific or as elaborate as you choose, but the GM must agree that the condition is reasonable and has a likelihood of coming to pass. The conditions can be based on a creature’s name, identity, or deity but otherwise must be based on observable actions or qualities and not based on intangibles such as level, class, or hit points.

% A \textit{dispel magic} spell can end the spell only if it is cast as a 9th-level spell, targeting either the prison or the special component used to create it.

% You can use a particular special component to create only one prison at a time. If you cast the spell again using the same component, the target of the first casting is immediately freed from its binding.

% \subsection{Incendiary Cloud}

% _8th-level conjuration_

% **Casting Time:** 1 action

% **Range:** 150 feet

% **Components:** V, S

% **Duration:** Concentration, up to 1 minute

% A swirling cloud of smoke shot through with white* hot embers appears in a 20-foot-radius sphere centered on a point within range. The cloud spreads around corners and is heavily obscured. It lasts for the duration or until a wind of moderate or greater speed (at least 10 miles per hour) disperses it.

% When the cloud appears, each creature in it must make a Dexterity saving throw. A creature takes 10d8 fire damage on a failed save, or half as much damage on a successful one. A creature must also make this saving throw when it enters the spell’s area for the first time on a turn or ends its turn there.

% The cloud moves 10 feet directly away from you in a direction that you choose at the start of each of your turns.

\subsection{Irresistible Dance}
\subparagraph*{Tags} Arcane, Lesser, Concentration
\subparagraph*{Cast Time} 1 action

Choose one creature that you can see within 30 feet. The target begins a comic dance in place: shuffling, tapping its feet, and capering for one minute. Creatures that can’t be charmed are immune to this spell.

A dancing creature must use all its movement to dance without leaving its space and has disadvantage on Dexterity saving throws and attack rolls. While the target is affected by this effect, other creatures have advantage on attack rolls against it. As an action, a dancing creature makes a Wisdom saving throw to regain control of itself. On a successful save, the effect ends.

% \subsection{Magic Jar}

% _6th-level necromancy_

% **Casting Time:** 1 minute

% **Range:** Self

% **Components:** V, S, M (a gem, crystal, reliquary, or some other ornamental container worth at least 500 gp)

% **Duration:** Until dispelled

% Your body falls into a catatonic state as your soul leaves it and enters the container you used for the spell’s material component. While your soul inhabits the container, you are aware of your surroundings as if you were in the container’s space. You can’t move or use reactions. The only action you can take is to project your soul up to 100 feet out of the container, either returning to your living body (and ending the spell) or attempting to possess a humanoids body.

% You can attempt to possess any humanoid within 100 feet of you that you can see (creatures warded by a \textit{protection from evil and good} or \textit{magic circle} spell can’t be possessed). The target must make a Charisma saving throw. On a failure, your soul moves into the target’s body, and the target’s soul becomes trapped in the container. On a success, the target resists your efforts to possess it, and you can’t attempt to possess it again for 24 hours.

% Once you possess a creature’s body, you control it. Your game statistics are replaced by the statistics of the creature, though you retain your alignment and your Intelligence, Wisdom, and Charisma scores. You retain the benefit of your own class features. If the target has any class levels, you can’t use any of its class features.

% Meanwhile, the possessed creature’s soul can perceive from the container using its own senses, but it can’t move or take actions at all.

% While possessing a body, you can use your action to return from the host body to the container if it is within 100 feet of you, returning the host creature’s soul to its body. If the host body dies while you’re in it, the creature dies, and you must make a Charisma saving throw against your own spellcasting DC. On a success, you return to the container if it is within 100 feet of you. Otherwise, you die.

% If the container is destroyed or the spell ends, your soul immediately returns to your body. If your body is more than 100 feet away from you or if your body is dead when you attempt to return to it, you die. If another creature’s soul is in the container when it is destroyed, the creature’s soul returns to its body if the body is alive and within 100 feet. Otherwise, that creature dies.

% When the spell ends, the container is destroyed.

\subsection{Mass Heal}
\subparagraph*{Tags} Divine, Greater
\subparagraph*{Cast Time} 1 action

A flood of healing energy flows from you into injured creatures around you. You restore up to 700 hit points, divided as you choose among any number of creatures that you can see within 60 feet. Creatures healed by this spell are also cured of all poisons, diseases, any effect making them blinded or deafened, as well as any curse. This spell has no effect on undead or constructs.

% \subsection{Mass Suggestion}

% _6th-level enchantment_

% **Casting Time:** 1 action

% **Range:** 60 feet

% **Components:** V, M (a snake’s tongue and either a bit of honeycomb or a drop of sweet oil)

% **Duration:** 24 hours

% You suggest a course of activity (limited to a sentence or two) and magically influence up to twelve creatures of your choice that you can see within range and that can hear and understand you. Creatures that can’t be charmed are immune to this effect. The suggestion must be worded in such a manner as to make the course of action sound reasonable. Asking the creature to stab itself, throw itself onto a spear, immolate itself, or do some other obviously harmful act automatically negates the effect of the spell.

% Each target must make a Wisdom saving throw. On a failed save, it pursues the course of action you described to the best of its ability. The suggested course of action can continue for the entire duration. If the suggested activity can be completed in a shorter time, the spell ends when the subject finishes what it was asked to do.

% You can also specify conditions that will trigger a special activity during the duration. For example, you might suggest that a group of soldiers give all their money to the first beggar they meet. If the condition isn’t met before the spell ends, the activity isn’t performed.

% If you or any of your companions damage a creature affected by this spell, the spell ends for that creature.

% **_At Higher Levels._** When you cast this spell using a 7th-level spell slot, the duration is 10 days. When you use an 8th-level spell slot, the duration is 30 days. When you use a 9th-level spell slot, the duration is a year and a day.

\subsection{Maze}
\subparagraph*{Tags} Arcane, Greater, Concentration
\subparagraph*{Cast Time} 1 action

You banish a creature that you can see within range into a labyrinthine demiplane. The target remains there for one minute or until it escapes the maze.

The target can use its action to attempt to escape. When it does so, it makes an Intelligence check against the effect's DC. If it succeeds, it escapes, and the spell ends.

When the spell ends, the target reappears in the space it left or, if that space is occupied, in the nearest unoccupied space.

\subsection{Meteor Swarm}
\subparagraph*{Tags} Arcane or Primal, Greater
\subparagraph*{Cast Time} 1 action

Blazing orbs of fire plummet to the ground at four different points you can see within 1 mile. Each creature in a 40-foot-radius sphere centered on each point you choose must make a Dexterity saving throw. The sphere spreads around corners. A creature takes 20d6 fire damage and 20d6 plus 2x your level bludgeoning damage  on a failed save, or half as much damage on a successful one. A creature in the area of more than one fiery burst is affected only once. This effect ignores resistance to fire or bludgeoning damage and any effect that would reduce the damage to half on a failed save or ignore it on a success (such as a rogue's Evasion).

The spell destroys Large or smaller objects and ignites flammable objects in the area that aren’t being worn or carried.

\subsection{Mind Blank}
\subparagraph*{Tags} Arcane or Divine, Greater
\subparagraph*{Cast Time} 1 action

One willing creature you touch is immune to psychic damage, any effect that would sense its emotions or read or influence its thoughts or actions, divination spells, and the charmed condition for 24 hours. The spell even foils legendary effects used to affect the target’s mind or to gain information about the target.

\subsection{Mirage Arcane}
\subparagraph*{Tags} Arcane, Lesser, Concentration
\subparagraph*{Cast Time} 1 minute

You make terrain in an area up to 1 mile square look, sound, smell, and even feel like some other sort of terrain. The terrain's general shape remains the same, however. Open fields or a road could be made to resemble a swamp, hill, crevasse, or some other difficult or impassable terrain. A pond can be made to seem like a grassy meadow, a precipice like a gentle slope, or a rock-strewn gully like a wide and smooth road.

Similarly, you can alter the appearance of structures, or add them where none are present. The spell doesn't disguise, conceal, or add creatures.

The illusion includes audible, visual, tactile, and olfactory elements, so it can turn clear ground into difficult terrain (or vice versa) or otherwise impede movement through the area. Any piece of the illusory terrain (such as a rock or stick) that is removed from the spell's area disappears immediately.

Creatures with truesight can see through the illusion to the terrain's true form; however, all other elements of the illusion remain, so while the creature is aware of the illusion's presence, the creature can still physically interact with the illusion.

\subsection{Power Word Kill}
\subparagraph*{Tags} Arcane or Divine, Greater
\subparagraph*{Cast Time} 1 action

You utter a word of power that can compel one weaker creature you can see within 60 feet to die instantly and weakens those too strong to die outright. If the creature you choose has 100 hit points or fewer, it dies. Otherwise, the creature takes damage equal to 3x your level. This damage cannot be reduced or eliminated in any way.

\subsection{Power Word Stun}
\subparagraph*{Tags} Divine, Greater
\subparagraph*{Casting Time} 1 action

You speak a word of power that can overwhelm the mind of one creature you can see within range, leaving it dumbfounded. If the target has 150 hit points or fewer, it is stunned. Otherwise, the target is staggered.

The stunned or staggered target must make a Constitution saving throw at the end of each of its turns. On a successful save, this stunning or staggering effect ends.

% \subsection{Prismatic Spray}

% _7th-level evocation_

% **Casting Time:** 1 action

% **Range:** Self (60-foot cone)

% **Components:** V, S

% **Duration:** Instantaneous

% Eight multicolored rays of light flash from your hand. Each ray is a different color and has a different power and purpose. Each creature in a 60-foot cone must make a Dexterity saving throw. For each target, roll a d8 to determine which color ray affects it.

% 1.  **_Red._** The target takes 10d6 fire damage on a failed save, or half as much damage on a successful one.
% 2.  **_Orange._** The target takes 10d6 acid damage on a failed save, or half as much damage on a successful one.
% 3.  **_Yellow._** The target takes 10d6 lightning damage on a failed save, or half as much damage on a successful one.
% 4.  **_Green._** The target takes 10d6 poison damage on a failed save, or half as much damage on a successful one.
% 5.  **_Blue._** The target takes 10d6 cold damage on a failed save, or half as much damage on a successful one.
% 6.  **_Indigo._** On a failed save, the target is restrained. It must then make a Constitution saving throw at the end of each of its turns. If it successfully saves three times, the spell ends. If it fails its save three times, it permanently turns to stone and is subjected to the petrified condition. The successes and failures don’t need to be consecutive; keep track of both until the target collects three of a kind.
% 7.  **_Violet._** On a failed save, the target is blinded. It must then make a Wisdom saving throw at the start of your next turn. A successful save ends the blindness. If it fails that save, the creature is transported to another plane of existence of the GM’s choosing and is no longer blinded. (Typically, a creature that is on a plane that isn’t its home plane is banished home, while other creatures are usually cast into the Astral or  Border Shadows.)
% 8.  **_Special._** The target is struck by two rays. Roll twice more, rerolling any 8.

% \subsection{Prismatic Wall}

% _9th-level abjuration_

% **Casting Time:** 1 action

% **Range:** 60 feet

% **Components:** V, S

% **Duration:** 10 minutes

% A shimmering, multicolored plane of light forms a vertical opaque wall—up to 90 feet long, 30 feet high, and 1 inch thick—centered on a point you can see within range. Alternatively, you can shape the wall into a sphere up to 30 feet in diameter centered on a point you choose within range. The wall remains in place for the duration. If you position the wall so that it passes through a space occupied by a creature, the spell fails, and your action and the spell slot are wasted.

% The wall sheds bright light out to a range of 100 feet and dim light for an additional 100 feet. You and creatures you designate at the time you cast the spell can pass through and remain near the wall without harm. If another creature that can see the wall moves to within 20 feet of it or starts its turn there, the creature must succeed on a Constitution saving throw or become blinded for 1 minute.

% The wall consists of seven layers, each with a different color. When a creature attempts to reach into or pass through the wall, it does so one layer at a time through all the wall’s layers. As it passes or reaches through each layer, the creature must make a Dexterity saving throw or be affected by that layer’s properties as described below.

% The wall can be destroyed, also one layer at a time, in order from red to violet, by means specific to each layer. Once a layer is destroyed, it remains so for the duration of the spell. A \textit{rod of cancellation} destroys a \textit{prismatic wall}, but an \textit{antimagic field} has no effect on it.

% 1.  **_Red._** The creature takes 10d6 fire damage on a failed save, or half as much damage on a successful one. While this layer is in place, nonmagical ranged attacks can’t pass through the wall. The layer can be destroyed by dealing at least 25 cold damage to it.
% 2.  **_Orange._** The creature takes 10d6 acid damage on a failed save, or half as much damage on a successful one. While this layer is in place, magical ranged attacks can’t pass through the wall. The layer is destroyed by a strong wind.
% 3.  **_Yellow._** The creature takes 10d6 lightning damage on a failed save, or half as much damage on a successful one. This layer can be destroyed by dealing at least 60 force damage to it.
% 4.  **_Green._** The creature takes 10d6 poison damage on a failed save, or half as much damage on a successful one. A \textit{passwall} spell, or another spell of equal or greater level that can open a portal on a solid surface, destroys this layer.
% 5.  **_Blue._** The creature takes 10d6 cold damage on a failed save, or half as much damage on a successful one. This layer can be destroyed by dealing at least 25 fire damage to it.
% 6.  **_Indigo._** On a failed save, the creature is restrained. It must then make a Constitution saving throw at the end of each of its turns. If it successfully saves three times, the spell ends. If it fails its save three times, it permanently turns to stone and is subjected to the petrified condition. The successes and failures don’t need to be consecutive; keep track of both until the creature collects three of a kind. While this layer is in place, spells can’t be cast through the wall. The layer is destroyed by bright light shed by a \textit{daylight} spell or a similar spell of equal or higher level.
% 7.  **_Violet._** On a failed save, the creature is blinded. It must then make a Wisdom saving throw at the start of your next turn. A successful save ends the blindness. If it fails that save, the creature is transported to another plane of the GM’s choosing and is no longer blinded. (Typically, a creature that is on a plane that isn’t its home plane is banished home, while other creatures are usually cast into the Astral or  Border Shadows.) This layer is destroyed by a \textit{dispel magic} spell or a similar spell of equal or higher level that can end spells and magical effects.

\subsection{Regenerate}
\subparagraph*{Tags} Primal, Lesser
\subparagraph*{Cast Time} 1 minute

You touch a creature and stimulate its natural healing ability. The target regains 4d8 + 2x your level hit points. In addition, the target can spend hit dice as an action. If the result is less than half the maximum value of the hit dice, the die is not expended.

The target’s severed body members (fingers, legs, tails, and so on), if any, as well as any lingering or internal injuries, are restored after 2 minutes. If you have the severed part and hold it to the stump, the spell instantaneously causes the limb to knit to the stump.

\subsection{Reverse Gravity}
\subparagraph*{Tags} Arcane or Primal, Lesser, Concentration
\subparagraph*{Cast Time} 1 action

This effect reverses gravity in a 50-foot-radius, 100-foot high cylinder centered on a point within 100 feet. All creatures and objects that aren’t somehow anchored to the ground in the area fall upward and reach the top of the area when you cast this effect. A creature can make a Dexterity saving throw to grab onto a fixed object it can reach, thus avoiding the fall.

If some solid object (such as a ceiling) is encountered in this fall, falling objects and creatures strike it just as they would during a normal downward fall. If an object or creature reaches the top of the area without striking anything, it remains there, oscillating slightly, for the duration.

At the end of the duration, affected objects and creatures fall back down.

% \subsection{Sequester}

% _7th-level transmutation_

% **Casting Time:** 1 action

% **Range:** Touch

% **Components:** V, S, M (a powder composed of diamond, emerald, ruby, and sapphire dust worth at least 5,000 gp, which the spell consumes)

% **Duration:** Until dispelled

% By means of this spell, a willing creature or an object can be hidden away, safe from detection for the duration. When you cast the spell and touch the target, it becomes invisible and can’t be targeted by divination spells or perceived through scrying sensors created by divination spells.

% If the target is a creature, it falls into a state of suspended animation. Time ceases to flow for it, and it doesn’t grow older.

% You can set a condition for the spell to end early. The condition can be anything you choose, but it must occur or be visible within 1 mile of the target. Examples include “after 1,000 years” or “when the tarrasque awakens.” This spell also ends if the target takes any damage.

\subsection{Shapechange}
\subparagraph*{Tags} Primal, Greater, Concentration
\subparagraph*{Cast Time} 1 action

You assume the form of a different creature for the duration. The new form can be of any creature with a challenge rating equal to your level or lower. The creature can’t be a construct, an undead, or a unique creature and you must have seen the sort of creature at least once. You transform into an average example of that creature, one without any class levels or the Spellcasting trait. If you choose a humanoid lineage, you transform into an average member of that lineage and gain no special abilities except those common to all members of that lineage.

Your game statistics are replaced by the statistics of the chosen creature, though you retain your Intelligence, Wisdom, and Charisma scores and personality. You also retain all of your skill and saving throw proficiencies, in addition to gaining those of the creature. If the creature has the same proficiency as you and the bonus listed in its statistics is higher than yours, use the creature’s bonus in place of yours. You can’t use any legendary actions or lair actions of the new form. You can only speak if the creature can normally speak one or more languages, although you retain your normal languages.

You assume the hit points and Hit Dice of the new form. When you revert to your normal form, you return to the number of hit points you had before you transformed. If you revert as a result of dropping to 0 hit points, any excess damage carries over to your normal form. As long as the excess damage doesn’t reduce your normal form to 0 hit points, you aren’t knocked unconscious. 

When you transform, you choose whether your equipment falls to the ground, merges into the new form, or is worn by it. Worn equipment functions as normal. The GM determines whether it is practical for the new form to wear a piece of equipment, based on the creature’s shape and size. Your equipment doesn’t change shape or size to match the new form, and any equipment that the new form can’t wear must either fall to the ground or merge into your new form. Equipment that merges has no effect in that state. Other magical effects on you are suppressed while you function normally if they can affect your new form.

During this spell’s duration, you can use your action to assume a different form following the same restrictions and rules for the original form, with two exceptions: first, if your new form has more hit points than your current one, your hit points remain at their current value. Second, any effect granted by your previous form ends once you transform. 

% \subsection{Storm of Vengeance}

% _9th-level conjuration_

% **Casting Time:** 1 action

% **Range:** Sight

% **Components:** V, S

% **Duration:** Concentration, up to 1 minute

% A churning storm cloud forms, centered on a point you can see and spreading to a radius of 360 feet. Lightning flashes in the area, thunder booms, and strong winds roar. Each creature under the cloud (no more than 5,000 feet beneath the cloud) when it appears must make a Constitution saving throw. On a failed save, a creature takes 2d6 thunder damage and becomes deafened for 5 minutes.

% Each round you maintain concentration on this spell, the storm produces additional effects on your turn.

% **_Round 2._** Acidic rain falls from the cloud. Each creature and object under the cloud takes 1d6 acid damage.

% **_Round 3._** You call six bolts of lightning from the cloud to strike six creatures or objects of your choice beneath the cloud. A given creature or object can’t be struck by more than one bolt. A struck creature must make a Dexterity saving throw. The creature takes 10d6 lightning damage on a failed save, or half as much damage on a successful one.

% **_Round 4._** Hailstones rain down from the cloud. Each creature under the cloud takes 2d6 bludgeoning damage.

% **_Round 5–10._** Gusts and freezing rain assail the area under the cloud. The area becomes difficult terrain and is heavily obscured. Each creature there takes 1d6 cold damage. Ranged weapon attacks in the area are impossible. The wind and rain count as a severe distraction for the purposes of maintaining concentration on spells. Finally, gusts of strong wind (ranging from 20 to 50 miles per hour) automatically disperse fog, mists, and similar phenomena in the area, whether mundane or magical.

\subsection{Sunburst}
\subparagraph*{Tags} Divine, Greater
\subparagraph*{Cast Time} 1 action

Brilliant sunlight flashes in a 60-foot radius centered on a point you choose within 150 feet. Each creature in that light must make a Constitution saving throw. On a failed save, a creature takes 12d6 radiant damage and is blinded for 1 minute. On a successful save, it takes half as much damage and isn’t blinded by this spell. Undead and oozes have disadvantage on this saving throw.

A creature blinded by this spell makes another Constitution saving throw at the end of each of its turns. On a successful save, it is no longer blinded.

This spell dispels any darkness in its area that was created by a spell.

\subsection{Symbol}
\subparagraph*{Tags} Generic, Lesser
\subparagraph*{Cast Time} 1 minute
\subparagraph*{Other requirements} You must expend powdered diamond and opal with a total value of at least 1,000 gp

When you cast this effect, you inscribe a harmful glyph either on a surface (such as a section of floor, a wall, or a table) or within an object that can be closed to conceal the glyph (such as a book, a scroll, or a treasure chest). If you choose a surface, the glyph can cover an area of the surface no larger than 10 feet in diameter. If you choose an object, that object must remain in its place; if the object is moved more than 10 feet from where you cast this effect (including into or out of interdimensional spaces or between planes), the glyph is broken, and the effect ends without being triggered.

The glyph is nearly invisible, requiring an Intelligence (Investigation) check against your legendary effect DC to find it.

You decide what triggers the glyph when you cast the effect. For glyphs inscribed on a surface, the most typical triggers include touching or stepping on the glyph, removing another object covering it, approaching within a certain distance of it, or manipulating the object that holds it. For glyphs inscribed within an object, the most common triggers are opening the object, approaching within a certain distance of it, or seeing or reading the glyph.

You can further refine the trigger so the effect is activated only under certain circumstances or according to a creature’s physical characteristics (such as height or weight), or physical kind (for example, the ward could be set to affect hags or shapechangers). You can also specify creatures that don’t trigger the glyph, such as those who say a certain password.

When you inscribe the glyph, choose one of the options below for its effect. Once triggered, the glyph glows, filling a 60-foot-radius sphere with dim light for 10 minutes, after which time the effect ends. Each creature in the sphere when the glyph activates is targeted by its effect, as is a creature that enters the sphere for the first time on a turn or ends its turn there.

**_Death._** Each target must make a Constitution saving throw, taking 10d10 necrotic damage on a failed save, or half as much damage on a successful save.

**_Discord._** Each target must make a Constitution saving throw. On a failed save, a target bickers and argues with other creatures for 1 minute. During this time, it is incapable of meaningful communication and has disadvantage on attack rolls and ability checks.

**_Fear._** Each target must make a Wisdom saving throw and becomes frightened for 1 minute on a failed save. While frightened, the target drops whatever it is holding and must move at least 30 feet away from the glyph on each of its turns, if able.

**_Hopelessness._** Each target must make a Charisma saving throw. On a failed save, the target is overwhelmed with despair for 1 minute. During this time, it can’t attack or target any creature with harmful abilities, spells, or other magical effects.

**_Insanity._** Each target must make an Intelligence saving throw. On a failed save, the target is driven insane for 1 minute. An insane creature can’t take actions, can’t understand what other creatures say, can’t read, and speaks only in gibberish. The GM controls its movement, which is erratic.

**_Pain._** Each target must make a Constitution saving throw and becomes incapacitated with excruciating pain for 1 minute on a failed save.

**_Sleep._** Each target must make a Wisdom saving throw and falls unconscious for 10 minutes on a failed save. A creature awakens if it takes damage or if someone uses an action to shake or slap it awake.

**_Stunning._** Each target must make a Wisdom saving throw and becomes stunned for 1 minute on a failed save.

\subsection{Time Stop}

_9th-level transmutation_

**Casting Time:** 1 action

**Range:** Self

**Components:** V

**Duration:** Instantaneous

You briefly stop the flow of time for everyone but yourself. No time passes for other creatures, while you take 1d4 + 1 turns in a row, during which you can use actions and move as normal.

This effect ends if one of the actions you use during this period, or any effects that you create during this period, affects a creature other than you or an object being worn or carried by someone other than you. In addition, the effect ends if you move to a place more than 1,000 feet from the location where you cast it.

\subsection{Weird}

_9th-level illusion_

**Casting Time:** 1 action

**Range:** 120 feet

**Components:** V, S

**Duration:** Concentration, up to one minute

Drawing on the deepest fears of a group of creatures, you create illusory creatures in their minds, visible only to them. Each creature in a 30-foot-radius sphere centered on a point of your choice within range must make a Wisdom saving throw. On a failed save, a creature becomes frightened for the duration. The illusion calls on the creature’s deepest fears, manifesting its worst nightmares as an implacable threat. At the end of each of the frightened creature’s turns, it must succeed on a Wisdom saving throw or take 4d10 psychic damage. On a successful save, the effect ends for that creature.

\subsection{Wind Walk}

_6th-level transmutation_

**Casting Time:** 1 minute

**Range:** 30 feet

**Components:** V, S, M (fire and holy water)

**Duration:** 8 hours

You and up to ten willing creatures you can see within range assume a gaseous form for the duration, appearing as wisps of cloud. While in this cloud form, a creature has a flying speed of 300 feet and has resistance to damage from nonmagical weapons. The only actions a creature can take in this form are the Dash action or to revert to its normal form. Reverting takes 1 minute, during which time a creature is incapacitated and can’t move. Until the effect ends, a creature can revert to cloud form, which also requires the 1-minute transformation.

If a creature is in cloud form and flying when the effect ends, the creature descends 60 feet per round for 1 minute until it lands, which it does safely. If it can’t land after 1 minute, the creature falls the remaining distance.

\subsection{Anyspell}
\subparagraph*{Tags} Generic, Greater
\subparagraph*{Cast Time} 1 action

This effect duplicates any spell or incantation. You don’t need to meet any requirements in that spell, including costly components. The spell or incantation simply takes effect. Spells created by this effect count as legendary effects for the purpose of bypassing spells or effects that care.

\subsection{Word of Recall}
\subparagraph*{Tags} Divine, Lesser
\subparagraph*{Casting Time} 1 action

You and up to five willing creatures within 10 feet of you instantly teleport to a previously designated sanctuary. You and any creatures that teleport with you appear in the nearest unoccupied space to the spot you designated when you prepared your sanctuary (see below). If you cast this effect without first preparing a sanctuary, the effect does nothing.

You must designate a sanctuary by casting this effect within a location, such as a temple, dedicated to or strongly linked to your deity or Ascended patron. If you attempt to cast the effect in this manner in an area that isn’t dedicated to your deity, the effect has no effect.

This effect ignores all non-legendary effects that preclude or inhibit teleportation.



\DndSpellHeader{Move Earth}
{(6)}
{1 action}
{120 feet}
{V, S, M (an iron blade and a small bag containing a mixture of soils—clay, loam, and sand)}
{Concentration, up to 2 hours}

Choose an area of terrain no larger than 40 feet on a side within range. You can reshape dirt, sand, or clay in the area in any manner you choose for the duration. You can raise or lower the area's elevation, create or fill in a trench, erect or flatten a wall, or form a pillar. The extent of any such changes can't exceed half the area's largest dimension. So, if you affect a 40-foot square, you can create a pillar up to 20 feet high, raise or lower the square's elevation by up to 20 feet, dig a trench up to 20 feet deep, and so on. It takes 10 minutes for these changes to complete.

At the end of every 10 minutes you spend concentrating on the effect, you can choose a new area of terrain to affect.

Because the terrain's transformation occurs slowly, creatures in the area can't usually be trapped or injured by the ground's movement.

This effect can't manipulate natural stone or stone construction. Rocks and structures shift to accommodate the new terrain. If the way you shape the terrain would make a structure unstable, it might collapse.

Similarly, this effect doesn't directly affect plant growth. The moved earth carries any plants along with it.

\label{spell:heroes-feast}
\DndSpellHeader{Heroes' Feast}
{(6)}
{10 minutes}
{30 feet}
{V, S , M (a gem-encrusted bowl worth at least 1,000 gp, which the effect consumes)}
{Instantaneous}

You bring forth a great feast, including magnificent food and drink. The feast takes 1 hour to consume and disappears at the end of that time, and the beneficial effects don't set in until this hour is over. Up to twelve other creatures can partake of the feast.

A creature that partakes of the feast gains several benefits. The creature is cured of all diseases and poison, becomes immune to poison and being frightened, and makes all Wisdom saving throws with advantage. Its hit point maximum also increases by 2d10, and it gains the same number of hit points. These benefits last for 24 hours.

\label{effect:modify-memory}
\DndSpellHeader{Modify Memory}
{12 AET}
{1 action}
{30 feet}
{V, S}
{Concentration, up to 1 minute}

You attempt to reshape another creature's memories. One creature that you can see must make a Wisdom saving throw. If you are fighting the creature, it has advantage on the saving throw. On a failed save, the target becomes charmed by you for the duration. The charmed target is incapacitated and unaware of its surroundings, though it can still hear you. If it takes any damage or is targeted by another spell, this spell ends, and none of the target's memories are modified.

While this charm lasts, you can affect the target's memory of an event that it experienced within the last 24 hours and that lasted no more than 10 minutes. You can permanently eliminate all memory of the event, allow the target to recall the event with perfect clarity and exacting detail, change its memory of the details of the event, or create a memory of some other event.

You must speak to the target to describe how its memories are affected, and it must be able to understand your language for the modified memories to take root. Its mind fills in any gaps in the details of your description. If the spell ends before you have finished describing the modified memories, the creature's memory isn't altered. Otherwise, the modified memories take hold when the spell ends.

A modified memory doesn't necessarily affect how a creature behaves, particularly if the memory contradicts the creature's natural inclinations, alignment, or beliefs. An illogical modified memory, such as implanting a memory of how much the creature enjoyed dousing itself in acid, is dismissed, perhaps as a bad dream. The GM might deem a modified memory too nonsensical to affect a creature in a significant manner.

A \textit{remove curse} spell or \textit{greater restoration} incantation cast on the target restores the creature's true memory.
{If you cast this spell using a spell slot of 6th level or higher, you can alter the target's memories of an event that took place up to 7 days ago (6th level), 30 days ago (7th level), 1 year ago (8th level), or any time in the creature's past (9th level).}

\label{spell:tree-stride}
\DndSpellHeader{Tree Stride}
{12 AET}
{1 action}
{Self}
{V, S}
{Concentration, up to 1 minute}

You gain the ability to enter a tree and move from inside it to inside another tree of the same kind within 500 feet. Both trees must be living and at least the same size as you. You must use 5 feet of movement to enter a tree. You instantly know the location of all other trees of the same kind within 500 feet and, as part of the move used to enter the tree, can either pass into one of those trees or step out of the tree you're in. You appear in a spot of your choice within 5 feet of the destination tree, using another 5 feet of movement. If you have no movement left, you appear within 5 feet of the tree you entered.

You can use this transportation ability once per round for the duration. You must end each turn outside a tree.

\label{spell:arcane-hand}
\DndSpellHeader{Arcane Hand}
{13 AET, damage + forced movement + movement restraint}
{1 action}
{120 feet}
{V, S, M (an eggshell and a snakeskin glove)}
{Concentration, up to 1 minute}

You create a Large hand of shimmering, translucent force in an unoccupied space that you can see within range. The hand lasts for the spell's duration, and it moves at your command, mimicking the movements of your own hand.

The hand is an object that has AC 20 and hit points equal to your hit point maximum. If it drops to 0 hit points, the spell ends. It has a Strength of 26 (+8) and a Dexterity of 10 (+0). The hand doesn't fill its space.

When you cast the spell and as a bonus action on your subsequent turns, you can move the hand up to 60 feet and then cause one of the following effects with it.

\subparagraph*{Clenched Fist} The hand strikes one creature or object within 5 feet of it. Make a melee spell attack for the hand using your game statistics. On a hit, the target takes 4d8 force damage.

\subparagraph*{Forceful Hand} The hand attempts to push a creature within 5 feet of it in a direction you choose.

Make a check with the hand's Strength contested by the Strength (Athletics) check of the target. If the target is Medium or smaller, you have advantage on the check. If you succeed, the hand pushes the target up to 5 feet plus a number of feet equal to five times your spellcasting ability modifier. The hand moves with the target to remain within 5 feet of it.

\subparagraph*{Grasping Hand} The hand attempts to grapple a Huge or smaller creature within 5 feet of it. You use the hand's Strength score to resolve the grapple. If the target is Medium or smaller, you have advantage on the check. While the hand is grappling the target, you can use a bonus action to have the hand crush it. When you do so, the target takes bludgeoning damage equal to 2d6 + your spellcasting ability modifier.

\subparagraph*{Interposing Hand} The hand interposes itself between you and a creature you choose until you give the hand a different command. The hand moves to stay between you and the target, providing you with half cover against the target. The target can't move through the hand's space if its Strength score is less than or equal to the hand's Strength score. If its Strength score is higher than the hand's Strength score, the target can move toward you through the hand's space, but that space is difficult terrain for the target.

\subparagraph*{Overcast} When you cast this spell using at least 15 AET, the damage from the clenched fist option increases by 2d8 and the damage from the grasping hand increases by 2d6 for every 3 extra AET you spend.