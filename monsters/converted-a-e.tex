\subsection{Ankheg}
\begin{DndMonster}[float*=b,width=\textwidth + 8pt]{Ankheg}
\begin{multicols}{2}
\DndMonsterType{Large monstrosity}
\DndMonsterBasics[armor-class={14 (natural armor), 11 while prone}, hit-points={39 (6d10 + 6)}, speed={30 ft., burrow 10 ft.}]
\MonsterStats{+3}{+0}{+1}{-5}{+1}{-2}
\DndMonsterDetails[saving-throws={}, skills={}, damage-immunities={}, damage-resistances={}, damage-vulnerabilities={}, condition-immunities={}, senses={darkvision 60 ft., tremorsense 60 ft., passive Perception 11}, languages={—}, challenge={2 (450 XP)}]
\DndMonsterSection{Actions}
\DndMonsterMelee[
    name=Bite,
    mod=+5,
    dmg=\DndDice(2d6+3),
    dmg-type=slashing,
    plus-dmg=\DndDice(1d6),
    plus-dmg-type=acid,
    extra={If the target is a Large or smaller creature, it is grappled (escape DC 13). Until this grapple ends, the ankheg can bite only the grappled creature and has advantage on attack rolls to do so.}
]
\DndMonsterAction{Spit} The ankheg spits acid in a line that is 30 feet long and 5 feet wide, provided that it has no creature grappled. Each creature in that line must make a DC 13 Dexterity saving throw, taking 10 (3d6) acid damage on a failed save, or half as much damage on a successful one.
\end{multicols}
\end{DndMonster}

\subsection{Azer}
\begin{DndMonster}[float*=b,width=\textwidth + 8pt]{Azer}
\begin{multicols}{2}
\DndMonsterType{Medium elemental}
\DndMonsterBasics[armor-class={17 (natural armor, shield)}, hit-points={39 (6d8 + 12)}, speed={30 ft.}]
\MonsterStats{+3}{+1}{+2}{+1}{+1}{+0}
\DndMonsterDetails[saving-throws={Con +4}, skills={}, damage-immunities={fire, poison}, damage-resistances={}, damage-vulnerabilities={}, condition-immunities={poisoned}, senses={passive Perception 11}, languages={Ignan}, challenge={2 (450 XP)}]
\DndMonsterAction{Heated Body} A creature that touches the azer or hits it with a melee attack while within 5 feet of it takes 5 (1d10) fire damage.

\DndMonsterAction{Heated Weapons} When the azer hits with a metal melee weapon, it deals an extra 3 (1d6) fire damage (included in the attack).

\DndMonsterAction{Illumination} The azer sheds bright light in a 10-foot radius and dim light for an additional 10 feet.

\DndMonsterSection{Actions}
\DndMonsterMelee[
    name=Warhammer,
    mod=+5,
    dmg=\DndDice{1d8+3},
    dmg-type=bludgeoning,
    plus-dmg=\DndDice{1d6},
    plus-dmg-type=fire
]
\end{multicols}
\end{DndMonster}

\subsection{Balor}
\begin{DndMonster}[float*=b,width=\textwidth + 8pt]{Balor}
\begin{multicols}{2}
\DndMonsterType{Huge fiend (demon)}
\DndMonsterBasics[armor-class={19 (natural armor)}, hit-points={262 (21d12 + 126)}, speed={40 ft., fly 80 ft.}]
\MonsterStats{+8}{+2}{+6}{+5}{+3}{+6}
\DndMonsterDetails[saving-throws={Str +14, Con +12, Wis +9, Cha +12}, skills={}, damage-immunities={fire, poison}, damage-resistances={cold, lightning; bludgeoning, piercing, and slashing from nonmagical attacks}, damage-vulnerabilities={}, condition-immunities={poisoned}, senses={truesight 120 ft., passive Perception 13}, languages={Abyssal, telepathy 120 ft.}, challenge={19 (22,000 XP)}]
\DndMonsterAction{Death Throes} When the balor dies, it explodes, and each creature within 30 feet of it must make a DC 20 Dexterity saving throw, taking 70 (20d6) fire damage on a failed save, or half as much damage on a successful one. The explosion ignites flammable objects in that area that aren’t being worn or carried, and it destroys the balor’s weapons.

\DndMonsterAction{Fire Aura} At the start of each of the balor’s turns, each creature within 5 feet of it takes 10 (3d6) fire damage, and flammable objects in the aura that aren’t being worn or carried ignite. A creature that touches the balor or hits it with a melee attack while within 5 feet of it takes 10 (3d6) fire damage.

\DndMonsterAction{Magic Resistance} The balor has advantage on saving throws against spells and other magical effects.

\DndMonsterAction{Magic Weapons} The balor’s weapon attacks are magical.

\DndMonsterSection{Actions}
\DndMonsterAction{Multiattack} The balor makes two attacks: one with its longsword and one with its whip.
\DndMonsterAction[
    name=Longsword,
    mod=+14,
    reach=10,
    dmg=\DndDice{3d8+8},
    dmg-type=slashing,
    plus-dmg=\DndDice{3d8},
    plus-dmg-type=lightning,
    extra={If the balor scores a critical hit, it rolls damage dice three times, instead of twice.}
]
\DndMonsterMelee[
    name=Whip,
    mod=+14,
    reach=30,
    dmg=\DndDice{2d6+8},
    dmg-type=slashing,
    plus-dmg=\DndDice{3d6},
    plus-dmg-type=fire,
    extra={, and the target must succeed on a DC 20 Strength saving throw or be pulled up to 25 feet toward the balor.}
]
\DndMonsterAction{Teleport} The balor magically teleports, along with any equipment it is wearing or carrying, up to 120 feet to an unoccupied space it can see.
\end{multicols}
\end{DndMonster}

\subsection{Basilisk}
\begin{DndMonster}[float*=b,width=\textwidth + 8pt]{Basilisk}
\begin{multicols}{2}
\DndMonsterType{Medium monstrosity}
\DndMonsterBasics[armor-class={15 (natural armor)}, hit-points={52 (8d8 + 16)}, speed={20 ft.}]
\MonsterStats{+3}{-1}{+2}{-4}{-1}{-2}
\DndMonsterDetails[saving-throws={}, skills={}, damage-immunities={}, damage-resistances={}, damage-vulnerabilities={}, condition-immunities={}, senses={darkvision 60 ft., passive Perception 9}, languages={—}, challenge={3 (700 XP)}]
\DndMonsterAction{Petrifying Gaze} If a creature starts its turn within 30 feet of the basilisk and the two of them can see each other, the basilisk can force the creature to make a DC 12 Constitution saving throw if the basilisk isn’t incapacitated. On a failed save, the creature magically begins to turn to stone and is restrained. It must repeat the saving throw at the end of its next turn. On a success, the effect ends. On a failure, the creature is petrified until freed by the _greater restoration_ spell or other magic.\\nA creature that isn’t surprised can avert its eyes to avoid the saving throw at the start of its turn. If it does so, it can’t see the basilisk until the start of its next turn, when it can avert its eyes again. If it looks at the basilisk in the meantime, it must immediately make the save.\\nIf the basilisk sees its reflection within 30 feet of it in bright light, it mistakes itself for a rival and targets itself with its gaze.

\DndMonsterSection{Actions}
**_Bite_**. _Melee Weapon Attack:_ +5 to hit, reach 5 ft., one target. H_it:_ 10 (2d6 + 3) piercing damage plus 7 (2d6) poison damage.
\end{multicols}
\end{DndMonster}
\subsection{Behir}
\begin{DndMonster}[float*=b,width=\textwidth + 8pt]{Behir}
\begin{multicols}{2}
\DndMonsterType{Huge monstrosity}
\DndMonsterBasics[armor-class={17 (natural armor)}, hit-points={168 (16d12 + 64)}, speed={50 ft., climb 40 ft.}]
\MonsterStats{+6}{+3}{+4}{-2}{+2}{+1}
\DndMonsterDetails[saving-throws={}, skills={Perception +6, Stealth +7}, damage-immunities={lightning}, damage-resistances={}, damage-vulnerabilities={}, condition-immunities={}, senses={darkvision 90 ft., passive Perception 16}, languages={Draconic}, challenge={11 (7,200 XP)}]
\DndMonsterSection{Actions}
\DndMonsterAction{Multiattack} The behir makes two attacks: one with its bite and one to constrict.
**_Bite_**. _Melee Weapon Attack:_ +10 to hit, reach 10 ft., one target. _Hit:_ 22 (3d10 + 6) piercing damage.
**_Constrict_**. _Melee Weapon Attack:_ +10 to hit, reach 5 ft., one Large or smaller creature. _Hit:_ 17 (2d10 + 6) bludgeoning damage plus 17 (2d10 + 6) slashing damage. The target is grappled (escape DC 16) if the behir isn’t already constricting a creature, and the target is restrained until this grapple ends.
The behir exhales a line of lightning that is 20 feet long and 5 feet wide. Each creature in that line must make a DC 16 Dexterity saving throw, taking 66 (12d10) lightning damage on a failed save, or half as much damage on a successful one.
The behir makes one bite attack against a Medium or smaller target it is grappling. If the attack hits, the target is also swallowed, and the grapple ends. While swallowed, the target is blinded and restrained, it has total cover against attacks and other effects outside the behir, and it takes 21 (6d6) acid damage at the start of each of the behir’s turns. A behir can have only one creature swallowed at a time.\\nIf the behir takes 30 damage or more on a single turn from the swallowed creature, the behir must succeed on a DC 14 Constitution saving throw at the end of that turn or regurgitate the creature, which falls prone in a space within 10 feet of the behir. If the behir dies, a swallowed creature is no longer restrained by it and can escape from the corpse by using 15 feet of movement, exiting prone.
\end{multicols}
\end{DndMonster}
\subsection{Bugbear}
\begin{DndMonster}[float*=b,width=\textwidth + 8pt]{Bugbear}
\begin{multicols}{2}
\DndMonsterType{Medium humanoid (goblinoid)}
\DndMonsterBasics[armor-class={16 (hide armor, shield)}, hit-points={27 (5d8 + 5)}, speed={30 ft.}]
\MonsterStats{+2}{+2}{+1}{-1}{+0}{-1}
\DndMonsterDetails[saving-throws={}, skills={Stealth +6, Survival +2}, damage-immunities={}, damage-resistances={}, damage-vulnerabilities={}, condition-immunities={}, senses={darkvision 60 ft., passive Perception 10}, languages={Common, Goblin}, challenge={1 (200 XP)}]
\DndMonsterAction{Brute} A melee weapon deals one extra die of its damage when the bugbear hits with it (included in the attack).

\DndMonsterAction{Surprise Attack} If the bugbear surprises a creature and hits it with an attack during the first round of combat, the target takes an extra 7 (2d6) damage from the attack.

\DndMonsterSection{Actions}
**_Morningstar_**. _Melee Weapon Attack:_ +4 to hit, reach 5 ft., one target. _Hit:_ 11 (2d8 + 2) piercing damage.
**_Javelin_**. _Melee or Ranged Weapon Attack:_ +4 to hit, reach 5 ft. or range 30/120 ft., one target. _Hit:_ 9 (2d6 + 2) piercing damage in melee or 5 (1d6 + 2) piercing damage at range.
\end{multicols}
\end{DndMonster}
\subsection{Bulette}
\begin{DndMonster}[float*=b,width=\textwidth + 8pt]{Bulette}
\begin{multicols}{2}
\DndMonsterType{Large monstrosity}
\DndMonsterBasics[armor-class={17 (natural armor)}, hit-points={94 (9d10 + 45)}, speed={40 ft., burrow 40 ft.}]
\MonsterStats{+4}{+0}{+5}{-4}{+0}{-3}
\DndMonsterDetails[saving-throws={}, skills={Perception +6}, damage-immunities={}, damage-resistances={}, damage-vulnerabilities={}, condition-immunities={}, senses={darkvision 60 ft., tremorsense 60 ft., passive Perception 16}, languages={—}, challenge={5 (1,800 XP)}]
\DndMonsterAction{Standing Leap} The bulette’s long jump is up to 30 feet and its high jump is up to 15 feet, with or without a running start.

\DndMonsterSection{Actions}
**_Bite_**. _Melee Weapon Attack:_ +7 to hit, reach 5 ft., one target. _Hit:_ 30 (4d12 + 4) piercing damage.
If the bulette jumps at least 15 feet as part of its movement, it can then use this action to land on its feet in a space that contains one or more other creatures. Each of those creatures must succeed on a DC 16 Strength or Dexterity saving throw (target’s choice) or be knocked prone and take 14 (3d6 + 4) bludgeoning damage plus 14 (3d6 + 4) slashing damage. On a successful save, the creature takes only half the damage, isn’t knocked prone, and is pushed 5 feet out of the bulette’s space into an unoccupied space of the creature’s choice. If no unoccupied space is within range, the creature instead falls prone in the bulette’s space.
\end{multicols}
\end{DndMonster}
\subsection{Centaur}
\begin{DndMonster}[float*=b,width=\textwidth + 8pt]{Centaur}
\begin{multicols}{2}
\DndMonsterType{Large monstrosity}
\DndMonsterBasics[armor-class={12}, hit-points={45 (6d10 + 12)}, speed={50 ft.}]
\MonsterStats{+4}{+2}{+2}{-1}{+1}{+0}
\DndMonsterDetails[saving-throws={}, skills={Athletics +6, Perception +3, Survival +3}, damage-immunities={}, damage-resistances={}, damage-vulnerabilities={}, condition-immunities={}, senses={passive Perception 13}, languages={Elvish, Sylvan}, challenge={2 (450 XP)}]
\DndMonsterAction{Charge} If the centaur moves at least 30 feet straight toward a target and then hits it with a pike attack on the same turn, the target takes an extra 10 (3d6) piercing damage.

\DndMonsterSection{Actions}
\DndMonsterAction{Multiattack} The centaur makes two attacks: one with its pike and one with its hooves or two with its longbow.
**_Pike_**. _Melee Weapon Attack:_ +6 to hit, reach 10 ft., one target. _Hit:_ 9 (1d10 + 4) piercing damage.
**_Hooves_**. _Melee Weapon Attack:_ +6 to hit, reach 5 ft., one target. _Hit:_ 11 (2d6 + 4) bludgeoning damage.
**_Longbow_**. _Ranged Weapon Attack:_ +4 to hit, range 150/600 ft., one target. _Hit:_ 6 (1d8 + 2) piercing damage.
\end{multicols}
\end{DndMonster}
\subsection{Chimera}
\begin{DndMonster}[float*=b,width=\textwidth + 8pt]{Chimera}
\begin{multicols}{2}
\DndMonsterType{Large monstrosity}
\DndMonsterBasics[armor-class={14 (natural armor)}, hit-points={114 (12d10 + 48)}, speed={30 ft., fly 60 ft.}]
\MonsterStats{+4}{+0}{+4}{-4}{+2}{+0}
\DndMonsterDetails[saving-throws={}, skills={Perception +8}, damage-immunities={}, damage-resistances={}, damage-vulnerabilities={}, condition-immunities={}, senses={darkvision 60 ft., passive Perception 18}, languages={understands Draconic but can’t speak}, challenge={6 (2,300 XP)}]
\DndMonsterSection{Actions}
\DndMonsterAction{Multiattack} The chimera makes three attacks: one with its bite, one with its horns, and one with its claws. When its fire breath is available, it can use the breath in place of its bite or horns.
**_Bite_**. _Melee Weapon Attack:_ +7 to hit, reach 5 ft., one target. _Hit:_ 11 (2d6 + 4) piercing damage.
**_Horns_**. _Melee Weapon Attack:_ +7 to hit, reach 5 ft., one target. _Hit:_ 10 (1d12 + 4) bludgeoning damage.
**_Claws_**. _Melee Weapon Attack:_ +7 to hit, reach 5 ft., one target. _Hit:_ 11 (2d6 + 4) slashing damage.
The dragon head exhales fire in a 15-foot cone. Each creature in that area must make a DC 15 Dexterity saving throw, taking 31 (7d8) fire damage on a failed save, or half as much damage on a successful one.
\end{multicols}
\end{DndMonster}
\subsection{Chuul}
\begin{DndMonster}[float*=b,width=\textwidth + 8pt]{Chuul}
\begin{multicols}{2}
\DndMonsterType{Large aberration}
\DndMonsterBasics[armor-class={16 (natural armor)}, hit-points={93 (11d10 + 33)}, speed={30 ft., swim 30 ft.}]
\MonsterStats{+4}{+0}{+3}{-3}{+0}{-3}
\DndMonsterDetails[saving-throws={}, skills={Perception +4}, damage-immunities={poison}, damage-resistances={}, damage-vulnerabilities={}, condition-immunities={poisoned}, senses={darkvision 60 ft., passive Perception 14}, languages={understands Deep Speech but can’t speak}, challenge={4 (1,100 XP)}]
\DndMonsterAction{Amphibious} The chuul can breathe air and water.

\DndMonsterAction{Sense Magic} The chuul senses magic within 120 feet of it at will. This trait otherwise works like the _detect magic_ spell but isn’t itself magical.

\DndMonsterSection{Actions}
\DndMonsterAction{Multiattack} The chuul makes two pincer attacks. If the chuul is grappling a creature, the chuul can also use its tentacles once.
**_Pincer_**. _Melee Weapon Attack:_ +6 to hit, reach 10 ft., one target. _Hit:_ 11 (2d6 + 4) bludgeoning damage. The target is grappled (escape DC 14) if it is a Large or smaller creature and the chuul doesn’t have two other creatures grappled.
One creature grappled by the chuul must succeed on a DC 13 Constitution saving throw or be poisoned for 1 minute. Until this poison ends, the target is paralyzed. The target can repeat the saving throw at the end of each of its turns, ending the effect on itself on a success.
\end{multicols}
\end{DndMonster}
\subsection{Cloaker}
\begin{DndMonster}[float*=b,width=\textwidth + 8pt]{Cloaker}
\begin{multicols}{2}
\DndMonsterType{Large aberration}
\DndMonsterBasics[armor-class={14 (natural armor)}, hit-points={78 (12d10 + 12)}, speed={10 ft., fly 40 ft.}]
\MonsterStats{+3}{+2}{+1}{+1}{+1}{+2}
\DndMonsterDetails[saving-throws={}, skills={Stealth +5}, damage-immunities={}, damage-resistances={}, damage-vulnerabilities={}, condition-immunities={}, senses={darkvision 60 ft., passive Perception 11}, languages={Deep Speech, Undercommon}, challenge={8 (3,900 XP)}]
\DndMonsterAction{Damage Transfer} While attached to a creature, the cloaker takes only half the damage dealt to it (rounded down), and that creature takes the other half.

\DndMonsterAction{False Appearance} While the cloaker remains motionless without its underside exposed, it is indistinguishable from a dark leather cloak.

\DndMonsterAction{Light Sensitivity} While in bright light, the cloaker has disadvantage on attack rolls and Wisdom (Perception) checks that rely on sight.

\DndMonsterSection{Actions}
\DndMonsterAction{Multiattack} The cloaker makes two attacks: one with its bite and one with its tail.
**_Bite_**. _Melee Weapon Attack:_ +6 to hit, reach 5 ft., one creature. _Hit:_ 10 (2d6 + 3) piercing damage, and if the target is Large or smaller, the cloaker attaches to it. If the cloaker has advantage against the target, the cloaker attaches to the target’s head, and the target is blinded and unable to breathe while the cloaker is attached. While attached, the cloaker can make this attack only against the target and has advantage on the attack roll. The cloaker can detach itself by spending 5 feet of its movement. A creature, including the target, can take its action to detach the cloaker by succeeding on a DC 16 Strength check.
**_Tail_**. _Melee Weapon Attack:_ +6 to hit, reach 10 ft., one creature. _Hit:_ 7 (1d8 + 3) slashing damage.
Each creature within 60 feet of the cloaker that can hear its moan and that isn’t an aberration must succeed on a DC 13 Wisdom saving throw or become frightened until the end of the cloaker’s next turn. If a creature’s saving throw is successful, the creature is immune to the cloaker’s moan for the next 24 hours
The cloaker magically creates three illusory duplicates of itself if it isn’t in bright light. The duplicates move with it and mimic its actions, shifting position so as to make it impossible to track which cloaker is the real one. If the cloaker is ever in an area of bright light, the duplicates disappear.\\nWhenever any creature targets the cloaker with an attack or a harmful spell while a duplicate remains, that creature rolls randomly to determine whether it targets the cloaker or one of the duplicates. A creature is unaffected by this magical effect if it can’t see or if it relies on senses other than sight.\\nA duplicate has the cloaker’s AC and uses its saving throws. If an attack hits a duplicate, or if a duplicate fails a saving throw against an effect that deals damage, the duplicate disappears.
\end{multicols}
\end{DndMonster}
\subsection{Cockatrice}
\begin{DndMonster}[float*=b,width=\textwidth + 8pt]{Cockatrice}
\begin{multicols}{2}
\DndMonsterType{Small monstrosity}
\DndMonsterBasics[armor-class={11}, hit-points={27 (6d6 + 6)}, speed={20 ft., fly 40 ft.}]
\MonsterStats{-2}{+1}{+1}{-4}{+1}{-3}
\DndMonsterDetails[saving-throws={}, skills={}, damage-immunities={}, damage-resistances={}, damage-vulnerabilities={}, condition-immunities={}, senses={darkvision 60 ft., passive Perception 11}, languages={—}, challenge={1/2 (100 XP)}]
\DndMonsterSection{Actions}
**_Bite_**. _Melee Weapon Attack:_ +3 to hit, reach 5 ft., one creature. _Hit:_ 3 (1d4 + 1) piercing damage, and the target must succeed on a DC 11 Constitution saving throw against being magically petrified. On a failed save, the creature begins to turn to stone and is restrained. It must repeat the saving throw at the end of its next turn. On a success, the effect ends. On a failure, the creature is petrified for 24 hours.
\end{multicols}
\end{DndMonster}
\subsection{Couatl}
\begin{DndMonster}[float*=b,width=\textwidth + 8pt]{Couatl}
\begin{multicols}{2}
\DndMonsterType{Medium celestial}
\DndMonsterBasics[armor-class={19 (natural armor)}, hit-points={97 (13d8 + 39)}, speed={30 ft., fly 90 ft.}]
\MonsterStats{+3}{+5}{+3}{+4}{+5}{+4}
\DndMonsterDetails[saving-throws={Con +5, Wis +7, Cha +6}, skills={}, damage-immunities={psychic; bludgeoning, piercing, and slashing from nonmagical attacks}, damage-resistances={radiant}, damage-vulnerabilities={}, condition-immunities={}, senses={truesight 120 ft., passive Perception 15}, languages={all, telepathy 120 ft.}, challenge={4 (1,100 XP)}]
\DndMonsterAction{Innate Spellcasting} The couatl’s spellcasting ability is Charisma (spell save DC 14). It can innately cast the following spells, requiring only verbal components:\\nAt will: _detect evil and good_, _detect magic_, _detect thoughts_\\n3/day each: _bless_, _create food and water_, _cure wounds_, _lesser restoration_, _protection from poison_, _sanctuary_, _shield_\\n1/day each: _dream_, _greater restoration_, _scrying_

\DndMonsterAction{Magic Weapons} The couatl’s weapon attacks are magical.

\DndMonsterAction{Shielded Mind} The couatl is immune to scrying and to any effect that would sense its emotions, read its thoughts, or detect its location.

\DndMonsterSection{Actions}
**_Bite_**. _Melee Weapon Attack:_ +8 to hit, reach 5 ft., one creature. _Hit:_ 8 (1d6 + 5) piercing damage, and the target must succeed on a DC 13 Constitution saving throw or be poisoned for 24 hours. Until this poison ends, the target is unconscious. Another creature can use an action to shake the target awake.
**_Constrict_**. _Melee Weapon Attack:_ +6 to hit, reach 10 ft., one Medium or smaller creature. _Hit:_ 10 (2d6 + 3) bludgeoning damage, and the target is grappled (escape DC 15). Until this grapple ends, the target is restrained, and the couatl can’t constrict another target.
The couatl magically polymorphs into a humanoid or beast that has a challenge rating equal to or less than its own, or back into its true form. It reverts to its true form if it dies. Any equipment it is wearing or carrying is absorbed or borne by the new form (the couatl’s choice).\\nIn a new form, the couatl retains its game statistics and ability to speak, but its AC, movement modes, Strength, Dexterity, and other actions are replaced by those of the new form, and it gains any statistics and capabilities (except class features, legendary actions, and lair actions) that the new form has but that it lacks. If the new form has a bite attack, the couatl can use its bite in that form.
\end{multicols}
\end{DndMonster}
\subsection{Darkmantle}
\begin{DndMonster}[float*=b,width=\textwidth + 8pt]{Darkmantle}
\begin{multicols}{2}
\DndMonsterType{Small monstrosity}
\DndMonsterBasics[armor-class={11}, hit-points={22 (5d6 + 5)}, speed={10 ft., fly 30 ft.}]
\MonsterStats{+3}{+1}{+1}{-4}{+0}{-3}
\DndMonsterDetails[saving-throws={}, skills={Stealth +3}, damage-immunities={}, damage-resistances={}, damage-vulnerabilities={}, condition-immunities={}, senses={blindsight 60 ft., passive Perception 10}, languages={—}, challenge={1/2 (100 XP)}]
\DndMonsterAction{Echolocation} The darkmantle can’t use its blindsight while deafened.

\DndMonsterAction{False Appearance} While the darkmantle remains motionless, it is indistinguishable from a cave formation such as a stalactite or stalagmite.

\DndMonsterSection{Actions}
**_Crush_**. _Melee Weapon Attack:_ +5 to hit, reach 5 ft., one creature. _Hit:_ 6 (1d6 + 3) bludgeoning damage, and the darkmantle attaches to the target. If the target is Medium or smaller and the darkmantle has advantage on the attack roll, it attaches by engulfing the target’s head, and the target is also blinded and unable to breathe while the darkmantle is attached in this way.\\nWhile attached to the target, the darkmantle can attack no other creature except the target but has advantage on its attack rolls. The darkmantle’s speed also becomes 0, it can’t benefit from any bonus to its speed, and it moves with the target.\\nA creature can detach the darkmantle by making a successful DC 13 Strength check as an action. On its turn, the darkmantle can detach itself from the target by using 5 feet of movement.
A 15-foot radius of magical darkness extends out from the darkmantle, moves with it, and spreads around corners. The darkness lasts as long as the darkmantle maintains concentration, up to 10 minutes (as if concentrating on a spell). Darkvision can’t penetrate this darkness, and no natural light can illuminate it. If any of the darkness overlaps with an area of light created by a spell of 2nd level or lower, the spell creating the light is dispelled.
\end{multicols}
\end{DndMonster}
\subsection{Devil, Barbed}
\begin{DndMonster}[float*=b,width=\textwidth + 8pt]{Devil, Barbed}
\begin{multicols}{2}
\DndMonsterType{Medium fiend (devil)}
\DndMonsterBasics[armor-class={15 (natural armor)}, hit-points={110 (13d8 + 52)}, speed={30 ft.}]
\MonsterStats{+3}{+3}{+4}{+1}{+2}{+2}
\DndMonsterDetails[saving-throws={Str +6, Con +7, Wis +5, Cha +5}, skills={Deception +5, Insight +5, Perception +8}, damage-immunities={fire, poison}, damage-resistances={cold; bludgeoning, piercing, and slashing from nonmagical attacks that aren’t silvered}, damage-vulnerabilities={}, condition-immunities={poisoned}, senses={darkvision 120 ft., passive Perception 18}, languages={Infernal, telepathy 120 ft.}, challenge={5 (1,800 XP)}]
\DndMonsterAction{Barbed Hide} At the start of each of its turns, the barbed devil deals 5 (1d10) piercing damage to any creature grappling it.

\DndMonsterAction{Devil’s Sight} Magical darkness doesn’t impede the devil’s darkvision.

\DndMonsterAction{Magic Resistance} The devil has advantage on saving throws against spells and other magical effects.

\DndMonsterSection{Actions}
\DndMonsterAction{Multiattack} The devil makes three melee attacks: one with its tail and two with its claws. Alternatively, it can use Hurl Flame twice.
**_Claw_**. _Melee Weapon Attack:_ +6 to hit, reach 5 ft., one target. _Hit:_ 6 (1d6 + 3) piercing damage.
**_Tail_**. _Melee Weapon Attack:_ +6 to hit, reach 5 ft., one target. _Hit:_ 10 (2d6 + 3) piercing damage.
Ranged Spell Attack:_ +5 to hit, range 150 ft., one target. _Hit:_ 10 (3d6) fire damage. If the target is a flammable object that isn’t being worn or carried, it also catches fire.
\end{multicols}
\end{DndMonster}
\subsection{Devil, Bearded}
\begin{DndMonster}[float*=b,width=\textwidth + 8pt]{Devil, Bearded}
\begin{multicols}{2}
\DndMonsterType{Medium fiend (devil)}
\DndMonsterBasics[armor-class={13 (natural armor)}, hit-points={52 (8d8 + 16)}, speed={30 ft.}]
\MonsterStats{+3}{+2}{+2}{-1}{+0}{+0}
\DndMonsterDetails[saving-throws={Str +5, Con +4, Wis +2}, skills={}, damage-immunities={fire, poison}, damage-resistances={cold; bludgeoning, piercing, and slashing from nonmagical attacks that aren’t silvered}, damage-vulnerabilities={}, condition-immunities={poisoned}, senses={darkvision 120 ft., passive Perception 10}, languages={Infernal, telepathy 120 ft.}, challenge={3 (700 XP)}]
\DndMonsterAction{Devil’s Sight} Magical darkness doesn’t impede the devil’s darkvision.

\DndMonsterAction{Magic Resistance} The devil has advantage on saving throws against spells and other magical effects.

\DndMonsterAction{Steadfast} The devil can’t be frightened while it can see an allied creature within 30 feet of it.

\DndMonsterSection{Actions}
\DndMonsterAction{Multiattack} The devil makes two attacks: one with its beard and one with its glaive.
**_Beard_**. _Melee Weapon Attack:_ +5 to hit, reach 5 ft., one creature. _Hit:_ 6 (1d8 + 2) piercing damage, and the target must succeed on a DC 12 Constitution saving throw or be poisoned for 1 minute. While poisoned in this way, the target can’t regain hit points. The target can repeat the saving throw at the end of each of its turns, ending the effect on itself on a success.
**_Glaive_**. _Melee Weapon Attack:_ +5 to hit, reach 10 ft., one target. _Hit:_ 8 (1d10 + 3) slashing damage. If the target is a creature other than an undead or a construct, it must succeed on a DC 12 Constitution saving throw or lose 5 (1d10) hit points at the start of each of its turns due to an infernal wound. Each time the devil hits the wounded target with this attack, the damage dealt by the wound increases by 5 (1d10). Any creature can take an action to stanch the wound with a successful DC 12 Wisdom (Medicine) check. The wound also closes if the target receives magical healing.
\end{multicols}
\end{DndMonster}
\subsection{Devil, Bone}
\begin{DndMonster}[float*=b,width=\textwidth + 8pt]{Devil, Bone}
\begin{multicols}{2}
\DndMonsterType{Large fiend (devil)}
\DndMonsterBasics[armor-class={19 (natural armor)}, hit-points={142 (15d10 + 60)}, speed={40 ft., fly 40 ft.}]
\MonsterStats{+4}{+3}{+4}{+1}{+2}{+3}
\DndMonsterDetails[saving-throws={Int +5, Wis +6, Cha +7}, skills={Deception +7, Insight +6}, damage-immunities={fire, poison}, damage-resistances={cold; bludgeoning, piercing, and slashing from nonmagical attacks that aren’t silvered}, damage-vulnerabilities={}, condition-immunities={poisoned}, senses={darkvision 120 ft., passive Perception 12}, languages={Infernal, telepathy 120 ft.}, challenge={9 (5,000 XP)}]
\DndMonsterAction{Devil’s Sight} Magical darkness doesn’t impede the devil’s darkvision.

\DndMonsterAction{Magic Resistance} The devil has advantage on saving throws against spells and other magical effects.

\DndMonsterSection{Actions}
\DndMonsterAction{Multiattack} The devil makes three attacks: two with its claws and one with its sting.
**_Claw_**. _Melee Weapon Attack:_ +8 to hit, reach 10 ft., one target. _Hit:_ 8 (1d8 + 4) slashing damage.
**_Sting_**. _Melee Weapon Attack:_ +8 to hit, reach 10 ft., one target. _Hit:_ 13 (2d8 + 4) piercing damage plus 17 (5d6) poison damage, and the target must succeed on a DC 14 Constitution saving throw or become poisoned for 1 minute. The target can repeat the saving throw at the end of each of its turns, ending the effect on itself on a success.
\end{multicols}
\end{DndMonster}
\subsection{Devil, Horned}
\begin{DndMonster}[float*=b,width=\textwidth + 8pt]{Devil, Horned}
\begin{multicols}{2}
\DndMonsterType{Large fiend (devil)}
\DndMonsterBasics[armor-class={18 (natural armor)}, hit-points={148 (17d10 + 55)}, speed={20 ft., fly 60 ft.}]
\MonsterStats{+6}{+3}{+5}{+1}{+3}{+3}
\DndMonsterDetails[saving-throws={Str +10, Dex +7, Wis +7, Cha +7}, skills={}, damage-immunities={fire, poison}, damage-resistances={cold; bludgeoning, piercing, and slashing from nonmagical attacks not made with silvered weapons}, damage-vulnerabilities={}, condition-immunities={poisoned}, senses={darkvision 120 ft., passive Perception 13}, languages={Infernal, telepathy 120 ft.}, challenge={11 (7,200 XP)}]
\DndMonsterAction{Devil’s Sight} Magical darkness doesn’t impede the devil’s darkvision.

\DndMonsterAction{Magic Resistance} The devil has advantage on saving throws against spells and other magical effects.

\DndMonsterSection{Actions}
\DndMonsterAction{Multiattack} The devil makes three melee attacks: two with its fork and one with its tail. It can use Hurl Flame in place of any melee attack.
**_Fork_**. _Melee Weapon Attack:_ +10 to hit, reach 10 ft., one target. _Hit:_ 15 (2d8 + 6) piercing damage.
**_Tail_**. _Melee Weapon Attack:_ +10 to hit, reach 10 ft., one target. _Hit:_ 10 (1d8 + 6) piercing damage. If the target is a creature other than an undead or a construct, it must succeed on a DC 17 Constitution saving throw or lose 10 (3d6) hit points at the start of each of its turns due to an infernal wound. Each time the devil hits the wounded target with this attack, the damage dealt by the wound increases by 10 (3d6). Any creature can take an action to stanch the wound with a successful DC 12 Wisdom (Medicine) check. The wound also closes if the target receives magical healing.
Ranged Spell Attack:_ +7 to hit, range 150 ft., one target. _Hit:_ 14 (4d6) fire damage. If the target is a flammable object that isn’t being worn or carried, it also catches fire.
\end{multicols}
\end{DndMonster}
\subsection{Devil, Ice}
\begin{DndMonster}[float*=b,width=\textwidth + 8pt]{Devil, Ice}
\begin{multicols}{2}
\DndMonsterType{Large fiend (devil)}
\DndMonsterBasics[armor-class={18 (natural armor)}, hit-points={180 (19d10 + 76)}, speed={40 ft.}]
\MonsterStats{+5}{+2}{+4}{+4}{+2}{+4}
\DndMonsterDetails[saving-throws={Dex +7, Con +9, Wis +7, Cha +9}, skills={}, damage-immunities={cold, fire, poison}, damage-resistances={bludgeoning, piercing, and slashing from nonmagical attacks that aren’t silvered}, damage-vulnerabilities={}, condition-immunities={poisoned}, senses={blindsight 60 ft., darkvision 120 ft., passive Perception 12}, languages={Infernal, telepathy 120 ft.}, challenge={14 (11,500 XP)}]
\DndMonsterAction{Devil’s Sight} Magical darkness doesn’t impede the devil’s darkvision.

\DndMonsterSection{Actions}
\end{multicols}
\end{DndMonster}
\subsection{Doppelganger}
\begin{DndMonster}[float*=b,width=\textwidth + 8pt]{Doppelganger}
\begin{multicols}{2}
\DndMonsterType{Medium monstrosity (shapechanger)}
\DndMonsterBasics[armor-class={14}, hit-points={52 (8d8 + 16)}, speed={30 ft.}]
\MonsterStats{+0}{+4}{+2}{+0}{+1}{+2}
\DndMonsterDetails[saving-throws={}, skills={Deception +6, Insight +3}, damage-immunities={}, damage-resistances={}, damage-vulnerabilities={}, condition-immunities={charmed}, senses={darkvision 60 ft., passive Perception 11}, languages={Common}, challenge={3 (700 XP)}]
\DndMonsterAction{Shapechanger} The doppelganger can use its action to polymorph into a Small or Medium humanoid it has seen, or back into its true form. Its statistics, other than its size, are the same in each form. Any equipment it is wearing or carrying isn’t transformed. It reverts to its true form if it dies.

\DndMonsterAction{Ambusher} The doppelganger has advantage on attack rolls against any creature it has surprised.

\DndMonsterAction{Surprise Attack} If the doppelganger surprises a creature and hits it with an attack during the first round of combat, the target takes an extra 10 (3d6) damage from the attack.

\DndMonsterSection{Actions}
\DndMonsterAction{Multiattack} The doppelganger makes two melee attacks.
**_Slam_**. _Melee Weapon Attack:_ +6 to hit, reach 5 ft., one target. _Hit:_ 7 (1d6 + 4) bludgeoning damage.
The doppelganger magically reads the surface thoughts of one creature within 60 feet of it. The effect can penetrate barriers, but 3 feet of wood or dirt, 2 feet of stone, 2 inches of metal, or a thin sheet of lead blocks it. While the target is in range, the doppelganger can continue reading its thoughts, as long as the doppelganger’s concentration isn’t broken (as if concentrating on a spell). While reading the target’s mind, the doppelganger has advantage on Wisdom (Insight) and Charisma (Deception, Intimidation, and Persuasion) checks against the target.
\end{multicols}
\end{DndMonster}
\subsection{Dretch}
\begin{DndMonster}[float*=b,width=\textwidth + 8pt]{Dretch}
\begin{multicols}{2}
\DndMonsterType{Small fiend (demon)}
\DndMonsterBasics[armor-class={11 (natural armor)}, hit-points={18 (4d6 + 4)}, speed={20 ft.}]
\MonsterStats{+0}{+0}{+1}{-3}{-1}{-4}
\DndMonsterDetails[saving-throws={}, skills={}, damage-immunities={poison}, damage-resistances={cold, fire, lightning}, damage-vulnerabilities={}, condition-immunities={poisoned}, senses={darkvision 60 ft., passive Perception 9}, languages={Abyssal, telepathy 60 ft. (works only with creatures that understand Abyssal)}, challenge={1/4 (50 XP)}]
\DndMonsterSection{Actions}
\DndMonsterAction{Multiattack} The dretch makes two attacks: one with its bite and one with its claws.
**_Bite_**. _Melee Weapon Attack:_ +2 to hit, reach 5 ft., one target. _Hit:_ 3 (1d6) piercing damage.
**_Claws_**. _Melee Weapon Attack:_ +2 to hit, reach 5 ft., one target. _Hit:_ 5 (2d4) slashing damage.
A 10-foot radius of disgusting green gas extends out from the dretch. The gas spreads around corners, and its area is lightly obscured. It lasts for 1 minute or until a strong wind disperses it. Any creature that starts its turn in that area must succeed on a DC 11 Constitution saving throw or be poisoned until the start of its next turn. While poisoned in this way, the target can take either an action or a bonus action on its turn, not both, and can’t take reactions.
\end{multicols}
\end{DndMonster}
\subsection{Dragon Turtle}
\begin{DndMonster}[float*=b,width=\textwidth + 8pt]{Dragon Turtle}
\begin{multicols}{2}
\DndMonsterType{Gargantuan dragon}
\DndMonsterBasics[armor-class={20 (natural armor)}, hit-points={341 (22d20 + 110)}, speed={20 ft., swim 40 ft.}]
\MonsterStats{+7}{+0}{+5}{+0}{+1}{+1}
\DndMonsterDetails[saving-throws={Dex +6, Con +11, Wis +7}, skills={}, damage-immunities={}, damage-resistances={fire}, damage-vulnerabilities={}, condition-immunities={}, senses={darkvision 120 ft., passive Perception 11}, languages={Aquan, Draconic}, challenge={17 (18,000 XP)}]
\DndMonsterAction{Amphibious} The dragon turtle can breathe air and water.

\DndMonsterSection{Actions}
\DndMonsterAction{Multiattack} The dragon turtle makes three attacks: one with its bite and two with its claws. It can make one tail attack in place of its two claw attacks.
**_Bite_**. _Melee Weapon Attack:_ +13 to hit, reach 15 ft., one target. _Hit:_ 26 (3d12 + 7) piercing damage.
**_Claw_**. _Melee Weapon Attack:_ +13 to hit, reach 10 ft., one target. _Hit:_ 16 (2d8 + 7) slashing damage.
**_Tail_**. _Melee Weapon Attack:_ +13 to hit, reach 15 ft., one target. _Hit:_ 26 (3d12 + 7) bludgeoning damage. If the target is a creature, it must succeed on a DC 20 Strength saving throw or be pushed up to 10 feet away from the dragon turtle and knocked prone.
The dragon turtle exhales scalding steam in a 60-foot cone. Each creature in that area must make a DC 18 Constitution saving throw, taking 52 (15d6) fire damage on a failed save, or half as much damage on a successful one. Being underwater doesn’t grant resistance against this damage.
\end{multicols}
\end{DndMonster}
\subsection{Drider}
\begin{DndMonster}[float*=b,width=\textwidth + 8pt]{Drider}
\begin{multicols}{2}
\DndMonsterType{Large monstrosity}
\DndMonsterBasics[armor-class={19 (natural armor)}, hit-points={123 (13d10 + 52)}, speed={30 ft., climb 30 ft.}]
\MonsterStats{+3}{+3}{+4}{+1}{+2}{+1}
\DndMonsterDetails[saving-throws={}, skills={Perception +5, Stealth +9}, damage-immunities={}, damage-resistances={}, damage-vulnerabilities={}, condition-immunities={}, senses={darkvision 120 ft., passive Perception 15}, languages={Elvish,}, challenge={6}]
\DndMonsterAction{Fey Ancestry} The drider has advantage on saving throws against being charmed, and magic can’t put the drider to sleep.

\DndMonsterAction{Innate Spellcasting} The drider’s innate spellcasting ability is Wisdom (spell save DC 13). The drider can innately cast the following spells, requiring no material components:\\nAt will: _dancing lights_\\n1/day each: _darkness_, _faerie fire_

\DndMonsterAction{Spider Climb} The drider can climb difficult surfaces, including upside down on ceilings, without needing to make an ability check.

\DndMonsterAction{Sunlight Sensitivity} While in sunlight, the drider has disadvantage on attack rolls, as well as on Wisdom (Perception) checks that rely on sight.

\DndMonsterAction{Web Walker} The drider ignores movement restrictions caused by webbing.

\DndMonsterSection{Actions}
\DndMonsterAction{Multiattack} The drider makes three attacks, either with its longsword or its longbow. It can replace one of those attacks with a bite attack.
**_Bite_**. _Melee Weapon Attack:_ +6 to hit, reach 5 ft., one creature. _Hit:_ 2 (1d4) piercing damage plus 9 (2d8) poison damage.
**_Longsword_**. _Melee Weapon Attack:_ +6 to hit, reach 5 ft., one target. _Hit:_ 7 (1d8 + 3) slashing damage, or 8 (1d10 + 3) slashing damage if used with two hands.
**_Longbow_**. _Ranged Weapon Attack:_ +6 to hit, range 150/600 ft., one target. _Hit:_ 7 (1d8 + 3) piercing damage plus 4 (1d8) poison damage.
\end{multicols}
\end{DndMonster}
\subsection{Dryad}
\begin{DndMonster}[float*=b,width=\textwidth + 8pt]{Dryad}
\begin{multicols}{2}
\DndMonsterType{Medium fey}
\DndMonsterBasics[armor-class={11 (16 with _barkskin_)}, hit-points={22 (5d8)}, speed={30 ft.}]
\MonsterStats{+0}{+1}{+0}{+2}{+2}{+4}
\DndMonsterDetails[saving-throws={}, skills={Perception +4, Stealth +5}, damage-immunities={}, damage-resistances={}, damage-vulnerabilities={}, condition-immunities={}, senses={darkvision 60 ft., passive Perception 14}, languages={Primordial, one other (usually ihmisi or common)}, challenge={1}]
\DndMonsterAction{Innate Spellcasting} The dryad’s innate spellcasting ability is Charisma (spell save DC 14). The dryad can innately cast the following spells, requiring no material components:\\nAt will: _druidcraft_\\n3/day each: _entangle_, _goodberry_\\n1/day each: _barkskin_, _pass without trace_, _shillelagh_

\DndMonsterAction{Magic Resistance} The dryad has advantage on saving throws against spells and other magical effects.

\DndMonsterAction{Speak with Beasts and Plants} The dryad can communicate with beasts and plants as if they shared a language.

\DndMonsterAction{Tree Stride} Once on her turn, the dryad can use 10 feet of her movement to step magically into one living tree within her reach and emerge from a second living tree within 60 feet of the first tree, appearing in an unoccupied space within 5 feet of the second tree. Both trees must be Large or bigger.

\DndMonsterSection{Actions}
**_Club_**. _Melee Weapon Attack:_ +2 to hit (+6 to hit with _shillelagh_), reach 5 ft., one target. _Hit:_ 2 (1d4) bludgeoning damage, or 8 (1d8 + 4) bludgeoning damage with _shillelagh_.
The dryad targets one humanoid or beast that she can see within 30 feet of her. If the target can see the dryad, it must succeed on a DC 14 Wisdom saving throw or be magically charmed. The charmed creature regards the dryad as a trusted friend to be heeded and protected. Although the target isn’t under the dryad’s control, it takes the dryad’s requests or actions in the most favorable way it can.\\nEach time the dryad or its allies do anything harmful to the target, it can repeat the saving throw, ending the effect on itself on a success. Otherwise, the effect lasts 24 hours or until the dryad dies, is on a different plane of existence from the target, or ends the effect as a bonus action. If a target’s saving throw is successful, the target is immune to the dryad’s Fey Charm for the next 24 hours.\\nThe dryad can have no more than one humanoid and up to three beasts charmed at a time.
\end{multicols}
\end{DndMonster}
\subsection{Dwarf, Twisted}
\begin{DndMonster}[float*=b,width=\textwidth + 8pt]{Dwarf, Twisted}
\begin{multicols}{2}
\DndMonsterType{Medium humanoid (dwarf)}
\DndMonsterBasics[armor-class={16 (scale mail, shield)}, hit-points={26 (4d8 + 8)}, speed={25 ft.}]
\MonsterStats{+2}{+0}{+2}{+0}{+0}{-1}
\DndMonsterDetails[saving-throws={}, skills={}, damage-immunities={}, damage-resistances={poison}, damage-vulnerabilities={}, condition-immunities={}, senses={darkvision 120 ft., passive Perception 10}, languages={Dwarvish}, challenge={1}]
\DndMonsterAction{Duergar Resilience} The duergar has advantage on saving throws against poison, spells, and illusions, as well as to resist being charmed or paralyzed.

\DndMonsterAction{Sunlight Sensitivity} While in sunlight, the duergar has disadvantage on attack rolls, as well as on Wisdom (Perception) checks that rely on sight.

\DndMonsterSection{Actions}
For 1 minute, the duergar magically increases in size, along with anything it is wearing or carrying. While enlarged, the duergar is Large, doubles its damage dice on Strength-based weapon attacks (included in the attacks), and makes Strength checks and Strength saving throws with advantage. If the duergar lacks the room to become Large, it attains the maximum size possible in the space available.
**_War Pick_**. _Melee Weapon Attack:_ +4 to hit, reach 5 ft., one target. _Hit:_ 6 (1d8 + 2) piercing damage, or 11 (2d8 + 2) piercing damage while enlarged.
**_Javelin_**. _Melee or Ranged Weapon Attack:_ +4 to hit, reach 5 ft. or range 30/120 ft., one target. _Hit:_ 5 (1d6 + 2) piercing damage, or 9 (2d6 + 2) piercing damage while enlarged.
The duergar magically turns invisible until it attacks, casts a spell, or uses its Enlarge, or until its concentration is broken, up to 1 hour (as if concentrating on a spell). Any equipment the duergar wears or carries is invisible with it.
\end{multicols}
\end{DndMonster}
\subsection{Elemental, Air}
\begin{DndMonster}[float*=b,width=\textwidth + 8pt]{Elemental, Air}
\begin{multicols}{2}
\DndMonsterType{Large elemental}
\DndMonsterBasics[armor-class={15}, hit-points={90 (12d10 + 24)}, speed={0 ft., fly 90 ft. (hover)}]
\MonsterStats{+2}{+5}{+2}{-2}{+0}{-2}
\DndMonsterDetails[saving-throws={}, skills={}, damage-immunities={poison}, damage-resistances={lightning, thunder; bludgeoning, piercing, and slashing from nonmagical attacks}, damage-vulnerabilities={}, condition-immunities={exhaustion, grappled, paralyzed, petrified, poisoned, prone, restrained, unconscious}, senses={darkvision 60 ft., passive Perception 10}, languages={Auran}, challenge={5 (1,800 XP)}]
\DndMonsterAction{Air Form} The elemental can enter a hostile creature’s space and stop there. It can move through a space as narrow as 1 inch wide without squeezing.

\DndMonsterSection{Actions}
\DndMonsterAction{Multiattack} The elemental makes two slam attacks.
**_Slam_**. _Melee Weapon Attack:_ +8 to hit, reach 5 ft., one target. _Hit:_ 14 (2d8 + 5) bludgeoning damage.
Each creature in the elemental’s space must make a DC 13 Strength saving throw. On a failure, a target takes 15 (3d8 + 2) bludgeoning damage and is flung up 20 feet away from the elemental in a random direction and knocked prone. If a thrown target strikes an object, such as a wall or floor, the target takes 3 (1d6) bludgeoning damage for every 10 feet it was thrown. If the target is thrown at another creature, that creature must succeed on a DC 13 Dexterity saving throw or take the same damage and be knocked prone.\\nIf the saving throw is successful, the target takes half the bludgeoning damage and isn’t flung away or knocked prone.
\end{multicols}
\end{DndMonster}
\subsection{Elemental, Earth}
\begin{DndMonster}[float*=b,width=\textwidth + 8pt]{Elemental, Earth}
\begin{multicols}{2}
\DndMonsterType{Large elemental}
\DndMonsterBasics[armor-class={17 (natural armor)}, hit-points={126 (12d10 + 60)}, speed={30 ft., burrow 30 ft.}]
\MonsterStats{+5}{-1}{+5}{-3}{+0}{-3}
\DndMonsterDetails[saving-throws={}, skills={}, damage-immunities={poison}, damage-resistances={bludgeoning, piercing, and slashing from nonmagical attacks}, damage-vulnerabilities={thunder}, condition-immunities={exhaustion, paralyzed, petrified, poisoned, unconscious}, senses={darkvision 60 ft., tremorsense 60 ft., passive Perception 10}, languages={Terran}, challenge={5 (1,800 XP)}]
\DndMonsterAction{Earth Glide} The elemental can burrow through nonmagical, unworked earth and stone. While doing so, the elemental doesn’t disturb the material it moves through.

\DndMonsterAction{Siege Monster} The elemental deals double damage to objects and structures.

\DndMonsterSection{Actions}
\DndMonsterAction{Multiattack} The elemental makes two slam attacks.
**_Slam_**. _Melee Weapon Attack:_ +8 to hit, reach 10 ft., one target. _Hit:_ 14 (2d8 + 5) bludgeoning damage.
\end{multicols}
\end{DndMonster}
\subsection{Elemental, Fire}
\begin{DndMonster}[float*=b,width=\textwidth + 8pt]{Elemental, Fire}
\begin{multicols}{2}
\DndMonsterType{Large elemental}
\DndMonsterBasics[armor-class={13}, hit-points={102 (12d10 + 36)}, speed={50 ft.}]
\MonsterStats{+0}{+3}{+3}{-2}{+0}{-2}
\DndMonsterDetails[saving-throws={}, skills={}, damage-immunities={fire, poison}, damage-resistances={bludgeoning, piercing, and slashing from nonmagical attacks}, damage-vulnerabilities={}, condition-immunities={exhaustion, grappled, paralyzed, petrified, poisoned, prone, restrained, unconscious}, senses={darkvision 60 ft., passive Perception 10}, languages={Ignan}, challenge={5 (1,800 XP)}]
\DndMonsterAction{Fire Form} The elemental can move through a space as narrow as 1 inch wide without squeezing. A creature that touches the elemental or hits it with a melee attack while within 5 feet of it takes 5 (1d10) fire damage. In addition, the elemental can enter a hostile creature’s space and stop there. The first time it enters a creature’s space on a turn, that creature takes 5 (1d10) fire damage and catches fire; until someone takes an action to douse the fire, the creature takes 5 (1d10) fire damage at the start of each of its turns.

\DndMonsterAction{Illumination} The elemental sheds bright light in a 30- foot radius and dim light in an additional 30 feet.

\DndMonsterAction{Water Susceptibility} For every 5 feet the elemental moves in water, or for every gallon of water splashed on it, it takes 1 cold damage.

\DndMonsterSection{Actions}
\DndMonsterAction{Multiattack} The elemental makes two touch attacks.
**_Touch_**. _Melee Weapon Attack:_ +6 to hit, reach 5 ft., one target. _Hit:_ 10 (2d6 + 3) fire damage. If the target is a creature or a flammable object, it ignites. Until a creature takes an action to douse the fire, the target takes 5 (1d10) fire damage at the start of each of its turns.
\end{multicols}
\end{DndMonster}
\subsection{Elemental, Water}
\begin{DndMonster}[float*=b,width=\textwidth + 8pt]{Elemental, Water}
\begin{multicols}{2}
\DndMonsterType{Large elemental}
\DndMonsterBasics[armor-class={Class 14 (natural armor)}, hit-points={114 (12d10 + 48)}, speed={30 ft., swim 90 ft.}]
\MonsterStats{+4}{+2}{+4}{-3}{+0}{-1}
\DndMonsterDetails[saving-throws={}, skills={}, damage-immunities={poison}, damage-resistances={acid; bludgeoning, piercing, and slashing from nonmagical attacks}, damage-vulnerabilities={}, condition-immunities={exhaustion, grappled, paralyzed, petrified, poisoned, prone, restrained, unconscious}, senses={darkvision 60 ft., passive Perception 10}, languages={Aquan}, challenge={5 (1,800 XP)}]
\DndMonsterAction{Water Form} The elemental can enter a hostile creature’s space and stop there. It can move through a space as narrow as 1 inch wide without squeezing.

\DndMonsterAction{Freeze} If the elemental takes cold damage, it partially freezes; its speed is reduced by 20 feet until the end of its next turn.

\DndMonsterSection{Actions}
\DndMonsterAction{Multiattack} The elemental makes two slam attacks.
Melee _Weapon Attack:_ +7 to hit, reach 5 ft., one target. _Hit:_ 13 (2d8 + 4) bludgeoning damage.
Each creature in the elemental’s space must make a DC 15 Strength saving throw. On a failure, a target takes 13 (2d8 + 4) bludgeoning damage. If it is Large or smaller, it is also grappled (escape DC 14). Until this grapple ends, the target is restrained and unable to breathe unless it can breathe water. If the saving throw is successful, the target is pushed out of the elemental’s space.\\nThe elemental can grapple one Large creature or up to two Medium or smaller creatures at one time. At the start of each of the elemental’s turns, each target grappled by it takes 13 (2d8 + 4) bludgeoning damage. A creature within 5 feet of the elemental can pull a creature or object out of it by taking an action to make a DC 14 Strength and succeeding.
\end{multicols}
\end{DndMonster}
\subsection{Erinyes}
\begin{DndMonster}[float*=b,width=\textwidth + 8pt]{Erinyes}
\begin{multicols}{2}
\DndMonsterType{Medium fiend (devil)}
\DndMonsterBasics[armor-class={18 (plate)}, hit-points={153 (18d8 + 72)}, speed={30 ft., fly 60 ft.}]
\MonsterStats{+4}{+3}{+4}{+2}{+2}{+4}
\DndMonsterDetails[saving-throws={Dex +7, Con +8, Wis +6, Cha +8}, skills={}, damage-immunities={fire, poison}, damage-resistances={cold; bludgeoning, piercing, and slashing from nonmagical attacks that aren’t silvered}, damage-vulnerabilities={}, condition-immunities={poisoned}, senses={truesight 120 ft., passive Perception 12}, languages={Infernal, telepathy 120 ft.}, challenge={12 (8,400 XP)}]
\DndMonsterAction{Hellish Weapons} The erinyes’s weapon attacks are magical and deal an extra 13 (3d8) poison damage on a hit (included in the attacks).

\DndMonsterAction{Magic Resistance} The erinyes has advantage on saving throws against spells and other magical effects.

\DndMonsterSection{Actions}
\DndMonsterAction{Multiattack} The erinyes makes three attacks.
**_Longsword_**. _Melee Weapon Attack:_ +8 to hit, reach 5 ft., one target. _Hit:_ 8 (1d8 + 4) slashing damage, or 9 (1d10 + 4) slashing damage if used with two hands, plus 13 (3d8) poison damage.
**_Longbow_**. _Ranged Weapon Attack:_ +7 to hit, range 150/600 ft., one target. _Hit:_ 7 (1d8 + 3) piercing damage plus 13 (3d8) poison damage, and the target must succeed on a DC 14 Constitution saving throw or be poisoned. The poison lasts until it is removed by the _lesser restoration_ spell or similar magic.
The erinyes adds 4 to its AC against one melee attack that would hit it. To do so, the erinyes must see the attacker and be wielding a melee weapon.
\end{multicols}
\end{DndMonster}
\subsection{Ettercap}
\begin{DndMonster}[float*=b,width=\textwidth + 8pt]{Ettercap}
\begin{multicols}{2}
\DndMonsterType{Medium monstrosity}
\DndMonsterBasics[armor-class={13 (natural armor)}, hit-points={44 (8d8 + 8)}, speed={30 ft., climb 30 ft.}]
\MonsterStats{+2}{+2}{+1}{-2}{+1}{-1}
\DndMonsterDetails[saving-throws={}, skills={Perception +3, Stealth +4, Survival +3}, damage-immunities={}, damage-resistances={}, damage-vulnerabilities={}, condition-immunities={}, senses={darkvision 60 ft., passive Perception 13}, languages={—}, challenge={2 (450 XP)}]
\DndMonsterAction{Spider Climb} The ettercap can climb difficult surfaces, including upside down on ceilings, without needing to make an ability check.

\DndMonsterAction{Web Sense} While in contact with a web, the ettercap knows the exact location of any other creature in contact with the same web.

\DndMonsterAction{Web Walker} The ettercap ignores movement restrictions caused by webbing.

\DndMonsterSection{Actions}
\DndMonsterAction{Multiattack} The ettercap makes two attacks: one with its bite and one with its claws.
**_Bite_**. _Melee Weapon Attack:_ +4 to hit, reach 5 ft., one creature. _Hit:_ 6 (1d8 + 2) piercing damage plus 4 (1d8) poison damage. The target must succeed on a DC 11 Constitution saving throw or be poisoned for 1 minute. The creature can repeat the saving throw at the end of each of its turns, ending the effect on itself on a success.
**_Claws_**. _Melee Weapon Attack:_ +4 to hit, reach 5 ft., one target. _Hit:_ 7 (2d4 + 2) slashing damage.
**_Web (Recharge 5–6)_**. _Ranged Weapon Attack:_ +4 to hit, range 30/60 ft., one Large or smaller creature. _Hit:_ The creature is restrained by webbing. As an action, the restrained creature can make a DC 11 Strength check, escaping from the webbing on a success. The effect also ends if the webbing is destroyed. The webbing has AC 10, 5 hit points, vulnerability to fire damage, and immunity to bludgeoning, poison, and psychic damage.
\end{multicols}
\end{DndMonster}
\subsection{Ettin}
\begin{DndMonster}[float*=b,width=\textwidth + 8pt]{Ettin}
\begin{multicols}{2}
\DndMonsterType{Large giant}
\DndMonsterBasics[armor-class={12 (natural armor)}, hit-points={85 (10d10 + 30)}, speed={40 ft.}]
\MonsterStats{+5}{-1}{+3}{-2}{+0}{-1}
\DndMonsterDetails[saving-throws={}, skills={Perception +4}, damage-immunities={}, damage-resistances={}, damage-vulnerabilities={}, condition-immunities={}, senses={darkvision 60 ft., passive Perception 14}, languages={Giant, Orc}, challenge={4 (1,100 XP)}]
\DndMonsterAction{Two Heads} The ettin has advantage on Wisdom (Perception) checks and on saving throws against being blinded, charmed, deafened, frightened, stunned, and knocked unconscious.

\DndMonsterAction{Wakeful} When one of the ettin’s heads is asleep, its other head is awake.

\DndMonsterSection{Actions}
\DndMonsterAction{Multiattack} The ettin makes two attacks: one with its battleaxe and one with its morningstar.
**_Battleaxe_**. _Melee Weapon Attack:_ +7 to hit, reach 5 ft., one target. _Hit:_ 14 (2d8 + 5) slashing damage.
**_Morningstar_**. _Melee Weapon Attack:_ +7 to hit, reach 5 ft., one target. _Hit:_ 14 (2d8 + 5) piercing damage.
\end{multicols}
\end{DndMonster}
