\section{Ranger}

Design Discussion: Half-caster primalist. Less support, more damage than a shaman. SDCT 4/7/4/5.

\subsection{Class Features}

As a ranger, you gain the following class features.

\subsection{Hit Points}

\textbf{Hit Dice:} 1d10 per ranger level

\textbf{Hit Points at 1st Level:} 10 + your Constitution modifier

\textbf{Hit Points at Higher Levels:} 1d10 (or 6) + your Constitution modifier per ranger level after 1st

\subsection{Proficiencies}

\textbf{Armor:} Light armor, medium armor, shields

\textbf{Weapons:} Simple weapons, martial weapons

\textbf{Tools:} None

\textbf{Saving Throws:} Strength, Dexterity

\textbf{Skills:} Choose three from Animal Handling, Athletics, Insight, Investigation, Nature, Perception, Stealth, and Survival

\subsection{Equipment}

You start with the following equipment, in addition to the equipment granted by your background:
\begin{itemize}
\item (\textit{a}) scale mail or (\textit{b}) leather armor
\item (\textit{a}) two shortswords or (\textit{b}) two simple melee weapons
\item (\textit{a}) a dungeoneer's pack or (\textit{b}) an explorer's pack
\item A longbow and a quiver of 20 arrows
\end{itemize}
\textit{The Ranger (table)}

\begin{DndTable}[header=The Ranger\label{tbl:ranger}]{XXXXXXXXX}
 Level & Proficiency Bonus & Features                                          & Spells Known & Focused Foe Bonus & Stamina & Aether & Aether Limit \\
 1st   & +2                & Focused Foe, Natural Explorer                     & -            & 1d4   & 1   & 2   & 1      \\
 2nd   & +2                & Weapon Specialization, Spellcasting               & 2            & 1d4   & 2   & 4   & 2      \\
 3rd   & +2                & Ranger Archetype, Primeval Awareness              & 3            & 1d4   & 3   & 6   & 3      \\
 4th   & +2                & Ability Score Improvement                         & 3            & 1d4   & 4   & 8   & 3      \\
 5th   & +3                & Extra Attack                                      & 4            & 1d6   & 5   & 10   & 4      \\
 6th   & +3                & Favored Enemy and Natural Explorer improvements   & 4            & 1d6   & 6   & 12   & 5      \\
 7th   & +3                & Ranger Archetype feature                          & 5            & 1d6   & 7   & 14   & 5      \\
 8th   & +3                & Ability Score Improvement, Land's Stride          & 5            & 1d6   & 8   & 16   & 6      \\
 9th   & +4                & Aether Strike                                     & 6            & 1d6   & 9   & 18   & 7      \\
 10th  & +4                & Natural Explorer Improvement                      & 6            & 1d6   & 10   & 20   & 7      \\
 11th  & +4                & Ranger Archetype feature                          & 7            & 1d8   & 11   & 22   & 8      \\
 12th  & +4                & Ability Score Improvement                         & 7            & 1d8   & 11   & 24   & 9      \\
 13th  & +5                & -                                                 & 8            & 1d8   & 12   & 26   & 9      \\
 14th  & +5                & Favored Enemy improvement, Vanish                 & 8            & 1d8   & 12   & 28   & 10      \\
 15th  & +5                & Ranger Archetype feature                          & 9            & 1d8   & 13   & 30   & 11      \\
 16th  & +5                & Ability Score Improvement                         & 9            & 1d8   & 13   & 32   & 11      \\
 17th  & +6                & -                                                 & 10           & 1d8   & 14   & 34   & 12      \\
 18th  & +6                & Feral Senses                                      & 10           & 1d8   & 14   & 36   & 13     \\
 19th  & +6                & Ability Score Improvement                         & 11           & 1d8   & 15   & 38   & 13      \\
 20th  & +6                & Foe Slayer                                        & 11           & 1d8   & 15   & 40   & 14      \\
\end{DndTable}
\subsection{Focused Foe}

Rangers have the ability to focus on a single enemy at a time, predicting their actions and striking their weak spots. As a bonus action, choose a creature you can see to become your Focused Foe. This lasts until the target dies or you select another target as your Focused Foe. While focused, a creature takes additional damage from your attacks equal to your Focused Foe die (originally a d4) once per turn when you hit them with an attack.

Additionally, focused foes cannot gain advantage on attacks against you and you cannot have disadvantage to attack them. This even works if they are invisible. You always know where they are if they are within 120 feet of you; if they are further away, you have advantage to track them.

\subsection{Natural Explorer}

You are particularly familiar with one type of natural environment and are adept at traveling and surviving in such regions. Choose one type of favored terrain: arctic, coast, desert, forest, grassland, mountain, swamp, or underground. When you make an Intelligence or Wisdom check related to your favored terrain, your proficiency bonus is doubled if you are using a skill that you're proficient in.

While traveling in your favored terrain, you gain the following benefits:
\begin{itemize}
\item Difficult terrain doesn't slow your group's travel.
\item Your group can't become lost except by magical means.
\item Even when you are engaged in another activity while traveling (such as foraging, navigating, or tracking), you remain alert to danger.
\item Your group can move stealthily at a normal pace.
\item When you forage, you find twice as much food as you normally would.
\item While tracking other creatures, you also learn their exact number, their sizes, and how long ago they passed through the area.
\end{itemize}

Additionally, when you are in your favored terrain, creatures have disadvantage on attempts to hide from you and you have advantage on Wisdom (Perception) checks made to spot hidden creatures and objects.

You choose additional favored terrain types at 6th and 10th level.

\subsection{Weapon Specialization}

You are especially adept at using particular properties. Choose one of the properties below; you gain the effect listed in addition to the property's normal effect. If an effect calls for a saving throw, the DC = 8 + your Strength modifier + your proficiency bonus.

\subparagraph*{Battering} Once per turn when you hit with a battering weapon, you can force the target to make a Strength saving throw. On a failed save, the target is knocked prone.

\subparagraph*{Cleaving} You can attempt to cleave even if you miss. If you do so, roll a new attack with the same modifiers and compare it to the new target's AC.

\subparagraph*{Heavy (Ranged only)} You can choose to forgo your proficiency bonus to the attack roll. If you still hit, you can add twice your proficiency bonus to the damage dealt.

\subparagraph*{Light} When you make the additional attack with a light weapon, you add your ability modifier to the damage dealt.

\subparagraph*{Loading} You ignore the normal effect of this property. Instead, when you hit with an attack from a loading weapon and drop the target to 0 HP, you can choose to have the bolt pass through at a creature behind the slain creature. The closest creature on a 5' wide line connecting you to the slain creature and extending 30' behind him acts as the new target. Make an attack at disadvantage against that creature. If it hits, it takes damage as normal from the attack.

\subparagraph*{Parrying} The bonus from this property increases to +4.

\subparagraph*{Precise} You score a critical hit on an 18, 19, or 20 instead of on a 19 or 20.

\subparagraph*{Reach} You can make opportunity attacks when a creature enters your range as well as leaves it.

\subparagraph*{Thrown} You can draw thrown weapons as part of the attack. In addition, the damage die increases by one step when thrown and you do not suffer disadvantage out to the long range of the attack.

\subparagraph*{Versatile} You get the increased damage die even when wielding it in one hand.

\subsection{Spellcasting}

By the time you reach 2nd level, you have learned to befriend the kami, teaching them to do magical tricks (in the form of spells) in exchange for your personal aether. See chapter 10 for the general rules of spellcasting and chapter 11 for the ranger spell list.

\subsubsection{Preparing and Casting Spells}

The ranger table shows how much aether (AET) you have to cast your spells and do other magical tasks. To cast a spell that requires aether, you must expend aether equal to its cost or greater. You regain all expended aether when you finish a long rest. It also shows your Aether Limit, which is the maximum aether you can expend on a single action.

You know a certain number of ranger spells, choosing from the ranger spell list. You can trade out any known spell for any other spell you can learn from that list when you finish a long rest. When you do so, choose a number of ranger spells equal to your Wisdom modifier + half your ranger level, rounded down (minimum of one spell). To prepare a spell you must be able to cast it without exceeding your Aether Limit.

\subsection{Spellcasting Ability}

Wisdom is your spellcasting ability for your ranger spells, since your magic draws on your attunement to nature. You use your Wisdom whenever a spell refers to your spellcasting ability. In addition, you use your Wisdom modifier when setting the saving throw DC for a ranger spell you cast and when making an attack roll with one.

\textbf{Spell save DC} = 8 + your proficiency bonus + your Wisdom modifier

\textbf{Spell attack modifier} = your proficiency bonus + your Wisdom modifier

\subsection{Ranger Archetype}

At 3rd level, you choose an archetype that you strive to emulate: Bounty Hunter or Monster Slayer, both detailed at the end of the class description. Your choice grants you features at 3rd level and again at 7th, 11th, and 15th level.

\subsection{Primeval Awareness}

Beginning at 3rd level, you can use your action to focus your awareness on the region around you. For a number of minutes equal to your proficiency bonus, you can sense whether the following types of creatures are present within 1 mile of you (or within up to 6 miles if you are in your favored terrain): aberrations, celestials, dragons, elementals, fey, fiends, and undead. This feature reveals the direction and approximate distance (very near, near, far, very far) as well as a general sense of the number (solitary, a group, a horde) of each distinct cluster of creatures detected. Once you use this feature, one hour must pass before you can use it again.

\subsection{Ability Score Improvement}

When you reach 4th level, and again at 8th, 12th, 16th, and 19th level, you can increase one ability score of your choice by 1. As normal, you can't increase an ability score above +5 using this feature.

You can also pick a Skill Trick but you must meet the prerequisites for skill tricks learned in this way. See \nameref{ch:skill-tricks} for that list. At levels 4 and 12, you learn two Skill Tricks instead of one.

\subsection{Extra Attack}

Beginning at 5th level, you can attack twice, instead of once, whenever you take the Attack action on your turn. In addition, you can expend 1 STA when you hit with an attack to deal your Favored Foe damage an additional time per turn. 

\subsection{Land's Stride}

Starting at 8th level, moving through difficult terrain costs you no extra movement. You can also pass through plants without being slowed by them and without taking damage from them if they have thorns, spines, or a similar hazard.

In addition, you have advantage on saving throws against plants that are magically created or manipulated to impede movement, such those created by the \textit{entangle} spell.

\subsection{Aether Strike}

Starting at 9th level, you can manipulate aether to strike multiple targets in a blur. As an action on your turn while you are wielding a weapon, expend 4+ AET and choose a number of targets equal to half the amount of Aether spent, rounded up. Depending on the type of weapon (ranged or melee) you are wielding, one of the following occurs.

\subparagraph*{Melee weapon} You teleport from target to target in an order you choose, moving no more than twice your speed in each jump. As you briefly appear next to each of them, make a melee weapon attack against the target. On a hit, the target takes normal damage from your weapon plus a number of d10s equal to your proficiency bonus. This damage counts as magical. After attacking the final target, you appear in an empty space within your reach of that target.

\subparagraph*{Ranged Weapon} Make a ranged weapon atack against each target in turn. On a hit, targets take normal damage from your weapon plus a number of d10s equal to your proficiency bonus. This damage counts as magical. Targets struck by this attack have disadvantage on their next attack.

\subsection{Vanish}

Starting at 14th level, you can use the Hide action as a bonus action on your turn. Also, you can't be tracked by nonmagical means, unless you choose to leave a trail.

\subsection{Feral Senses}

At 18th level, you gain preternatural senses that help you fight creatures you can't see. When you attack a creature you can't see, your inability to see it doesn't impose disadvantage on your attack rolls against it.

You are also aware of the location of any invisible creature within 30 feet of you, provided that the creature isn't hidden from you and you aren't blinded or deafened.

\subsection{Foe Slayer}
%fixme
At 20th level, you are particularly adept at finding the weak spots of your enemies. When you attack a target you've selected as your favored foe, you can choose to target a vital spot. Make the attack as normal. You score a critical hit on a roll of a 2 - 19, as long as the attack would hit. On a natural 20, you score a critical hit and the damage done is the maximum possible damage for that attack, including all additional damage. A natural 1 is still an automatic miss. Once you use this ability against a particular foe, you cannot use it again against that creature.

\subsection{Ranger Archetypes}

The ideal of the ranger has two classic expressions: the Bounty Hunter and the Monster Slayer.

\subsection{Bounty Hunter}
Bounty Hunters are those who specialize in tracking down and dealing with civilization's refuse. Those who harm others, those who prey on the weak, those who flout society's standards. Many of your preferred foes are humanoid, but more monstrous foes are not exempt. You are just as much at home in the cities and settled areas as you are in the wilds; your quarry goes to ground wherever they are most comfortable. A bounty hunter lives and dies by his reputation--it is his currency and one of his primary weapons.

\subsubsection{Information Gatherer}
At 3rd level, you gain proficiency in Charisma checks made to gain information. Additionally, you learn the secret signals that identify one as an ally of the various criminal organizations of the known world. This does not gain you any direct favors, but grants access to black markets, fences, and other less reputable establishments.

\subsubsection{Fearsome Reputation}
At 3rd level, you've begun to establish a reputation as a hunter to be feared. This lets you unsettle the minds of the weak. As a bonus action on your turn, you can attempt to intimidate a number of creatures that can see and hear you equal to your proficiency bonus. The targets must speak at least one language that you are proficient in. Each target must make a Wisdom saving throw against your spell save DC. On a failed save, they are frightened of you for one minute. A creature who ends its turn where it cannot see you can attempt the saving throw again, ending the effect on a success. Creatures that succeed on the saving throw are immune to this ability for 24 hours.

Additionally, you have advantage on Charisma (Intimidation) checks made against any creature who can see you and who speaks a language that you are proficient in.

\subsubsection{Debilitating Reputation}
At 7th level, your reputation has grown strong enough to make the weak unable to act against you. When you use your Fearsome Reputation feature, choose one creature that failed the saving throw. That creature is unable to make attacks against you while frightened of you.

\subsubsection{Razor Dance}
At 11th level, you've honed your skills at taking down a single target. When you use your Aether Strike ability, instead of choosing multiple targets, you can choose to distribute the same number of attacks between fewer targets. For example, if you spent 6 AET (and thus could target 3 creatures), you can choose to make 3 attacks against a single target or 2 attacks against one target and 1 against another.

\subsubsection{Wordless Reputation}
Starting at 15th level, your reputation and the aura that surrounds you are such that you can use your Fearsome Reputation ability against any foe, whether it can understand you or not. In addition, you can choose to use Debilitating Reputation against all creatures that fail their saving throws instead of just one.

\subsection{Monster Slayer}
Monster Slayers protect civilization from the horrific things that lurk on the frontier. Most at home in the wilderness, they are rangers in the truest sense--ranging the borders of civilization. Many of the things they hunt are twisted monstrosities and aberrations...but many of the worst monsters present a civilized face. 

\subsubsection{Hunter's Prey}

At 3rd level, you gain the following features, but can only apply one of them per turn.

\subparagraph*{Colossus Slayer.} Your tenacity can wear down the most potent foes. When you hit a creature with a weapon attack, the creature takes an extra 1d8 damage if it's below its hit point maximum. You can deal this extra damage only once per turn.

\subparagraph*{Giant Killer.} When a Large or larger creature within 5 feet of you hits or misses you with an attack, you can use your reaction to attack that creature immediately after its attack, provided that you can see the creature.

\subparagraph*{Horde Breaker.} Once on each of your turns when you make a weapon attack, you can make another attack with the same weapon against a different creature that is within 5 feet of the original target and within range of your weapon.

\subsubsection{Defensive Tactics}

At 7th level, you gain the following features, but can only apply one of them per turn.

\subparagraph*{Escape the Horde.} Opportunity attacks against you are made with disadvantage.

\subparagraph*{Multiattack Defense.} When a creature hits you with an attack, you gain a +4 bonus to AC against all subsequent attacks made by that creature for the rest of the turn.

\subsubsection{Steel Will.} Starting at 7th level, you have advantage on saving throws against being frightened.

\subsubsection{Multiattack}

At 11th level, you gain the following features.

\subparagraph*{Volley.} When you use your Aether Strike ability with a ranged weapon, you can target one creature for every aether spent instead of one target per two aether spent.

\subparagraph*{Whirlwind Attack.} When you use your Aether Strike ability with a melee weapon, you can choose to perform a whirlwind attack at any point along the chain of attacks. If you do so, all creatures within your reach at that point count as targets for your Aether Strike. Make a separate attack for each one. You may only perform a single whirlwind attack per use of Aether Strike.

\subsubsection{Superior Hunter's Defense}

At 15th level, you gain the following features.

\subparagraph*{Evasion.} When you are subjected to an effect, such as a red dragon's fiery breath or a \textit{lightning bolt} spell, that allows you to make a Dexterity saving throw to take only half damage, you instead take no damage if you succeed on the saving throw, and only half damage if you fail.

\subparagraph*{Stand Against the Tide.} When a hostile creature misses you with a melee attack, you can use your reaction to force that creature to repeat the same attack against another creature (other than itself) of your choice.

\subparagraph*{Uncanny Dodge.} When an attacker that you can see hits you with an attack, you can use your reaction to halve the attack's damage against you.

\subsection{Ranger Spell List}
The Ranger Spell List table contains a short summary of the spells available to all Oathbound, ordered by aether cost. The Details column may contain any of the following symbols:
\begin{itemize}
	\item \copyright : the spell requires concentration.
	\item $\dagger$ : the spell requires an attack roll.
	\item \textbf{STR|DEX|CON|INT|WIS|CHA} : the spell requires a saving throw of the indicated type.
	\item $\odot$ : The spell affects an area around yourself.
	\item $\bigcirc$ : The spell affects an area around a point you can choose. Note: this may be a square, circle, or cylinder.
	\item $\triangleleft$: The spell affects a cone.
	\item $\rightarrow$: the spell affects a line.
\end{itemize}


\begin{DndTable}[header=Ranger Spell List\label{lst:ranger-spells}]{XXX}
    \textbf{AET} & \textbf{Name} & \textbf{Description} \\
    1 & \nameref{spell:cure-wounds} & 1d8+MOD healing, touch \\
    1 & \nameref{spell:true-strike} & Buff next attack, BA \\
    2 & \nameref{spell:bane} & \textminus 1d4 from attacks and saves, 3 creatures, \copyright 1 minute. \\
    2 & \nameref{spell:burning-hands} & 3d6 fire, $\triangleleft$ \\
    2 & \nameref{spell:disguise-self} & Change appearance, self, 1 hour \\
    2 & \nameref{spell:entangle} & Restrain in area, $\bigcirc$, \copyright 1 minute \\
    2 & \nameref{spell:faerie-fire} & Advantage on attacks, $\bigcirc$, \textbf{DEX}, \copyright 1 minute \\
    2 & \nameref{spell:flash-freeze} & 4d6 cold, 1 target, \textbf{CON} \\
    2 & \nameref{spell:fog-cloud} & Obscure area, $\bigcirc$, \copyright 1 hour\\
    2 & \nameref{spell:grease} & Slippery area, $\bigcirc$, 1 minute \\
    2 & \nameref{spell:longstrider} & +10ft move, touch, 1 hour \\
    2 & \nameref{spell:sleep} & Put people to sleep, $\bigcirc$, \copyright 1 minute \\
    2 & \nameref{spell:thunderwave} & Cube of push + 2d8 thunder, $\odot$, \textbf{CON}\\
    3 & \nameref{spell:alter-self} & Minor shapechanging, self, \copyright 1 hour.\\
    3 & \nameref{spell:barkskin} & AC/THP boost, touch, 1 hour. \\
    3 & \nameref{spell:blindness-deafness} & blind or deafen, 1 target, \textbf{CON}, 1 minute \\
    3 & \nameref{spell:darkvision} & Grant darkvision, 8 hours \\
    3 & \nameref{spell:detect-thoughts} & Scan for thoughts, \copyright 1 minute \\
    3 & \nameref{spell:find-traps} & Detect harmful areas \\
    3 & \nameref{spell:flame-blade} & Create firey scimitar, self, \copyright 10 minutes \\
    3 & \nameref{spell:invisibility} & Become invisible, touch, \copyright 1 hour \\
    3 & \nameref{spell:pass-without-trace} & Boost stealth, $\odot$, \copyright 1 hour \\
    3 & \nameref{spell:see-invisibility} & See invisible, self, 1 hour \\
    3 & \nameref{spell:silence} & Create zone of silence, $\bigcirc$, \copyright 10 minutes \\
    3 & \nameref{spell:spike-growth} & Nasty area of spikes, $\bigcirc$, \copyright 10 minutes \\
    3 & \nameref{spell:web} & Entangle area, $\bigcirc$, \copyright 1 hour \\
    4 & \nameref{spell:hold-person} & Paralyze humanoid, 1 target, \textbf{WIS}, \copyright 1 minute \\
    5 & \nameref{spell:bestow-curse} & Curse target, touch, \copyright 1 minute \\
    5 & \nameref{spell:clairvoyance} & Create immobile sensor, \copyright 10 minutes \\
    5 & \nameref{spell:conjure-animals} & Summon beasts, \copyright 1 hour \\
    5 & \nameref{spell:plant-growth} & Large area of difficult terrain, $\bigcirc$ \\
    5 & \nameref{spell:wind-wall} & Create wall of air, $\bigcirc$, \copyright 1 minute \\
    6 & \nameref{spell:blight} & 10d8 necrotic, 1 target \\
    6 & \nameref{spell:call-lightning} & 3d10 repeatable, $\bigcirc$, \copyright 10 minutes \\
    7 & \nameref{spell:wall-of-thorns} & Create wall of thorns, $\bigcirc$, \copyright 10 minutes \\
    8 & \nameref{spell:conjure-woodland-beings} & Summon fey, \copyright 1 hour \\
    8 & \nameref{spell:dominate-beast} & Take control of beast, \copyright 1 minute \\
    8 & \nameref{spell:faithful-hound} & Create phantom hound, 8 hours \\
    8 & \nameref{spell:freedom-of-movement} & Immune to movement reducers, touch, 1 hour \\
    8 & \nameref{spell:giant-insect} & Transform insects, \copyright 10 minutes\\
    8 & \nameref{spell:greater-invisibility} & Non-breaking invisibility, \copyright 1 minute\\
    12 & \nameref{spell:contagion} & Cause disease, $\dagger$, \textbf{CON}, 7 days \\
    12 & \nameref{spell:hold-monster} & Paralyze anything, \textbf{WIS}, \copyright 1 minute\\
    13 & \nameref{spell:chain-lightning} & 10d8 lightning, 4 targets 
\end{DndTable}
