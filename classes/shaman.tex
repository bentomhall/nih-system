\section{Shaman}

\subsection{Class Features}

As a shaman, you gain the following class features.

\subsection{Hit Points}

\textbf{Hit Dice:} 1d8 per shaman level

\textbf{Hit Points at 1st Level:} 8 + your Constitution modifier

\textbf{Hit Points at Higher Levels:} 1d8 (or 5) + your Constitution modifier per shaman level after 1st

\subsection{Proficiencies}

\textbf{Armor:} Light armor, medium armor, shields (shamans will not wear armor or use shields made of metal)

\textbf{Weapons:} Clubs, daggers, darts, javelins, maces, quarterstaffs, scimitars, sickles, slings, spears

\textbf{Tools:} Herbalism kit

\textbf{Saving Throws:} Intelligence, Wisdom

\textbf{Skills:} Choose two from Arcana, Animal Handling, Insight, Medicine, Nature, Perception, Religion, and Survival

\subsection{Equipment}

You start with the following equipment, in addition to the equipment granted by your background:
\begin{itemize}
\item (*a*) a wooden shield or (*b*) any simple weapon
\item (*a*) a scimitar or (*b*) any simple melee weapon
\item Leather armor, an explorer’s pack, and a shamanic focus
\end{itemize}

\textbf{The Shaman (table)}
\begin{DndTable}[header=The Shaman\label{tbl:shaman}]{XXXXXXXXXXXXX}
 Level & Proficiency Bonus & Features                                          & Cantrips Known & 1st & 2nd & 3rd & 4th & 5th & 6th & 7th & 8th & 9th \\
 1st   & +2                & Spellcasting                             & 2              & 2   & -   & -   & -   & -   & -   & -   & -   & -   \\
 2nd   & +2                & Shaman Circle                          & 2              & 3   & -   & -   & -   & -   & -   & -   & -   & -   \\
 3rd   & +2                & -                                                 & 2              & 4   & 2   & -   & -   & -   & -   & -   & -   & -   \\
 4th   & +2                & Ability Score Improvement & 3              & 4   & 3   & -   & -   & -   & -   & -   & -   & -   \\
 5th   & +3                & -                                                 & 3              & 4   & 3   & 2   & -   & -   & -   & -   & -   & -   \\
 6th   & +3                & Shaman Circle feature                              & 3              & 4   & 3   & 3   & -   & -   & -   & -   & -   & -   \\
 7th   & +3                & -                                                 & 3              & 4   & 3   & 3   & 1   & -   & -   & -   & -   & -   \\
 8th   & +3                & Ability Score Improvement & 3              & 4   & 3   & 3   & 2   & -   & -   & -   & -   & -   \\
 9th   & +4                & -                                                 & 3              & 4   & 3   & 3   & 3   & 1   & -   & -   & -   & -   \\
 10th  & +4                & Shaman Circle feature                              & 4              & 4   & 3   & 3   & 3   & 2   & -   & -   & -   & -   \\
 11th  & +4                & -                                                 & 4              & 4   & 3   & 3   & 3   & 2   & 1   & -   & -   & -   \\
 12th  & +4                & Ability Score Improvement                         & 4              & 4   & 3   & 3   & 3   & 2   & 1   & -   & -   & -   \\
 13th  & +5                & -                                                 & 4              & 4   & 3   & 3   & 3   & 2   & 1   & 1   & -   & -   \\
 14th  & +5                & Shaman Circle feature                              & 4              & 4   & 3   & 3   & 3   & 2   & 1   & 1   & -   & -   \\
 15th  & +5                & -                                                 & 4              & 4   & 3   & 3   & 3   & 2   & 1   & 1   & 1   & -   \\
 16th  & +5                & Ability Score Improvement                         & 4              & 4   & 3   & 3   & 3   & 2   & 1   & 1   & 1   & -   \\
 17th  & +6                & -                                                 & 4              & 4   & 3   & 3   & 3   & 2   & 1   & 1   & 1   & 1   \\
 18th  & +6                & Timeless Body                       & 4              & 4   & 3   & 3   & 3   & 3   & 1   & 1   & 1   & 1   \\
 19th  & +6                & Ability Score Improvement                         & 4              & 4   & 3   & 3   & 3   & 3   & 2   & 1   & 1   & 1   \\
 20th  & +6                &                                         & 4              & 4   & 3   & 3   & 3   & 3   & 2   & 2   & 1   & 1   \\
\end{DndTable}

\subsection{Spellcasting}

Drawing on the divine essence of nature itself, you can cast spells to shape that essence to your will.

\subsection{Cantrips}

At 1st level, you know two cantrips of your choice from the shaman spell list. You learn additional shaman cantrips of your choice at higher levels, as shown in the Cantrips Known column of the Shaman table.

\subsubsection{Preparing and Casting Spells}

\nameref{tbl:shaman} table shows how much aether (AET) you have to cast your spells and do other magical tasks. To cast a spell that requires aether, you must expend aether equal to its cost or greater. You regain all expended aether when you finish a long rest. It also shows your Aether Limit, which is the maximum aether you can expend on a single action.

You know a certain number of shaman spells, choosing from the shaman spell list. You can trade out any known spell for any other spell you can learn from that list when you finish a long rest. When you do so, choose a number of shaman spells equal to your Wisdom modifier + your shaman level (minimum of one spell). To prepare a spell you must be able to cast it without exceeding your Aether Limit.

\subsection{Spellcasting Ability}

Wisdom is your spellcasting ability for your shaman spells. The power of your spells comes from your devotion to your deity. You use your Wisdom whenever a shaman spell refers to your spellcasting ability. In addition, you use your Wisdom modifier when setting the saving throw DC for a shaman spell you cast and when making an attack roll with one.

\textbf{Spell save DC} = 8 + your proficiency bonus + your Wisdom modifier

\textbf{Spell attack modifier} = your proficiency bonus + your Wisdom modifier

\subsection{Ritual Casting}

You learn a common incantation (see \nameref{ch:incantations} for the list) of your choice. When you reach 5th level, you learn an uncommon incantation of your choice, and at 11th level you learn a rare incantation of your choice. You can cast any incantation you learned from this feature without needing a Ritual Scroll in hand.

\subsection{Spellcasting Focus}

You can use a shamanic focus (see chapter 5, “Equipment”) as a spellcasting focus for your shaman spells.

\subsection{Shaman Circle}

At 2nd level, you choose to identify with a circle of shamans: the Circle of the Land or the Circle of the Moon, both detailed at the end of the class description. Your choice grants you features at 2nd level and again at 6th, 10th, and 14th level.

\subsection{Ability Score Improvement}

When you reach 4th level, and again at 8th, 12th, 16th, and 19th level, you can increase one ability score of your choice by 2, or you can increase two ability scores of your choice by 1. As normal, you can’t increase an ability score above 20 using this feature.

\subsection{Timeless Body}

Starting at 18th level, the primal magic that you wield causes you to age more slowly. For every 10 years that pass, your body ages only 1 year.

\subsection{Shaman Circles}

\subsection{Circle of the Land}

The Circle of the Land is made up of mystics and sages who safeguard ancient knowledge and rites through a vast oral tradition. These shamans meet within sacred circles of trees or standing stones to whisper primal secrets in Shamanic. The circle’s wisest members preside as the chief priests of communities that hold to the Old Faith and serve as advisors to the rulers of those folk. As a member of this circle, your magic is influenced by the land where you were initiated into the circle’s mysterious rites.

\subsection{Bonus Cantrip}

When you choose this circle at 2nd level, you learn one additional shaman cantrip of your choice.

\subsection{Natural Recovery}

Starting at 2nd level, you can regain some of your magical energy by sitting in meditation and communing with nature. During a short rest, you choose expended spell slots to recover. The spell slots can have a combined level that is equal to or less than half your shaman level
(rounded up), and none of the slots can be 6th level or higher. You can’t use this feature again until you finish a long rest.

For example, when you are a 4th-level shaman, you can recover up to two levels worth of spell slots. You can recover either a 2nd-level slot or two 1st-level slots.

\subsection{Circle Spells}

Your mystical connection to the land infuses you with the ability to cast certain spells. At 3rd, 5th, 7th, and 9th level you gain access to circle spells connected to the land where you became a shaman. Choose that land—arctic, coast, desert, forest, grassland, mountain, or swamp—and consult the associated list of spells.

Once you gain access to a circle spell, you always have it prepared, and it doesn’t count against the number of spells you can prepare each day. If you gain access to a spell that doesn’t appear on the shaman spell list, the spell is nonetheless a shaman spell for you.

\textbf{Arctic (table)}

 Shaman Level & Circle Spells                     
 3rd         & hold person, spike growth         
 5th         & sleet storm, slow                 
 7th         & freedom of movement, ice storm    
 9th         & commune with nature, cone of cold 

\textbf{Coast (table)}

 Shaman Level & Circle Spells                      
 3rd         & mirror image, misty step           
 5th         & water breathing, water walk        
 7th         & control water, freedom of movement 
 9th         & conjure elemental, scrying         

\textbf{Desert (table)}

 Shaman Level & Circle Spells                                 
 3rd         & blur, silence                                 
 5th         & create food and water, protection from energy 
 7th         & blight, hallucinatory terrain                 
 9th         & insect plague, wall of stone                  

\textbf{Forest (table)}

 Shaman Level & Circle Spells                    
 3rd         & barkskin, spider climb           
 5th         & call lightning, plant growth     
 7th         & divination, freedom of movement  
 9th         & commune with nature, tree stride 

\textbf{Grassland (table)}

 Shaman Level & Circle Spells                    
 3rd         & invisibility, pass without trace 
 5th         & daylight, haste                  
 7th         & divination, freedom of movement  
 9th         & dream, insect plague             

\textbf{Mountain (table)}

 Shaman Level & Circle Spells                   
 3rd         & spider climb, spike growth      
 5th         & lightning bolt, meld into stone 
 7th         & stone shape, stoneskin          
 9th         & passwall, wall of stone         

\textbf{Swamp (table)}

 Shaman Level & Circle Spells                        
 3rd         & acid arrow, darkness                 
 5th         & water walk, stinking cloud           
 7th         & freedom of movement, locate creature 
 9th         & insect plague, scrying               

\subsection{Land’s Stride}

Starting at 6th level, moving through nonmagical difficult terrain costs you no extra movement. You can also pass through nonmagical plants without being slowed by them and without taking damage from them if they have thorns, spines, or a similar hazard.

In addition, you have advantage on saving throws against plants that are magically created or manipulated to impede movement, such those created by the *entangle* spell.

\subsection{Nature’s Ward}

When you reach 10th level, you can’t be charmed or frightened by elementals or fey, and you are immune to poison and disease.

\subsection{Nature’s Sanctuary}

When you reach 14th level, creatures of the natural world sense your connection to nature and become hesitant to attack you. When a beast or plant creature attacks you, that creature must make a Wisdom saving throw against your shaman spell save DC. On a failed save, the creature must choose a different target, or the attack automatically misses. On a successful save, the creature is immune to this effect for 24 hours.

The creature is aware of this effect before it makes its attack against you.

\begin{DndComment}{Sacred Plants and Wood}

A shaman holds certain plants to be sacred, particularly alder, ash, birch, elder, hazel, holly, juniper, mistletoe, oak, rowan, willow, and yew. Shamans often use such plants as part of a spellcasting focus, incorporating lengths of oak or yew or sprigs of mistletoe.

 Similarly, a shaman uses such woods to make other objects, such as weapons and shields. Yew is associated with death and rebirth, so weapon handles for scimitars or sickles might be fashioned from it. Ash is associated with life and oak with strength. These woods make excellent hafts or whole weapons, such as clubs or quarterstaffs, as well as shields. Alder is associated with air, and it might be used for thrown weapons, such as darts or javelins.

 Shamans from regions that lack the plants described here have chosen other plants to take on similar uses. For instance, a shaman of a desert region might value the yucca tree and cactus plants.
\end{DndComment}

\begin{DndComment}{Shamans and the Gods}
 Some shamans venerate the forces of nature themselves, but most shamans are devoted to one of the many nature deities worshiped in the multiverse (the lists of gods in appendix B include many such deities). The worship of these deities is often considered a more ancient tradition than the faiths of clerics and urbanized peoples.
\end{DndComment}