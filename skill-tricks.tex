\chapter{Skill Tricks}
\label{ch:skill-tricks}

Those who are particularly adept at certain aspects of adventuring often learn ways to use their talents to perform tricks that seem supernatural or magical to outside observers. While they are not magical in the same sense as spells or invocations, per se, they do produce effects not normally possible.

Each skill trick detailed below shares some common characteristics:
\begin{itemize}
	\item \textbf{A cost}. Every skill trick costs something, whether expending a replaceable tool, damaging a weapon or armor, or (most commonly) expending Stamina or Aether or both.
	\item \textbf{An ability score}. Every skill trick is tied to a particular ability score. That ability score sets its DC.
	\item \textbf{A prerequisite}. Every skill trick has one or more prerequisites before it can be learned. These are generally either a particular level of proficiency (numerical value, which does not includes expertise) for those that are tied to a particular skill or tool, or a character level for those marked as General.
	\item \textbf{A target or targets}. Many skill tricks target either an object or one or more creatures. A few target a particular area.
	\item \textbf{An effect}. The text of the skill trick describes the effect, as well as any saving throws required.
\end{itemize}

\subsection{Skill Trick DCs}
The DC for any saving throws required by skill tricks is given by

\begin{center}
\textbf{8 + the relevant ability score + your proficiency}
\end{center}

regardless of whether the trick involves a proficiency or not. If you have expertise in the relevant skill or tool, targets have disadvantage on the saving throw.

\subsection{Acquiring Skill Tricks}
\label{subsec:acquiring-skill-tricks}

Some classes get native access to Skill Tricks as a class feature. If they grant access to more advanced skill tricks beyond the basic ones at particular levels, that access overrides any prerequisites in the skill trick. Everyone else can choose a skill trick that they qualify for whenever they acquire an Ability Score Improvement from their class. At the same time, they can trade out one skill trick they've learned for a different one they qualify for.

\section{Basic Skill Tricks}
\label{sec:skill-tricks-basic}

Basic skill tricks only require a +2 proficiency or level 1 characters.

\section{Advanced Skill Tricks}
\label{sec:skill-tricks-advanced}

Advanced skill tricks require a +4 proficiency or level 9 characters. 

\section{Expert Skill Tricks}
\label{sec:skill-tricks-expert}

Expert skill tricks require a +5 proficiency or level 13 characters.

\section{Master Skill Tricks}
\label{sec:skill-tricks-master}

Master skill tricks require a +6 proficiency or level 17 characters.