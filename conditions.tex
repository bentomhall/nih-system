\section{Conditions}
\label{sec:conditions}
Conditions alter a creature's capabilities in a variety of ways and can arise as a result of a spell, a class feature, a monster's attack, or other effect. Most conditions, such as blinded, are impairments, but a few, such as invisible, can be advantageous.

A condition lasts either until it is countered (the prone condition is countered by standing up, for example) or for a duration specified by the effect that imposed the condition.

If multiple effects impose the same condition on a creature, each instance of the condition has its own duration, but the condition's effects don't get worse. A creature either has a condition or doesn't.

The following definitions specify what happens to a creature while it is subjected to a condition.

\subsection{Blinded\label{condition:blinded}}
\begin{itemize}
    \item A blinded creature can't see and automatically fails any ability check that requires sight.
    \item Attack rolls against the creature have advantage, and the creature's attack rolls have disadvantage.
\end{itemize}

\subsection{Bloodied\label{condition:bloodied}}
Bloodied, by itself, does little. It is automatically applied to any creature whose hit points are below half of its maximum.
\begin{itemize}
    \item  Deflect, exert, and focus cost double.
\end{itemize}

\subsection{Broken\label{condition:broken}}
The default way of imposing the broken condition is by the creature failing a Wisdom saving throw against Morale effects. Some other abilities may impose it. It lasts for one round at minimum, or more if appropriate. Mindless creatures are immune to being broken.
\begin{itemize}
	\item A broken creature focuses their attention on personal survival. This may mean surrendering, fleeing, or taking the Total Defense action if they can't get away.
	\item Broken creatures rarely attack, but if they do they do so at disadvantage.
	\item Attacks against broken creatures are at advantage.
\end{itemize}


\subsection{Charmed\label{condition:charmed}}
\begin{itemize}
\item A charmed creature can't attack the charmer or target the charmer with harmful abilities or magical effects.
\item The charmer has advantage on any ability check to interact socially with the creature.
\end{itemize}

\subsection{Deafened\label{condition:deafened}}
\begin{itemize}
\item A deafened creature can't hear and automatically fails any ability check that requires hearing.
\end{itemize}

\subsection{Exhaustion\label{condition:exhaustion}}

Some special abilities and environmental hazards, such as starvation and the long-term effects of freezing or scorching temperatures, can lead to a special condition called exhaustion. Exhaustion is measured in six levels. An effect can give a creature one or more levels of exhaustion, as specified in the effect's description.

Each level of exhaustion (up to five levels) adds a cumulative -1 modifier to all checks, saves, and attacks. It also decreases the save DC of your abilities and spells by 1. Upon taking a 6th level of exhaustion, you are incapacitated until you no longer have 6 levels.

If an already exhausted creature suffers another effect that causes exhaustion, its current level of exhaustion increases by the amount specified in the effect's description.

An effect that removes exhaustion reduces its level as specified in the effect's description, with all exhaustion effects ending if a creature's exhaustion level is reduced below 1.

Finishing a long rest reduces a creature's exhaustion level by 1, provided that the creature has also ingested some food and drink (or was incapacitated with 6 levels).

\subsection{Frightened\label{condition:frightened}}
\begin{itemize}
\item A frightened creature has disadvantage on ability checks and attack rolls while the source of its fear is within line of sight.
\item The creature can't willingly move closer to the source of its fear.
\end{itemize}

\subsection{Grappled\label{condition:grappled}}
\begin{itemize}
\item A grappled creature's speed becomes 0, and it can't benefit from any bonus to its speed.
\item The condition ends if the grappler is incapacitated (see \nameref{condition:incapacitated}).
\item The condition also ends if an effect removes the grappled creature from the reach of the grappler or grappling effect, such as when a creature is hurled away by the \textit{thunder-wave} spell. The reverse also holds--effects that move the grappler out of reach of the grappled creature end the grapple as well.
\end{itemize}

\subsection{Hidden}\label{condition:hidden}
Hidden is a \textit{relative} condition--you can be hidden from one creature and not from another. To become hidden, you must either be impossible to detect by the other creature---have total concealment, be unable to be heard (such as by the \nameref{spell:silence} spell), and be unable to be smelled or otherwise sensed (via senses like tremorsense or blindsight)--or have successfully taken the Hide action (which requires total concealment) and beaten that creature's Passive Perception.
\begin{itemize}
	\item A hidden creature has advantage on their first attack against a creature they're hidden from. Hit or miss, the hidden condition then ends for any creature that could see the attack happen.
	\item A hidden creature is automatically revealed if they move out of total concealment.
\end{itemize}

\subsection{Incapacitated\label{condition:incapacitated}}
\begin{itemize}
\item An incapacitated creature can't take actions or reactions. They can still move and speak.
\end{itemize}

\subsection{Invisible\label{condition:invisible}}
An invisible creature is impossible to see without the aid of magic or a special sense. If they are unseen as a result, see the Unseen Attackers rules. 

\subsection{Paralyzed\label{condition:paralyzed}}
\begin{itemize}
\item A paralyzed creature is incapacitated (see the \nameref{condition:incapacitated}) and can't move or speak.
\item The creature automatically fails Strength and Dexterity saving throws.
\item Attack rolls against the creature have advantage.
\item Any attack that hits the creature is a critical hit if the attacker is within 5 feet of the creature.
\end{itemize}

\subsection{Petrified\label{condition:petrified}}
\begin{itemize}
\item A petrified creature is transformed, along with any nonmagical object it is wearing or carrying, into a solid inanimate substance (usually stone). Its weight increases by a factor of ten, and it ceases aging.
\item The creature is incapacitated (see the condition), can't move or speak, and is unaware of its surroundings.
\item Attack rolls against the creature have advantage.
\item The creature automatically fails Strength and Dexterity saving throws.
\item The creature has resistance to all damage.
\item The creature is immune to poison and disease, although a poison or disease already in its system is suspended, not neutralized.
\end{itemize}

\subsection{Poisoned\label{condition:poisoned}}
\begin{itemize}
\item A poisoned creature has disadvantage on attack rolls and ability checks.
\end{itemize}

\subsection{Prone\label{condition:prone}}
\begin{itemize}
\item A prone creature's only movement option is to crawl, unless it stands up and thereby ends the condition.
\item The creature has disadvantage on attack rolls.
\item An attack roll against the creature has advantage if the attacker is within 5 feet of the creature. Otherwise, the attack roll has disadvantage.
\end{itemize}

\subsection{Restrained\label{condition:restrained}}
\begin{itemize}
\item A restrained creature's speed becomes 0, and it can't benefit from any bonus to its speed.
\item Attack rolls against the creature have advantage, and the creature's attack rolls have disadvantage.
\item The creature has disadvantage on Dexterity saving throws.
\end{itemize}

\subsection{Shaken\label{condition:shaken}}
\begin{itemize}
	\item A shaken creature cannot take reactions.
	\item A shaken creature's speed is reduced to half.
\end{itemize}

\subsection{Staggered\label{condition:staggered}}
\begin{itemize}
	\item A staggered creature has disadvantage on attack rolls 
	\item A staggered creature has disadvantage on Dexterity saving throws and ability checks.
\end{itemize}

\subsection{Stunned\label{condition:stunned}}
\begin{itemize}
\item A stunned creature is incapacitated (see \nameref{condition:incapacitated}), can't move, and can speak only falteringly.
\item The creature automatically fails Strength and Dexterity saving throws.
\item Attack rolls against the creature have advantage.
\end{itemize}

\subsection{Surprised}\label{condition:surprised}
This condition is applied only on the first round of combat and automatically ends at the end of the affected person's first turn.
\begin{itemize}
	\item A surprised creature cannot take actions or reactions and cannot move. 
	\item Surprised creatures still roll for initiative normally and take their turn. They can speak, but this speech cannot clear the surprised condition for other creatures in that encounter.
\end{itemize}

\subsection{Unconscious\label{condition:unconscious}}
\begin{itemize}
\item An unconscious creature is incapacitated (see \nameref{condition:incapacitated}), can't move or speak, and is unaware of its surroundings
\item The creature drops whatever it's holding and falls prone.
\item The creature automatically fails Strength and Dexterity saving throws.
\item Attack rolls against the creature have advantage.
\item Any attack that hits the creature is a critical hit if the attacker is within 5 feet of the creature.
\end{itemize}