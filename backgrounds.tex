\section{Backgrounds}
\label{sec:backgrounds}

Every story has a beginning. Your character's background reveals where you came from, how you became an adventurer, and your place in the world. Your fighter might have been a courageous knight or a grizzled soldier. Your wizard could have been a sage or an artisan. Your rogue might have gotten by as a guild thief or commanded audiences as a jester.

Choosing a background provides you with important story cues about your character's identity. The most important question to ask about your background is \textit{what changed}? Why did you stop doing whatever your background describes and start adventuring? Where did you get the money to purchase your starting gear, or, if you come from a wealthy background, why don't you have \textit{more} money? How did you learn the skills of your class? What sets you apart from ordinary people who share your background?

The sample backgrounds in this chapter provide both concrete benefits (features, proficiencies, and languages) and roleplaying suggestions.

\subsection{Proficiencies}

Each background gives a character proficiency in two skills (described in \nameref{ch:using-ability-scores}).

In addition, most backgrounds give a character proficiency with one or more tools (detailed in \nameref{ch:equipment}).

If a character would gain the same proficiency from two different sources, he or she can choose a different proficiency of the same kind (skill or tool) instead.

\subsection{Languages}

Some backgrounds also allow characters to learn additional languages beyond those given by race. See “\nameref{sec:languages}.”

\subsection{Equipment}

Each background provides a package of starting equipment. This is in addition to the starting equipment listed for your class.

\subsection{Questions to Consider}
Each sample background presents a list of questions you can use as a springboard to guide you in creating your character's history before they started adventuring. These are the sort of thing to discuss with your GM so that your story can be woven into the world and the ongoing campaign's story.

\subsection{Skill Trick}

A background gives access to a single skill trick (see \ref{ch:skill-tricks} for more details and the complete list). As you customize your background, you can substitute the listed skill trick for any other that uses one of the proficiencies granted by the background.

\subsection{Customizing a Background} 

You might want to tweak some of the features of a background so it better fits your character or the campaign setting. To customize a background, you can replace one feature with any other one, choose any two skills, and choose a total of two tool proficiencies or languages from the sample backgrounds. You can either use the equipment package from your background or spend coin on gear as described in the equipment section. If you spend coin, you can't also take the equipment package suggested for your class. Finally, choose two personality traits, one ideal, one bond, and one flaw. If you can't find a feature that matches your desired background, work with your GM to create one.

\subsubsection{Personality Traits}
Personality traits are small "tics" and mannerisms--things that your character does naturally--or other neutral facets other people can observe. These may be verbal tics, mannerisms, habits, etc. Generally these are small and neutral, just reminders for you to play off of. Unlike the bond, ideal, and flaw, they're less about being sources of plot hooks and more about characterization.

Ideas:
\begin{itemize}
	\item You like to use overly-large words.
	\item You talk with your hands, which can get dangerous if you happen to be armed.
	\item Given a choice, you'll never take the shortest path between two points.
	\item You have a horrible sense of direction and frequently get lost if left to your own devices.
	\item You are fastidious about your grooming, getting annoyed if you're muddy or unable to bathe frequently.
\end{itemize}

\subsubsection{Ideal}
Your ideal is something that you hold to deeply, something that you want to bring about. Different ideals can be interpreted many different ways--for one person, freedom might be being free from external restraints and going with the wind. For another, it might be about having enough power that no one tells you what to do. Choose something that your character will attempt to follow through on and seek for, but be careful about choosing things that strongly conflict with the ideals of fellow party members. Conflict of ideals isn't necessarily \textit{wrong}, but it should be discussed openly out of character before the campaign begins so all the players are ok with it.

\begin{itemize}
	\item Freedom--you believe that everyone should be left to do as they please as long as it doesn't hurt anyone.
	\item Freedom--you don't ever want anyone to tell you what to do.
	\item Order--you believe it's best if everyone has a place and job and everyone stays in their place and does their job.
	\item Order--you believe it's best if everyone listens to you.
	\item Altruism--you believe that self-sacrifice for others is the noblest of goals.
	\item Profit--you believe that money makes the world go round.
\end{itemize}

\subsubsection{Bond}
While ideals are abstract, bonds are concrete. A person, organization, or place that you strongly want to protect, defend, regain, or help. Work with your GM to flesh out the details. A bond is supposed to act as a "plot eyebolt"--a place you've agreed with the GM that he or she can attach plot hooks and you'll bite.

\begin{itemize}
	\item Family--your family means everything to you, but you're not strong enough to defend them. Which is why you're adventuring.
	\item Family--it was taken from you/you never had it. So you're out looking for family of your own.
	\item Family--they kicked you out. So now you're looking to make your own or show them that they were wrong to do so.
	\item A mentor--you act to further his or her legacy, fulfil their last wishes, etc.
	\item A guild, company, etc.--you are an agent of a group that has sent you out on an adventure. You owe much to them.
	\item A guild, company, etc.--they kicked you out and you want to prove them wrong/revenge/get back into their graces.
\end{itemize}

\subsubsection{Flaw}
A flaw is something that gets you in trouble. Like the bond, this is a plot eyebolt. Flaws should be noticeable and make you do things that aren't "optimal", but shouldn't make the game un-fun for others by putting the campaign in jeopardy senselessly. "I attack my fellows in blind rage" is not, generally, an appropriate flaw. Neither is "I never back down".

\begin{itemize}
	\item I tend to drink more than I should.
	\item I'm an inveterate skirt-chaser (of the appropriate gender). A pretty person will turn my head and make me stop thinking as clearly.
	\item I can stop gambling whenever I choose...
	\item I'm in debt to a criminal organization.
	\item I did something really stupid, and people are after me.
\end{itemize}

\subsection{Regional Origin}
Every character comes from somewhere. And that origin makes a big difference. Who you know, what you know, who knows you, and what things you're good at are all part of your regional origin. Some regions have peculiarities that make some backgrounds inapposite or at least require significantly more explanation--for example, the entirely land-locked Uulan Confederacy doesn't exactly have many sailors. And the kritocracy of Byssia doesn't have nobility, but a similar background can be framed around one of the more influential (in practice, if not in law) merchant or religious families.

In the default setting (Dreams of Hope's Federated Nations area), the regions of orign are:

\subparagraph*{Byssia}
Byssia lies in the southern portion of the Nocthian Caldera and parts south, including a large portion of Gap Tooth Bay. The Byssians are mostly human and ihmisi, with a very few others mingled in. The nation is a highly-decentralized state governed by judges elected from the towns and villages, although their powers are mostly executive and judicial rather than legislative. The Church of Night Reborn (worshiping the Ascendant Nocthis) and the Home of the Elements (a monastic order of elementalists and sages) both have significant de facto influence. There are no nobility, and the nation does not have a standing army. It has many militias; some of them (such as the Caldera Militia), are more organized and perpetual than others. Byssians traditionally use little metal (the costal part has no significant sources of workable metal) and venerate ascendant ancestors, Nocthis, and the kami rather than the true gods. There are very few true priests in the area, and almost no organized religion other than the Church of Night Reborn. They do not have a strong academic tradition.

\subparagraph*{Giant Spine and Barrier Mountains}
The mountain folk of Shinevog, Zhapai Karmap, the Tuura Adam, and the Uulan Confederacy are hardy and self-sufficient. Shinevog and Zhapai Karmap are known for their "anything goes" pursuit of knowledge and money, respectively, and are home to many peoples. The Tuura Adam and the Uulan Confederacy are both traditionalist states, the former the home of most jazuu and the later home to most of the dwarven clans. Neither of the latter has any water access, and neither really has "nobility" per se. But similar backgrounds can be constructed. All three are heavily craft oriented, but much more guild and individual-oriented than the industrialized forces of Wyrmhold. None of them have standing militaries.

\subparagraph*{Jungle of Fangs}
The Jungle lies mostly isolated from the south end of the Sea of Grass down to the Moon Sea. There are 3 nations here (Sha'slar, Asai'ka, and the Serpent Dominion), but they share most of the culture. Humans, half-elves (scaled), and ophidians are the dominant lineages, although dwarves are not uncommon. Aristocracy, merchant castes, and (in the far south), sailing are the big features. The area is uniformly religious, with heavy worship of the Queen Ascendant. Criminal organizations are quite common.

\subparagraph*{Sea of Grass}
The Sea of Grass is a wide plain encompassing most of the largest cities. It borders Lake Coy'in heavily, with significant water traffic. The merchant nation of Rauviz, the gwerin-influenced Crisial Kingdom, the rugged Duarchy of Kotimaa, and the theocratic Holy Kaelthian Republic, the latter 3 of which are the most expansive nations in the area, are the dominant powers. Humans, halflings, gwerin, and dwarves are the dominant lineages. Any background is appropriate here.

\subparagraph*{Wyrmhold}
Wyrmhold is a highly militarized and industrial neighbor occupying the eastern flank of the Nocthian Caldera and the adjoining Kairen Mountains, as well as the southern part of the Fiach Wood and western part of the Lupaus Plains. Populated primarily by dragonborn, orcs, and goblins, there are jazuu in the high mountains as well. Other lineages only occur as migrants in the last decade or so. A clan-based aristocracy ruled by a queen, the nation has a strong and proud military tradition. They are also the second-most technologically advanced nation of the area--Shinevog beats them out. But they have a much deeper industrial base (albeit mostly military focused until very recently). They have little water access and are not generally known for their trading or merchant prowess.

\subsection{Sample Backgrounds}
\subsubsection{Aristocrat}
\BackgroundBlock{History, Persuasion, one gambling set, land or water vehicles}{Any language of your choice}{A set of fine clothes, a signet ring with your house's seal, a pouch containing 15 gp}{\nameref{st:scholar-history} OR \nameref{st:diplomat}}

True nobles are rare in the Federated Nations--the noble houses are few and far between and mostly very small--and not generally suited for adventuring. But aristocrats (formal or informal), the scions of landed gentry, rich merchant princes, influential families, etc? Those are many, even in the more egalitarian nations. And second and third children often make a name for themselves as adventurers.

\subparagraph*{Questions to consider}
\begin{itemize}
	\item What role did your family play in the nation you grew up in? Did they have a formal title or just significant influence? Maybe they had an old name, but had fallen on hard times?
	\item Are you still in good favor with your family? Or are you estranged?
	\item Is there anyone trying to get you to return and play a bigger role in the family affairs? Are you running from any arranged marriages?
	\item How do you view the "common folk"? How familiar are you with their ways and traditions? How cloistered in your high status were you?
\end{itemize}

\subsubsection{Crafter}
\BackgroundBlock{Investigation, two crafting tools of your choice}{Dwarven}{A set of crafting tools you are proficient in, 10gp of materials for that work, and a pouch containing 15 gp}{\nameref{st:craft-apprentice}}

Most of the Federated Nations works on a guild apprenticeship basis, with individual crafters learning under masters and striking out on their own to provide services. You were one of those crafters.

\subparagraph*{Questions to consider}
\begin{itemize}
	\item Did you complete your apprenticeship? If not, is your master still looking for you? If so, what is your relationship with your former master?
	\item Are you a member in good standing with any of the crafting guilds? If not, are you at odds with any?
	\item What variety of crafting did you do (e.g. fine metal work, pots, structural metal, armor/weapon smithing, etc)?
\end{itemize}

\subsubsection{Criminal}
\BackgroundBlock{Intimidation OR Deception, Stealth, Thieves tools}{One common language of your choice}{Thieves tools, a set of dark clothing, and a pouch containing 15 gp}{\nameref{st:misdirect} OR \nameref{st:feint}}

Criminal organizations, as well as individuals who live and operate outside the law, are common throughout the lands of the Federated Nations. The city state of Rauviz and the oligarchy of Asai'ka are most notorious for harboring criminal organizations, but "guilds" are present in most areas. Before you were an adventurer, you lived such a life.

\subparagraph*{Questions to Consider}
\begin{itemize}
	\item Were you part of an organized group? Or a freelancer?
	\item What was your specialty? Armed thuggery? Muscle? A pick-pocket? A confidence-man? A smuggler?
	\item Do you still have connections with the underground where you came from? Are they friendly? Or did you flee in haste? If so, why?
	\item What did you do that you regretted, if anything? What secrets do you have that might come back to bite you or your party?
\end{itemize}

\subsubsection{Entertainer}
\BackgroundBlock{Acrobatics, Performance, Disguise Kit, one musical instrument of your choice}{One common language of your choice}{A disguise kit, a set of performers clothes, a musical instrument, and a pouch containing 10 gp}{\nameref{st:tumble}}

Entertainers are minstrels, traveling players, actors, actresses, temple dancers, street performers, etc. They often move from place to place in search of work and new audiences.

\subparagraph*{Questions to Consider}
\begin{itemize}
	\item What kind of entertainer were you?
	\item Were you part of a troupe or band, or were you solo?
	\item What kinds of audiences did you favor? The common folk in taverns? Busking for coins on the street? The high society parties?
	\item Was there any particular person or people you had drama with?
\end{itemize}

\subsubsection{Farmer}
\BackgroundBlock{Animal Handling, Survival, Nature, one crafting tool of your choice or Land Vehicles}{One common language of your choice}{A small pet (CR 0) such as a dog, cat, or squirrel that will do simple tricks, a pouch containing 5 gp}{\nameref{st:soothe-domesticated-animal} OR (if proficient in a crafting tool) \nameref{st:craft-apprentice}}

The majority of the population of the Federated Nations is involved in farming or livestock handling to one degree or another. Without the rural folks, no one eats. Not even the greatest. You were part of that agricultural backbone...until you took up the adventuring life.

\begin{itemize}
	\item What did you or your family/village specialize in? Livestock? Grains? Did you have an orchard?
	\item What kind of village or town did you live in? Maybe an isolated hamlet of a few dozen souls? maybe the outskirts of a large city?
	\item Where did you get the resources and learning to pursue your adventuring career? A fighter might have inherited the sword and armor from an ancestor, etc.
	\item Do you still have family back on the farm? Were they ok with you leaving?
	\item What were you known for as a youth? Any particular events stand out?
\end{itemize}

\subsubsection{Merchant}
\BackgroundBlock{Deception or Persuasion, Insight, land vehicles, one gambling set of your choice}{One common language of your choice}{A book, a quill pen, and ink. A pouch containing 20 gp}{\nameref{st:diplomat} OR \nameref{st:haggler}}

Merchants run the gamut from the great merchant princes of Rauviz or Asai'ka to the humble traveling peddlers wandering among the settlements on the fringes of the known world. Before taking up adventuring, you were living this life, trading goods produced by others for coin.

\begin{itemize}
	\item Were you a solo practitioner? Part of a family business? Or part of a larger conglomerate?
	\item What did your business specialize in, if anything? Were you a purveyor of particular goods or a general peddler? What kind of customers were your norm?
	\item What contacts do you still have?
	\item What kind of reputation did you have? A fair dealer, accepted if not liked? A fly-by-night operator? Someone who could find what the customer wants...at a price? A shady, high pressure operator?
\end{itemize}

\subsubsection{Sailor}
\BackgroundBlock{Perception, Survival, water vehicles and cartographer's tools}{None}{A compass, a dagger, a pouch containing 10gp}{\nameref{st:alert} OR \nameref{st:lung-capacity}}

The Federated Nations is mostly landlocked. Four major bodies of water, plus a selection of riverine routes are the major outlets for the sailing dreams of mortalkind. The cold and misty Sea of Dreams to the north, the placid, freshwater, and deep Lake Coy'in (more the size of an inland sea), the shallow and stormy Gap-tooth Bay near Byssia, and the pirate-infested, island-dotted tropical Moon Sea south of the Jungle of Fangs. Each one has their own maritime tradition. You were crew aboard a ship, or maybe a solo fisherman.

\begin{itemize}
	\item Was your ship a large cargo vessel, a fast courier, or a fishing boat?
	\item Was the business of your ship entirely above-board? Or was smuggling a factor?
	\item What's your ship doing now?
	\item Were you willingly part of that life? Or were you snatched up and bound to service?
	\item Was your ship one big family or wer the captain and officers tyrants? Or maybe you were a tyrant?
\end{itemize}

\subsubsection{Scholar}
\BackgroundBlock{two of Arcana, History, Nature, or Religion}{Any two languages of your choice}{A quill, ink pot, and a notebook, a pouch containing 10 gp}{\nameref{st:linguist} OR \nameref{st:sense-baleful-magic}}

Many of the nations have a scholarly tradition, although not all have an \textit{academic} tradition. Some scholars are bound up in a library, such as the Four Towers just outside Crisial City, others wander. Many, if not most, scholars have some association with the Sages Guild, the international union of "civilized" scholars. For some, however, that association is negative--they reject the hidebound and conservative traditions of the Sages.

\begin{itemize}
	\item What was your specialty? The history of nations? Natural sciences? Arcana? The planes beyond? The practical matters of alchemy? Something even more esoteric or forbidden?
	\item What is your relationship with the Sages Guild? A member in good standing? An outcast? Something in between?
	\item If you were cloistered in a library or laboratory...what brought you out of that life? If you were a wanderer...what made you take up your particular practices (your class)?
	\item Do you have a reputation in the scholarly community?
\end{itemize}

\subsubsection{Shrine-keeper}
\BackgroundBlock{Religion, Persuasion, Wood-carver's tools}{Sylvan OR Lucian}{A set of vestments, a holy symbol, and a pouch containing 10 gp}{\nameref{st:scholar-religion} OR \nameref{st:diplomat}}

Shrine-keeper, priest, cleric, wise one, witch. Those that tend to the shrines of the Ascendants and kami that dot the landscape have many names. Most have no particular gifts of power; those that do often take up a calling like that of the priest class. But the number of those that simply tend the shrines, participating in the veneration and worship while not having an official standing with the Power in question, is legion. You were among that latter number whether by birth or choice.

\begin{itemize}
	\item What kind of shrine did you serve at? A small obscure rural shrine to a kami? Or a major temple in the heart of a city? Or maybe a private shrine frequented by the powerful?
	\item Were you part of a larger group of keepers and priests? Or was this your duty alone?
	\item Did you choose this life? Or was it thrust upon you by family obligations?
	\item Do you still have faith in that Power? What's your relationship with Them now?
	\item What event made you leave the service of the shrine and take up the life of a wandering adventurer? An oracle from your Power? Or the destruction/desecration of the shrine? Or maybe just a desire for a new life?
\end{itemize}

\subsubsection{Soldier}
\BackgroundBlock{Athletics, Medicine, one game set, smiths' tools}{One common language of your choice}{One weapon you are proficient in, a rank insignia, and a pouch containing 10 gp}{\nameref{st:frighten} OR \nameref{st:medic}}

Most of the Federated Nations don't have large standing armies--it's been a time of peace for quite a while. Wyrmhold is the key exception. But they all have militias of one sort or another, and every nation has specialist forces such as Crisial's Scout Corps or the Caldera Wardens of Byssia. There are also many private guard companies (mercenary companies by another name) doing caravan and local security, as well as private "armies" of the various wealthy and/or aristocratic families, as well as the clan guards common in the more tribal areas. You served in one of these groups.

\begin{itemize}
	\item What kind of organization did you serve with?
	\item Were you just a grunt or were you an officer?
	\item What role did you play in that organization?
	\item Did you leave on good terms? Or did you leave under a cloud (earned or not)?
	\item Were there any particular (small-scale) actions you participated in?
\end{itemize}

\subsubsection{Street Kid}
\BackgroundBlock{Perception, Stealth OR Intimidation, thieves tools, one gambling set of your choice}{None}{A gambling set, thieves tools, a pouch containing 5 gp}{\nameref{st:alert} OR \nameref{st:frighten}}

Every city and substantial town has those who fall through the cracks. Kids, especially, who grow up among the poor and lack stable homes. Not all of them are orphans, but all of them share the desperate struggle for day-to-day survival. Working odd jobs, stealing food, joining a street gang, begging--these are the occupations of such children. You were part of that life, but unlike most, you broke out and gained enough training to adventure.

\begin{itemize}
	\item Were you an orphan? Why did you end up on the streets?
	\item How did you survive? Did you try to play by the rules, or were you part of the underground (even if unofficially)? Were you part of a gang?
	\item How did you gain the training and resources for your class? Was there a kind (or cruel!) mentor or benefactor? A happy accident? A particularly big score?
	\item What kind of a town did you grow up in? Was it a big city? A medium-sized town?
\end{itemize}