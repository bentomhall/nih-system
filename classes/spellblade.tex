\section{Spellblade}\label{class:spellblade}

The spellblade is half-rogue, half-arcanist. They weave distraction, illusion, and weapon-play to confuse their foes, warping reality around them to hamper their ability to strike others. Unique among the classes, they have the ability to mix weapons and spells directly, gaining the ability to cast a spell on their weapon and have it take effect when they strike an enemy.

Design Discussion: The half-rogue, half-arcanist gish. Skill tricks, mixing spell and magic. UCT is Arcane Manipulation (inverse bardic inspiration) + Channeling (cast spell as part of attack). SDC 3/6/6/5.

\subsection{Class Features}

As a spellblade, you gain the following class features.

\subsection{Hit Points}

\textbf{Hit Dice:} 1d8 per spellblade level

\textbf{Hit Points at 1st Level:} 8 + your Constitution modifier

\textbf{Hit Points at Higher Levels:} 1d8 (or 5) + your Constitution modifier per spellblade level after 1st

\subsection{Proficiencies}

\textbf{Armor:} Light armor

\textbf{Weapons:} Simple weapons, hand crossbows, longswords, rapiers, shortswords

\textbf{Tools:} Three musical instruments of your choice

\textbf{Saving Throws:} Dexterity, Charisma

\textbf{Skills:} Choose any three

\subsection{Equipment}

You start with the following equipment, in addition to the equipment granted by your background:

\begin{itemize}
\item (\textit{a}) a rapier, (\textit{b}) a longsword, or (\textit{c}) any simple weapon
\item (\textit{a}) a diplomat’s pack or (\textit{b}) an entertainer’s pack or (\textit{c}) an explorer's pack
\item (\textit{a}) a lute or (\textit{b}) any other musical instrument or (\textit{c}) any tool
\item Leather armor and a dagger
\end{itemize}

\begin{figure*}[htb]
\begin{DndTable}[header=Spellblade\label{tbl:spellblade}]{llXcccccc}
 \textbf{Level} & \textbf{Proficiency} & \textbf{Features} & \textbf{Spells} & \textbf{Cantrips} & \textbf{Skill Tricks} & \textbf{Stamina} & \textbf{Aether} & \textbf{Aether Limit} \\ 
 1st   & +2 & Spellcasting, Arcane Manipulation (d6)        	& 1 + CHA      & 2              & --- & 1   & 2 & 1   \\  
 2nd   & +2 & Skill Tricks, Expertise                 				& 1 + CHA      & 2              & 1   & 2   & 4 & 2   \\   
 3rd   & +2 & Spellblade Focus                              	& 2 + CHA      & 2              & 1   & 3   & 6 & 3   \\   
 4th   & +2 & Ability Score Improvement                     	& 2 + CHA      & 2              & 2   & 4   & 8 & 3   \\   
 5th   & +3 & Arcane Manipulation (d8), Renewed Manipulation	& 3 + CHA      & 3              & 2   & 5   & 10 & 4  \\  
 6th   & +3 & Channeling (Cantrip)               							& 3 + CHA      & 3              & 2   & 6   & 12 & 5  \\   
 7th   & +3 & Spellblade Focus Feature                        & 4 + CHA      & 3              & 2   & 7   & 14 & 5  \\   
 8th   & +3 & Ability Score Improvement                       & 4 + CHA      & 3              & 3   & 8   & 16 & 6  \\   
 9th   & +4 & Counterweave                                    & 5 + CHA      & 4              & 3   & 9   & 18 & 7  \\   
 10th  & +4 & Arcane Manipulation (d10), Expertise 						& 5 + CHA      & 4              & 3   & 10  & 20 & 7  \\   
 11th  & +4 & Channeling (3)                                  & 6 + CHA      & 4              & 3   & 11  & 22 & 8  \\
 12th  & +4 & Ability Score Improvement                       & 6 + CHA      & 4              & 4   & 12  & 24 & 9  \\
 13th  & +5 & Channeling (5)                                  & 7 + CHA      & 5              & 4   & 13  & 26 & 9  \\
 14th  & +5 & Spellblade Focus Feature                				& 7 + CHA      & 5              & 4   & 14  & 28 & 10  \\
 15th  & +5 & Arcane Manipulation (d12)                       & 8 + CHA      & 5              & 4   & 15  & 30 & 11  \\   
 16th  & +5 & Ability Score Improvement                       & 8 + CHA      & 5              & 5   & 16  & 32 & 11  \\   
 17th  & +6 & Channeling (8)                                  & 9 + CHA      & 6              & 5   & 17  & 34 & 12 \\   
 18th  & +6 & Spellblade Focus Feature                 				& 9 + CHA      & 6              & 5   & 18  & 36 & 13 \\   
 19th  & +6 & Ability Score Improvement                       & 10 + CHA     & 6              & 6   & 19  & 38 & 13 \\   
 20th  & +6 & Superior Inspiration                            & 10 + CHA     & 6              & 6   & 20  & 40 & 14 \\   
\end{DndTable}
\end{figure*}

\subsection{Spellcasting}

You have learned to untangle and reshape the fabric of reality in harmony with your wishes.

\subsection{Cantrips}

You know two cantrips of your choice from the spellblade spell list. You learn additional spellblade cantrips of your choice at higher levels, as shown in the Cantrips Known column of the Spellblade table.

\subsubsection{Preparing and Casting Spells}

The spellblade table shows how much aether (AET) you have to cast your spells and do other magical tasks. To cast a spell that requires aether, you must expend aether equal to its cost or greater. You regain all expended aether when you finish a long rest. It also shows your Aether Limit, which is the maximum aether you can expend on a single action.

You know a certain number of spellblade spells, choosing from the spellblade spell list. You can trade out any known spell for any other spell you can learn from that list when you finish a long rest. When you do so, choose a number of spellblade spells equal to your Charisma modifier + half your spellblade level, rounded down (minimum of one spell). To prepare a spell you must be able to cast it without exceeding your Aether Limit.

\subsubsection{Spellcasting Ability}

Charisma is your spellcasting ability for your spellblade spells, since their power derives from the strength of your will. You use your Charisma whenever a spell refers to your spellcasting ability. In addition, you use your Charisma modifier when setting the saving throw DC for a spellblade spell you cast and when making an attack roll with one.

\textbf{Spell save DC} = 8 + your proficiency bonus + your Charisma modifier

\textbf{Spell attack modifier} = your proficiency bonus + your Charisma modifier

\subsection{Ritual Casting}

You learn one common incantation (see \nameref{ch:incantations}) of your choice. At 5th level, you learn one Uncommon incantation of your choice. At 11th level you learn one Rare incantation of your choice. You can perform any incantation you know through this feature without a Ritual Scroll in hand.

\subsection{Spellcasting Focus}

You can use a weapon as a spellcasting focus for your spellblade spells.

\subsection{Arcane Manipulation}

You can magically distort the minds of your foes. To do so, you use a bonus action on your turn to choose one creature other than yourself within 60 feet of you who can hear you. That creature must make a Wisdom saving throw against your Spellcasting DC. On a failure, they are marked for ill-luck. Once within the next minute when the creature makes an attack roll, saving throw, or ability check, you can invoke the mark without using an action. If you do so, the creature must roll a d6 (hereafter an Arcane Manipulation die) and subtract it from the die result before applying modifiers. This can convert a critical hit into a regular hit or miss. Any creature can only be marked with one manipulation die at a time.

You can use this feature a number of times equal to your Charisma modifier (a minimum of once). You regain any expended uses when you finish a long rest.

Your Arcane Manipulation die changes when you reach certain levels in this class. The die becomes a d8 at 5th level, a d10 at 10th level, and a d12 at 15th level.

\subsection{Expertise}

At 2nd level, choose one of your skill proficiencies. Your proficiency bonus is doubled for any ability check you make that uses the chosen proficiencie.

At 10th level, you can choose another skill proficiency to gain this benefit.

\subsubsection{Skill Tricks}

Starting at 2nd level, you've learned additional ways to employ your abilities. You learn one \nameref{sec:skill-tricks-basic} of your choice, even if you don't have proficiency in that skill. See \nameref{ch:skill-tricks} for more details and the rules governing skill tricks.

When you gain access to a new Skill Trick, you can also swap any Skill Trick you know for a new one you could otherwise learn at that point.

\subsection{Spellblade Focus}

At 3rd level, you delve into the advanced techniques of a spellblade focus of your choice and choose to focus either on Inspiration or War, both detailed at the end of the class description. Your choice grants you features at 3rd level and again at 6th and 14th level.

\subsection{Ability Score Improvement}

When you reach 4th level, and again at 8th, 12th, 16th, and 19th level, you can increase one ability score of your choice by 1. As normal, you can't increase an ability score above +5 using this feature.

You can also pick a Skill Trick (included in the skill tricks column of the spellblade table) but you must meet the prerequisites for skill tricks learned in this way. See \nameref{ch:skill-tricks} for that list.

\subsection{Renewed Manipulation}

Beginning when you reach 5th level, you regain all of your expended uses of Arcane Manipulation when you finish a short or long rest.

\subsection{Channeling}
Starting at 6th level, you have learned to weave your spells into your weapon use. When you take the Attack action on your turn, you can also cast a cantrip of your choice that targets another creature and infuse your weapon strike with it. If you hit with the attack, the target suffers the effect of both the regular weapon damage and the cantrip. If the cantrip targets two or more creatures, the additional creatures must be within 5 feet of the target of the weapon attack.

Starting at 11th level, you can infuse any spell that costs 3 AET or less by paying the requisite cost. It does not require a separate action. The spell must target a creature or a point in space. If it targets a single creature, the target of your weapon attack is the target of the spell and must make any requisite saving throws, although they do so at disadvantage. If it targets a point in space, the target is centered on the creature struck. The limit on spell cost increases at 13th level (to 5 AET) and 17th level (to 8 AET).

If you miss with an infused attack, the spell fizzles and has no effect.

\subsection{Counterweave}

Starting at 9th level, you gain the ability to use musical notes or words of power to disrupt magical effects. As a reaction when someone within 60 ft of begins casting a spell or magical effect, you can spend 2 AET to attempt to counter it. The target must make a Charisma saving throw against your Spellcasting DC. They gain a +1 bonus for every 2 CR above 9 they are. On a failed save, the spell or magical effect is cancelled and has no effect. You can spend additional AET up to your limit; for every additional AET, the DC increases by 1.

Additionally, you learn \textit{\nameref{spell:unbind}} if you do not already know it and can cast it using 2 AET instead of its normal cost. It does not count against your spells known.

\subsection{Superior Manipulation}

At 20th level, targets marked by your Arcane Manipulation must subtract the die from every attack roll or ability check they make for one minute. Only one saving throw is affected, regardless.

\section{Spellblade Focuses}
\subsection{Focus: Inspiration}
Basically "bard, as a subclass.

\subsubsection{Inspiring Manipulation}
Starting at 3rd level when you pick this focus, you can use your Arcane Manipulation to benefit allies as well. When you target a willing creature with your Arcane Manipulation, the target instead \textit{adds} the die to one ability check or damage roll they make within the next minute. No saving throw is required. The target can use the die after they see the result of the check. If they add it to a damage roll, it is multiplied by critical hits and deals the same damage as the underlying source (the weapon or the spell or ability) deals.

\subsubsection{Beneficial Channel}
Starting at 7th level, when you take the Attack action and choose \textit{not} to apply your Channeling ability, you can instead cast any Spellblade spell with a cost of 2 AET or less that has a cast time of 1 action and targets a willing creature other than yourself as a bonus action instead.

\subsubsection{Shielding Counterweave}
Starting at 14th level, when you use your Counterweave ability against a magical effect that causes damage and the target succeeds on the saving throw, you can choose a number of willing targets equal to your proficiency bonus. Those targets have resistance to the damage dealt by the magical effect.

\subsubsection{Legendary Effect: Heroes' Feast}
At 14th level, you learn the legendary effect \nameref{spell:heroes-feast} and can use it once per day.

\subsubsection{Improved Inspiring Manipulation}
Starting at 18th level, when you use your Inspiring Manipulation ability, the friendly target can add the die to an attack roll or saving throw as well as an ability check or damage roll.

\subsubsection{Legendary Effect: Abjure Arcana}
Starting at 18th level, you learn the legendary effect \nameref{spell:abjure-aether-manipulation} and can use it once per day.

\subsection{Focus: War}
Goes all in on combat.
\subsubsection{Piercing Manipulation}
Starting at 3rd level when you pick this focus, you can use your Arcane Manipulation to pierce the defenses of a foe you attack. When you make an attack roll, you can expend one use of Arcane Manipulation to add the die result to your attack roll. Alternatively, when you cast a spell or use an ability (other than Arcane Manipulation) that requires a saving throw from an enemy, you can expend one use of Arcane Manipulation to subtract the die result from the target's saving throw.

\subsubsection{Steady Channeling}
Starting at 7th level, when you deal damage with your Channeling ability and roll below half the maximum damage on the spell's damage, you can expend 1 STA to instead deal half of the spell's maximum damage.

\subsubsection{Rebounding Counterweave}
Starting at 14th level, when you use your Counterweave ability against a magical effect that causes damage and the target fails the saving throw, the target takes force damage equal to your level in addition to the regular effects of a failed save.

\subsubsection{Legendary Effect: Globe of Invulnerability}
At 14th level, you learn the legendary effect \nameref{spell:globe-of-invulnerability} and can use it once per day.

\subsubsection{Improved Piercing Manipulation}
Starting at 18th level, targets of your Piercing Manipulation take additional force damage equal to your level, regardless of whether the triggering attack hits or the triggering ability takes effect.

\subsubsection{Legendary Effect: Power Word Kill}
At 18th level, you learn the legendary effect \nameref{spell:power-word-kill} and can use it once per day.

\subsection{Spellblade Spell List}
The Spellblade Spell List table contains a short summary of the spells available to all spellblades, ordered by aether cost.
\begin{figure*}[!ht]
\begin{DndTable}[header=Spellblade Spell List]{rlXrl}
	\textbf{Aether Cost} & \textbf{Name} & & \textbf{Aether Cost} & \textbf{Name} \\
    0 & \nameref{spell:acid-burst} & & 3 & \nameref{spell:invisibility}\\
    0 & \nameref{spell:dancing-lights} & & 3 & \nameref{spell:levitate}\\
    0 & \nameref{spell:grave-touch} & & 3 & \nameref{spell:misty-step}\\
    0 & \nameref{spell:light} & & 3 & \nameref{spell:ray-of-enfeeblement} \\
		0 & \nameref{spell:mage-hand} & & 3 & \nameref{spell:scorching-ray}\\
    0 & \nameref{spell:prestidigitation} & & 3 & \nameref{spell:shatter}\\
    0 & \nameref{spell:produce-flame} & & 4 & \nameref{spell:hold-person}\\
    0 & \nameref{spell:ray-of-frost} & & 4 & \nameref{spell:vampiric-touch}\\
    0 & \nameref{spell:resistance} & & 5 & \nameref{spell:bestow-curse}\\
    0 & \nameref{spell:shocking-grasp} & & 5 & \nameref{spell:blindness-deafness}\\
    1 & \nameref{spell:true-strike} & & 5 & \nameref{spell:blink}\\
    2 & \nameref{spell:bane} & & 5 & \nameref{spell:fear}\\
    2 & \nameref{spell:burning-hands} & & 5 & \nameref{spell:fireball}\\
    2 & \nameref{spell:color-spray} & & 5 & \nameref{spell:hypnotic-pattern}\\
    2 & \nameref{spell:disguise-self} & & 5 & \nameref{spell:slow}\\
    2 & \nameref{spell:expeditious-retreat} & & 5 & \nameref{spell:stinking-cloud}\\
    2 & \nameref{spell:faerie-fire} & & 6 & \nameref{spell:blight}\\
    2 & \nameref{spell:false-life} & & 8 & \nameref{spell:arcane-eye}\\
    2 & \nameref{spell:flash-freeze} & & 8 & \nameref{spell:banishment}\\
    2 & \nameref{spell:fog-cloud} & & 8 & \nameref{spell:black-tentacles}\\
    2 & \nameref{spell:inflict-wounds} & & 8 & \nameref{spell:confusion}\\
    2 & \nameref{spell:guiding-bolt} & & 8 & \nameref{spell:dimension-door}\\
    2 & \nameref{spell:hideous-laughter} & & 8 & \nameref{spell:dominate-beast}\\
    2 & \nameref{spell:shield} & & 8 & \nameref{spell:phantasmal-killer}\\
    2 & \nameref{spell:sleep} & & 10 & \nameref{spell:antilife-shell}\\
    2 & \nameref{spell:thunderwave} & & 10 & \nameref{spell:cone-of-cold}\\
    3 & \nameref{spell:acid-arrow} & & 12 & \nameref{spell:contagion}\\
    3 & \nameref{spell:blur} & & 12 & \nameref{spell:hold-monster}\\
    3 & \nameref{spell:darkness} & & 12 & \nameref{spell:mislead}\\
    3 & \nameref{spell:enhance-ability} & & 12 & \nameref{spell:telekinesis}\\
    3 & \nameref{spell:enlarge-reduce} & & & \\
    3 & \nameref{spell:flaming-sphere} & & &\\
    3 & \nameref{spell:heat-metal} & & & 
\end{DndTable}
\end{figure*}