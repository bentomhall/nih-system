\chapter{Skill Tricks}
\label{ch:skill-tricks}

Those who are particularly adept at certain aspects of adventuring often learn ways to use their talents to perform tricks that seem supernatural or magical to outside observers. While they are not magical in the same sense as spells or invocations, per se, they do produce effects not normally possible.

Each skill trick detailed below shares some common characteristics:
\begin{itemize}
	\item \textbf{A cost}. Most skill tricks cost something, whether expending a replaceable tool, damaging a weapon or armor, or (most commonly) expending Stamina or Aether or both.
	\item \textbf{An ability score}. Every skill trick is tied to a particular ability score. That ability score sets its DC.
	\item \textbf{A prerequisite}. Every skill trick has one or more prerequisites before it can be learned. These are generally either a particular level of proficiency (numerical value, which does not includes expertise) for those that are tied to a particular skill or tool, or a character level for those marked as General.
	\item \textbf{A target or targets}. Many skill tricks target either an object or one or more creatures. A few target a particular area.
	\item \textbf{An effect}. The text of the skill trick describes the effect, as well as any saving throws required.
\end{itemize}

\subsection{Skill Trick DCs}
The DC for any saving throws required by skill tricks is given by

\begin{center}
\textbf{8 + the relevant ability score + your proficiency}
\end{center}

regardless of whether the trick involves a proficiency or not. If you have expertise in the relevant skill or tool, targets have disadvantage on the saving throw.

\subsection{Acquiring Skill Tricks}
\label{subsec:acquiring-skill-tricks}

Some classes get native access to Skill Tricks as a class feature. If they grant access to more advanced skill tricks beyond the basic ones at particular levels, that access overrides any prerequisites in the skill trick. Everyone else can choose a skill trick that they qualify for whenever they acquire an Ability Score Improvement from their class. At the same time, they can trade out one skill trick they've learned for a different one they qualify for.

\section{Basic Skill Tricks}
\label{sec:skill-tricks-basic}

Basic skill tricks only require a +2 proficiency or level 4 characters.

\subsection{Alert}
\textit{Wisdom(Perception) Basic Skill Trick}
You have advantage on Wisdom (Perception) checks against being surprised.

\subsection{Arcane Initiate}
\textit{Intelligence Basic Skill Trick}
You learn one cantrip of your choice from the Arcanist list, as well as one spell costing no more than 2 AET from that same list. Intelligence is your casting ability for these spells. You can pick this skill trick more than once. Each time you do, pick a different cantrip and spell.

\subsection{Divine Initiate}
\textit{Wisdom{Religion} Basic Skill Trick}
You learn one cantrip of your choice from the Priest list, as well as one spell costing no more than 2 AET from that same list. Wisdom is your casting ability for these spells. You can pick this skill trick more than once. Each time you do, pick a different cantrip and spell.

\subsection{Feint}
\textit{Charisma(Deception) Basic Skill Trick}
You fake an attack as a bonus action, trying to misdirect the enemy. Expend 1 STA. The opponent must make a Wisdom saving throw. On a failed save, you have advantage on your next attack roll against them.

\subsection{Frighten} \label{st:frighten}
\textit{Charisma(Intimidation) Basic Skill Trick}
As an action, you can expend 1 STA to threaten one enemy that can hear you. The target must make a Charisma saving throw. On a failed save, they are frightened of you until the end of your next turn.

\subsection{Jump}
\textit{Strength(Athletics) Basic Skill Trick}
You always count as having a running start when jumping. Additionally, you can fall an additional 10 ft. before taking fall damage. Start counting fall damage from 20 ft = 1d6 instead of 10 ft = 1d6.

\subsection{Linguist}
\textit{Intelligence Skill Trick}
When you listen to conversation in a language you don't speak for at least 10 minutes, you can pick up the rudiments. Enough to be understood, but not enough to convey subtle details.

Additionally, you can make out the basic sense of any text written in a script for which you are fluent in at least one language. This does not help you decipher intentionally obfuscated or encoded messages.

\subsection{Lung Capacity}
\textit{Constitution Skill Trick}
You can hold your breath for twice as long. In addition, you can spend 1 STA when you are exposed to a source of poison gas (such as \nameref{spell:cloudkill} or a dretch's Stench ability) that requires a Constitution saving throw to gain advantage on the Constitution saving throw. 

\subsection{Medic}
\textit{Wisdom(Medicine) Basic Skill Trick}
When you make a Wisdom (Medicine) check to stabilize someone at 0 HP and succeed, the target regains 1 hit point and is conscious instead.

\subsection{Misdirect}
\textit{Dexterity(Stealth) Basic Skill Trick}
When you are hidden, you can spend 1 STA to force a number of creatures equal to your proficiency bonus to make a Wisdom saving throw. On a failure, they do not notice you even if you move out of heavy obscurement, as long as you end your turn behind heavy obscurement.

\subsection{Primal Initiate}
\textit{Wisdom Basic Skill Trick}
You learn one cantrip of your choice from the Shaman list, as well as one spell costing no more than 2 AET. Wisdom is your casting ability for these spells. You can pick this skill trick more than once. Each time you do, pick a different cantrip and spell.

\subsection{Scholar: Religion}
\textit{Intelligence(Religion) Basic Skill Trick}
You automatically recognize holy symbols of currently-active ascendants and know at least the basic tenants of that religion. Additionally, when you make an Intelligence (Religion) check to know information about dead or obscure religions or their worshippers, you have advantage on the check.

\subsection{Sense Baleful Magic}
\textit{Intelligence(Arcana) Basic Skill Trick}
You are sensitive to the presence of hostile magics in your proximity. When you are within 30 ft. of a magical trap, spell glyph, or other hostile magical environment, you can use your passive Intelligence (Arcana) instead of your passive Wisdom (Perception) to determine their location and nature.

Additionally, you have advantage on checks made to determine the spell being cast.

\subsection{Shield Bash}
\textit{Strength(Shield) Basic Skill Trick}
You lash out with your shield. Expend 1 STA and make an attack with a proficient melee weapon. On a hit, the opponent takes 1d4 bludgeoning damage and is \nameref{condition:staggered} until the end of their next turn. If you score a critical hit, the target is \nameref{condition:staggered} until the end of your next turn. This can replace an attack when you take the Attack action.

\subsection{Soothe Domesticated Animal}
\textit{Animal Handling Basic Skill Trick}
You can make a Wisdom (Animal Handling) check against a DC of 10 to alter the disposition of a domesticated animal to friendly toward you or prevent a domesticated animal from panicking. Trained guard animals have a DC of 15 if they were hostile toward you. This effect lasts for one hour unless you or your allies attack the animals or their friends.

\subsection{Tumble}
\textit{Dexterity (Acrobatics) Basic Skill Trick}
You can move through opponents spaces if they are only one size larger than you, not two by expending 1 STA. They count as difficult terrain and you cannot willingly end your movement in their space.

\section{Advanced Skill Tricks}
\label{sec:skill-tricks-advanced}

Advanced skill tricks require a +4 proficiency or level 9 characters.

\subsection{Arcane Journeyman}
\textit{Intelligence Basic Skill Trick}
You learn one cantrip of your choice from the Arcanist list, as well as one spell costing no more than 3 AET from the same list. Intelligence is your casting ability for these spells. You can pick this skill trick more than once. Each time you do, pick a different cantrip and spell.

\subsection{Athlete}
\textit{Strength(Athletics) Advanced Skill Trick}
You can climb at full speed without making checks even on surfaces with few handholds or slick surfaces. You can expend 1 STA to climb even magically slick surfaces without needing hands or a check; if you are still on this surface at the start of your next turn, you must expend additional STA or use your hands.

Additionally, you gain a swimming speed equal to your normal speed and no longer need to make checks to swim even in very rough or fast waters.

Additionally, the distance you can jump doubles and the height at which you start taking fall damage increases to 30 ft (taking 1d6 for the first 30 ft you fall and 1d6 for every 10 ft above that).

\subsection{Befriend Wild Animal}
\textit{Wisdom(Animal Handling) Advanced Skill Trick}
As an action, you can attempt to soothe an angry creature that does not speak any language or befriend a wary one. The creature must make a Charisma saving throw, at advantage if it is actively hostile to you. On a failure, the creature becomes friendly. Originally non-hostile creatures may follow you and protect you as long as you feed them and do not harm them, although they are still wild animals and they are not under your control.

\subsection{Bond Breaker}
\textit{Strength Advanced Skill Trick}
You can spend 2 STA to break any non-magical shackles or bonds without a check. If the shackles are magical, you gain +10 on the Strength check to break free.

\subsection{Delay Unconsciousness}
\textit{Constitution Advanced Skill Trick}
As a reaction when you are brought to zero hit points, you can expend 3 STA and gain a level of exhaustion. If you do, you do not gain the \nameref{condition:unconscious} condition and can act normally. You still make death saving throws as normal, including when you take damage. If you are still at 0 HP at the end of your next turn, you go unconscious at that point.

\subsection{Demoralize}\label{st:demoralize}
\textit{Charisma(Intimidation) Advanced Skill Trick}
As an action, you expend 3 STA. You can either threaten a single enemy that can hear and see you or a group. If you threaten a single enemy, they must make a Wisdom saving throw. On a failed save, they suffer the consequences of failing a morale check and are \nameref{condition:broken}. If you threaten a group of creature, they all are affected as if you used the \nameref{st:frighten} skill trick on them.

\subsection{Divine Journeyman}
\textit{Wisdom Basic Skill Trick}
You learn one cantrip of your choice from the Priest list, as well as one spell costing no more than 3 AET from the same list. Wisdom is your casting ability for these spells. You can pick this skill trick more than once. Each time you do, pick a different cantrip and spell.

\subsection{Fascinate}
\textit{Performance Advanced Skill Trick}
As an action, you begin a distracting performance. Expend 1 STA. Any number of creatures of your choice within 60 ft of you that can hear and see you must make a Wisdom saving throw. On a failure, they can't focus on anything but you and are effectively blinded and deafened to all other occurrences. Taking any damage breaks the effect, as does being shaken awake by someone else as an action. This effect lasts until you stop performing (using your action each round to maintain the distraction).

\subsection{Find Weakness}
\textit{Intelligence(Investigation) Advanced Skill Trick}
As a bonus action, you can search for flaws in your opponent. Expend 2 STA and make an Intelligence (Investigation) check against a DC of 10 + half the target's CR. On a success, you learn three of the following of your choice.
\begin{itemize}
	\item Their highest and lowest saving throw modifiers
	\item Any resistances or immunities they have.
	\item Any vulnerabilities they have (whether ot damage particularly or things like Sunlight Sensitivity)
	\item Their current goals
\end{itemize}

Alternatively on a success, you can temporarily remove any one damage resistance you know about by informing your allies how to bypass it.

\subsection{Like a Solid Snake}
\textit{Dexterity(Stealth) Advanced Skill Trick}
You can attempt to hide even if only lightly obscured. Additionally, missing with an attack does not remove the hidden or invisible status.

\subsection{Mental Toughness}
\textit{Charisma Advanced Skill Trick}
When you are afflicted by the \nameref{condition:charmed}, \nameref{condition:frightened}, or \nameref{condition:incapacitated} conditions at the beginning of your turn due to an effect that caused a Wisdom saving throw, you can spend 2 STA to ignore the effects of those conditions until the end of your turn.

\subsection{People Whisperer}
\textit{Wisdom(Insight) Advanced Skill Trick}
When you make a Wisdom (Insight) check and the result is above a 15, you gain one pertinent, specific detail about the target's mental or emotional state for every 5 higher you rolled (ie 1 at 15, 2 at 20, etc.).

\subsection{Primal Journeyman}
\textit{Wisdom Basic Skill Trick}
You learn one cantrip of your choice from the Shaman list, as well as one spell costing no more than 3 AET from the same list. Wisdom is your casting ability for these spells. You can pick this skill trick more than once. Each time you do, pick a different cantrip and spell.

\subsection{Pocket Sand}
\textit{Dexterity(Sleight of Hand) Advanced Skill Trick}
As a bonus action, you can expend 1 STA and attempt to throw sand or dust into an opponent's eyes. The target must make a Dexterity saving throw. On a failed save, they are blinded until the end of your next turn. This does not work on targets that have non-standard vision (ie don't use eyes to see).

\subsection{Resuccitation}
\textit{Wisdom(Medicine) Advanced Skill Trick}
As an action, you can attempt to resuccitate someone who died within the last minute. Make a Wisdom (Medicine) check against a DC of 15 + the number of rounds since they died. On a success, the creature is restored to 1 HP and any mortal wounds are closed, but gains a permanent injury. Both you and the target gain one level of \nameref{condition:exhaustion}. 

\subsection{Snow Job}
\textit{Charisma(Deception) Advanced Skill Trick}
When you make a Charisma (Deception) check to convince someone you know something you don't or are someone you are not, you do so at advantage. In addition, if you succeed by more than 5, the target willingly tells you the missing information.

\subsection{Sunder}
\textit{Strength(Carpentry, Mason's, or Blacksmith's Tools) Advanced Skill Trick}
When you make an attack against an unattended object and hit, you ignore its Damage Threshold and deal double damage. 

Alternatively, you can target attended objects as follows, expending 2 STA:
\begin{itemize}
	\item \textbf{Armor}: Make an attack against the target's AC. On a hit, the target takes half damage from the attack but any other attacks against the target have advantage until the target uses an action to realign the damaged piece.
	\item \textbf{Weapons}: Make an attack against the target's AC. On a hit, the target takes half damage from the attack and has disadvantage on all attacks made with that weapon.
	\item \textbf{Wielded Spell Foci or other objects in hand}: Make an attack at disadvantage against the target's AC. On a hit, the focus is knocked from their grasp and lands 1d6 \texttimes 5 ft away in a random direction.
\end{itemize}

\subsection{Wrestler}
\textit{Strength(Athletics) Advanced Skill Trick}
You can grapple and shove creatures two sizes larger than yourself. If you expend 1 STA, you can remove the size limit entirely.

Additionally, when you start your turn with a creature grappled, you can expend 1 STA and attempt a second grapple check. If you succeed, the target is restrained until the grapple ends.

\section{Expert Skill Tricks}
\label{sec:skill-tricks-expert}

Expert skill tricks require a +5 proficiency or level 13 characters.

\subsection{Blindfighter}
\textit{Wisdom(Perception) Expert Skill Trick}
Invisible or unseen enemies no longer have advantage to hit you. In addition, you do not have disadvantage to hit invisible or unseen enemies and gain blindsight out to 10 ft. If you already have blindsight or gain it later, this stacks.

\subsection{Break Will}
\textit{Charisma(Intimidation) Expert Skill Trick}
This skill trick acts like \nameref{st:demoralize} except that you can force any number of creatures that can see and hear you to make a Wisdom saving throw, becoming \nameref{condition:broken} on a failure and \nameref{condition:frightened} of you on a success. The frightened state lasts for 1 minute.

\subsection{Comprehend Dweomer}
\textit{Intelligence(Arcana) Expert Skill Trick}
You gain the following benefits:
\begin{itemize}
	\item You can determine the nature of any arcane phenomena you encounter, including spell glyphs, illusions, etc.
	\item You can detect spellcasting within 60 ft. of you even if there are no components.
	\item You can expend 3 AET as a reaction to attempt to disrupt spellcasting by a creature you can see within 60 ft. of you. Make an Intelligence (Arcana) check against a DC of 10 + the spell level. On a success, the spell fails. If you have the Spellcasting feature, you can substitute your spellcasting ability for Intelligence.
\end{itemize}

\subsection{Dungeoncrasher}
\textit{Strength(Athletics) Expert Skill Trick}
When you move at least 10 feet before attempting to shove a creature, you can shove any size of creature. In addition, the distance you can shove creatures increases by 5 ft. for every STA you expend when making the check. If the creature is stopped short of the full distance by a hard surface, they must make a Constitution saving throw. On a failure, they are stunned until the end of their next turn. If they are stopped by running into a creature of their size or smaller, the other creature is knocked prone unless they succeed on a Strength saving throw.

\subsection{Find Portal}
\textit{Wisdom(Survival) Expert Skill Trick}
You search for signs of an accessible planar portal within 1 mile of you. Make a Wisdom (Survival) check and expend 2 AET, with advantage if you also are proficient in Arcana. The result determines your success:

\begin{DndTable}[header=Find Portal Results]{XX}
	\textbf{Check Total}  & \textbf{Result} \\
	< 10 & You find a portal to a plane of the DM's choosing. The location on that plane that it leads to is dangerous. \\
	10-14 & You find a portal to a plane of the DM's choosing. The location on that plane that it leads to is not inherently dangerous. \\
	15-19 & You find a portal to a plane of your choosing. The location on that plane that it leads to is chosen by the DM, but is not inherently dangerous. \\
	20+ & You find a portal to a plane of your choosing. You can choose the approximate location that it leads to.
\end{DndTable}

\subsection{Like a Ghost}
\textit{Dexterity(Stealth) Expert Skill Trick}
When you are hidden, you no longer need to be concealed to remain hidden, but you must expend 1 STA for every turn you spend hidden without concealment. Missing with an attack no longer breaks stealth, although casting a spell with verbal components or hitting with an attack does reveal your position.

\subsection{Slippery}
\textit{Dexterity(Acrobatics) Expert Skill Trick}
You can no longer be grappled or restrained by non-legendary effects as long as you spend 1 STA when the effect is applied.

\section{Master Skill Tricks}
\label{sec:skill-tricks-master}

Master skill tricks require a +6 proficiency or level 17 characters.
\subsection{Balance on Thin Air}
\textit{Dexterity(Acrobatics) Master Skill Trick}
You can expend 1 AET per turn to walk on air as if it was solid ground.

\subsection{Dragon's Fear}
\textit{Charisma(Intimidation) Master Skill Trick}
This skill trick works as Break Will, but with the addition that creatures whose CR is less than half your level automatically fail the saving throw and creatures higher than that have disadvantage on the saving throw.

\subsection{Friend to All}
\textit{Charisma(Persuasion) Master Skill Trick}
As an action, you can magically call for a truce even during combat, expending 3 AET. All creatures that can understand your language must make a Wisdom saving throw. On a failure, they become non-hostile until someone takes a hostile action. This ability does not work on mindless creatures, including zombies and skeletons.

Alternatively, if you are not actively fighting someone, you can force them to make a Wisdom saving throw. On a failure, their disposition to you increases by one step for 10 minutes or until you take hostile action against them. Once the effect ends, they will still follow through on any bargains struck during that time, but their disposition toward you may change.

\subsection{Force Portal}
\textit{Wisdom(Survival) Master Skill Trick}
As an action and expending 4 AET, you force a micro-fissure in the fabric of the planes to form a full-fledged portal that lasts for 6 seconds. You can choose what plane it exits to, but only a rough description of the location on that plane.

\subsection{Healing Hands}
\textit{Wisdom(Medicine) Master Skill Trick}
As an action you can expend 4 AET to magically do one of the following to a creature you touch:
\begin{itemize}
	\item Heal the creature to half of its maximum hit points.
	\item Remove any condition from a living creature.
	\item Cure any disease and remove any poison or curse affecting the creature.
	\item Break one spell affecting the creature.
	\item Restore a dead body to life as long as it has been dead less than 1 hour. Taking this benefit causes both you and the target 3 levels of exhaustion.
\end{itemize}