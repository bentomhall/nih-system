\chapter{Introduction}
\section{Core Principles of NIH}

\begin{DndSidebar}[float=b]{What is Magic?}
    That word is used throughout these documents, and deserves a little more reflection. "Magic", as used here, is all those things that separate the fictional world in which the game takes place from the real world in which the players act. Spells? Magic. Dragons? Magic. Heroic mortals breaking "normal" constraints? Magic. \textbf{It's magic all the way down.} Unlike Dungeons and Dragons, I attempt to be more precise in my use of this term. So abilities that counter \textit{specific forms} of magic will be written precisely. Thus, \textit{dispel magic} (the Dungeons and Dragons spell) becomes \nameref{spell:unbind}, an spell that specifically focuses on undoing active aetheric effects (a subset of all magic including spell effects, magic items, and sustained wards of various types).
\end{DndSidebar}

\subsection{Worldbuilding}
The core assumptions are that:
\begin{enumerate}
\item \textbf{The world is thoroughly magical.} Magic is in and through everything and everyone. There is no mundane (in the sense of "bound by all the principles of real world"), at least as far as adventurers and adventuring goes.
\item \textbf{Adventuring is normalized.} Adventurers are a known social "group", even if a disreputable one.
\item \textbf{The world responds to mortal efforts.} In these sorts of worlds, training really hard physically \textit{is} a form of access to magical power just as much as learning magic gestures and words in a book. But this plateaus.
\item \textbf{The world has its own logic, not real-world science.} Players should be able to rely on \textit{surface} similarity between the real world and the fictional one. Things will fall when dropped. Water flows downhill. Paper burns and fire hurts. Fire turns water to steam. If it looks like a cow, walks like a cow, and moos like a cow, it's probably a cow. But the \textit{reasons} why these happen are not at all guaranteed to be the same. The further and deeper you get into modern scientific understanding of the world, the less this will apply to the fictional world. A fire spell will burn flammable things...but not because of exothermic oxidation reactions. There might not even be oxygen, and fires might burn just fine even in a "vacuum"...if such a thing even exists. Atoms, molecules, cells, DNA--these sorts of things are not at all guaranteed to exist.
\item \textbf{PCs are among the few that break the normal (soft) limits.} Most people in the world are relatively low power, and will never grow much further. PCs are not bounded in the same way.
\item \textbf{World-ending threats are rare, but problems are common.} Most threats a party will face don't have whole-world-changing consequences. But they do change the local world.
\item \textbf{PCs change the world...but not by pushing buttons.} It's expected that the outcome of the PCs adventure will be changes to the status quo. The world \textit{should} react to their actions. But the PCs don't have powers that allow them to directly do that on the large scale. Large-scale changes happen because of the relationships the PCs form with others, the movements they support, and the people they affect.
\item \textbf{PCs are usually the underdogs.} Either due to numbers (the enemy has an army) or due to individual power. PCs generally win against significant foes not by overwhelming the opponents with bigger numbers but by teamwork, guile, good strategy, finding the opponents' weak spots, building alliances with others, etc. 
\end{enumerate}