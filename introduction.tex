\chapter{Introduction}
\section{Core Principles of NIH}

\begin{DndSidebar}[float=b]{What is Magic?}
    That word is used throughout these documents, and deserves a little more reflection. "Magic", as used here, is all those things that separate the fictional world in which the game takes place from the real world in which the players act. Spells? Magic. Dragons? Magic. Heroic mortals breaking "normal" constraints? Magic. \textbf{It's magic all the way down.} Unlike Dungeons and Dragons, I attempt to be more precise in my use of this term. So abilities that counter \textit{specific forms} of magic will be written precisely. Thus, \textit{dispel magic} (the Dungeons and Dragons spell) becomes \nameref{spell:unbind}, an spell that specifically focuses on undoing active aetheric effects (a subset of all magic including spell effects, magic items, and sustained wards of various types).
\end{DndSidebar}

\subsection{Core Assumptions}
The core assumptions are that:
\begin{enumerate}
\item \textbf{The world is thoroughly magical.} Magic is in and through everything and everyone. There is no mundane (in the sense of "bound by all the principles of real world"), at least as far as adventurers and adventuring goes.
\item \textbf{Adventuring is normalized.} Adventurers are a known social "group", even if a disreputable one.
\item \textbf{The world responds to mortal efforts.} In these sorts of worlds, training really hard physically \textit{is} a form of access to magical power just as much as learning magic gestures and words in a book. But this plateaus.
\item \textbf{The world has its own logic, not real-world science.} Players should be able to rely on \textit{surface} similarity between the real world and the fictional one. Things will fall when dropped. Water flows downhill. Paper burns and fire hurts. Fire turns water to steam. If it looks like a cow, walks like a cow, and moos like a cow, it's probably a cow. But the \textit{reasons} why these happen are not at all guaranteed to be the same. The further and deeper you get into modern scientific understanding of the world, the less this will apply to the fictional world. A fire spell will burn flammable things...but not because of exothermic oxidation reactions. There might not even be oxygen, and fires might burn just fine even in a "vacuum"...if such a thing even exists. Atoms, molecules, cells, DNA--these sorts of things are not at all guaranteed to exist.
\item \textbf{PCs are among the few that break the normal (soft) limits.} Most people in the world are relatively low power, and will never grow much further. PCs are not bounded in the same way.
\item \textbf{World-ending threats are rare, but problems are common.} Most threats a party will face don't have whole-world-changing consequences. But they do change the local world.
\item \textbf{PCs change the world...but not by pushing buttons.} It's expected that the outcome of the PCs adventure will be changes to the status quo. The world \textit{should} react to their actions. But the PCs don't have powers that allow them to directly do that on the large scale. Large-scale changes happen because of the relationships the PCs form with others, the movements they support, and the people they affect.
\item \textbf{PCs are usually the underdogs.} Either due to numbers (the enemy has an army) or due to individual power. PCs generally win against significant foes not by overwhelming the opponents with bigger numbers but by teamwork, guile, good strategy, finding the opponents' weak spots, building alliances with others, etc. 
\end{enumerate}

\subsection{How to Play}
NIH requires one player to assume the role of Game Master (hereafter GM), while the remaining players (usually 2-4 in number) create and control individual characters, called Player Characters (PCs) or "the party". It's assumed that the PCs will work together cooperatively--the basic unit of the game is the party, not the individual. The GM's role is several-fold:
\begin{enumerate}
    \item He voices and decides the actions for all the non-Player Character characters (NPCs, for short). If the party is fighting someone, the GM makes decisions for that antagonist. This doesn't mean the GM is antagonistic toward the \textit{players} or is trying to kill the PCs, but the characters he or she controls most certainly may be antagonists.
    \item The GM is the voice of the setting and the narrative. Nothing happens in-game until he narrates it, and he is the eyes through which the players experience the world and the story they are collectively creating. Which makes it imperative that the GM is not biased toward or against any of the players and should, when speaking in his "voice of the world"'s role, never lie to the players. NPCs may lie and try to deceive, the GM as the GM should not. Of course, when illusions and compulsions are in play, what a PC sees or experiences may not be real. But what is described should be what the PC experiences.
    \item The GM is the rules engine for the game. These rules are inputs and guidelines, not mandates. It's the GM's role to decide how, when, and even if the rules apply and to adjust on the fly. Many GMs rely on group consensus for rule modifications and rulings, but at the end of the day, the GM is the final decider. If a player says that his PC acts a certain way, the GM is responsible for deciding how to execute that action and what success or failure looks like, as well as narrating the result. 
    \item Often, the GM is responsible for coming up with the world and/or the set of events surrounding the PCs. Unlike a video game, there may not be a "main quest" that the PCs must follow, but the GM is the one placing things in the world for them to find and interact with.
\end{enumerate}

\subsubsection{The Basic Game Loop}
The most basic, most generic pattern of play is as follows:
\begin{itemize}
    \item The GM describes a scenario, including what the PCs see, hear, smell, etc.
    \item He or she then asks either a single player or the group "What do you do"? (or something similar).
    \item That player or group of players then describes what action or actions they want their PC to attempt. This description may be vague ("I attack the giant with my sword!") or specific ("Gerrold lunges forward, slashing upward at the giant's leg with his sword"), but must always provide
    \begin{enumerate}
        \item An indication of what the player wants to have happen (a goal)
        \item An indication of how the player character is achieving that goal (a method)
        \item And any pertinent facts, such as abilities being used, amounts of bribes being offered, etc. that may change the resolution of the action.
    \end{enumerate}
    \item The GM, often after discussing details with the player, decides how the action will be resolved. This often involves some sort of die roll for randomization purposes. These rules are full of resolution mechanics such as attack rolls, ability checks, saving throws, etc. Many actions don't need any explicit resolution method--they just succeed. It is not expected that you have to roll to tie your shoes in the morning. Actions that have little opposition (chance of failure), are strongly in-character (the sailor can climb masts), or where failure doesn't have meaningful consequences that change the situation (picking a practice lock in the safety of your home) rarely, if ever, call for active resolution.
    \item Once the action is resolved, the GM narrates the changes to the situation and the loop continues.
\end{itemize}

Many times, multiple actions can be resolved simultaneously and the GM may ask multiple players for their actions and decide how to order their resolutions. The first person to speak doesn't necessarily go first--that depends on the entire situation.

At times when exact sequencing is important (such as combat), the GM may call for Initiative. When in initiative order, players take actions from highest initiative downward. In the game world, they're all acting within the same short period of time--one complete pass through the initiative order, called a \textbf{round} represents about 6 seconds of game time, but they are sequenced for ease of play.
