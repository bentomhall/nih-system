\section{Arcanist}

Design goals: The arcanist is the basic full-caster arcane magic user, replacing the sorcerer and wizard. Their UCT is Metamagic, which will be expanded to do a whole lot more. SDC: 1/6/3

Subclasses:
\begin{itemize}
	\item Awakened: This is the self-taught "savant", whose powers are more instinctual than trained. They'll get more aether and some CHA-based abilities.
	\item Book Mage: This is the closest to your classic "wizard". They'll get the ability to write in spells to have them always known and not counting against their limits, plus Ritual Caster.
	\item Warmage: This tradition focuses on attacking. Gets motes when using metamagic, which do various things. Best at amplifying damage of spells. Gets light armor and some weapons.
\end{itemize}

\subsection{Class Features}

As a arcanist, you gain the following class features.

\subsection{Hit Points}

\textbf{Hit Dice:} 1d6 per arcanist level

\textbf{Hit Points at 1st Level:} 6 + your Constitution modifier

\textbf{Hit Points at Higher Levels:} 1d6 (or 4) + your Constitution modifier per arcanist level after 1st

\subsection{Proficiencies}

\textbf{Armor:} None

\textbf{Weapons:} Daggers, darts, slings, quarterstaffs, light crossbows

\textbf{Tools:} None

\textbf{Saving Throws:} Constitution, Intelligence

\textbf{Skills:} Choose two from Arcana, Deception, Insight, Intimidation, Persuasion, and Religion

\subsection{Equipment}

You start with the following equipment, in addition to the equipment granted by your background:
\begin{itemize}
\item (\textit{a}) a light crossbow and 20 bolts or (\textit{b}) any simple weapon
\item (\textit{a}) a component pouch or (\textit{b}) an arcane focus
\item (\textit{a}) a dungeoneer's pack or (\textit{b}) an explorer's pack
\item Two daggers
\end{itemize}

\begin{DndTable}[header=The Arcanist\label{tbl:arcanist}]{XXXXXXXX}
 Level & Proficiency Bonus & Features                       & Cantrips Known & Spells Known & Stamina & Aether & Aether Limit \\
 1st   & +2                & Spellcasting, Arcane Tradition & 4              & 2            & 1   & 6   & 2 \\
 2nd   & +2                & Font of Magic                  & 4              & 3            & 1   & 8   & 2 \\
 3rd   & +2                & Metamagic                      & 4              & 4            & 2   & 14   & 3 \\
 4th   & +2                & Ability Score Improvement      & 5              & 5            & 2   & 18   & 3 \\
 5th   & +3                & Advanced Metamagic             & 5              & 6            & 3   & 22   & 5 \\
 6th   & +3                & Arcane Tradition Feature       & 5              & 7            & 3   & 27   & 5 \\
 7th   & +3                & -                              & 5              & 8            & 4   & 35   & 8 \\
 8th   & +3                & Ability Score Improvement      & 5              & 9            & 4   & 40   & 8 \\
 9th   & +4                & Superior Metamagic             & 5              & 10           & 5   & 52   & 12 \\
 10th  & +4                &                                & 6              & 11           & 5   & 60   & 12 \\
 11th  & +4                & Arcane Secrets (1)             & 6              & 12           & 6   & 70   & 14 \\
 12th  & +4                & Ability Score Improvement      & 6              & 12           & 6   & 70   & 14 \\
 13th  & +5                & Arcane Secrets (2)             & 6              & 13           & 7   & 80   & 16 \\
 14th  & +5                & Arcane Tradition Feature       & 6              & 13           & 7   & 80   & 16 \\
 15th  & +5                & Arcane Secrets (3)             & 6              & 14           & 8   & 90   & 18 \\
 16th  & +5                & Ability Score Improvement      & 6              & 14           & 8   & 90   & 18 \\
 17th  & +6                & Supreme Arcane Secrets         & 6              & 15           & 9   & 100   & 20 \\
 18th  & +6                & Arcane Tradition Feature       & 6              & 15           & 9   & 100   & 20 \\
 19th  & +6                & Ability Score Improvement      & 6              & 15           & 10   & 110   & 22 \\
 20th  & +6                & Sorcerous Restoration          & 6              & 15           & 10   & 110   & 22 \\
\end{DndTable}

\subsection{Spellcasting}

An event in your past, or in the life of a parent or ancestor, left an indelible mark on you, infusing you with arcane magic. This font of magic, whatever its origin, fuels your spells.

\subsection{Cantrips}

At 1st level, you know four cantrips of your choice from the arcanist spell list. You learn additional arcanist cantrips of your choice at higher levels, as shown in the Cantrips Known column of the Arcanist table.

\subsubsection{Preparing and Casting Spells}

\nameref{tbl:arcanist} table shows how much aether (AET) you have to cast your spells and do other magical tasks. To cast a spell that requires aether, you must expend aether equal to its cost or greater. You regain all expended aether when you finish a long rest. It also shows your Aether Limit, which is the maximum aether you can expend on a single action.

You know a certain number of arcanist spells, choosing from the arcanist spell list. You can trade out any known spell for any other spell you can learn from that list when you finish a long rest. When you do so, choose a number of arcanist spells from your list as shown on the \nameref{tbl:arcanist} table. To prepare a spell you must be able to cast it without exceeding your Aether Limit.

\subsection{Spellcasting Ability}

Intelligence is your spellcasting ability for your arcanist spells, since the power of your magic relies on your ability to understand and recall the complex patterns of arcane magic. You use your Intelligence whenever a spell refers to your spellcasting ability. In addition, you use your Intelligence modifier when setting the saving throw DC for a arcanist spell you cast and when making an attack roll with one.

\textbf{Spell save DC} = 8 + your proficiency bonus + your Intelligence modifier

\textbf{Spell attack modifier} = your proficiency bonus + your Intelligence modifier

\subsection{Spellcasting Focus}

You can use an arcane focus as a spellcasting focus for your arcanist spells.

\subsection{Arcane Tradition}

Choose a arcane tradition, which describes the source of your magical training: Awakened Mage, Book Mage, or Warmage, all detailed at the end of the class description.

Your choice grants you features when you choose it at 1st level and again at 6th, 14th, and 18th level.

\subsection{Font of Magic}

At 2nd level, you tap into a deep wellspring of magic within yourself. When you finish a short rest, you can recover aether equal to your arcanist level, rounded up. Once you use this feature, you can't use it again until you finish a long rest.

\subsection{Metamagic}

At 3rd level, you gain the ability to twist your spells to suit your needs. You gain two of the following Metamagic options of your choice. You gain another one at 10th and 17th level.

You can use only one Metamagic option on a spell when you cast it, unless otherwise noted.

\subsection{Careful Spell}

When you cast a spell that forces other creatures to make a saving throw, you can protect some of those creatures from the spell's full force. To do so, you spend 1 sorcery point and choose a number of those creatures up to your Intelligence modifier (minimum of one creature). A chosen creature automatically succeeds on its saving throw against the spell.

\subsection{Distant Spell}

When you cast a spell that has a range of 5 feet or greater, you can spend 1 sorcery point to double the range of the spell.

When you cast a spell that has a range of touch, you can spend 1 sorcery point to make the range of the spell 30 feet.

\subsection{Empowered Spell}

When you roll damage for a spell, you can spend 1 sorcery point to reroll a number of the damage dice up to your Intelligence modifier (minimum of one). You must use the new rolls.

You can use Empowered Spell even if you have already used a different Metamagic option during the casting of the spell.

\subsection{Extended Spell}

When you cast a spell that has a duration of 1 minute or longer, you can spend 1 sorcery point to double its duration, to a maximum duration of 24 hours.

\subsection{Heightened Spell}

When you cast a spell that forces a creature to make a saving throw to resist its effects, you can spend 3 sorcery points to give one target of the spell disadvantage on its first saving throw made against the spell.

\subsection{Quickened Spell}

When you cast a spell that has a casting time of 1 action, you can spend 2 sorcery points to change the casting time to 1 bonus action for this casting.

\subsection{Subtle Spell}

When you cast a spell, you can spend 1 sorcery point to cast it without any somatic or verbal components.

\subsection{Twinned Spell}

When you cast a spell that targets only one creature and doesn't have a range of self, you can spend a number of sorcery points equal to the spell's level to target a second creature in range with the same spell (1 sorcery point if the spell is a cantrip).

To be eligible, a spell must be incapable of targeting more than one creature at the spell's current level. For example, \textit{magic missile* and *scorching ray* aren't eligible, but *ray of frost* and *chromatic or}* are.

\subsection{Ability Score Improvement}

When you reach 4th level, and again at 8th, 12th, 16th, and 19th level, you can increase one ability score of your choice by 2, or you can increase two ability scores of your choice by 1. As normal, you can't increase an ability score above 20 using this feature.

\subsection{Sorcerous Restoration}

At 20th level, you regain 4 expended sorcery points whenever you finish a short rest.

\subsection{Arcane Traditions}

Different arcanists claim different origins for their innate magic. Although many variations exist, most of these origins fall into two categories: a draconic bloodline and wild magic.

\subsection{Draconic Bloodline}

Your innate magic comes from draconic magic that was mingled with your blood or that of your ancestors. Most often, arcanists with this origin trace their descent back to a mighty arcanist of ancient times who made a bargain with a dragon or who might even have claimed a dragon parent. Some of these bloodlines are well established in the world, but most are obscure. Any given arcanist could be the first of a new bloodline, as a result of a pact or some other exceptional circumstance.

\subsection{Dragon Ancestor}

At 1st level, you choose one type of dragon as your ancestor. The damage type associated with each dragon is used by features you gain later.

\begin{DndTable}[header=Draconic Ancestry]{XX}
    
 Dragon & Damage Type \\
  Black  & Acid        \\
  Blue   & Lightning   \\
  Brass  & Fire        \\
  Bronze & Lightning   \\
  Copper & Acid        \\
  Gold   & Fire        \\
  Green  & Poison      \\
  Red    & Fire        \\
  Silver & Cold        \\
  White  & Cold        \\
\end{DndTable}

You can speak, read, and write Draconic. Additionally, whenever you make a Intelligence check when interacting with dragons, your proficiency bonus is doubled if it applies to the check.

\subsection{Draconic Resilience}

As magic flows through your body, it causes physical traits of your dragon ancestors to emerge. At 1st level, your hit point maximum increases by 1 and increases by 1 again whenever you gain a level in this class.

Additionally, parts of your skin are covered by a thin sheen of dragon-like scales. When you aren't wearing armor, your AC equals 13 + your Dexterity modifier.

\subsection{Elemental Affinity}

Starting at 6th level, when you cast a spell that deals damage of the type associated with your draconic ancestry, you can add your Intelligence modifier to one damage roll of that spell. At the same time, you can spend 1 sorcery point to gain resistance to that damage type for 1 hour.

\subsection{Dragon Wings}

At 14th level, you gain the ability to sprout a pair of dragon wings from your back, gaining a flying speed equal to your current speed. You can create these wings as a bonus action on your turn. They last until you dismiss them as a bonus action on your turn.

You can't manifest your wings while wearing armor unless the armor is made to accommodate them, and clothing not made to accommodate your wings might be destroyed when you manifest them.

\subsection{Draconic Presence}

Beginning at 18th level, you can channel the dread presence of your dragon ancestor, causing those around you to become awestruck or frightened. As an action, you can spend 5 sorcery points to draw on this power and exude an aura of awe or fear (your choice) to a distance of 60 feet. For 1 minute or until you lose your concentration (as if you were casting a concentration spell), each hostile creature that starts its turn in this aura must succeed on a Wisdom saving throw or be charmed (if you chose awe) or frightened (if you chose fear) until the aura ends. A creature that succeeds on this saving throw is immune to your aura for 24 hours.